\setuppapersize[A4]
\setuphead[section][style=\bfa]
\mainlanguage[de]
\setupbodyfont[times,10pt]
\starttext
\startpagemakeup[align=center,pagestate=start]
This should be the long title, Andreas' solution.
\stoppagemakeup

\startsection[title={Stetigkeit}]
Sei $(X, d)$ ein metrischer Raum und $(x_{n})$ eine Folge in $X$.
$(x_{n})$ heißt konvergent gegen $a \in X$, falls
\startformula
  \forall \varepsilon > 0, \forall N \in \mathbb{N}, \forall n \ge N\colon d(x_{n}, a) < \varepsilon.
\stopformula
Wir definieren $A \colonequals \bigcup_{n \in \mathbb{N}}{A_{n}}$ und $A_{0} = \emptyset$.  Wegen $A_{i -
  1} \subseteq A_{i}$ und $\sigma$-Additivität, es gilt
\startformula
  \mathbb{P}(A) = \sum_{i=1}^{\infty}{\mathbb{P}(A_{i} \setminus A_{i-1})}.
\stopformula
  Wir betrachten die partielle Summe von der Reihe, dann erhalten wir
  \startformula \sum_{i=1}^{\infty}{\mathbb{P}(A_{i} \setminus A_{i-1})} = \lim_{n \to \infty}{\sum_{i=1}^{n}{\mathbb{P}(A_{i}
        \setminus A_{i-1})}} = \lim_{n \to \infty}{\mathbb{P}(A_{n})}.\stopformula
    Damit gilt die \quotation{Behauptung}.
    \startquotation
      \input knuth
    \stopquotation
    \startformula
      \lVert f \rVert
    f(T), f\left( T \right),
    \int_{a}^{b} f\left( x \right) d x, \fraction{1}{T},
    \iiint_{a}^{b}
    \stopformula
\stopsection
\startsection[title={Einteilung der reellen Zahlen}]
  Zur Bezeichung der Menge aller reelen Zahlen wird das Symbol $\mathbb{R}$
  oder auch $\mathbf{R}$ verwendet.  Die reelen Zahlen umfassen
  \startitemize
  \item rationale Zahlen
    \startformula
      \mathbb{Q} = \left\{\ldots, -\frac21, -\frac12, 0, \frac11, \frac12, \ldots\right\}
      = 
      \{\frac{p}{q} \mid p \in \mathbb{Z}, q \in \mathbb{N} \setminus \{0\}\}
    \stopformula
  \stopitemize
  
\stopsection
\stoptext
