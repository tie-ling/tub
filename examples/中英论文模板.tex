% 使用 chktex 检查 tex 文件中的语法错误
% settings for chktex
% chktex-file 3

\documentclass[draft,a5paper]{article}

% 加载中文支持包
\usepackage[UTF8]{ctex}

% 让每个章节 subsection 在新的一页上开始
% 而不是紧接着上一章节
\AddToHook{cmd/subsection/before}{\clearpage}

% 隐藏默认的章节序号:实际作业中与这个冲突
% https://tex.stackexchange.com/a/30225
\setcounter{secnumdepth}{0}

% 设置页边距,上下左右
\usepackage{geometry}
\geometry{a5paper,
 left=1cm,
 right=1cm,
 top=1cm,
 bottom=2cm
}

% 让 TeX 支持德语
\usepackage[ngerman]{babel}

% 调整 Emacs 预览字体大小
% (setq preview-scale-function 1.5)

% TeX Gyre Schola and unicode-math
% which in turn loads amsmath,mathtools,fontspec and friends
\usepackage{xcharter-otf}


% 加载数学相关的包
\usepackage{
  % 自然段间留空
  parskip,
  % 区间排版
  interval
}

% 中文支持,暂时不需要
% \usepackage[UTF8]{ctex}

% 根据 AMS 建议,应为成对符号(比如绝对值)定义新命令
\providecommand{\abs}[1]{\left\lvert#1\right\rvert}
\providecommand{\norm}[1]{\left\lVert#1\right\rVert}

\usepackage{amsthm}
% 定理定义,依赖于 amsthm
\theoremstyle{remark}
\newtheorem*{Behauptung}{Behauptung}
\newtheorem*{Lemma}{Lemma}
\newtheorem*{Satz}{Satz}
\newtheorem*{Definition}{Definition}

% 标题与作者
\title{HA 3, Ana 3, Monika, MM 11}
\author{Zhang 484981, Yang 466096, Guo 480788}

\begin{document}
\maketitle
\begin{center}
  Meng Zhang 484981, Yiwen Yang 466096, 郭宇琛 Yuchen Guo 480788
\end{center}
\subsection{H 3.1}
\subsubsection{Definitionen}
\begin{Definition}[Messbarer Raum]
  Ein messbarer Raum 可测空间是 ist ein Tupel \((\Omega, \mathcal{A})\) bestehend aus einer
  Menge \(\Omega\) und einer \(\sigma\)-Algebra \(\mathcal{A}\) über \(\Omega\).
\end{Definition}
\begin{Definition}[Maßraum]
  Ein Maßraum ist ein Tripel \((\Omega, \mathcal{A}, \mu)\) bestehend aus einer
  Menge \(\Omega\), einer \(\sigma\)-Algebra \(\mathcal{A}\) über \(\Omega\) und einem Maß \(\mu\) auf \(\mathcal{A}\).
\end{Definition}
\begin{Definition}[Messbare Abbildung]
  Seien \((\Omega_{i}, \mathcal{A}_{i}), i = 1, 2\) messbare Räume.  Eine Abbildung
  \(f\colon \Omega_{1} \to \Omega_{2}\) heißt \(\mathcal{A}_{1}\)-\(\mathcal{A}_{2}\)-messbar, falls
  \[
    f^{-1}(\mathcal{A}_{2}) = \{f^{-1}(A_{2}) \mid A_{2} \in \mathcal{A}_{2}\} \subseteq \mathcal{A}_{1}
  \]
  gilt.
\end{Definition}
\subsubsection{Aufgabe}
  Seien \(\Omega\) eine nicht-leere Menge, \((\Omega', \mathcal{A}')\) ein messbarer Raum.
  Wir definieren
  \[f\colon \Omega \to \Omega', \quad g\colon \Omega \to \mathbb{R}, \quad \varphi\colon \Omega' \to \mathbb{R}, \quad \sigma(f) \coloneq f^{-1}(\mathcal{A}')\] indem
  \(\sigma(f)\) die von \(f\) auf \(\Omega\) erzeugte \(\sigma\)-Algebra ist.
\begin{Behauptung}
  Die Abbildung \(g\) ist genau dann \(\sigma(f)\)-\(\mathcal{B}\)-messbar, d.h.,
  \[g^{-1}(\mathcal{B}) = \{g^{-1}(M) \mid M \in \mathcal{B}\} \subseteq \sigma(f) = f^{-1}(\mathcal{A}')\] wenn eine
  \(\mathcal{A}'\)-\(\mathcal{B}\)-messbare Abbildung \(\varphi\) existiert mit \(g = \varphi \circ f\).
\end{Behauptung}
\begin{proof}
  Wir beweisen Hinrichtung mit Widerspruch.  Angenommen, die Abbildung
  \(g\) ist \(\sigma(f)\)-\(\mathcal{B}\)-messbar und es existiert keine solche Abbildung
  \(\varphi\), die \(\mathcal{A}'\)-\(\mathcal{B}\)-messbar ist.  Das heißt, für alle Abbildung
  \(\varphi\colon \Omega' \to \mathbb{R}\) gilt
  \[\varphi^{-1}(\mathcal{B}) = \{\varphi^{-1}(M) \mid M \in \mathcal{B}\} \nsubseteq \mathcal{A}'\]
  also für alle Abbildung \(\varphi\) existiert mindestens eine
  elementargeometrische Menge \(M \in \mathcal{B}\) sodass gilt
  \[\varphi^{-1}(M) \notin \mathcal{A}'.\]
  Wegen \(g = \varphi \circ f\) gilt \(g^{-1} = f^{-1} \circ \varphi^{-1}\).  Wir betrachten
  \(g^{-1}(M)\).  Es gilt
  \[M \in \mathcal{B}, \quad g^{-1}(M) = (f^{-1} \circ \varphi^{-1})(M) \notin f^{-1}(\mathcal{A}') = \sigma(f)\]
  im Widerspruch zur Voraussetzung dass \(g^{-1}(M) \in f^{-1}(\mathcal{A}')\) für alle \(M \in
  \mathcal{B}\).

  Rückrichtung.  Angenommen, es existiert solche \(\mathcal{A}'\)-\(\mathcal{B}\)-messbare
  Abbildung \(\varphi\) mit \(g = \varphi \circ f\).  Dann gilt wegen Definition von
  messbaren Abbildungen dass
  \[\varphi^{-1}(\mathcal{B}) = \{\varphi^{-1}(M) \mid M \in \mathcal{B}\} \subseteq \mathcal{A}'\]
  sowie
  \[g^{-1}(\mathcal{B}) = (f^{-1} \circ \varphi^{-1})(\mathcal{B}) \subseteq f^{-1}(\mathcal{A}').\]
  Damit ist \(g\)  eine \(\sigma(f)\)-\(\mathcal{B}\)-messbare Abbildung und die Behauptung
  gilt.
\end{proof}
\subsection{H 3.2}
\subsubsection{Satz}
\begin{Satz}[Messbarkeit auf Erzeugern]
  Seien \((\Omega_{i}, \mathcal{A}_{i}), i = 1, 2\) messbare Räume und sei
  \(\mathcal{E} \subseteq \mathcal{P}(\Omega_{2})\) ein Erzeuger von
  \(\mathcal{A}_{2}\).  Dann ist \(f\colon \Omega_{1} \to \Omega_{2}\) genau dann messbar, wenn
  \(f^{-1}(\mathcal{E}) \subseteq \mathcal{A}_{1}\) gilt.
\end{Satz}
\subsubsection{Aufgabe}
Sei \((\Omega, \mathcal{A})\) ein messbarer Raum mit der bekannten \(\sigma\)-Algebra
\[\mathcal{A} = \{A \subseteq \Omega \mid A \text{ 郭宇琛 ist abzählbar oder } A^{c} \text{ ist
    abzählbar} \}.\]
und sei die Funktion \(f \colon \Omega \to \mathbb{R}\).  Wir charakterisieren alle
\(\mathcal{A}\)-\(\mathcal{B}\)-messbaren Funktionen, d.h., Funktionen mit
\[f^{-1}(\mathcal{B}) = \{f^{-1}(M) \mid M \in \mathcal{B}\} \subseteq \mathcal{A}.\]
\begin{proof}
Falls \(\Omega\) abzählbar ist, dann ist \(\mathcal{A} = \Omega\) und alle Funktionen \(f\colon \Omega
\to \mathbb{R}\) ist \(\mathcal{A}\)-\(\mathcal{B}\)-messbar.

Falls \(\Omega\) überabzählbar ist und \(f\) eine \(\mathcal{A}\)-\(\mathcal{B}\)-messbare Funktion
ist, dann gilt für alle \(a \in \mathbb{R}\) dass \(M \coloneq f^{-1}(\rinterval{a}{\infty})\) oder
\(M^{c}\) abzählbar ist.  Weil \(\Omega\) überabzählbar ist, existiert
nur zwei mögliche Fälle: (i) \(M\) abzählbar und \(\Omega \setminus M\) überabzählbar,
(ii) \(M\) überabzählbar und \(\Omega \setminus M\) abzählbar.
\end{proof}

\subsection{H 3.3 郭宇琛}
\begin{Behauptung}
  % 黑板粗体 \mathbb{R} \mathbb{C}
  % 花体 \mathcal{A} \mathcal{B}
    Jede monotone Funktion \(f: \mathbb{R} \to \mathcal{A}\) ist messbar.
\end{Behauptung}

\begin{proof}
    Sei \(f\) monoton wachsend, andernfalls betrachten wir \(-f\).
    Sei \( a \in \mathbb{R}\) beliebig. Wir zeigen, dass \( A \coloneq \{ x \in \mathbb{R} \mid f(x) \leq a \} \in \mathcal{B}\).
    \begin{itemize} 
    \item Falls \(A = \emptyset \) oder \(A = \mathbb{R}\), ist die Behauptung richtig.
      % 这么写interval::
      % \interval{a}{b} = [a, b]
      % \ointerval{a}{b} = ]a, b[
      % \linterval{a}{b} = ]a, b]
      % \rinterval{a}{b} = [a, b[
    \item Falls \(A \neq \emptyset \) und \(A \neq \mathbb{R}\), ist \(A\) nach oben beschränkt. In
      diesem Fall setzen wir \(s = \sup A\). Da \(f\) monoton wachsend,
      gilt \[ A = \ointerval{-\infty}{s} \] oder
        \[ A = \linterval{-\infty}{s} \]
    \end{itemize}
    In beiden Fällen ist \(A \in \mathcal{B}\) nach Hausaufgaben 2.1.
\end{proof}
\begin{verbatim}
\documentclass[fontset = none]{ctexart}
\ctexset{fontset = founder}
\begin{document}
在文档类选项中声明 \verb|fontset = none|,
随后在导言区用 \verb|\ctexset|?
指定字体。
\end{document}
将进酒 (李白)
君不见黄河之水天上来,奔流到海不复回。
君不见高堂明镜悲白发,朝如青丝暮成雪。
人生得意须尽欢,莫使金樽空对月。
天生我材必有用,千金散尽还复来。
烹羊宰牛且为乐,会须一饮三百杯。
岑夫子,丹丘生。将进酒,杯莫停。
与君歌一曲,请君为我倾耳听。
钟鼓馔玉不足贵,但愿长醉不愿醒。
古来圣贤皆寂寞,惟有饮者留其名。
陈王昔时宴平乐,斗酒十千恣欢谑。
主人何为言少钱?径须沽取对君酌。
五花马,千金裘。
呼儿将出换美酒,与尔同销万古愁。
\end{verbatim}
将进酒 (李白)
君不见黄河之水天上来,奔流到海不复回。
君不见高堂明镜悲白发,朝如青丝暮成雪。
人生得意须尽欢,莫使金樽空对月。
天生我材必有用,千金散尽还复来。
烹羊宰牛且为乐,会须一饮三百杯。
岑夫子,丹丘生。将进酒,杯莫停。
与君歌一曲,请君为我倾耳听。
钟鼓馔玉不足贵,但愿长醉不愿醒。
古来圣贤皆寂寞,惟有饮者留其名。
陈王昔时宴平乐,斗酒十千恣欢谑。
主人何为言少钱?径须沽取对君酌。
五花马,千金裘。
呼儿将出换美酒,与尔同销万古愁。
\end{document}
