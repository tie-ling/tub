% Insert environment
\starttext
% Insert item in itemize environment
\startitemize
\item This is the first item.
\stopitemize
% Insert section

\section{First Level Title}

% Now this title should show up in Emacs menu bar,
% first add TOC to Emacs menu.

% Close environment
\startitemize
\stopitemize

% More key bindings available in C-h m.

% Insert macro
% Config.  AUCTeX defaults to ConTeXt mkii.  Set ConTeXt-Mark-version
% to "IV" for IV and above.

% Demo.  Electric dollar for inline math.  Press dollar key \$ for a
% pair of dollars, then place the cursor (point in Emacs jargon)
% between the dollars.

$a + b = c$.  Therefore we obtain $E = mc^{2}$, q.e.d.

% Demo.  Electric braces.  Automatically insert \{\} and \left[\right]
$ a \in \{a,b,c\}$ or $\left(\frac{1}{2} + \frac{1}{2}\right) = 1$.
% As you can see, left*right* pairs are electric, very helpful when
% entering such pairs in math environment.

% Demo.  Electric sub and superscript.
$ c^{ab}_{cd} $.  Automatically insert {} pairs immediately after sub
and superscripts, then place the cursor inside the braces.

% Demo.  Prettify symbols mode.  Display TeX math commands as Unicode
% symbols.
$ \alpha \beta \mathbb{R}$.

% a full list of prettified symbols can be viewed at
% tex--prettify-symbols-alist variable.

% Demo.  Keyboard shortcuts for math symbols.
% Press acute ` key, followed by another key, such as a, will type \alpha.
$ \alpha $
% A full list of all shortcuts is available in LaTeX-math-default
% variable.  Customizable via LaTeX-math-list variable.
% Alternatively, you can insert and view available math symbols from
% Emacs menu bar.
$\Subset$

% Config.  Emacs/AUCTeX does not recognize start/stop formula
% environment as math mode by default.  We need to fix this, or else
% math shortcuts will not work properly.

% Now we can insert a formula environment.
\startformula
  \alpha + \beta = \rho \gamma \sigma.
\stopformula

% Finally, compile the document with C-c C-c.
% Interrupt the compilation with C-c C-k.
% Type C-c C-c again to launch the PDF viewer.
% Type C-c C-l to view logs.

Happy \ConTeXt{}ing!
\stoptext
