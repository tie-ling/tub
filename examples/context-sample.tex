% "Hello world!" document for the ConTeXt typesetting system
%
% === History ===
% 2006-12-29  Sanjoy Mahajan  <sanjoy@mit.edu>
%     * Created
% 
% This document is the public domain (no copyright).

\setupcolors[state=start]       % otherwise you get greyscale
\definecolor[headingcolor][r=1,b=0.4]
\usemodule[newmat]

% for the document info/catalog (reported by 'pdfinfo', for example)
\setupinteraction[state=start,  % make hyperlinks active, etc.
  title={Hello world!},
  subtitle={A ConTeXt template},
  author={Sanjoy Mahajan},
  keyword={template}]

% useful urls
\useURL[author-email][mailto:a.u.thor@somewhere.edu][][a.u.thor@somewhere.edu]
\useURL[wiki][http://wiki.contextgarden.net][][\ConTeXt\ wiki]
\useURL[sanjoy][mailto:sanjoy@mit.edu][][sanjoy@mit.edu]

% for US paper; the sensible default is [A4][A4] (A4 typesetting,
% printed on A4 paper)
\setuppapersize[letter][letter]
\setuplayout[topspace=0.5in, backspace=1in, header=24pt, footer=36pt,
  height=middle, width=middle]
% uncomment the next line to see the layout
% \showframe

% headers and footers
\setupfooter[style=\it]
\setupfootertexts[\date\hfill \ConTeXt\ template]
\setuppagenumbering[location={header,right}, style=bold]

\setupbodyfont[times,11pt]            % default is 12pt

\setuphead[section,chapter,subject][color=headingcolor]
\setuphead[section,subject][style={\ss\bfa},
  before={\bigskip\bigskip}, after={}]
\setuphead[chapter][style={\ss\bfd}]
\setuphead[title][style={\ss\bfd},
  before={\begingroup\setupbodyfont[14.4pt]},
  after={\leftline{\ss\tfa A. U. Thor $\langle$\from[author-email]$\rangle$}
         \bigskip\bigskip\endgroup}]

\setupitemize[inbetween={}, style=bold]

% set inter-paragraph spacing
\setupwhitespace[medium]
% comment the next line to not indent paragraphs
\setupindenting[medium, yes]

\starttext

\title{Hello, world!}

Here is a hello-world template document to illustrates a few \ConTeXt\
features.  Have fun.  You can find a lot more information at
\from[wiki]; the preceding text should be colored and clickable, and
clicking it should take you to the wiki.

\subject{A list}

Here is an example of a list.

\startitemize[a]                % tags are lowercase letters
\item first
\item second
\item third
\stopitemize

\subject{Math}

An equation can be typeset inline like $e^{\pi i}+1=0$, or as a
displayed formula:
\startformula
\int_0^\infty t^4 e^{-t}\,dt = 24.
\stopformula
% don't use $$...$$ (the plain TeX equivalent)
You can also have numbered equations:
\placeformula[eq:factorial-example]\startformula
\int_0^\infty t^5 e^{-t}\,dt = 120.
\stopformula
And you can refer to them by name. I called the previous equation {\tt
factorial-example}, and it is equation \in[eq:factorial-example].
\ConTeXt\ figures out the number for you.  And with interaction turned
on, you can click on the equation number to get to the equation.

\subject{Text with figures}

Now text with a few figures.  The first figure goes on the right, with
the paragraph flowing around it.

\placefigure[right,none]{}{\externalfigure[dummy]}

\input tufte

The next figure will go inline, like a displayed formula:
\placefigure[here,none]{}{\externalfigure[dummy]}
\input tufte

Here's another reference to the numbered equation -- equation
\in[eq:factorial-example] on \at{page}[eq:factorial-example], so that
you can test clicking on it or on the page reference.

Inline math: $x = \frac{y}{2z} + x_{\text{center}}$.
\startformula
\delta_{ij} =
 \startmathcases
 \NC 1 \TC if \m{i = j} \NR
 \NC 0 \TC otherwise \NR
 \stopmathcases
\stopformula

Display math:
\startformula
q = \delta \frac{\partial p}{\partial x} = 
\delta(\phi) p_{vsat}(\theta) \frac{\partial \phi}{\partial x} = 
\left[ \frac{\delta_a}{\mu(\theta)} p_{vsat}(\theta) \right] \frac{\partial \phi}{\partial x} = 
k \frac{\partial \phi}{\partial x}
\stopformula
\definemathfunction [MyFunc]
\startformula
  \startalign
    \NC \tan \alpha a v w \nu g \NC = \pi \NR
    \NC \MyFunc \beta \NC = \ker \sigma \NR
  \stopalign
\stopformula


% most plain TeX commands work
\vfill

\noindent 
\framed[corner=round, width=\textwidth,height=1in,
backgroundcolor=gray,background=color]
{This document is in the public domain, so that you can improve it, share
it, and otherwise do what you want with it.  
Suggestions are welcome.  You can send them to me
at \from[sanjoy] (Sanjoy Mahajan).}

Electric math $a b c = d$.

Electric

\startformula
  \alpha \beta \Omega 
\stopformula
Demostration of AUCTeX LaTeX features in ConTeXt mode.

All ConTeXt mode functionality is configured by a single use-package
invocation.

Demo 1.  Electric math: key in single dollar sign for a pair of
dollars:
$a + b = c$.

Demo 2.  Prettify symbols minor mode:
$\alpha \beta \gamma \mathbb{P} \mathbb{Q} \subseteq \mathbb{R}$.

Demo 3. LaTeX-math minor mode.
Press accent key to access all mathematical symbols
$\alpha \beta \sigma \mathbb{R} $

LaTeX-math minor mode is customizable:

Symbols are also accessible via Menu (M-`) -> Math.

To get AUCTeX to recognize \startformula\ldots\stopformula pairs as math
environment, we need to change this variable:

\startformula
  \alpha + \beta = \gamma.
\stopformula

Demo 4.  Electric sub and super script:
$a_{b}^{c}=d$.

Demo 5.  Electric left and right brace:
$a \in \{ a, b, c\}$
$\frac12 \in \left\{ a, b, c\right\}$

$\left(\right)$
$\left[\right]$
$\left\{\right\}$
$\{\}$ for sets.



\stoptext
