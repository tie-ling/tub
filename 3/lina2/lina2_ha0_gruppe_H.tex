% !TeX program = LuaLaTeX
% 使用 LuaLaTeX,支持所有 Unicode 符号

% !TeX spellcheck = de_DE
% 用德语词典检查拼写

% settings for chktex
% chktex-file 3

% metadata, for compliance with PDF/A
% for more options, see document of pdfx package on ctan
\begin{filecontents*}{\jobname.xmpdata}
\Title{Hausaufgabe}
\Author{Guo 宇琛\sep Kuan \sep Yu}
\Language{de-DE}
\Keywords{math\sep assignment}
\Publisher{铁岭人民出版社}
\end{filecontents*}

% 使用 KOMA-Script 模板
\documentclass[12pt]{scrartcl}

% pdf-archive, must be the first package loaded, lowest version of
% PDF/A standard is recommended for maximum compatibility.
%
% this also loads hyperref, later configure hyperref with hypersetup
% command instead
% emacs preview-latex also depends on this package
%\usepackage[a-1b]{pdfx}
%\hypersetup{bookmarksnumbered,colorlinks=true}

% adjust emacs scaling
% (setq preview-scale-function 2)
% run C-x C-e here

% 使用德语标点
\usepackage[ngerman]{babel}

% 加载数学相关的包
\usepackage{amsmath,mathtools,fontspec,amsthm,interval,mathtools}

% 字体,使用便于阅读的 TeX Gyre Schola 字体,
% 基于 Century Schoolbook 设计。
\usepackage[libertinus]{fontsetup}

% theorems
\theoremstyle{remark}
\newtheorem*{Behauptung}{Behauptung}
\newtheorem*{Lemma}{Lemma}

\title{HA 0, LinA 2, Gruppe H}
\author{Kuan 480169, Yu 478912, Guo 480788}
\providecommand{\abs}[1]{\left\lvert#1\right\rvert}
\providecommand{\norm}[1]{\left\lVert#1\right\rVert}
\DeclareMathOperator{\Span}{Span}
\DeclareMathOperator{\Ker}{Ker}
\DeclareMathOperator{\Img}{Im}

\begin{document}
\maketitle
\begin{center}
  Ting Yu Kuan 480169, Shilong Yu 478912, Yuchen Guo 480788
\end{center}

\newpage

\section*{Injektiv, Surjektiv, Bijektiv}
\subsection*{Aufgabe 0.1}
Es gilt \(\sin(\mathbb{R}) = \interval{-1}{1}\), wegen [Amann, Analysis I,
Theorem III.6.16.vi].

\begin{enumerate}
\item \(f\colon \mathbb{R} \to \mathbb{R}, x \mapsto \sin(x)\) ist nicht injektiv, wegen
  \(f(0) = f(2\pi) = 0\).
  Die ist auch nicht surjektiv, denn es existiert kein
  \(a \in \mathbb{R}\) sodass \(f(a) >1\) gilt.
\item \(f\colon \mathbb{R} \to \interval{-1}{1}, x \mapsto \sin(x)\) ist nicht injektiv, wegen
  \(f(0) = f(2\pi) = 0\).
  Die ist surjektiv, wie oben erwähnt.
\item Es gilt
  \begin{align*}
    f\colon \ointerval{0}{\infty} \to \mathbb{R}, x \mapsto
  \log(x) = \frac{\log(x)}{\log(2)}
  \end{align*}
  und
  \begin{align*}
    \log(x)\coloneq\exp^{-1}(x).
  \end{align*}
  Es gilt auch, dass
  \(\exp(x)\colon \mathbb{R} \to \ointerval{0}{\infty}\) strikt monoton steigend ist.
  Daraus folgt, dass
  \(\log(x)\colon \ointerval{0}{\infty} \to \mathbb{R}\) strikt monoton steigend ist.  Daraus
  folgt, dass \(f\) bijektiv.
\item \(f\colon \mathbb{N} \to \mathbb{N}, n \mapsto 2n\) ist injektiv, denn aus
  \begin{align*}
    f(a) = f(b) = 2a = 2b, \quad a, b \in \mathbb{N}
  \end{align*}
  folgt, dass \(a = b\). Die ist nicht surjektiv, denn für ungerade
  \(a \in \mathbb{N}\) gibt es kein Urbild \(a/2 \in \mathbb{N}\).
\item \(f\colon \mathbb{R}^{2} \to \mathbb{R}, (x, y) \mapsto \frac{x}{y^{2}+1}\) ist nicht
  injektiv, denn es gilt \(f(0, 1) = f(0, -1) = 0\). Die ist
  surjektiv, denn für alle \(a \in \mathbb{R}\) gilt \(f(a, 0) = a\).
\end{enumerate}
\subsection*{Aufgabe 0.2}
\begin{enumerate}
\item \(f\colon \mathbb{R} \to \mathbb{R}, f(x) = x^{2}\) nicht injektiv, wegen \(f(x) =
  f(-x)\) für alle \(x \in \mathbb{R}\).  Nicht surjektiv, wegen \(f(\mathbb{R}) =
  \rinterval{0}{\infty}\).
\item
  \(g\colon (\mathbb{N} \setminus \{0\})^{2} \to \mathbb{N} \setminus \{0\}, g((a, b)) = kgv(a, b)\) nicht
  injektiv, wegen \(g((a, b)) = g((b, a))\) für alle
  \((a, b) \in (\mathbb{N} \setminus \{0\})^{2}\).  Surjektiv, denn sei \(b \in \mathbb{N}\setminus\{0\}\)
  beliebig.  Dann gilt \(g((1, b)) = b\).
\end{enumerate}
\begin{Behauptung}
  Eine bijektive Funktion hat ein eindeutiges Inverses.
\end{Behauptung}
\begin{proof}
  Sei \(A, B\) Mengen, \(f\colon A \to B\) bijektiv.  Dann existiert zu jedem
  \(y \in B\) genau ein \(x \in A\) mit \(f(x) = y\).  Wir definieren
  \(f^{-1}(y) = x\).  Damit ist \((f \circ f^{-1})(y) = y\) und \((f^{-1}
  \circ f)(x) = x\).  Damit haben wir die Existenz von dem Inverses
  gezeigt.

  Wir zeigen die Eindeutigkeit.  Angenommen, es existiert ein \(g\colon B
  \to A\) mit \(g \ne f^{-1}\), also es existiert mindestens ein \(y \in B\)
  sodass \(g(y) \ne f^{-1}(y)\).  Damit gilt \(g(y) \ne x\) und \((f \circ
  g)(y) \ne x \).  Also ist \(g\) kein Inverse von \(f\).
\end{proof}
\section*{Gruppen}
\subsection*{Aufgabe 0.3}
\begin{enumerate}
\item Definition von Gruppe.  Sei \(G\) eine Menge und \(+\colon G \times G \to G,
  (a, b) \mapsto (a + b)\) eine Funktion.  Dann ist \((G, +)\) eine Gruppe,
  falls gilt:
  \begin{itemize}
  \item Assoziativ, also \((a+b)+c = a+(b+c)\).
  \item Existenz des neutralen Elements.  Existiert \(e \in G\) mit \(e
    + a = a\) für alle \(a \in G\).
  \item Existenz des Inverses.  Zu jedem \(a \in G\) existiert \((-a) \in
    G\) mit \((-a) + a = e\).
  \item Falls \(+\) kommutativ ist, dann heißt die Gruppe abelsch.
  \end{itemize}
\item Behauptung.  \((\mathbb{Z}, +)\) ist eine Gruppe.
  \begin{proof}
    Für alle \(a, b, c \in \mathbb{Z}\) gilt
    \((a+b)+c=a+(b+c)=a+b+c\), also ist die Addition auf
    \(\mathbb{Z}\) kommutativ.  Es gilt für alle \(a\in \mathbb{Z}\) dass
    \(0+a=a\). Zu jedem \(a \in \mathbb{Z}\) existiert \((-a) \in \mathbb{Z}\) mit \((-a) +
    a = 0\).  Damit ist \((\mathbb{Z}, +)\) eine Gruppe.
  \end{proof}

\item Behauptung.  Für jedes \(k \in \mathbb{N}\) ist \(k\mathbb{Z} \coloneq \{kz \mid z \in \mathbb{Z}\}\)
  eine Untergruppe von \((\mathbb{Z}, +)\).
  \begin{proof}
    Sei \(k \in \mathbb{N}\) beliebig.  Zuerst gilt \(k\mathbb{Z} \subsetneq \mathbb{Z}\).  Sei \(a, b, c \in
    k\mathbb{Z}\).  Dann existiert \(p, q, r \in \mathbb{Z}\) mit \(a = kp, b = kq, c =
    kr\).

    Es gilt \((a+b)+c = a+(b+c) = k(p+q+r) \in k\mathbb{Z}\).  Es gilt
    \(0 \in k\mathbb{Z}\) und \(0+a = a\).  Es gilt
    \(-a = k(-p) \in k\mathbb{Z}\).  Damit gilt die Behauptung.
  \end{proof}
\item Sei \(k \in \mathbb{N}\).  Sei
  \(f\colon \mathbb{Z} \to k\mathbb{Z}\) eine Homomorphismus.  Also gilt
  \(f(a + b) = f(a) \oplus f(b)\) für alle \(a, b \in \mathbb{Z}\).  Sei
  \(f(1) = m\), dann existiert \(p \in \mathbb{Z}\) mit
  \(f(1) = m = kp\).  Weil wir alle \(a \in \mathbb{Z}\) durch einen Vielfach der
  Anwendung von Addition oder Subtraktion (Addition der additiven
  Inverses) von Eins erhalten kann, gilt
  \begin{align*}
    f(a+b) = f(a) \oplus f(b)
    = \underbrace{(f(1) \oplus \cdots \oplus f(1))}_{a\text{-mal}} \oplus \underbrace{(f(1) \oplus \cdots \oplus f(1))}_{b\text{-mal}}
   = (a+b) \cdot kp.
  \end{align*}
\end{enumerate}
\subsection*{Aufgabe 0.4}
Quotientengruppe?
\subsection*{Aufgabe 0.5}
Sei \((G, \oplus)\) eine Gruppe und \(U \subseteq G, U \ne \emptyset\).
\begin{Behauptung}
  \((U, \oplus)\) ist genau dann eine Untergruppe von \(G\), wenn \[R \coloneq
  \{(a, b) \in G^{2} \mid a \oplus b^{-1} \in U\}\] eine Äquivalenzrelation ist.
\end{Behauptung}
\begin{proof}
  Hinrichtung.  Sei \((U, \oplus)\) eine Untergruppe von \(G\).  Sei nun
  \((a, b) \in R\).  Wir betrachten die drei Kriterien von
  Äquivalenzrelation.

  \(R\) ist reflexiv.  Weil \(U\) eine Untergruppe ist, existiert für
  \(a \oplus b^{-1} \in U\) ein Inverse \((a \oplus b^{-1})^{-1} = b \oplus a^{-1} \in
  U\).  Damit gilt \((b, a) \in R\).

  \(R\) ist symmetrisch.   Weil \(U\) eine Untergruppe ist, gilt \(e \in
  U\).  Daraus folgt, dass \(a \oplus a^{-1} = e \in U\) für alle \(a \in G\).
  Damit gilt \((a, a) \in R\).

  \(R\) ist transitiv.  Sei \((a, b), (b, c) \in R\).  Wir zeigen, dass
  \((a, c) \in R\) gilt.  Wegen \((a, b) \in R\) gilt \(a \oplus b^{-1} \in U\).
  Wegen \((b, c) \in R\) gilt \(b \oplus c^{-1} \in U\).  Es gilt \(b^{-1} \oplus b
  \in U\).  Daraus folgt, dass
  \begin{align*}
    a \oplus c^{-1} = (a \oplus b^{-1}) \oplus (b \oplus c^{-1}) \in U.
  \end{align*}

  Rückrichtung.  Wir zeigen, dass die drei Kriterien von Gruppen
  erfüllt sind.  Dann gilt für alle \(a, b, c \in U \subseteq G\) dass
  \((a \oplus b) \oplus c = a \oplus (b \oplus c)\), weil \(G\) bereits eine Gruppe ist.
  Wegen der Symmetrie existiert \(e \in U\) dass \(e \oplus a = a\).  Wegen
  Reflexivität existiert \(a^{-1} \in U\) mit \(a \oplus a^{-1} = e\).
\end{proof}

\section*{Kern und Bild}
\subsection*{Aufgabe 0.6}
\begin{enumerate}
\item In Treppennormalform.
  \begin{align*}
    \begin{bmatrix}
      1 & 0 & 0 & -5/11 \\
      0 & 1 & 0 & 5/11 \\
      0 & 0 & 1 & 14/11 \\
      0 & 0 & 0 & 0
    \end{bmatrix}
  \end{align*}
  Der Bildraum einer Matrix ist der Spaltenraum.  Also, \[\Span([1, 0,
  0, 0], [0, 1, 0, 0], [0, 0, 1, 0]).\] Der Kern ist der Lösungsraum
  von \(A x = o\).  Dies gilt genau dann, wenn \[x \in \Span([5/11,
  -5/11, -14/11, 1])\] ist.
\item In Treppennormalform.
  \begin{align*}
    \begin{bmatrix}
      1 & 0 & -1 \\
      0 & 1 & 2 \\
      0 & 0 & 0
    \end{bmatrix}
  \end{align*}
  Der Bildraum einer Matrix ist der Spaltenraum.  Also,
  \(\Span([1, 0, 0], [0, 1, 0])\). Der Kern ist der Lösungsraum von
  \(A x = o\).  Dies gilt genau dann, wenn
  \(x \in \Span([1, -2, 1])\) ist.
\end{enumerate}
\subsection*{Aufgabe 0.7}
Sei \(A \in \mathbb{R}^{n \times n}\) eine Matrix, \(n \in \mathbb{N}\).
\begin{enumerate}
\item Sei \(B \in \mathbb{R}^{n \times n}\) invertierbar.
  \begin{Behauptung}
    Dann gilt
    \[\dim \Ker B A = \dim \Ker A = \dim \Ker A B.\]
  \end{Behauptung}
  \begin{proof}
    Sei \(x \in \Ker A\).  Weil \(B\) invertierbar ist, ist \(\Ker B =
    \{o\}\).  Also ist \(x \in \Ker B A\) genau dann, wenn \(x \in \Ker
    A\) gilt.  Damit ist die erste Gleichheit bewiesen.

    Wir zeigen, dass die zweite Gleichheit gilt.  Zuerst bemerken wir,
    dass \[\dim \Ker A B = \dim \Ker B^{-1} A.\]  Wegen \(B^{-1} \in
    \mathbb{R}^{n \times n}\) invertierbar, folgt die zweite Gleichheit aus der
    ersten Gleichheit.
  \end{proof}

\item Sei \(\tilde{A}\) eine Matrix, die aus \(A\) durch elementare
  Umformungen entsteht.
  \begin{Behauptung}
    Es gilt \(\dim \Ker A = \dim \Ker \tilde{A}\).
  \end{Behauptung}
  \begin{proof}
    Jede elementare Umformung ist invertierbar.  Dann ist die
    Komposition von alle solche Umformungen invertierbar.  Das heißt,
    es existiert \(B \in \mathbb{R}^{n \times n}\) invertierbar mit \(\tilde{A} = B
    A\).  Nun folgt die Behauptung aus Teil Eins dieses Aufgabes.
  \end{proof}
\item Sei \(\tilde{A}\) in Zeilenstufenform.  Bestimme \(\dim \Ker
  \tilde{A}\).
\item Beschreibe Algorithmus zur Berechnung von \(\dim \Ker A\).
\end{enumerate}
\section*{Vektorräume und Basen}
\subsection*{Aufgabe 0.8}
Ein Vektorraum \(V\) über dem Körper \(K\) ist eine Menge von Vektoren
mit zwei Abbildungen, Addition und skalare Multiplikation, für die die
folgende Regeln erfüllt sind: (i) \((V, +)\) kommutative Gruppe. (ii)
Für die skalaren Multiplikation gilt die distributiv Gesetze. (iii)
Die skalaren Multiplikation ist assoziativ. (iv) Existiert Einsvektor
\(1\) sodass \(1v=v\) für alle \(v \in V\).
\begin{enumerate}
\item \(V \coloneq \{(x, y) \in \mathbb{R}^{2} \mid 2x+y=0\}\) ist ein Vektorraum.
\item Gruppe \(\mathbb{Z}_{10}\) mit Skalierung durch Zahlen in \(\mathbb{Z}_{5}\)?
\item
  \(V \coloneq \{f \colon \mathbb{R} \to \mathbb{R} \mid f \text{ bijektiv}\}\) über Körper
  \(\mathbb{R}\), mit Komposition als Vektoraddition, ist kein Vektorraum. (Tut
  05 L).
\item Die Gruppe \((\mathbb{R}, +)\) mit Skalierung durch Zahlen in \(\mathbb{Q}\)?
\end{enumerate}
\subsection*{Aufgabe 0.9}
\begin{enumerate}
\item Basen für \(\{f \in \mathbb{R}[x] \mid \deg(f) \le 4\}\) über \(\mathbb{R}\): \[\{1, x,
    x^{2}, x^{3}, x^{4}\}, \{3, x-1, x^{2}+2, 3x^{3} - 5, 4x^{4}\}\]
\item Basen für \(\mathbb{R}^{2}\) über \(\mathbb{R}\): \[\{(1, 0), (0, 1)\}, \{(\pi, e),
    (e, \pi)\}\]
\item Basen für \(\{(x, y, z) \in \mathbb{R}^{3} \mid x + 2y - z = 0\}\) über \(\mathbb{R}\).
  Wegen \(\dim \Ker A = \dim \mathbb{R}^{3} - \dim \Img A = 2\), sind die
  Basen:
  \[\{(1, 0, 1), (0, 1, 2)\}, \{(2, -1, 0), (0, 1, 2)\}\]
\end{enumerate}
\subsection*{Aufgabe 0.10}
Sei \(V\) ein \(\mathbb{C}\)-Vektorraum mit Basis \([b_{1}, \ldots, b_{n}]\).  Indem
wir die Skalarmultiplikation auf Zahlen in \(\mathbb{R}\) einschränken, wird
\(V\) zu einem \(\mathbb{R}\)-Vektorraum.  Finde eine Basis von \(V\) als
\(\mathbb{R}\)-Vektorraum.

\section*{Dimensionsformeln}
\subsection*{Aufgabe 0.11}
\begin{Behauptung}
  Sei \(V\) ein endlich dimensionaler \(K\)-Vektorraum und sei \(U\)
  ein Unterraum von \(V\).  Angenommen, \(\dim U = \dim V\), dann
  folgt \(U = V\).
\end{Behauptung}
\begin{proof}
  Beweis durch Widerspruch.  Wir nehmen an, dass \(U \ne V\) gilt.  Sei
  \(B_{V}\) eine Basis von \(V\).  Wegen \(U \subsetneq V\), existiert eine echte
  Teilmenge \(B_{U}\) von \(B_{V}\), sodass \(B_{U}\) eine Basis von
  \(U\) ist.  Daraus folgt, dass \(\dim U < \dim V\).  Widerspruch.
\end{proof}
\begin{proof}
  Alternativer Beweis.
\end{proof}
\subsection*{Aufgabe 0.12}
\begin{enumerate}
\item Bestimme alle lienaren Abbildungen von \(\mathbb{R}\) nach \(\mathbb{R}\).

  Behauptung.  Sei \(f\colon \mathbb{R} \to \mathbb{R}\) linear.  Dann muss \(f\colon x \mapsto px, p \in
  \mathbb{R}\) sein.

  Beweis?  Denn nur Matrizen mit einer Eintrag ist linear.
  \[
    \begin{bmatrix}
      p
    \end{bmatrix}
    \cdot x =
    \begin{bmatrix}
      px
    \end{bmatrix}.
  \]
\item Wir suchen nach eine lineare Abbildung \(f\colon \mathbb{R}^{2} \to \mathbb{R}^{2}\) mit
  \(\Img f = \Ker f\).  Wegen \[\dim \Ker f + \dim \Img f = \dim
    \mathbb{R}^{2} = 2 \]
  folgt, dass \(\dim \Ker f = \dim \Img f = 1\).  Wir betrachten die
  Abbildung
  \[
    f\colon
    \begin{bmatrix}
      1 & 0 \\
      0 & 0
    \end{bmatrix}
    \times
    \begin{bmatrix}
      a \\ b
    \end{bmatrix}
  \]
  dann gilt \(\Ker f = \Span\{(0, 1)\}\) und \(\Img f = \Span\{(1,
  0)\}\).  Damit folgt die Behauptung.

\item Wir können eine lineare Abbildung \(\mathbb{R}^{3} \to \mathbb{R}^{3}\) mit oben
  genannten Eigenschaften nicht finden.
\end{enumerate}
\section*{Matrizen und Koordinatentransformation}
\subsection*{Aufgabe 0.13}
\begin{align*}
  T_{A \to B} =
  \begin{bmatrix}
    0 & 1 & 1 & 0 \\
    1 & 0 & 0 & 1 \\
    1 & 0 & 1 & 1 \\
    0 & 0 & 1 & 1
  \end{bmatrix},
  \quad
  T_{B \to A} =
  \begin{bmatrix}
    0 & 2 & -1 & -1 \\
    1 & 1 & -1 & 0 \\
    0 & -1 & 1 & 0 \\
    0 & 1 & -1 & 1
  \end{bmatrix}.
\end{align*}
\subsection*{Aufgabe 0.14}
Bestimmen bezügilch der Standardbasis \(\{(1, 0), (0, 1)\}\) im
\(\mathbb{R}^{2}\) die Matrix der linearen Abbildung
\(f\colon \mathbb{R}^{2} \to \mathbb{R}^{2}\), die jedes \(u \in \mathbb{R}^{2}\)
\begin{enumerate}
\item an der \(x\)-Achse spiegelt, \((x, y) \mapsto (x, -y)\),

  um \(\pi/2\) im Uhrzeigersinn dreht, \((x, -y) \mapsto (-y, -x)\),

  in Richtung der \(y\)-Achse um den Faktor \(a\) streckt, \((-y, -x)
  \mapsto (-y, -ax)\)

  und dann erneut an der \(x\)-Achse spiegelt, \((-y, -ax) \mapsto (-y,
  ax)\).
  \[
    \begin{bmatrix}
      0 & -1 \\
      a & 0
    \end{bmatrix}
  \]
\item um \(\pi\) dreht, \((x, y) \mapsto (-x, -y)\),

  in Richtung der \(x\)-Achse um den Faktor \(a\), \((-x, -y) \mapsto (-ax,
  -y)\)

  in Richtung der \(y\)-Achse um den Faktor \(b\) streckt, \((-ax,
  -y) \mapsto (-ax, -by)\).
  \[
    \begin{bmatrix}
      -a & 0 \\
      0 & -b
    \end{bmatrix}
  \]
\end{enumerate}
\subsection*{Aufgabe 0.15}
\begin{enumerate}
\item \(f: \mathbb{R}^{4} \to \mathbb{R}^{4}, f(a, b, c, d) = (4a+b, -c-3d, 6a-d,
  a+b+c+d)^{\top}\)
  \[
    \begin{bmatrix}
      4 & 1 & 0 & 0 \\
      0 & 0 & -1 & -3 \\
      6 & 0 & 0 & -1 \\
      1 & 1 & 1 & 1
    \end{bmatrix}
  \]
\item \(f\colon \mathbb{R}^{3} \to \mathbb{R}^{5}, f(x, y, z) = (x, z, y, x + z, x + y)^{\top}\)
  \[
    \begin{bmatrix}
      1 & 0 & 0 \\
      0 & 0 & 1 \\
      0 & 1 & 0 \\
      1 & 0 & 1 \\
      1 & 1 & 0
    \end{bmatrix}
  \]
\item \(f\colon \mathbb{R}[x]_{n} \to \mathbb{R}[x]_{n}, n \in \mathbb{N}\),
  \[f\left(\sum_{i = 0}^{n}{p_{i}x^{i}}\right) =
    \sum_{i=1}^{n}(p_{i}+p_{i-1})x^{i} + (p_{0} + p_{n})\]
  mit Basis \([1, x, x^{2}, \ldots, x^{n}]\).
  \[
    \begin{bmatrix}
      1 & 0 & 0 & 0 & \cdots & 0 & 1 \\
      1 & 1 & 0 & 0 & \cdots & 0 & 0 \\
      0 & 1 & 1 & 0 & \cdots & 0 & 0 \\
      \vdots \\
      0 & 0 & 0 & 0 & \cdots & 1 & 1
    \end{bmatrix}
    \in \{0, 1\}^{(n+1) \times (n+1)}.
  \]
\item \(f\colon \mathbb{R}[x]_{n} \to \mathbb{R}[x]_{n}, n \in \mathbb{N}\),
  \[f\left(\sum_{i=0}^{n}p_{i}x^{i}\right) = p_{0} + \cdots + p_{n}\]
  mit Basis \([1-x, x-x^{2}, x^{2} - x^{3}, \ldots, x^{n-1} - x^{n},
  x^{n}]\).
  [Tut 09].
\end{enumerate}
\section*{Determinanten}
\subsection*{Aufgabe 0.16}
\begin{enumerate}
\item \(2\).
\item Nicht invertierbar. \(0\).
\item \(ad-bc\).
\end{enumerate}
\subsection*{Aufgabe 0.17}
Berechnung des Determinants mittels elementaren Umformungen:
\begin{align*}
  \begin{bmatrix}
    1 & 2 & 1 \\
    0 & 2 & 1 \\
    -1 & 0 & 1
  \end{bmatrix} \to
  \begin{bmatrix}
    1 & 2 & 1 \\
    0 & 2 & 1 \\
    0 & 2 & 2
  \end{bmatrix} \to
  \begin{bmatrix}
    1 & 2 & 1 \\
    0 & 2 & 1 \\
    0 & 0 & 1
  \end{bmatrix}
\end{align*}
Wegen [Kowalsky, Satz 5.4.5] gilt \(\det A = 1 \times 2 \times 1 = 2\).
Mittels Laplace-Entwicklung an der zweiten Spalte:
\begin{align*}
  \det A = 2 \cdot (-1)^{1+2}\cdot 1 + 2 \cdot (-1)^{2+2} \cdot 2 = 2.
\end{align*}
\subsection*{Aufgabe 0.18}
Eine lineare Abbildung ist genau dann invertierbar, falls der Kern nur
aus Nullvektor besteht, genau dann, falls der Determinant ungleich
Null ist.
\end{document}
