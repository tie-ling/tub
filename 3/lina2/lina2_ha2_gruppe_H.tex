% 使用 chktex 检查 tex 文件中的语法错误
% settings for chktex
% chktex-file 3

\documentclass[draft,a5paper]{article}

\input{../../preamble.tex}


\usepackage{amsthm}
% 定理定义,依赖于 amsthm
\theoremstyle{remark}
\newtheorem*{Behauptung}{Behauptung}
\newtheorem*{Lemma}{Lemma}
\newtheorem*{Satz}{Satz}
\newtheorem*{Definition}{Definition}

% 定义新函数,依赖于AMSmath
\DeclareMathOperator{\card}{card}
\DeclareMathOperator{\Span}{Span}
\DeclareMathOperator{\Ker}{Ker}
\DeclareMathOperator{\Img}{Im}
\DeclareMathOperator{\rg}{rg}

% 画图软件
\usepackage{tikz}
\usetikzlibrary{matrix}

% 标题与作者
\title{HA 2, LinA 2, Gruppe H}
\author{Kuan 480169, Yu 478912, Guo 480788}

\begin{document}
\maketitle
\begin{center}
  Ting Yu Kuan 480169, Shilong Yu 478912, Yuchen Guo 480788
\end{center}
\subsection{Aufgabe 2.1}
Wir zeigen, ob die Matrizen diagonalisierbar sind.  Falls nicht, dann
trigonalisieren wir sie.  Eine Matrix ist genau dann diagonalisierbar,
falls (1) das charakteristische Polynom \(\chi\) in Linearfaktoren zerfällt
und (2) die geometrische Vielfachheit gleich der algebraischen
Vielfachheit ist.

\subsubsection{Aufgabe 2.1, Teil a}
Sei die Matrix
\begin{align*}
  A \coloneq
  \begin{bmatrix}
    3 & 4 & 3 \\
    -1 & 0 & -1 \\
    1 & 2 & 3
  \end{bmatrix}.
\end{align*}
Dann ist das charakteristische Polynom
\[\chi_{A}(t) = \det(E \cdot t - A) =(t-2)^{3}. \] Damit zerfällt die
Matrix in Linearfaktoren und \(\lambda = 2\) ist das einzigen Eigenwert von
\(A\).  Wir zeigen, dass die geometrische Vielfachheit nicht gleich
der algebraischen Vielfachheit ist.  Aus
\begin{align*}
  \begin{bmatrix}
    -1 & -4 & -3 \\
    1 & 2 & 1 \\
    -1 & -2 & -1
  \end{bmatrix}
  \cdot
  \begin{bmatrix}
    x_{1} \\ x_{2} \\ x_{3}
  \end{bmatrix}
  =
  \begin{bmatrix}
    0 \\ 0 \\ 0
  \end{bmatrix}
\end{align*}
folgt der einzige Eigenvektor \([1, -1, 1]\).  Damit ist
\[\dim \Ker (E \cdot 2 - A)  = 1 \ne 3\]
und die Matrix \(A\) ist nicht diagonalisierbar.  Sie ist aber
trigonalisierbar.

Trigonalisierbarverfahren, erster Schritt.  Sei \[B_{1} \coloneq \{[1, 0,
  0], [0, 1, 0], [0, 0, 1]\}\] eine Basis der Vektorraum \(\mathbb{R}^{3}\).
Nach dem Austauschlemma tauschen wir der ersten Basisvektor gegen
den einzigen Eigenvektor \([1, -1, 1]\), sodass
\[B_{2} \coloneq \{[1, -1, 1], [0, 1, 0], [0, 0, 1]\}\]
wieder eine Basis von \(\mathbb{R}^{3}\) ist.  Wir betrachten die
Transformationsmatrix
\[T_{1}^{-1} \coloneq T_{B_{1}}^{B_{2}}\]
mit der Basis \(B_{2}\) als Spalten.  Dann ist
\begin{equation*}
  A_{2} \coloneq T_{1} \cdot A \cdot T_{1}^{-1} =  \begin{tikzpicture}[baseline={(m.center)}]
    \matrix [matrix of math nodes,left delimiter={[},right delimiter={]}] (m)
    {
      2 & 4 & 3 \\
      0 & 4 & 2 \\
      0 & -2 & 0 \\
    };
    \draw (m-2-2.north west) rectangle (m-3-2.south east-|m-2-3.north east);
  \end{tikzpicture}.
\end{equation*}
Trigonalisierbarverfahren, zweiter Schritt.  Wir bezeichnen die
Matrix im Rechteck als \(A_{2}'\) und berechnen deren
charakteristischen Polynom als
\[\chi_{A_{2}} = (t-2)^{2}\]
mit Eigenvektor \([-1,1]\).  Dann ist
\[B_{3} \coloneq \{[1, -1, 1], [0, -1, 1], [0, 0, 1]\}\]
wieder eine Basis von \(\mathbb{R}^{3}\) ist.  Wir betrachten die
Transformationsmatrix
\[T_{2}^{-1} \coloneq T_{B_{2}}^{B_{3}}\]
mit der Basis \(B_{3}\) als Spalten.
Damit erhalten wir
\[ A_{3} \coloneq T_{2} \cdot A_{2} \cdot T_{2}^{-1}  =
  \begin{bmatrix}
    2 & 1 & 3 \\
    0 & 2 & 2 \\
    0 & 0 & 2
  \end{bmatrix}.
\]
\subsubsection{Aufgabe 2.1, Teil b}
Sei die Matrix
\begin{align*}
  B \coloneq
  \begin{bmatrix}
    0 & 0 & -1 \\
    1 & 3 & 1 \\
    2 & 0 & 2
  \end{bmatrix}.
\end{align*}
Dann ist das charakteristische Polynom
\[\chi_{B}(t) = \det(E \cdot t - B) =(t-3)(t^{2}-2t+2) \] mit
Lösungen \[t= \frac{2\pm \sqrt{-4}}{2}\] und daher nicht trigonalisierbar
oder diagonalisierbar.
\subsection{Aufgabe 2.2}
Sei die Matrix
\begin{align*}
  A_{x} =
  \begin{bmatrix}
    0 & 0 & 1 \\
    0 & 0 & 1 \\
    x & 1 & 1
  \end{bmatrix}
  \in \mathbb{R}^{3 \times 3}, \quad \chi_{A}(t) \coloneq t(t^{2}-t-x+1).
\end{align*}
Wir berechnen alle reelle Lösungen von \(\chi_{A}(t)=0\).  Der zweite
Faktor
\[t^{2}-t-x+1=0\] hat genau reelle Lösungen bzw. Eigenwerte
\[p,q=\frac{1\pm\sqrt{1-4(-x+1)}}{2}\] falls \(1-4(-x+1) \ge 0\) gilt.
In diesem Fall gilt \(x \ge 3/4\).
Falls \(x < 3/4\) dann zerfällt das
charakteristische Polynom nicht in Linearfaktoren und die Matrix ist
nicht diagonalisierbar oder trigonalisierbar.

Falls \(x \ge 3/4\).  Dann zerfällt das charakteristische Polynom in
Linearfaktoren und die Matrix ist trigonalisierbar.  Wir untersuchen
deren Diagonalisierbarkeit.

Nun sei \(\chi_{A}(t)=t(t-p)(t-q)\).  Wir berechnen die algebraische und
geometrische Vielfachheit.  Falls \(p \ne 0, q \ne 0, p \ne q\).  Dann
\begin{align*}
  \dim \Ker (E \cdot p - A_{x})
  &= \dim \mathbb{R}^{3} - \dim \Img (E \cdot p - A_{x}) \\
  &=3 - \rg
    \begin{bmatrix}
      p & 0 & -1 \\
      0 & p & -1 \\
      -x & -1 & p - 1
    \end{bmatrix}.
\end{align*}
Wir zeigen, dass die dritte Zeile eine lineare Kombination von der
anderen zwei Zeilen ist.  Denn sei
\[a
  \begin{bmatrix}
    p & 0 & -1
  \end{bmatrix}
  + b
  \begin{bmatrix}
    0 & p & -1
  \end{bmatrix}
  =
  \begin{bmatrix}
    -x & -1 & p-1
  \end{bmatrix}.
\]
erhalten wir
\[p^{2}-p-x-1=0; \quad p = \frac{1 \pm \sqrt{1-4(-x-1)}}{2}\] welche mit
\(x \ge -5/4\) lösbar ist.  Wegen Voraussetzung dass \(x \ge 3/4\) ist diese
Bedingung erfüllt.  Damit ist
\[\dim \Ker (E \cdot p - A_{x}) = 3 - 2 = 1\]
und die algebraische Vielfachheit und die geometrische Vielfachheit
sind gleich.  Die Matrix ist diagonalisierbar.  Den Fall mit \(q \ne 0\)
ist analog.

Falls \(p = q = 1/2\).  Dann \(x = 3/4\) und
\[\dim \Ker (E \cdot 1/2 - A_{3/4}) = \dim \mathbb{R}^{3} - \dim \Img (E \cdot 1/2 -
  A_{3/4}) = 0 \ne 1.\]  Damit nicht diagonalisierbar.

Falls \(p = 0, q = 1\). Dann \(x = 1\) und
\[\dim \Ker (E \cdot 1 - A_{1}) = \dim \mathbb{R}^{3} - \dim \Img (E \cdot 1 -
  A_{1}) = 0 \ne 1.\] Damit nicht diagonalisierbar.
\subsection{Aufgabe 2.3}
Satz.  Ähnliche Matrizen beschreibt dieselben lineare Abbildungen und
daher besitzen die dieselben Eigenwerte.
Seien
\[A =
  \begin{bmatrix}
    1 & 2 & 2 \\
    0 & 0 & 2 \\
    0 & 2 & 0
  \end{bmatrix},
  \quad
  B =
  \begin{bmatrix}
    2 & 2 & 1 \\
    2 & 2 & 2 \\
    2 & 1 & 0
  \end{bmatrix}
\]
zwei Matrizen.
\subsubsection{Aufgabe 2.3, Teil a}

Seien \(A, B \in \mathbb{R}^{3\times3}\).  Die zwei Matrizen sind nicht ähnlich,
denn ihre Eigenwerte stimmen nicht überein.  Es gilt
\[\chi_{A} = (\lambda-2)(\lambda-1)(\lambda+2); \quad \chi_{B}= \lambda^{3}-4\lambda^{2}-4\lambda-2.\]
Damit sind die Eigenwerte verschieden und die Matrizen \(A\) und \(B\)
sind wie oben erwähnt nicht ähnlich.
\subsection{Aufgabe 2.4}
Sei \(V\) der \(\mathbb{R}\)-Vektorraum der auf \(\mathbb{R}\) unendlich oft diffbaren
Funktionen.  Weiter sei
\[\Phi\colon V \to V, \quad f(x) \mapsto Df(x)\]
sowie
\[U= \Span(\{1, x, x^{2}, x^{3}, \sin(x) - 1, 2\cos(x), x^{3} -
  \sin(x) + 3 \cos(x)\}).\]
\subsubsection{Aufgabe 2.4, Teil a}
\begin{Behauptung}
  Es gilt \(\Phi(U) \subseteq U\).
\end{Behauptung}

\begin{proof}
  Sei \(v \in U\) beliebig.  Dann gilt
  \begin{align*}
    v = &~ \alpha_{0} \cdot 1 + \alpha_{1} x + \alpha_{2} x^{2} + \alpha_{3} x^{3} \\
        &+ \alpha_{4} (\sin  x - 1) + \alpha_{5} (2 \cos x) + \alpha_{6} (x^{3} -
          \sin x + 3 \cos x)
  \end{align*}
  sowie wegen Ergebnisse aus Analysis
  \begin{align*}
    \Phi(v) = &~ \alpha_{1} + 2 \alpha_{2} x + 3 \alpha_{3} x^{2} \\
           &+ \alpha_{4} \cos x + \alpha_{5}  (-2 \sin x) + \alpha_{6} (3x^{2} - \cos
             x + -3 \sin x).
  \end{align*}
  Damit ist \(\Phi(v)\) wieder eine lineare Kombination von Basisvektoren
  von \(U\) und \(\Phi(v) \in U\).
\end{proof}
\subsubsection{Aufgabe 2.4, Teil b}
Wir wählen Basis von \(U\)
\[B \coloneq \{1, x, x^{2}, x^{3}, \sin x, \cos x\}\]
\begin{proof}
  Daraus folgt, dass
  \begin{align*}
    M(\Phi\vert_{U}) \coloneq
    \begin{bmatrix}
      0 & 0 & 0 & 0 & 0 & 0\\
      0 & 1 & 0 & 0 & 0 & 0 \\
      0 & 0 & 2 & 0 & 0 & 0\\
      0 & 0 & 0 & 3 & 0 & 0\\
      0 & 0 & 0 & 0 & 0 & -1 \\
      0 & 0 & 0 & 0 & 1 &  0
    \end{bmatrix}.
  \end{align*}
\end{proof}
\subsubsection{Aufgabe 2.4, Teil c}
Das charakteristische Polynom ist
\begin{align*}
  \chi(t) = t(t-3)(t-2)(t-1)(t^{2}+1)
\end{align*}
\subsection{Zusatzaufgabe 2.5}
Seien \(a(0) = x, a(1) = y, a(2) = z, x,y,z \in \mathbb{R}\) und
\[a(n) = 2a(n-1) + a(n-2) - 2a(n-3), \quad n \in \mathbb{N}, n \ge 2.\]
\begin{Behauptung}
  Es gilt
  \[a(n) = 2^{n-1}(z-y) + 2y -z.\]
\end{Behauptung}
\begin{proof}
  Wegen Voraussetzung gilt
  \[a(n) - 2a(n-1) = a(n-2) - 2a(n-3) = \cdots = a(2) - 2a(1) = z - 2y.\]
  Wir setzen \(p \coloneq (z - 2y)\).  Daraus folgt, dass
  \[a(n) = p + 2a(n-1).\]
  Es gilt
  \begin{align*}
    a(2) &= p + 2a(1) \\
    a(3) &= p + 2a(2) = p + (p + 2a(1)) \\
    a(4) &= p + 2a(3) = p + 2(p + 2a(2)) = p + 2(p + 2(p + 2a(1))) \\
         &= 2^{4-1}a(1) + p(1+2(1+2)).
  \end{align*}
  Die Folge \((b_{n})\) mit \(b_{1} = 1, b_{n} = 1 + 2b_{n-1}\) hat die
  folgende Formel
  \[b_{n} = 2^{n} - 1.\]
  Induktionsanfang \(n = 1\).  Die Behauptung gilt.
  Induktionsvoraussetzung, die Behauptung gelte für ein festes \(n \in
  \mathbb{N}\).  Induktionsschritt \(n \to n+1\).  Es gilt
  \[b_{n+1} = 1 + 2b_{n} = 1 + 2(2^{n} - 1) = 2^{n+1} - 2 + 1.\] Damit
  gilt die Formel.  Daraus folgt
  \begin{align*}
        a(n) &= p + 2a(n-1) \\
         &= 2^{n-1}a(1) + p \cdot b_{n-1} \\
         &= 2^{n-1}y + (2^{n-1} - 1)p \\
         &= 2^{n-1}y + 2^{n-1}(z-2y) + 2y - z \\
         &= 2^{n-1}(z-y) + 2y -z.
  \end{align*}
\end{proof}
\end{document}
