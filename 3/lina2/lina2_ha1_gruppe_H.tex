% settings for chktex
% chktex-file 3
\documentclass[draft,a5paper]{article}
%\usepackage[a-1b]{pdfx}
%\hypersetup{bookmarksnumbered,colorlinks=true}
% https://tex.stackexchange.com/a/30225
\setcounter{secnumdepth}{0}
% bolder computer modern
\usepackage{xcharter-otf}
\usepackage{xcharter-otf}

\usepackage[a5paper, margin=2cm]{geometry}

% adjust emacs scaling
% (setq preview-scale-function 2)
% run C-x C-e here

% 使用德语标点
\usepackage[ngerman]{babel}

% 加载数学相关的包
\usepackage{amsmath,mathtools,fontspec,amssymb,amsthm,interval,mathtools}

% theorems
\theoremstyle{remark}
\newtheorem*{Behauptung}{Behauptung}
\newtheorem*{Lemma}{Lemma}

% math commands
\providecommand{\abs}[1]{\left\lvert#1\right\rvert}
\providecommand{\norm}[1]{\left\lVert#1\right\rVert}
\DeclareMathOperator{\Span}{Span}
\DeclareMathOperator{\Ker}{Ker}
\DeclareMathOperator{\Img}{Im}
\DeclareMathOperator{\rg}{rg}


\DeclareMathOperator{\id}{id}
\DeclareMathOperator{\tr}{tr}
\DeclareMathOperator{\charPol}{char\,Pol}

\usepackage{tikz,pgfplots}
\pgfplotsset{
  compat=1.18,
  every axis/.append style={
    axis x line=middle,
    axis y line=middle,
    xlabel={\(x\)},
    ylabel={\(y\)},
    axis line style={->},
  },
  grid style={dotted,gray},
}
\begin{document}
\maketitle
\begin{center}
  Ting Yu Kuan 480169, Shilong Yu 478912, Yuchen Guo 480788
\end{center}

\newpage
\subsection*{Aufgabe 1.1}
Sei \(A_{x, y} \in \mathbb{C}^{3\times3}\) eine Matrix mit
\begin{align*}
  \begin{bmatrix}
    1 & 0 & 1 \\
    0 & 1 & -1 \\
    0 & x & y
  \end{bmatrix}.
\end{align*}
\begin{enumerate}
\item Das charakteristische Polynom von \(A_{x,y}\) ist bestimmt durch
  \begin{align*}
    \charPol_{A_{x, y}}(p) &= \det(E \cdot p - A_{x, y}) = \det
    \begin{bmatrix}
      p - 1 & 0 & -1 \\
      0 & p - 1 & 1 \\
      0 & -x & p - y
    \end{bmatrix} \\
                           &= (p-1)^{2}(p-y) + (p-1)x \\
                           &= (p-1)[p^{2} - (y+1)p + x + y].
  \end{align*}
\item Genau dann ist \(f \in \mathbb{C}\) ein Eigenwert der Matrix \(A\), wenn \(f\)
  eine Nullstelle des charakteristischen Polynomes \(\charPol_{A}(p)\)
  von \(A\) ist.  Daraus folgt, dass
  \[1, \quad \frac{(1+y) \pm \sqrt{(1+y)^{2} - 4(x+y)}}{2}\]
  die Eigenwerte der Matrix \(A\) ist.
\item Wir bestimmen alle Paare \((x, y) \in \mathbb{R}^{2}\), sodass die Matrix \(A\)
  über \(\mathbb{R}\) nur den Eigenwert Eins hat.  Dazu betrachten wir die
  Lösungsmenge von
  \[(1+y)^{2} - 4(x+y) = y^{2} - 2y - 4x + 1 < 0.\]
  Daraus folgt, dass
  \(\frac{y^{2} - 2y + 1}{4} < x\).
  \begin{figure}[ht]
    \centering
    \begin{tikzpicture}
      \begin{axis}[
        grid=both,
        axis equal,
        xmin=-1,xmax=4,
        ymin=-4,ymax=5,
        ]
        % \addplot[smooth,domain = -6:6]{2-x^2};
        \addplot[domain=-4:5, smooth](x^2/4 - x/2 + 1/4, x);
        \addplot[draw=none,
        fill=red!20,
        semitransparent,
        domain=-5:7,
        smooth]
        (x^2/4 - x/2 + 1/4, x);
      \end{axis}
    \end{tikzpicture}
    \caption{Aufgabe 1.1, Teil iii}
  \end{figure}
\end{enumerate}
\subsection*{Aufgabe 1.2}
Seien die Matrizen \(A, B \in \mathbb{R}^{3 \times 3}\) mit
\begin{align*}
  A \coloneq
  \begin{bmatrix}
    1 & 2 & 3 \\
    4 & 5 & 6 \\
    0 & 0 & 1
  \end{bmatrix}, \quad
  B \coloneq
  \begin{bmatrix}
    1 & 0 & 1 \\
    0 & 1 & 1 \\
    0 & 2 & 4
  \end{bmatrix}.
\end{align*}
\begin{enumerate}
\item Wir zeigen, dass \(A\) und \(B\) nicht ähnlich sind, indem wir
  zeigen, dass die beide Matrizen nicht dieselben Eigenwerte
  besitzen. [Kowalsky, Satz 6.1.4.]
  Analog zu erster Aufgabe gilt
  \[\charPol_{A}(p) = (p-1)^{2}(p-5) - 8(p-1), \quad \charPol_{B}(p) =
    (p-1)^{2}(p-4) - 2(p-1).\]
  mit Eigenwerten
  \[\left\{1, 3 \pm \sqrt{12}\right\}, \quad \left\{1, \frac{5
        \pm\sqrt{17}}{2}\right\}.\]
\item Seien \(C, D \in \mathbb{R}^{n \times n}\) Matrizen.  Wegen Voraussetzung besitzt
  die Matrizen \(C\) und \(D\) die gleiche, paarweise verschiedene
  \(n\)-Eigenwerte \(\lambda_{1}, \ldots, \lambda_{n}\).  Wegen [Definition von Eigenwerte]
  und [Kowalsky, Satz 6.2.1] existieren die linear unabhängige
  Spaltenvektoren \(v_{1}, \ldots, v_{n}, w_{1}, \ldots, w_{n} \in \mathbb{R}^{n}\) sodass gilt
  \[C \cdot v_{i} = v_{i} \lambda_{i}, \quad D \cdot w_{i} = w_{i} \lambda_{i}; \quad 1 \le i \le n.\]
  Weil \(C, D \in \End(\mathbb{R}^{n})\) gilt und
  \((v_{i})_{1 \le i \le n}\) und \((w_{i})_{1 \le i \le n}\) linear unabhängig
  sind, definieren wir Basen von \(\mathbb{R}^{n}\) mit
  \[M \coloneq \{v_{i} \mid 1 \le i \le n \}, \quad N \coloneq \{w_{i} \mid 1 \le i \le n\}.\] Damit
  existiert eine invertierbare Matrix des Basiswechsels
  \(P \in \End(\mathbb{R}^{n})\) von \(M\) nach \(N\) mit
  \[D = P \cdot C \cdot P^{-1}.\] Damit sind die Matrizen \(C, D\) ähnlich.
\end{enumerate}
\subsection*{Aufgabe 1.3}
Sei die Matrix \(M \in \mathbb{R}^{3 \times 3}\) mit
\begin{align*}
  M \coloneq
  \begin{bmatrix}
    1 & 2 & 3 \\
    0 & 5 & 0 \\
    0 & 3 & -1
  \end{bmatrix}.
\end{align*}
\begin{enumerate}
\item Es gilt \[\charPol_{M}(p) = (p-1)(p+1)(p-5).\]  Daraus folgt, dass
  die Eigenwerte \(\{\pm 1, 5\}\) sind.
\item Wir berechnen \(\Ker(E \cdot f - M)\), wobei \(f \in \{\pm 1, 5\}\) gilt.
  Dann sind die entsprechende Eigenvektoren
  \[[1, 0, 0], [-3, 0, 2], [7, 8, 4]\]
\item Ja, die Eigenvektoren bilden eine Basis von \(\mathbb{R}^{3}\), denn die
  drei Vektoren linear unabhängig sind.
  \begin{align*}
    \begin{bmatrix}
      1 & 0 & 0 \\
      0 & 1 & 0 \\
      0 & 0 & 1
    \end{bmatrix}
  \end{align*}
  Die Matrix \(M\) ist diagonalisierbar.
\end{enumerate}
\end{document}
