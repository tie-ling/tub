% 使用 chktex 检查 tex 文件中的语法错误
% settings for chktex
% chktex-file 3

\documentclass[draft,a5paper]{article}

% 让每个章节 subsection 在新的一页上开始
% 而不是紧接着上一章节
\AddToHook{cmd/subsection/before}{\clearpage}

% 隐藏默认的章节序号:实际作业中与这个冲突
% https://tex.stackexchange.com/a/30225
\setcounter{secnumdepth}{0}

% 设置页边距,上下左右
\usepackage{geometry}
\geometry{a5paper,
 left=1cm,
 right=1cm,
 top=1cm,
 bottom=2cm
}

% 让 TeX 支持德语
\usepackage[ngerman]{babel}

% 调整 Emacs 预览字体大小
% (setq preview-scale-function 1.5)

% TeX Gyre Schola and unicode-math
% which in turn loads amsmath,mathtools,fontspec and friends
\usepackage{xcharter-otf}


% 加载数学相关的包
\usepackage{
  % 自然段间留空
  parskip,
  % 区间排版
  interval
}

% 中文支持,暂时不需要
% \usepackage[UTF8]{ctex}

% 根据 AMS 建议,应为成对符号(比如绝对值)定义新命令
\providecommand{\abs}[1]{\left\lvert#1\right\rvert}
\providecommand{\norm}[1]{\left\lVert#1\right\rVert}

\usepackage{amsthm}
% 定理定义,依赖于 amsthm
\theoremstyle{remark}
\newtheorem*{Behauptung}{Behauptung}
\newtheorem*{Lemma}{Lemma}
\newtheorem*{Satz}{Satz}
\newtheorem*{Definition}{Definition}

% 定义新函数,依赖于AMSmath
\DeclareMathOperator{\card}{card}

\newcommand{\wt}{\widetilde}
\newcommand{\dd}{\,\mathrm{d}}
% 标题与作者
\title{HA 4, Ana 3, Monika, MM 11}
\author{Zhang 484981, Yang 466096, Guo 480788}

\begin{document}
\maketitle
\begin{center}
  Meng Zhang 484981, Yiwen Yang 466096, Yuchen Guo 480788
\end{center}
\subsection{H 4.1}
Sei der Maßraum \((\mathbb{N}, \mathcal{P}(\mathbb{N}), \mu)\) mit dem Zählmaß \(\mu\).  Eine Funktion \(f\colon
\mathbb{N} \to \mathbb{R}\) ist genau dann bezüglich \(\mu\) integrierbar, wenn die Reihe
\[ \sum_{n \in \mathbb{N}}{f(n)} \tag{1} \]
absolut konvergiert ist und
\[ \int_{\mathbb{N}}{f \dd \mu} = \sum_{n \in \mathbb{N}}{f(n)} \tag{2}\]
gilt.
\begin{proof}
  Hinrichtung, aus der Integrierbarkeit folgt absolute Konvergenz.
  Wegen [Ü 4] ist \(f\) genau dann integrierbar, wenn \(\abs{f}\)
  integrierbar ist.  Weil \(\abs{f}\) integrierbar ist, existiert eine
  Folge von Elementarfunktionen \((f_{n})_{n \in \mathbb{N}}\) mit
  \[\abs{f} = \sup_{n \in \mathbb{N}} f_{n}.\]

  Daraus folgt,  dass
  \begin{align*}
    \infty &> \int{\abs{f}\dd\mu} = \sup_{n \in \mathbb{N}} \int f_{n} \dd \mu \\
      &= \sup_{n \in \mathbb{N}} \int f_{n} \chi_{\mathbb{N}} \dd \mu \\
      &= \sup_{n \in \mathbb{N}} \sum_{i=1}^{n} \alpha_{i} \mu(f_{n}^{-1}(\alpha_{i})) \\
      &\ge \sum_{i=1}^{n}\abs{f(i)}\mu(\{i\}) \\
    &= \sum_{i=1}^{n}\abs{f(i)} \cdot 1
  \end{align*}
  Damit konvergiert (1) absolut.

  Rückrichtung. Aus absoluter Konvergenz folgt die Integrierbarkeit.
  Denn, es gilt
  \begin{align*}
    \infty &> \sum_{i = 1}^{\infty}{\abs{f(i)}} = \sum_{i=1}^{\infty}{\abs{f(i)} \mu(\{i\})}
  \end{align*}
  Weil \(\abs{f}\) messbar ist, existiert die Funktionenfolge \(f_{n} \uparrow
  \abs{f}\) mit
  \begin{align*}
    \sum_{i=1}^{\infty}{\abs{f(i)} \mu(\{i\})} = \sup_{n \in \mathbb{N}}{f_{n}(i) \mu(\{i\})}
    < \infty
  \end{align*}
  damit ist \(\abs{f}\) auf \(\mathbb{N}\) integrierbar wegen der Definition von
  Integrierbarkeit.
\end{proof}
\begin{Behauptung}
  Es gilt
  \[\int f \dd \mu = \sum_{n=1}^{\infty}f(n)\]
\end{Behauptung}
\begin{proof}
  Denn es gilt
  \begin{align*}
    \int f \dd \mu &= \int f^{+} \dd \mu - \int f^{-} \dd \mu \\
              &= ???
  \end{align*}
\end{proof}
\subsection{H 4.2}
Seien \((\Omega, \mathcal{A}, \mu)\) ein Maßraum, \((\wt{\Omega}, \wt{\mathcal{A}})\) ein messbarer Raum
und \(g\colon \Omega \to \wt{\Omega}\) eine messbare Funktion.

\subsubsection{Teil i}

\begin{Behauptung}
  Das Bildmaß
  \[\mu \circ g^{-1}\colon \wt{\mathcal{A}} \to \interval{0}{\infty}, \quad A \mapsto \mu(g^{-1}(A)) \]
  ist ein auf \(\wt{\Omega}, \wt{\mathcal{A}}\) definierte Maß.
\end{Behauptung}
\begin{proof}
  Zuerst gilt
  \[(\mu \circ g^{-1})(\emptyset) = \mu(\emptyset) = 0.\]
  Es seien \((A_{i})_{i\in \mathbb{N}} \subseteq \wt{\mathcal{A}}\)  paarweise disjunkt.  Dann sind
  die Urbilder \((g^{-1}(A_{i}))_{i \in \mathbb{N}}\) paarweise disjunkt.  Damit
  gilt
  \[g^{-1}(\bigcupdot_{i \in \mathbb{N}}{A_{i}}) = \bigcupdot_{i \in
      \mathbb{N}}{g^{-1}(A_{i})},\] weil \(\mu\) eine \(\sigma\)-additive Funktion ist,
  gilt
  \begin{align*}
    (\mu \circ g^{-1})(\bigcupdot_{i \in \mathbb{N}}{A_{i}}) = \mu(\bigcupdot_{i \in
    \mathbb{N}} g^{-1}({A_{i}})) = \sum_{i \in \mathbb{N}}{\mu \circ g^{-1} (A_{i})}.
  \end{align*}
  Damit ist \(\mu \circ g^{-1}\) eine \(\sigma\)-additive Funktion.
\end{proof}

\subsubsection{Teil ii}

Sei \(f \colon \wt{\Omega} \to \overline{\mathbb{R}}\) messbar und \(A \in \wt{\mathcal{A}}\).
\begin{Behauptung}
  Die Identität
  \[\int_{g^{-1}(A)} f \circ g \dd \mu = \int_{A} f \dd(\mu \circ g^{-1})\]
  gilt, falls eines der beiden Integrale definiert ist.
\end{Behauptung}
\begin{proof}
  Angenommen, die linke Integral ist definiert.  Dann ist die Funktion
  \(f \circ g\colon \Omega \to \overline{\mathbb{R}}\) integrierbar über \(g^{-1}(A) \in \mathcal{A}\) mit
  \begin{align*}
    \int_{g^{-1}(A)} f \circ g \dd \mu
    &= \int_{\Omega}{(f \circ g) \chi_{g^{-1}(A)} \dd \mu} \\
    &= \lim_{n \to \infty}{\int{f_{n}}\dd\mu} \\
    &=\lim_{n \to \infty}\sum_{k=1}^{p_{n}}{\alpha_{k}\mu(g^{-1}(A))}  \\
    &= \int_{\Omega} f \chi_{A} \dd(\mu \circ g^{-1}) \\
    &= \int_{A}{f \dd(\mu \circ g^{-1})}
  \end{align*}
wobei
  \((f_{n})_{n \in \mathbb{N}}\) eine monoton wachsende Folge von
  Elementarfunktionen mit \(f_{n} \uparrow (f \circ g) \chi_{g^{-1}(A)}\) ist.
\end{proof}
\end{document}
