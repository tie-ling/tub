% 使用 chktex 检查 tex 文件中的语法错误
% settings for chktex
% chktex-file 3

\documentclass[draft,a5paper]{article}

% 让每个章节 subsection 在新的一页上开始
% 而不是紧接着上一章节
\AddToHook{cmd/subsection/before}{\clearpage}

% 隐藏默认的章节序号:实际作业中与这个冲突
% https://tex.stackexchange.com/a/30225
\setcounter{secnumdepth}{0}

% 设置页边距,上下左右
\usepackage{geometry}
\geometry{a5paper,
 left=1cm,
 right=1cm,
 top=1cm,
 bottom=2cm
}

% 让 TeX 支持德语
\usepackage[ngerman]{babel}

% 调整 Emacs 预览字体大小
% (setq preview-scale-function 1.5)

% TeX Gyre Schola and unicode-math
% which in turn loads amsmath,mathtools,fontspec and friends
\usepackage{xcharter-otf}


% 加载数学相关的包
\usepackage{
  % 自然段间留空
  parskip,
  % 区间排版
  interval
}

% 中文支持,暂时不需要
% \usepackage[UTF8]{ctex}

% 根据 AMS 建议,应为成对符号(比如绝对值)定义新命令
\providecommand{\abs}[1]{\left\lvert#1\right\rvert}
\providecommand{\norm}[1]{\left\lVert#1\right\rVert}

\usepackage{amsthm}
% 定理定义,依赖于 amsthm
\theoremstyle{remark}
\newtheorem*{Behauptung}{Behauptung}
\newtheorem*{Lemma}{Lemma}
\newtheorem*{Satz}{Satz}
\newtheorem*{Definition}{Definition}

% 定义新函数,依赖于AMSmath
\DeclareMathOperator{\card}{card}

\title{HA 1, Ana 3, Monika, MM 11}
\author{Zhang 484981, Yang 466096, Guo 480788}

\begin{document}
\maketitle
\begin{center}
  Meng Zhang 484981, Yiwen Yang 466096, Yuchen Guo 480788
\end{center}

\newpage

\subsection{H 1.1}
Siehe nächste Seite.
\newpage
\subsection{H 1.2}
Sei \(\Omega\) überabzählbar und sei die \(\sigma\)-Algebra
\[\mathcal{A} \coloneq \{A \subseteq \Omega \mid A \text{ oder } A^{c} \text{ ist abzählbar}\}.\]
\begin{enumerate}
\item Behauptung.  \(\mathcal{E} \coloneq \{ \{\omega\} \mid \omega \in \Omega\}\) erzeugt \(\mathcal{A}\).

  Beweis.  Wir zeigen, dass \(\sigma(\mathcal{E}) \subseteq \mathcal{A}\) und \(\mathcal{A} \subseteq \sigma(\mathcal{E})\) gelten.

  Es gilt  \(\sigma(\mathcal{E}) \subseteq \mathcal{A}\).  Denn sei \(A \in \sigma(\mathcal{E})\) beliebig.  Falls \(A\) oder
  \(A^{c}\) abzählbar ist, dann folgt \(A \in \mathcal{A}\).  Angenommen, \(A\) und
  \(A^{c}\) sind beide überabzählbar.  Dann können wir eine kleinste
  \(\sigma\)-Algebra \(\mathcal{A}'\) finden, die \(A\) und \(A^{c}\) nicht enthält, weil
  \(\sigma\)-Algebra nur bezüglich abzählbaren Vereinigungen stabil ist.

  Es gilt \(\mathcal{A} \subseteq \sigma(\mathcal{E})\).  Denn sei
  \(A \in \mathcal{A}\) beliebig.  Dann ist entweder \(A\) abzählbar oder
  \(A^{c}\) abzählbar.  Angenommen, \(A\) ist abzählbar, dann gilt
  \(A \in \sigma(\mathcal{E})\), weil \(\sigma(\mathcal{E})\) alle abzählbare Vereinigung von
  Einelementigen Mengen enthält.  Angenommen, \(A^{c}\) ist abzählbar.
  Dann ist wegen \(\Omega\) überabzählbar dass \(A\) überabzählbar.  Aber
  \(A\) ist trotzdem in \(\sigma(\mathcal{E})\) enthält, weil
  \(A^{c} \in \sigma(\mathcal{E})\) gilt wegen Stabilität bzgl.\ abzählbaren
  Vereinigungen, und \((A^{c})^{c} \in \sigma(\mathcal{E})\) gilt wegen Stabilität
  bzgl. Komplementarmengen.  Damit gilt \(\sigma(\mathcal{E}) = \mathcal{A}\).
\item Sei die Abbildung \(\mu\) definiert durch
  \begin{align*}
    \mu\colon \mathcal{A} \to \interval{0}{\infty}, \quad \mu(A) \coloneq
    \begin{cases}
      0, & \text{falls } A \text{ abzählbar ist;} \\
      1, & \text{falls } A^{c} \text{ abzählbar ist.}
    \end{cases}
  \end{align*}
  \begin{Behauptung}
    \(\mu\) ist wohldefiniert.  \(\mu\) ist ein Maß auf \(\mathcal{A}\).
  \end{Behauptung}
  \begin{proof}
    Wir zeigen wohldefiniertheit.  Das heißt, es existiert einen
    eindeutigen Funktionswert zu jeder \(A \in \mathcal{A}\).  Sei nun \(A \in \mathcal{A}\)
    beliebig.
    \begin{enumerate}
    \item Falls entweder \(A\) oder \(A^{c}\) abzählbar, dann ist \(\mu\)
      wohldefiniert.
    \item Falls beide Mengen \(A\) und \(A^{c}\) abzählbar sind, dann
      folgt, \(\Omega = A \cup A^{c}\) abzählbar im Widerspruch zur
      Voraussetzung dass \(\Omega\) überabzählbar ist.
    \item Falls beide Mengen \(A\) und \(A^{c}\) überabzählbar sind, dann
      folgt, dass \(A, A^{c} \notin \mathcal{A}\), also liegt nicht im Definitionsbereich.
    \end{enumerate}
    Damit ist die Abbildung \(\mu\colon \mathcal{A} \to \interval{0}{\infty}\) wohldefiniert.

    Wir zeigen, dass die Abbildung \(\mu\) ein Maß auf die
    \(\sigma\)-Algebra \(\mathcal{A}\) ist, indem wir zeigen, dass
    \(\mu\) eine \(\sigma\)-additive Abbildung ist.  Zuerst gilt
    \(\mu(\emptyset) = 0\), denn \(\emptyset\) ist abzählbar. Sei
    \((A_{k})_{k \in \mathbb{N}} \) eine beliebig gewählte Folge in
    \(\mathcal{A}\) mit \(A_{i} \cap A_{j} = \emptyset\) für
    \(i \ne j\), also die Folgengliedern sind paarweise disjunkt.  Wir
    betrachten zwei Fälle.
    \begin{itemize}
    \item Falls alle Folgenglieder abzählbar ist.  Die abzählbare
      Vereinigung von abzählbare Mengen ist wieder abzählbar.

      Daraus folgt, dass \(\mu(\cup_{k \in \mathbb{N}}A_{k}) = \sum_{k \in \mathbb{N}}\mu(A_{k}) = 0\).

    \item Falls mindestens ein Folgenglieder überabzählbar ist.  Wir
      nennen diese Folgenglieder \(A_{k}\).  Wegen
      \(A_{k} \in \mathcal{A}\) ist \(A_{k}^{c}\) abzählbar.  Weil die Folgenglieder
      paarweise disjunkt sind, gilt
      \[A_{i} \subseteq A_{k}^{c}, \quad i \ne k\] wobei \(A_{k}^{c}\) abzählbar ist.
      Daraus folgt, dass \(A_{i}\) abzählbar für alle \(i \ne k\) und
      \(\mu(A_{i}) = 0\) für alle \(i \ne k\).  Es folgt wegen
      \(A_{k}^{c}\) abzählbar dass \(\mu(A_{k}) = 1\).  Wir betrachten die
      Vereinigung \(\bigcup_{i \in \mathbb{N}}{A_{i}}\).  Aus Regel von De Morgan folgt,
      dass
      \[\left(\bigcup_{i \in \mathbb{N}}A_{i}\right)^{c} = \bigcap_{i \in \mathbb{N}}A_{i}^{c}\]
      wobei \(A_{k}^{c}\) im Durchschnitt abzählbar ist.  Daraus folgt,
      dass die linke Seite abzählbar ist und
      \(\mu\left(\bigcup_{i \in \mathbb{N}}A_{i}\right) = 1\).  Insgesamt folgt,
      \[ \mu{\left(\bigcup_{i \in \mathbb{N}}{A_{i}}\right)} = 1 = 0 + 0 + \cdots + 1 + \cdots + 0
        + 0 + \cdots = \sum_{i \in \mathbb{N}}{\mu(A_{i})}\] Damit ist
      \(\mu\) eine \(\sigma\)-additive Maß.
    \end{itemize}
  \end{proof}
\end{enumerate}
\newpage
\subsection{H 1.3}
Sei \((\Omega, \mathcal{A}, \mu)\) ein Maßraum.  Sei \(A, A_{1}, A_{2}, \ldots, \in \mathcal{A}\) und
definiere
\[ A_{n} \uparrow A \coloneq A_{1} \subseteq A_{2} \subseteq \ldots \text{ und } A = \bigcup_{n \in \mathbb{N}}{A_{n}},\]
sowie
\[ A_{n} \downarrow A \coloneq A_{1} \supseteq A_{2} \supseteq \ldots \text{ und } A = \bigcap_{n \in \mathbb{N}}{A_{n}}.\]
\begin{Behauptung}
  Aus \(A_{n} \uparrow A\) folgt \(\lim_{n \to \infty}{\mu(A_{n})} = \mu(A)\).
\end{Behauptung}
\begin{proof}
Wir konstruieren eine Folge von paarweisen disjunkte Mengen, sodass
das Kriterium von \(\sigma\)-Additivität des Maß \(\mu\) anwendbar wird.

Wir definieren
\[ B_{1} = A_{1}, \quad B_{k} = A_{k} \setminus A_{k - 1}, \quad k \in \mathbb{N}.\]
Wegen Voraussetzung dass \(A_{1} \subseteq A_{2} \subseteq \ldots\), gilt \(B_{i} \cap B_{j} = \emptyset
\) für alle \(i, j \in \mathbb{N}\) mit \(i \ne j\).  Weil \(\mathcal{A}\) eine \(\sigma\)-Algebra ist,
folgt
\[\mu\left(\bigcup_{k = 2}^{\infty}{B_{k}}\right) = \sum_{k = 2}^{\infty}{\mu(B_{k})}\]
sowie
\begin{align*}
  \mu(A) &= \mu(B_{1}) + \sum_{k = 2}^{\infty}{B_{k}}
       = \lim_{n \to \infty}\left(\mu(B_{1}) + \sum_{k=2}^{n}{\mu(B_{k})}\right)
       = \lim_{n \to \infty}\mu\left(B_{1} \cup \bigcup_{k = 2}^{n}{B_{k}}\right) \\
       &= \lim_{n \to \infty}\mu\left({\bigcup_{k =1}^{n}B_{k}}\right)
         = \lim_{n \to \infty}{\mu(A_{n})} \tag*{:wegen Def von \(B\)}
\end{align*}
damit ist die Behauptung bewiesen.
\end{proof}
\begin{Behauptung}
  Sei \(A_{n} \downarrow A\).  Dann ist die Behauptung von (1) falsch.
\end{Behauptung}
\begin{proof}
  Falls \(\mu(A_{i}) = \infty\) für alle \(i \in \mathbb{N}\) dann ist die Behauptung
  falsch.  Sei die Mengen und Zählmaß definiert durch
  \[A_{n} \coloneq \{k \in \mathbb{N} \mid k \ge n\}; \quad \mu(A_{n}) = \abs{A_{n}}\]
  dann ist \(\mu(A_{n}) = \infty\) für alle \(n \in \mathbb{N}\).  Aber die Durchschnitt von
  alle \(A_{n}\) ist die leere Menge \(A = \emptyset\).  Daraus folgt, dass
  \[\lim_{n \to \infty}{\mu(A_{n})} = \infty \ne 0 = \mu(A).\]
\end{proof}
\newpage
\subsection{H 1.4}
Sei \((\Omega, \mathcal{A}, \mu)\) ein Maßraum.  Für \(A, B \in \mathcal{A}\) wird \(A \sim B\) definiert
durch \(\mu(A \triangle B) = 0\).
\begin{enumerate}
\item Wir zeigen, dass \(A \sim B\) eine Äquivalenzrelation ist.  Die
  Relation ist symmetrisch, denn es gilt \(\mu(A \triangle A) = 0\).  Die Relation
  ist reflexiv, denn aus \(\mu(A \triangle B)  = 0\) folgt unmittelbar
  \[\mu(A \triangle B) = \mu((A \setminus B) \cup (B \setminus A)) = \mu((B \setminus A) \cup (A \setminus B)) = \mu(B
    \triangle A).\]
  Die Relation ist transitiv.  Denn sei \(\mu(A \triangle B) = \mu(B \triangle C) = 0\).
  Außerdem gilt
  \begin{align*}
    (A \cup B) \setminus (A \cap B) &= [A \setminus (B \cup C) \cup (A \cap C) \setminus B] \cup [B \setminus (A \cup C) \cup
                        (B \cap C) \setminus A] \\
    (B \cup C) \setminus (B \cap C) &= [B \setminus (C \cup A) \cup (B \cap A) \setminus C] \cup [C\setminus (B \cup A) \cap
                        (C \cap A) \setminus B] \\
    (A \cup C) \setminus (A \cap C) &= [A \setminus (C \cup B) \cup (A \cap B) \setminus C] \cup [C \setminus (A \cup B) \cup
                        (C \cap B) \setminus A] \\
                      &\text{ist Teilmenge von der Vereinigung der ersten zwei Mengen.}
  \end{align*}
  Nach [Bem 1.14] gilt dann \[\mu((A \cup C) \setminus (A \cap C)) \le \mu(((A \cup B) \setminus (A \cap B))
  \cup ((B \cup C) \setminus (B \cap C)))\] wobei die beiden Mengen nicht disjunkt mit
  \(\le \mu(A \triangle B) + \mu(B \triangle C) = 0\).  Wegen \(\mu\colon \mathcal{A} \to \interval{0}{\infty}\) ist die
  Relation transitiv.
\item Siehe nächste Seite.
\end{enumerate}
\end{document}
