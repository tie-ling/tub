% 使用 chktex 检查 tex 文件中的语法错误
% settings for chktex
% chktex-file 3

\documentclass[fleqn,draft,a5paper]{article}

% 让每个章节 subsection 在新的一页上开始
% 而不是紧接着上一章节
\AddToHook{cmd/subsection/before}{\clearpage}

% 隐藏默认的章节序号:实际作业中与这个冲突
% https://tex.stackexchange.com/a/30225
\setcounter{secnumdepth}{0}

% 设置页边距,上下左右
\usepackage{geometry}
\geometry{a5paper,
 left=1cm,
 right=1cm,
 top=1cm,
 bottom=2cm
}

% 让 TeX 支持德语
\usepackage[ngerman]{babel}

% 调整 Emacs 预览字体大小
% (setq preview-scale-function 1.5)

% TeX Gyre Schola and unicode-math
% which in turn loads amsmath,mathtools,fontspec and friends

% 让每个章节 subsection 在新的一页上开始
% 而不是紧接着上一章节
\AddToHook{cmd/subsection/before}{\clearpage}

% 隐藏默认的章节序号:实际作业中与这个冲突
% https://tex.stackexchange.com/a/30225
\setcounter{secnumdepth}{0}

% 设置页边距,上下左右
\usepackage{geometry}

\geometry{left=1cm,
 right=1cm,
 top=1cm,
 bottom=2cm}

% 让 TeX 支持德语
\usepackage[ngerman]{babel}

\usepackage{amsmath,mathtools,fontspec,amsthm,parskip,interval,unicode-math}
\setmainfont{texgyreschola}%
 [
  Extension = .otf ,
  UprightFont = *-regular,
  ItalicFont = *-italic,
  BoldFont = *-bold,
  BoldItalicFont = *-bolditalic,
  Ligatures=TeX,
 ]
\setmathfont{texgyreschola-math.otf}


% 加载数学相关的包
\usepackage{
  % 自然段间留空
  parskip,
  % 区间排版
  interval
}

% 中文支持,暂时不需要
% \usepackage[UTF8]{ctex}

% 根据 AMS 建议,应为成对符号(比如绝对值)定义新命令
\providecommand{\abs}[1]{\left\lvert#1\right\rvert}
\providecommand{\norm}[1]{\left\lVert#1\right\rVert}

\usepackage{amsthm}
% 定理定义,依赖于 amsthm
\theoremstyle{remark}
\newtheorem*{Behauptung}{Behauptung}
\newtheorem*{Lemma}{Lemma}
\newtheorem*{Satz}{Satz}
\newtheorem*{Definition}{Definition}

% 定义新函数,依赖于AMSmath
\DeclareMathOperator{\card}{card}

% 标题与作者
\title{HA 3, Ana 3, Monika, MM 11}
\author{Zhang 484981, Yang 466096, Guo 480788}

\begin{document}
\maketitle
\begin{center}
  Meng Zhang 484981, Yiwen Yang 466096, Yuchen Guo 480788
\end{center}
\subsection{H 3.1}
\subsubsection{Definitionen}
\begin{Definition}[Messbarer Raum]
  Ein messbarer Raum ist ein Tupel \((\Omega, \mathcal{A})\) bestehend aus einer
  Menge \(\Omega\) und einer \(\sigma\)-Algebra \(\mathcal{A}\) über \(\Omega\).
\end{Definition}
\begin{Definition}[Maßraum]
  Ein Maßraum ist ein Tripel \((\Omega, \mathcal{A}, \mu)\) bestehend aus einer
  Menge \(\Omega\), einer \(\sigma\)-Algebra \(\mathcal{A}\) über \(\Omega\) und einem Maß \(\mu\) auf \(\mathcal{A}\).
\end{Definition}
\begin{Definition}[Messbare Abbildung]
  Seien \((\Omega_{i}, \mathcal{A}_{i}), i = 1, 2\) messbare Räume.  Eine Abbildung
  \(f\colon \Omega_{1} \to \Omega_{2}\) heißt \(\mathcal{A}_{1}\)-\(\mathcal{A}_{2}\)-messbar, falls
  \[
    f^{-1}(\mathcal{A}_{2}) = \{f^{-1}(A_{2}) \mid A_{2} \in \mathcal{A}_{2}\} \subseteq \mathcal{A}_{1}
  \]
  gilt.
\end{Definition}
\subsubsection{Aufgabe}
  Seien \(\Omega\) eine nicht-leere Menge, \((\Omega', \mathcal{A}')\) ein messbarer Raum.
  Wir definieren
  \[f\colon \Omega \to \Omega', \quad g\colon \Omega \to \mathbb{R}, \quad \varphi\colon \Omega' \to \mathbb{R}, \quad \sigma(f) \coloneq f^{-1}(\mathcal{A}')\] indem
  \(\sigma(f)\) die von \(f\) auf \(\Omega\) erzeugte \(\sigma\)-Algebra ist.
\begin{Behauptung}
  Die Abbildung \(g\) ist genau dann \(\sigma(f)\)-\(\mathcal{B}\)-messbar, d.h.,
  \[g^{-1}(\mathcal{B}) = \{g^{-1}(M) \mid M \in \mathcal{B}\} \subseteq \sigma(f) = f^{-1}(\mathcal{A}')\] wenn eine
  \(\mathcal{A}'\)-\(\mathcal{B}\)-messbare Abbildung \(\varphi\) existiert mit \(g = \varphi \circ f\).
\end{Behauptung}
\begin{proof}
  Wir beweisen Hinrichtung mit Widerspruch.  Angenommen, die Abbildung
  \(g\) ist \(\sigma(f)\)-\(\mathcal{B}\)-messbar und es existiert keine solche Abbildung
  \(\varphi\), die \(\mathcal{A}'\)-\(\mathcal{B}\)-messbar ist.  Das heißt, für alle Abbildung
  \(\varphi\colon \Omega' \to \mathbb{R}\) gilt
  \[\varphi^{-1}(\mathcal{B}) = \{\varphi^{-1}(M) \mid M \in \mathcal{B}\} \nsubseteq \mathcal{A}'\]
  also für alle Abbildung \(\varphi\) existiert mindestens eine
  elementargeometrische Menge \(M \in \mathcal{B}\) sodass gilt
  \[\varphi^{-1}(M) \notin \mathcal{A}'.\]
  Wegen \(g = \varphi \circ f\) gilt \(g^{-1} = f^{-1} \circ \varphi^{-1}\).  Wir betrachten
  \(g^{-1}(M)\).  Es gilt
  \[M \in \mathcal{B}, \quad g^{-1}(M) = (f^{-1} \circ \varphi^{-1})(M) \notin f^{-1}(\mathcal{A}') = \sigma(f)\]
  im Widerspruch zur Voraussetzung dass \(g^{-1}(M) \in f^{-1}(\mathcal{A}')\) für alle \(M \in
  \mathcal{B}\).

  Rückrichtung.  Angenommen, es existiert solche \(\mathcal{A}'\)-\(\mathcal{B}\)-messbare
  Abbildung \(\varphi\) mit \(g = \varphi \circ f\).  Dann gilt wegen Definition von
  messbaren Abbildungen dass
  \[\varphi^{-1}(\mathcal{B}) = \{\varphi^{-1}(M) \mid M \in \mathcal{B}\} \subseteq \mathcal{A}'\]
  sowie
  \[g^{-1}(\mathcal{B}) = (f^{-1} \circ \varphi^{-1})(\mathcal{B}) \subseteq f^{-1}(\mathcal{A}').\]
  Damit ist \(g\)  eine \(\sigma(f)\)-\(\mathcal{B}\)-messbare Abbildung und die Behauptung
  gilt.
\end{proof}
\subsection{H 3.2}
Sei \((\Omega, \mathcal{A})\) ein messbarer Raum mit der bekannten \(\sigma\)-Algebra
\[\mathcal{A} = \{A \subseteq \Omega \mid A \text{ ist abzählbar oder } A^{c} \text{ ist
    abzählbar} \}.\]
und sei die Funktion \(f \colon \Omega \to \mathbb{R}\).  Wir charakterisieren alle
\(\mathcal{A}\)-\(\mathcal{B}\)-messbaren Funktionen, d.h., Funktionen mit
\[f^{-1}(\mathcal{B}) = \{f^{-1}(M) \mid M \in \mathcal{B}\} \subseteq \mathcal{A}.\]
\begin{proof}
Wir zeigen die Charakterisierung: Für alle \(A \in \mathcal{A}\) gilt:
\[f(A) \text{ konstant oder } f(A^{c}) \text{ konstant.}\]

 Falls \(\Omega\) abzählbar ist, dann ist \(\mathcal{A} = \Omega\) und alle Funktionen \(f\colon \Omega
    \to \mathbb{R}\) ist \(\mathcal{A}\)-\(\mathcal{B}\)-messbar.

 Falls \(\Omega\) überabzählbar ist. Wir betrachten den Bild \(I \coloneq f(\Omega)\).

 Falls \(I\) überabzählbar
    ist, dann existiert ein \(a \in \mathbb{R}\) sodass die beide Mengen
    \(I \cap \ointerval{-\infty}{a}\) und
    \(I \cap \rinterval{a}{\infty}\) überabzählbare Mengen sind.  Daraus folgt,
    dass
\[ \Omega = \left(f^{-1}(I \cap \ointerval{-\infty}{a}) \sqcup f^{-1}(I \cap
    \rinterval{a}{\infty})\right) \notin \mathcal{A}\] denn es gilt
\(f^{-1}(I) = \Omega\).  Damit ist \(f\) keine
\(\mathcal{A}\)-\(\mathcal{B}\)-messbaren Funktion, falls \(I\) überabzählbar ist.

 Falls \(I\) abzählbar.  Dann ist die disjunkte
Mengen der Urbild abzählbar, das heißt,
\[\bigsqcup_{y \in I}{f^{-1}(\{y\})} = \Omega, \quad \card(I) = \mathbb{N}.\]
Weil \(\Omega\) überabzählbar ist, existiert mindestens ein \(y \in I\) sodass
\(f^{-1}(\{y\}) \subseteq \Omega\) überabzählbar ist.

 Falls \(I\) abzählbar. Falls es mindestens zwei  \(y_{1}, y_{2} \in I\) mit disjunkte, überabzählbare
Urbild \(f^{-1}(\{y_{1}\}), f^{-1}(\{y_{2}\})\) existieren, dann
ist insbesondere
\[f^{-1}(\{y_{1}\}) \text{ überabzählbar und } f^{-1}(\{y_{2}\}) \subseteq
  \left(f^{-1}(\{y_{1}\})\right)^{c} \text{ überabzählbar.}\] Daraus
folgt, dass \(f\)
keine \(\mathcal{A}\)-\(\mathcal{B}\)-messbaren Funktion ist.

 Falls \(I\) abzählbar. Wir haben gezeigt, dass es genau ein
  \(y \in I\) mit überabzählbare Urbild \(f^{-1}(\{y\})\) existiert. Für
  deren Komplement gilt
\[\left(f^{-1}(\{y\}\right)^{c} = \bigsqcup_{k \in I \setminus
    \{y\}}{f^{-1}(\{k\})}\] indem alle Menge
\(f^{-1}(\{k\}), k \in I \setminus \{y\}\) abzählbar sind.  Abzählbare Vereinigung
von abzählbaren Mengen sind abzählbar.  Daraus folgt, dass
\[ \forall k \in I \setminus \{y\}, \quad f^{-1}(\{k\}) \in \mathcal{A} \text{ abzählbar} \]
sowie
\[ k = y, \quad \left(f^{-1}(\{k\})\right)^{c} \in \mathcal{A} \text{ abzählbar}\] Damit
ist \(f\) eine \(\mathcal{A}\)-\(\mathcal{B}\)-messbare Funktion.  Damit folgt die Charakterisierung.
\end{proof}

\subsection{H 3.3}
\begin{Behauptung}
  % 黑板粗体 \mathbb{R} \mathbb{C}
  % 花体 \mathcal{A} \mathcal{B}
    Jede monotone Funktion \(f: \mathbb{R} \to \mathbb{R}\) ist messbar.
\end{Behauptung}

\begin{proof}
    Sei \(f\) monoton wachsend, andernfalls betrachten wir \(-f\).
    Sei \( a \in \mathbb{R}\) beliebig. Wir zeigen, dass \( A \coloneq \{ x \in \mathbb{R} \mid f(x) \leq a \} \in \mathcal{B}\).
    \begin{itemize} 
    \item Falls \(A = \emptyset \) oder \(A = \mathbb{R}\), ist die Behauptung richtig.
      % 这么写interval::
      % \interval{a}{b} = [a, b]
      % \ointerval{a}{b} = ]a, b[
      % \linterval{a}{b} = ]a, b]
      % \rinterval{a}{b} = [a, b[
    \item Falls \(A \neq \emptyset \) und \(A \neq \mathbb{R}\), ist \(A\) nach oben beschränkt. In
      diesem Fall setzen wir \(s = \sup A\). Da \(f\) monoton wachsend,
      gilt \[ A = \ointerval{-\infty}{s} \] oder
        \[ A = \linterval{-\infty}{s} \]
    \end{itemize}
    In beiden Fällen ist \(A \in \mathcal{B}\) nach Hausaufgaben 2.1.
\end{proof}

\subsection{H 3.4}
\begin{Behauptung}
  Sei \(\mu\) ein endliches Maß auf \((\mathbb{R}, \mathcal{B})\).  Es existiert eine monoton
  wachsende und rechtsseitig stetige Funktion
  \(F\colon \mathbb{R} \to \mathbb{R}\) sodass
  \[\mu(\linterval{a}{b}) = F(b) - F(a)\]
  für alle \(a, b \in \mathbb{R}, a \le b\).
\end{Behauptung}
\begin{proof}
  Weil \(F\) auf ganze \(\mathbb{R}\) definiert ist, wählen wir ein \(a \in \mathbb{R}\) fest
  und definiere
  \[F(x) \coloneq \mu(\linterval{a}{x}) + F(a).\]
  Wir zeigen, dass \(F(x)\) monoton wachsend ist und rechtsseitig stetig
  ist.  Sei \(p, q \in \mathbb{R}, p \le q\).  Dann folgt
  \begin{align*}
    F(q) - F(p) &= \mu(\linterval{a}{q}) - \mu(\linterval{a}{p}) \\
    \intertext{weil die Funktion \(\mu\) eine \(\sigma\)-additive Funktion ist,}
    &= \mu(\linterval{a}{p}) + \mu(\linterval{p}{q}) - \mu(\linterval{a}{p})
    \\
    &= \mu(\linterval{p}{q}) \ge 0.
  \end{align*}
  Wir bemerken, dass wegen
  \[\mu(\emptyset) = \mu(\linterval{0}{0}) = 0 = F(0) - F(0)\] folgt \(F(0) = 0\)
  und
  \[F(x) \coloneq \mu(\linterval{0}{x}) + F(0) = \mu(\linterval{0}{x}).\]


  Wir zeigen, dass \(F(x)\) rechtsseitig stetig ist.  Falls \(x > 0\) und sei
  \((x_{n}) \subseteq \mathbb{R}_{>x}\) eine monoton fallende Folge mit
  \(\lim_{n \to \infty}{x_{n}} = x\).  Dann gilt
  \(\linterval{0}{x_{n+1}} \subseteq \linterval{0}{x_{n}}\) und
  \(\linterval{0}{x} = \bigcap_{n \in \mathbb{N}}{\linterval{0}{x_{n}}}\).  Daraus folgt,
  dass
  \[F(x) = \mu(\linterval{0}{x}) = \mu(\bigcap_{n \in \mathbb{N}}{\linterval{0}{x_{n}}}) =
    \lim_{n \to \infty}\mu(\linterval{0}{x_{n}}) = \lim_{n \to \infty}F(x_{n}).\]
  Falls \(x \le 0\), dann betrachten wir die eine monoton fallende Folge   \((x_{n}) \subseteq \mathbb{R}_{>x}\)  mit
  \(\lim_{n \to \infty}{x_{n}} = x\).  Dann gilt analog, dass
  \begin{align*}
    F(x) - F(x-1) &= \mu(\linterval{x-1}{x}) \\
                  &= \mu(\bigcap_{n\in \mathbb{N}}\linterval{x-1}{x_{n}}) \\
                  &= \lim_{n \to \infty}\mu(\linterval{x-1}{x_{n}}) \\
                  &= \lim_{n \to \infty}F(x_{n}) - F(x-1).  
  \end{align*}
  Damit ist \(F(x)\) rechtsseitig stetig auf \(\mathbb{R}\).
\end{proof}
\begin{Behauptung}
  Die Aussage in (i) stimmt noch, wenn man auf die Endlichkeit von \(\mu\) verzichtet.
\end{Behauptung}
\begin{proof}
  Denn nirgenwo haben wir diese Eigenschaft im obigen Beweis verwendet.
\end{proof}
\end{document}
