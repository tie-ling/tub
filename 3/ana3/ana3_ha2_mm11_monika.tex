% 使用 chktex 检查 tex 文件中的语法错误
% settings for chktex
% chktex-file 3

\documentclass[draft,a5paper]{article}

% 让每个章节 subsection 在新的一页上开始
% 而不是紧接着上一章节
\AddToHook{cmd/subsection/before}{\clearpage}

% 隐藏默认的章节序号:实际作业中与这个冲突
% https://tex.stackexchange.com/a/30225
\setcounter{secnumdepth}{0}

% 设置页边距,上下左右
\usepackage{geometry}
\geometry{a5paper,
 left=1cm,
 right=1cm,
 top=1cm,
 bottom=2cm
}

% 让 TeX 支持德语
\usepackage[ngerman]{babel}

% 调整 Emacs 预览字体大小
% (setq preview-scale-function 1.5)

% TeX Gyre Schola and unicode-math
% which in turn loads amsmath,mathtools,fontspec and friends
\usepackage{amsmath,mathtools,fontspec,amsthm,parskip,interval,unicode-math}
\setmainfont{texgyreschola}%
 [
  Extension = .otf ,
  UprightFont = *-regular,
  ItalicFont = *-italic,
  BoldFont = *-bold,
  BoldItalicFont = *-bolditalic,
  Ligatures=TeX,
 ]
\setmathfont{texgyreschola-math.otf}


% 加载数学相关的包
\usepackage{
  % 自然段间留空
  parskip,
  % 区间排版
  interval
}

% 中文支持,暂时不需要
% \usepackage[UTF8]{ctex}

% 根据 AMS 建议,应为成对符号(比如绝对值)定义新命令
\providecommand{\abs}[1]{\left\lvert#1\right\rvert}
\providecommand{\norm}[1]{\left\lVert#1\right\rVert}

\usepackage{amsthm}
% 定理定义,依赖于 amsthm
\theoremstyle{remark}
\newtheorem*{Behauptung}{Behauptung}
\newtheorem*{Lemma}{Lemma}
\newtheorem*{Satz}{Satz}
\newtheorem*{Definition}{Definition}

% 定义新函数,依赖于AMSmath
\DeclareMathOperator{\card}{card}

\providecommand{\cardi}[1]{\card\left(#1\right)}
\providecommand{\olA}{\overline{\mathcal{A}}}
\providecommand{\olm}{\overline{\mu}}


\title{HA 2, Ana 3, Monika, MM 11}
\author{Zhang 484981, Yang 466096, Guo 480788}

\begin{document}
\maketitle
\begin{center}
  Meng Zhang 484981, Yiwen Yang 466096, Yuchen Guo 480788
\end{center}
\subsection{H 2.1}
\subsubsection{Definitionen}
\begin{Definition}
  Sei \(\Omega\) eine Menge.  \(\mathcal{A} \subseteq \mathcal{P}(\Omega)\) ist eine \(\sigma\)-Algebra,  falls (i)
  \(\Omega \in \mathcal{A}\); (ii) komplementstabil; (iii)  stabil bzgl.  abzählbaren Vereinigungen.
\end{Definition}
\begin{Satz}
  Eine \(\sigma\)-Algebra \(\mathcal{A}\) über \(\Omega\) ist stabil bzgl. abzählbaren
  Durchschnitten. (Bem. 1.5)
\end{Satz}
\begin{Definition}
  Die von den elementargeom. Mengen \(\mathcal{R}_{\mathrm{EG}}^{n}\) erzeugte
  \(\sigma\)-Algebra über \(\mathbb{R}^{n}\) heißt Borelsche \(\sigma\)-Algebra \(\mathcal{B}^{n}\).
\end{Definition}
\subsubsection{Aufgabe}
\begin{Behauptung}
  Seien \(A \in \mathcal{B}^{n}, x \in \mathbb{R}^{n}\), dann gilt \(x + A \in \mathcal{B}^{n}\).
\end{Behauptung}
\begin{proof}
  Wir betrachten die Menge
  \[ C(x) \coloneq \{A \in \mathcal{B}^{n} \mid x + A \in \mathcal{B}^{n}\} \subseteq \mathcal{B}^{n}.\]
  Zuerst bemerken wir, dass
  \(\mathcal{R}_{\text{EG}}^{n} \subseteq C(x)\) gilt, denn jede verschobene endliche
  Vereinigung von Quadern ist natürlich wieder eine endliche
  Vereinigung von verschobenen Quadern.
  Danach zeigen wir, dass \(C(x)\) eine \(\sigma\)-Algebra über \(\mathbb{R}^{n}\) ist.
  \begin{itemize}
  \item Es gilt \(\mathbb{R}^{n} \in C(x)\).  Weil \(\mathcal{B}^{n}\) eine
    \(\sigma\)-Algebra ist, gilt
    \(\mathbb{R}^{n} \in \mathcal{B}^{n}\).  Es gilt auch, dass für alle
    \(x \in \mathbb{R}^{n}\) die Gleichheit
    \(x + \mathbb{R}^{n} = \{x\} \cup \mathbb{R}^{n} = \mathbb{R}^{n} \in \mathcal{B}^{n}\) gilt.  Daraus folgt,
    dass \(\mathbb{R}^{n} \in C(x)\).
  \item \(C(x)\) ist komplementstabil.  Denn sei \(A \in C(x)\) beliebig.  Wir
    zeigen, dass \(A^{c} \in C(x)\) gilt.  Sei \(x \in \mathbb{R}^{n}\) beliebig.  Es gibt
    zwei Fälle.
    \begin{itemize}
    \item Falls \(x \in A\).  Dann folgt, wegen \(A \in \mathcal{B}^{n}\) dass \(x \notin
      A^{c} \in \mathcal{B}^{n}\).  Weil \(\mathcal{B}^{n}\) bzgl. abzählbaren Vereinigungen
      stabil ist, gilt \(x + A^{c} = \{x\} \cup A^{c} \in \mathcal{B}^{n} \).  Daraus
      folgt, dass  \(A^{c} \in C(x)\).
    \item Falls \(x \in A^{c}\).  Dann folgt, dass \(x + A^{c} = \{x\} \cup
      A^{c} = A^{c}\) und wegen \(A \in \mathcal{B}^{n}\) dass \(A^{c} \in \mathcal{B}^{n}\).
      Daraus folgt, dass \(A^{c} \in C(x)\).
    \end{itemize}
  \item \(C(x)\) ist bzgl. abzählbaren Vereinigungen stabil.  Denn sei
    die Folge \((A_{n}) \subseteq C(x)\) beliebig.  Wir zeigen, dass
    \(\bigcup_{n\in\mathbb{N}}{A_{n}}\in C(x)\) gilt. Sei
    \(x \in \mathbb{R}^{n}\) beliebig.  Es gibt zwei Fälle.
    \begin{itemize}
    \item Falls \(x \in A_{n}\) für mindestens ein
      \(n \in \mathbb{N}\).  Dann gilt
      \[x + \bigcup_{n \in \mathbb{N}}{A_{n}} = \{x\} \cup \bigcup_{n \in \mathbb{N}}{A_{n}} = \bigcup_{n \in
          \mathbb{N}}{A_{n}} \in C(x).\]
    \item Falls \(x \in A^{c}_{n}\) für alle \(n \in \mathbb{N}\).  Wegen Voraussetzung
      dass \(A_{n} \in \mathcal{B}^{n}\) und \(x + A_{n} \in \mathcal{B}^{n}\) für alle \(n \in \mathbb{N}\)
      und \(\mathcal{B}^{n}\) bzgl. abzählbare Vereinigung stabil ist, gilt
      \[x + \bigcup_{n\in \mathbb{N}}{A_{n}} = \{x\} \cup \bigcup_{n\in \mathbb{N}}{A_{n}} = \bigcup_{n \in
          \mathbb{N}}{\left(\{x\} \cup A_{n}\right)} \in \mathcal{B}^{n}.\]
    \end{itemize}
    Damit ist \(C(x)\) bzgl. abzählbare Vereinigungen stabil.
  \end{itemize}
  Weil \(\mathcal{B}^{n}\) die von \(\mathcal{R}_{\text{EG}}^{n}\) erzeugte, kleinste
  \(\sigma\)-Algebra ist, gilt \(\mathcal{B}^{n} \subseteq C(x)\).  Damit gilt
  \[\mathcal{B}^{n} = \sigma(\mathcal{R}_{\text{EG}}^{n}) \subseteq C(x) \subseteq \mathcal{B}^{n}\]
  also \(C(x) = \mathcal{B}^{n}\) und daraus folgt die Behauptung.
\end{proof}
\subsection{H 2.2}
\subsubsection{Definitionen}
\begin{Definition}
  Sei \(\Omega\) eine Menge, \(\mathcal{R}\) eine Ring über \(\Omega\) und
  \(A \subseteq \Omega\).  \(A\) heißt \textbf{endlich messbar} bzgl.
  \(\mu^{*}\), falls (i) \(\mu^{*}(A) < \infty\) und (ii) existiert eine Folge
  \((E_{n})_{n \in \mathbb{N}} \subseteq \mathcal{R}\) sodass
  \(\lim_{n \to \infty}{d(E_{n}, A)}=0\) gilt.  Die Mengensystem aller endlich
  messbaren Teilmengen von \(\Omega\) ist mit \(\mathcal{M}_{\mu^{*}}\) bezeichnet.
\end{Definition}
\begin{Definition}
  Sei Voraussetzungen wie oben.  Die Menge \(A\) heißt \textbf{messbar}
  bzgl. \(\mu^{*}\), falls es existiert \((A_{n})_{n \in \mathbb{N}} \subseteq \mathcal{M}_{\mu^{*}}\)
  sodass \(A = \bigcup_{n \in \mathbb{N}}{A_{n}}\) gilt.  Die Mengensystem aller
  messbaren Teilmengen von \(\Omega\) ist mit \(\mathcal{A}_{\mu^{*}}\) bezeichnet.
\end{Definition}
\begin{Satz}
  Sei \(\lambda\) die elemetargeom. Inhalt und sei \(\lambda^{*}\) das von
  \(\lambda\) induzierte äußere Maß.  Die Mengensystem
  \(\mathcal{L} \coloneq \mathcal{A}_{\lambda^{*}}\) aller bzgl.
  \(\lambda^{*}\) messbaren Mengen heißt die \textbf{Lebesgue}
  \(\sigma\)\textbf{-Algebra}. (Korollar 1.31)
\end{Satz}
\subsubsection{Aufgabe}
\begin{Behauptung}
  Es gilt \(\cardi{\mathcal{L}} = \cardi{\mathcal{P}(\mathbb{R})}\).
\end{Behauptung}
\begin{proof}
  Wir zeigen, dass die Behauptung gilt, indem wir zeigen, dass
  \(\cardi{\mathcal{L}} \le \cardi{\mathcal{P}(\mathbb{R})}\) und \(\cardi{\mathcal{L}} \ge \cardi{\mathcal{P}(\mathbb{R})}\) gelten.
  Die Ungleichung \(\cardi{\mathcal{L}} \le \cardi{\mathcal{P}(\mathbb{R})}\) gilt wegen \(\mathcal{L} \subseteq \mathcal{P}(\mathbb{R})\).
  Sei
  \[C = \{ x \in \interval{0}{1} \colon x = \sum_{k \in \mathbb{N}}{x_{k} \cdot 3^{-k}}, x_{k}
    \in \{0, 2\}\}\] das Cantor'sche Diskontinuum.  Wegen Übung enthält
  \(C\) kein Intervall positiver Länge und ist nirgends dicht sowie
  total unzusammenhängend. Daraus folgt, dass jeder Teilmenge von
  \(C\) eine Nullmenge ist.  Daraus folgt, dass \(\mathcal{P}(C) \subseteq \mathcal{L}\).

  Weiter existiert eine bijektive Abbildung zwischen den Intervall
  \(\interval{0}{1}\) und die Menge der reellen Zahlen \(\mathbb{R}\).  Damit ist
  der Intervall und die Menge der reellen Zahlen gleichmächtig.  
  Außerdem ist jede Zahl \(a \in \interval{0}{1}\) binär darstellbar, das
  heißt, dass
  \[\interval{0}{1} = \{ x \in \interval{0}{1}\colon x = \sum_{k\in \mathbb{N}}{x_{k} \cdot
      2^{-k}}, x_{k} \in \{0, 1\}\}\]
  Damit gilt
  \[\cardi{C} = \cardi{\interval{0}{1}} = \cardi{\mathbb{R}}, \quad \mathcal{P}(C) \subseteq \mathcal{L}, \quad
    \cardi{\mathcal{P}(C)} = 2^{\cardi{C}}.\]
  Damit gilt
  \[\card(\mathcal{L}) \ge \card(\mathcal{P}(\mathbb{R})). \]
  Damit gilt die Behauptung.
\end{proof}
\subsection{H 2.3}
\subsubsection{Definitionen}
\begin{Definition}
  Ein \textbf{Maßraum} ist ein Tripel \((\Omega, \mathcal{A}, \mu)\) bestehend aus einer
  Menge \(\Omega\), einer \(\sigma\)-Algebra \(\mathcal{A}\) über \(\Omega\) und einem Maß \(\mu\) auf \(\mathcal{A}\).
\end{Definition}
\begin{Definition}
  Sei \((\Omega, \mathcal{A}, \mu)\) ein Maßraum.  Eine Menge \(N \in \mathcal{A}\) heißt Nullmenge,
  falls \(\mu(N) = 0\).  Eine Menge \(T \subseteq \Omega\) heißt vernachlässigbar, falls
  es eine Nullmenge \(N \in \mathcal{A}\) gibt, sodass \(T \subseteq N\) gilt.
  \(\mu\) bzw. \((\Omega, \mathcal{A}, \mu)\) heißt vollständig, falls jede vernachlässigbare
  Menge eine Nullmenge ist.
\end{Definition}
\subsubsection{Aufgabe}
Sei \((\Omega, \mathcal{A}, \mu)\) ein Maßraum.  Weiter definieren wir
\[\olA \coloneq \{A \cup N \mid A \in \mathcal{A}, N \text{ ist
  }\mu\text{-vernachlässigbar}\}, \quad \olm\colon \olA \to
  \interval{0}{\infty}, \quad \olm(A \cup N) \coloneq \mu(A).\]
\subsubsection{Teil i}
\begin{Behauptung}
  \(\olA\) ist eine \(\sigma\)-Algebra über \(\Omega\).
\end{Behauptung}
\begin{proof}
  Wir zeigen, dass \(\olA\) die drei Eigenschaften einer
  \(\sigma\)-Algebra besitzt.
  \begin{itemize}
  \item Es gilt \(\Omega \in \olA\).  Denn, es gilt
    \(\emptyset \in \mathcal{A}, \emptyset \subseteq \Omega, \mu(\emptyset) = 0, \emptyset \subseteq \emptyset\).  Daraus folgt, dass
    \(\emptyset\) ist \(\mu\)-vernachlässigbare Nullmenge.  Es gilt auch, dass
    \(\Omega \in \mathcal{A}\).  Wegen der Definition von \(\olA\) gilt
    \(\Omega \cup \emptyset = \Omega \in \olA\).  
  \item Wir bemerken, dass auch
    \(\mathcal{A} \subseteq \olA\) gilt.  Die Begründung ist dieselbe.
  \item \(\olA\) ist komplementstabil.  Denn sei
    \((A \cup N) \in \olA\) beliebig.  Dann gilt
    \((A \cup N)^{c} = (A^{c}\cap N^{c}) \).

    Es gibt zwei Fälle.  Falls \(N^{c} \in \mathcal{A}\).  Dann ist wegen
    \(A^{c} \in \mathcal{A}\) dass
    \(A^{c} \cap N^{c} \in \mathcal{A} \subseteq \olA\).  (Stabil bzgl. abzählbare
    Durchschnitten, Bemerkung 1.5.)

    Falls \(N^{c} \notin \mathcal{A}\). Weil \(N \subseteq \Omega\) eine
    \(\mu\)-vernachlässigbare Menge ist, existiert eine Menge
    \(M \in \mathcal{A}\) mit \(\mu(M) = 0\) und \(N \subseteq M\).  Daraus folgt, dass \(M^{c} \subseteq
    N^{c}\).  Dann gilt
    \[A^{c} \cap N^{c} = (A^{c} \cap M^{c}) \cup (A^{c} \cap (N^{c} \setminus M^{c})) =
      (A^{c} \cap M^{c}) \cup (A^{c} \cap (M \setminus N))\]
    indem \(A, M \in \mathcal{A}, N \subseteq M\) und \(\mu(M) = 0\) gelten.  Daraus folgt, dass
    \[(A^{c} \cap M^{c}) \in \mathcal{A}, \quad (A^{c} \cap (M \setminus N)) \subseteq M\] also ist
    \((A^{c} \cap (M \setminus N))\) vernachlässigbar.  Damit ist
    \[(A^{c} \cap M^{c}) \cup (A^{c} \cap (M \setminus N)) \in \olA\]
    und die Behauptung gilt.
  \item \(\olA\) ist stabil bzgl. abzählbar viele Vereinigungen.  Sei
    die Folge von Mengen \((A_{i} \cup N_{i})_{i \in \mathbb{N}} \subseteq \olA\) beliebig.
    Wir zeigen, dass
    \[\bigcup_{i \in \mathbb{N}}(A_{i} \cup N_{i}) \in \olA\]
    gilt.  Dies gilt, wegen
    \[\bigcup_{i\in\mathbb{N}}(A_{i} \cup N_{i}) = \left(\bigcup_{i\in\mathbb{N}}A_{i}\right) \cup
      \left(\bigcup_{i\in\mathbb{N}}N_{i}\right)\]
    wobei existiert die Folge von Nullmengen \((M_{i})\) sodass \(N_{i} \subseteq
    M_{i}\) und \(\mu(M_{i}) = 0\) für alle \(i \in \mathbb{N}\) dass
    \begin{align*}
      \bigcup_{i \in \mathbb{N}}A_{i} \in \mathcal{A} , \quad \bigcup_{i\in\mathbb{N}}N_{i} \subseteq \bigcup_{i \in \mathbb{N}}M_{i}
    \end{align*}
    also ist \(\bigcup_{i\in\mathbb{N}}N_{i}\) eine Teilmenge von eine Vereinigung von
    abzählbar viele Nullmengen. Wegen [Beispiel 1.33] ist eine
    abzählbare Vereinigung von Nullmengen wieder eine Nullmenge.
    Daraus folgt, dass \(\bigcup_{i\in\mathbb{N}}N_{i}\) vernachlässigbar ist.  Daraus
    folgt
  \[\bigcup_{i\in\mathbb{N}}(A_{i} \cup N_{i}) = \left(\bigcup_{i\in\mathbb{N}}A_{i}\right) \cup
      \left(\bigcup_{i\in\mathbb{N}}N_{i}\right) \in \olA.\]
  \end{itemize}
  Damit ist \(\olA\) eine \(\sigma\)-Algebra.
\end{proof}
\subsubsection{Teil ii}
\begin{Behauptung}
  \(\olm\) ist eine wohldefinierte Maß auf \((\Omega, \olA)\) mit
  \[\olm\colon \olA \to
  \interval{0}{\infty}, \quad \olm(A \cup N) \coloneq \mu(A).\]
\end{Behauptung}
\begin{proof}
  Wir zeigen Wohldefiniertheit.  Seien die vier Mengen
  \[A_{1}, A_{2} \in \mathcal{A}; \quad N_{1}, N_{2} \text{ sind } \mu\text{-vernachlässigbar}\]
  beliebig mit
  \[B = A_{1} \cup N_{1} = A_{2} \cup N_{2}.\]
  Wir zeigen, dass
  \[B = \olm(A_{1} \cup N_{1}) = \olm(A_{2} \cup N_{2})\]
  gelten.  O.B.d.A. seien
  \[A_{2} \setminus A_{1} \ne \emptyset, \quad N_{1} = N_{2} \cup (A_{2} \setminus A_{1}).\]
\end{proof}
\subsection{H 2.4}
\begin{Behauptung}
  Es existiert kein translationsinvariantes Maß auf \(\mathcal{P}(\mathbb{R})\) mit
  \(\mu(\linterval{0}{1}) = 1\).
\end{Behauptung}
\begin{proof}
  Wir definieren die Relation
  \(x \sim y \Leftrightarrow x - y \in \mathbb{Q}\) mit
  \(x, y \in \mathbb{R}\).  Wegen [Übung 2] ist diese Relation eine
  Äquivalenzrelation.  Wegen der Auswahlaxiom existiert
  \(K \subseteq \rinterval{0}{1}\) mit
  \[\cardi{K \cap (x+\mathbb{Q})} = 1\] für alle \(x \in \mathbb{R}\).   Es folgt dann, auch wegen [Übung 2],
  \[\mathbb{R} = \dot{\bigcup_{q \in \mathbb{Q}}} (K+q).\]
  Wir definieren \(Q \coloneq \mathbb{Q} \cap \rinterval{0}{1} \), dann ist \(Q\) abzählbar. Daraus folgt, dass
  \[\linterval{0}{1} \subset \bigcup_{q \in Q} (K+q) \subset
    \interval{0}{2}.\]
  Angenommen, es existiert ein translationsinvariantes Maß auf \(\mathcal{P}(\mathbb{R})\) mit
  \(\mu(\linterval{0}{1}) = 1\).  Falls \(\mu(K) = 0\).  Dann folgt,
  \[1 = \mu(\linterval{0}{1}) \le \mu(\bigcup_{q \in Q} (K+q)) = \sum_{q \in Q}\mu(K+q)
    = \sum_{q \in Q}\mu(K) = 0.\]
  Widerspruch!  Falls \(\mu(K) > 0\).  Dann folgt,
  \[2 = \mu(\linterval{0}{1}) + {\mu(\linterval{1}{2})}
     = \mu(\linterval{0}{2}) \ge \mu(\bigcup_{q \in Q} (K+q)) = \sum_{q \in Q}\mu(K+q)
     = \sum_{q \in Q}\mu(K) = \infty. \]
   Widerspruch!
 \end{proof}
\end{document}
