% 使用 chktex 检查 tex 文件中的语法错误
% settings for chktex
% chktex-file 3

\documentclass[draft,a5paper]{article}

% 让每个章节 subsection 在新的一页上开始
% 而不是紧接着上一章节
\AddToHook{cmd/subsection/before}{\clearpage}

% 隐藏默认的章节序号:实际作业中与这个冲突
% https://tex.stackexchange.com/a/30225
\setcounter{secnumdepth}{0}

% 设置页边距,上下左右
\usepackage{geometry}
\geometry{margin=1.5cm}

% 让 TeX 支持德语
\usepackage[ngerman]{babel}

% 调整 Emacs 预览字体大小
% (setq preview-scale-function 1.5)

% TeX Gyre Schola and unicode-math
% which in turn loads amsmath,mathtools,fontspec and friends
\usepackage{xcharter-otf}

% 加载数学相关的包
\usepackage{
  % 自然段间留空
  parskip,
  % 区间排版
  interval
}

% 中文支持,暂时不需要
% \usepackage[UTF8]{ctex}

% 根据 AMS 建议,应为成对符号(比如绝对值)定义新命令
\providecommand{\abs}[1]{\left\lvert#1\right\rvert}
\providecommand{\norm}[1]{\left\lVert#1\right\rVert}

\usepackage{amsthm}
% 定理定义,依赖于 amsthm
\theoremstyle{remark}
\newtheorem*{Behauptung}{Behauptung}
\newtheorem*{Lemma}{Lemma}
\newtheorem*{Satz}{Satz}
\newtheorem*{Definition}{Definition}

% 定义新函数,依赖于AMSmath
\DeclareMathOperator{\card}{card}

\newcommand{\wt}{\widetilde}
\newcommand{\dd}{\,\mathrm{d}}
% 标题与作者
\title{HA 7, Ana 3, Monika, MM 11}
\author{Zhang 484981, Yang 466096, Guo 480788}

\begin{document}
\maketitle
\begin{center}
  Meng Zhang 484981, Yiwen Yang 466096, Yuchen Guo 480788
\end{center}

\subsection{H 7.1}
\begin{Behauptung}
  Sei \((\Omega, \mathcal{A}, \mu)\) ein Maßraum und seien \(f_{1}, \ldots, f_{n}\colon \Omega \to
  \overline{\mathbb{R}}\) messbare Funktionen.  Seien \(r, p_{1}, \ldots, p_{n} \in
  \ointerval{0}{\infty}\) mit \(\sum_{i=1}^{n}{\frac{1}{p_{i}}} = \frac{1}{r}\).
  Dann gilt
  \[\left(\int_{\Omega}\prod_{i=1}^{n}{\abs{f_{i}}^{r}\dd\mu}\right)^{1/r} \le
    \prod_{i=1}^{n}\left(\int_{\Omega}\abs{f_{i}}^{p_{i}}\dd\mu\right)^{1/p_{i}}\le\infty.\]
\end{Behauptung}
\begin{proof}
  Falls existiert ein \(1 \le i \le n\) sodass gilt
  \[\int\abs{f_{i}}^{p_{i}}\dd\mu = \infty\]
  dann gilt wegen \(\infty^{s} \coloneq \infty\) für alle \(s >0\) dass
  \(\infty \le \infty\) und die Behauptung gilt.

  Falls für alle \(1\le i \le n\) dass
  \[f_{i} \in \mathcal{L}_{p_{i}}(\mu)\] gilt.  Vollständige Induktion über
  \(n\).  Induktionsanfang, \(n = 2\).  Die Behauptung gilt, der Beweis
  ist derselbe wie Hölder-Ungleichung.  Induktionsannahme. Die
  Behauptung gelte für ein festes \(n-1\).  Induktionsschritt,
  \(n -1 \to n\).  Wir betrachten
  \[\frac{1}{p_{1}}+\cdots+\frac{1}{p_{n-1}} + \frac{1}{p_{n}}
    = \frac{1}{r}.\]
  Dann ist wegen Hölder-Ungleichung im Fall \(n = 2\) für \(p, q =
  \frac{p_{n}}{p_{n}-r}, \frac{p_{n}}{r}\)dass
  \begin{align*}
    \int\abs{f_{1} \cdots f_{n-1}}^{r}\abs{f_{n}}^{r}\dd\mu \le
    \left(\int\abs{f_{1}\cdots f_{n-1}}^{\frac{rp_{n}}{p_{n}-r}}\dd\mu\right)^{\frac{p_{n}-r}{p_{n}}}\left(\int\abs{f_{n}}^{p_{n}}\right)^{\frac{r}{p_{n}}}.
  \end{align*}
  Nun gilt
  \[\frac{1}{p_{1}}+\cdots+\frac{1}{p_{n-1}}=\frac{1}{r} -
    \frac{1}{p_{n}}  = \frac{p_{n}-r}{rp_{n}}.\]
  Wegen Induktionsannahme, dass die Behauptung für \(n-1\) gelte, folgt
  die Behauptung.
\end{proof}

\subsection{H 7.4}
Seien \(p \in \interval{1}{\infty}\) und
\[A \coloneq L_{p}(\ointerval{0}{1}, \mathcal{B}(\ointerval{0}{1}), \lambda), \quad M = \{f \in A\mid f \ge 0
  \text{ fast überall}\}.\]

\subsubsection{Teil i}

\begin{Behauptung}
  \(M\) ist eine abgeschlossene Teilmenge von \(A\).
\end{Behauptung}
\begin{proof}
  Wir zeigen, dass \(M\) abgeschlossen ist, indem wir zeigen, dass
  \[M^{c}\coloneq\{f \in A \mid \forall N \in  \mathcal{B}(\ointerval{0}{1}), \exists y \in \ointerval{0}{1}
    \setminus N \colon f(y) < 0\}\]
  (wobei \(N\) eine Nullmenge) offen ist.  Sei \(g \in M^{c}\) beliebig.  Wir zeigen, dass es ein \(\varepsilon >
  0\) existiert, sodass \(U_{\varepsilon}(g) \subseteq M^{c}\) gilt.

  Es gilt
  \[\norm{f - g}_{p} = \left(\int\abs{f-g}^{p}\dd \lambda\right)^{1/p}.\]
  Wir erhalten das globale Minimum von \(\norm{f-g}_{p}\), genau dann
  \[f(x) = g(x) \text{ für alle } x \in \ointerval{0}{1} \setminus \{y\} \text{
      und } f(y) = 0.\]
  In diesem Fall gilt
  \[\norm{f - g}_{p} = \abs{g(x)} = -g(x) > 0.\]
  Sei \(0 < \varepsilon < -g(x)\), dann gilt \(U_{\varepsilon}(g) \subseteq M^{c}\).
\end{proof}
\end{document}
