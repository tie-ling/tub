% 使用 chktex 检查 tex 文件中的语法错误
% settings for chktex
% chktex-file 3

\documentclass[draft,a5paper]{article}

% 让每个章节 subsection 在新的一页上开始
% 而不是紧接着上一章节
\AddToHook{cmd/subsection/before}{\clearpage}

% 隐藏默认的章节序号:实际作业中与这个冲突
% https://tex.stackexchange.com/a/30225
\setcounter{secnumdepth}{0}

% 设置页边距,上下左右
\usepackage{geometry}
\geometry{a5paper,
 left=1cm,
 right=1cm,
 top=1cm,
 bottom=2cm
}

% 让 TeX 支持德语
\usepackage[ngerman]{babel}

% 调整 Emacs 预览字体大小
% (setq preview-scale-function 1.5)

% TeX Gyre Schola and unicode-math
% which in turn loads amsmath,mathtools,fontspec and friends
\usepackage{xcharter-otf}


% 加载数学相关的包
\usepackage{
  % 自然段间留空
  parskip,
  % 区间排版
  interval
}

% 中文支持,暂时不需要
% \usepackage[UTF8]{ctex}

% 根据 AMS 建议,应为成对符号(比如绝对值)定义新命令
\providecommand{\abs}[1]{\left\lvert#1\right\rvert}
\providecommand{\norm}[1]{\left\lVert#1\right\rVert}

\usepackage{amsthm}
% 定理定义,依赖于 amsthm
\theoremstyle{remark}
\newtheorem*{Behauptung}{Behauptung}
\newtheorem*{Lemma}{Lemma}
\newtheorem*{Satz}{Satz}
\newtheorem*{Definition}{Definition}

% 定义新函数,依赖于AMSmath
\DeclareMathOperator{\card}{card}

\title{HA 0, Ana 3, Monika, MM 11}
\author{Zhang 484981, Yang 466096, Guo 480788}

\begin{document}
\maketitle
\begin{center}
  Meng Zhang 484981, Yiwen Yang 466096, Yuchen Guo 480788
\end{center}

\newpage

\subsection*{H 0.1}
Seien \(I \coloneq \rinterval{t_{0}}{T} \subseteq \mathbb{R}\) und \(\alpha, \beta \in \mathcal{C}(I, \mathbb{R})\).
\subsubsection*{H 0.1, Teil i}
Sei \(\varphi \in \mathcal{C}^{1}(I, \mathbb{R})\) mit \(\varphi'(t) \le \alpha(t) + \beta(t)\varphi(t)\) für alle \(t \in
I\).

\begin{Behauptung}
  Es gilt für alle \(t \in I\), dass
  \[\varphi(t) \le \varphi(t_{0})e^{\int_{t_{0}}^{t}\beta(\tau)d\tau}+\int_{t_{0}}^{t}\alpha(s)e^{\int_{s}^{t}\beta(\tau)d\tau}ds.\]
\end{Behauptung}

\begin{proof}
  Sei \(u(t) \coloneq e^{-\int_{t_{0}}^{t}\beta(\tau)d\tau}\).  Wir zeigen, dass \((u\varphi)' \le
  u\alpha\) gilt.  Offensichtlich gilt wegen der Hauptsatz der
  Differential- und Integralrechnung dass
  \begin{align*}
    (u\varphi)' &= u' \varphi + u \varphi' \\
    &= [ u \cdot \left(-\int_{t_{0}}^{t}\beta(\tau)d\tau\right)'(t) ] \varphi + u
      \varphi' \\
    &= - u \beta \varphi + u \varphi'
  \end{align*}
  siehe auch [Amann, Analysis II, Theorem VI.4.12].

  Nun wegen Voraussetzung dass
  \[\varphi' \le \alpha + \beta \varphi\]
  gilt
  \[(u \varphi)' = - u \beta \varphi + u \varphi' \le u \alpha + u \beta \varphi - u \beta \varphi = u \alpha.\] Damit gilt
  \((u\varphi)' \le u \alpha\).

  Wir integrieren die Ungleichung, erhalten wir
  \begin{align*}
    \int_{t_{0}}^{t}{(u(s) \varphi(s))' ds} &\le \int_{t_{0}}^{t}{u(s) \alpha(s) ds} \\
    u(t)\varphi(t) - u(t_{0})\varphi(t_{0}) &\le \int_{t_{0}}^{t}{\alpha(s)
                                  e^{\int_{s}^{t_{0}}\beta(\tau)d\tau} ds}
  \end{align*}
  Es gilt \(u(t_{0}) = 1\) und \(u(t) > 0\) für alle
  \(t \in \mathbb{R}\).  Daraus folgt, dass
  \begin{align*}
    \varphi(t) &\le \frac{\varphi(t_{0})}{u(t)} + \int_{t_{0}}^{t}{\alpha(s)
           e^{\int_{s}^{t_{0}}\beta(\tau)d\tau} ds} \\
           &\le \frac{\varphi(t_{0})}{u(t)} + \int_{t_{0}}^{t}{\alpha(s)
             e^{\int_{t_{0}}^{t}\beta(\tau)d\tau + \int_{s}^{t_{0}}\beta(\tau)d\tau} ds} \\
         &= \varphi(t_{0})e^{\int_{t_{0}}^{t}\beta(\tau)d\tau}+\int_{t_{0}}^{t}\alpha(s)e^{\int_{s}^{t}\beta(\tau)d\tau}ds
  \end{align*}
  weil die Exponentialfunktion \(\exp(x)\) auf \(\mathbb{R}\) monoton steigend ist.
  Damit gilt die Behauptung.
\end{proof}

\end{document}
