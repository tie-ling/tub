% 使用 chktex 检查 tex 文件中的语法错误
% settings for chktex
% chktex-file 3

\documentclass[draft,a5paper]{article}

% 让每个章节 subsection 在新的一页上开始
% 而不是紧接着上一章节
\AddToHook{cmd/subsection/before}{\clearpage}

% 隐藏默认的章节序号:实际作业中与这个冲突
% https://tex.stackexchange.com/a/30225
\setcounter{secnumdepth}{0}

% 设置页边距,上下左右
\usepackage{geometry}

\geometry{left=1cm,
 right=1cm,
 top=1cm,
 bottom=2cm}

% 让 TeX 支持德语
\usepackage[ngerman]{babel}

% 调整 Emacs 预览字体大小
% (setq preview-scale-function 1.5)

\usepackage{amsmath,mathtools,fontspec,amsthm,parskip,interval,unicode-math}
\usepackage{xcharter-otf}

% 根据 AMS 建议,应为成对符号(比如绝对值)定义新命令
\providecommand{\abs}[1]{\left\lvert#1\right\rvert}
\providecommand{\norm}[1]{\left\lVert#1\right\rVert}

% 定理定义,依赖于 amsthm
\theoremstyle{remark}
\newtheorem*{Behauptung}{Behauptung}
\newtheorem*{Lemma}{Lemma}
\newtheorem*{Satz}{Satz}
\newtheorem*{Definition}{Definition}

% 定义新函数,依赖于AMSmath
\DeclareMathOperator{\card}{card}
\DeclareMathOperator{\Vol}{Vol}

\newcommand{\wt}{\widetilde}
\newcommand{\dd}{\,\mathrm{d}}
% 标题与作者
\title{HA 11, Ana 3, Monika, MM 11}
\author{Zhang 484981, Yang 466096, Guo 480788}

\begin{document}
\maketitle
\begin{center}
  Meng Zhang 484981, Yiwen Yang 466096, Yuchen Guo 480788
\end{center}
\subsection{Aufgabe 11.2}
Wir berechnen das Volumen von \(M \coloneq A \cap B\) für
\[
  A = \{(x, y, z) \in \mathbb{R}^{3} \mid x^{2} + y^{2} + z^{2} \le 1\} \text{ und }
  B = \{(x, y, z) \in \mathbb{R}^{3} \mid x^{2} + y^{2} \le 1/2\}.
\]

Es sei
\[
  M' \coloneq \{(x, y) \in \mathbb{R}^{2} \mid x^{2} + y^{2} \le 1/2 \}.
\]
Wegen Satz von Fubini erhalten wir
\begin{align*}
  \Vol(M) &= \int_{M'}\left(\int_{-\sqrt{1 - (x^{2} + y^{2})}}^{\sqrt{1 -
            (x^{2} + y^{2})}} 1 \dd z\right)\dd (x, y) \\
          &= 2\int_{M'}\sqrt{1 - (x^{2} + y^{2})} \dd (x, y).
\end{align*}
Wir betrachten nun die Polarkoordinatentransformation
\begin{align*}
  g(r, \varphi) =
  \begin{bmatrix}
    r \cos \varphi \\ r \sin \varphi
  \end{bmatrix}.
\end{align*}
Wir erhalten damit als Integrationsbereich
\begin{align*}
  r^{2} &\le 1/2, \\
  r^{2} &\le 1.
\end{align*}
Es sei zunächst \(r > 0\).  Lösen wir die beiden obigen Ungleichungen
nach \(r\) auf, so sehen wir, daß die zweite Ungleichung stets erfüllt
ist, sofern nur die erste gilt.  Nun bemerken wir, daß diese
Beobachtung trivialerweise auch für \(r = 0\) zutrifft.  Wir können die
zweite Ungleichung also in unseren weiteren Betrachtungen
vernachlässingen und erhalten mit der Achsensymmetrie des Cosinus
\begin{align*}
  \Vol(M) = 2 \int_{-\pi/2}^{\pi/2}{\int_{0}^{\cos \varphi}\sqrt{1-r^{2}}}\dd r \dd \varphi
\end{align*}
\end{document}
