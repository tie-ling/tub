% 使用 chktex 检查 tex 文件中的语法错误
% settings for chktex
% chktex-file 3

\documentclass[draft,a5paper]{article}

% 让每个章节 subsection 在新的一页上开始
% 而不是紧接着上一章节
\AddToHook{cmd/subsection/before}{\clearpage}

% 隐藏默认的章节序号:实际作业中与这个冲突
% https://tex.stackexchange.com/a/30225
\setcounter{secnumdepth}{0}

% 设置页边距,上下左右
\usepackage{geometry}
\geometry{margin=1.5cm}

% 让 TeX 支持德语
\usepackage[ngerman]{babel}

% 调整 Emacs 预览字体大小
% (setq preview-scale-function 1.5)

% TeX Gyre Schola and unicode-math
% which in turn loads amsmath,mathtools,fontspec and friends
\usepackage{xcharter-otf}

% 加载数学相关的包
\usepackage{
  % 自然段间留空
  parskip,
  % 区间排版
  interval
}

% 中文支持,暂时不需要
% \usepackage[UTF8]{ctex}

% 根据 AMS 建议,应为成对符号(比如绝对值)定义新命令
\providecommand{\abs}[1]{\left\lvert#1\right\rvert}
\providecommand{\norm}[1]{\left\lVert#1\right\rVert}

\usepackage{amsthm}
% 定理定义,依赖于 amsthm
\theoremstyle{remark}
\newtheorem*{Behauptung}{Behauptung}
\newtheorem*{Lemma}{Lemma}
\newtheorem*{Satz}{Satz}
\newtheorem*{Definition}{Definition}

% 定义新函数,依赖于AMSmath
\DeclareMathOperator{\card}{card}

\newcommand{\wt}{\widetilde}
\newcommand{\dd}{\,\mathrm{d}}
% 标题与作者
\title{HA 6, Ana 3, Monika, MM 11}
\author{Zhang 484981, Yang 466096, Guo 480788}

\begin{document}
\maketitle
\begin{center}
  Meng Zhang 484981, Yiwen Yang 466096, Yuchen Guo 480788
\end{center}

\subsection{H 6.1}
\subsubsection{Teil i}
Wegen Korollar 1.100:
\begin{align*}
  \int_{\interval{0}{1}^{2}}{x e^{-xy} \dd \lambda^{2}(x, y)}
  &= \int\left(\int_{\interval{0}{1}_{x}} {x e^{-xy} \dd \lambda(y)}\right) \dd
    \lambda(x) \\
  &= \int^{1}_{0}\left(\int_{0}^{1} x e^{-xy} \dd \lambda(y)\right) \dd \lambda(x) \\
  &= \int_{0}^{1}\left(-e^{-xy}\bigg\vert_{0}^{1}\right) \dd \lambda(x) \\
  &= \int_{0}^{1}{1-e^{-x}} \dd \lambda(x) \\
  &= (x + e^{-x}) \bigg\vert_{0}^{1} = e^{-1}.
\end{align*}
\subsubsection{Teil ii}
\begin{align*}
  \int_{\interval{0}{\pi/2}^{2}}{\sin (x+y) \dd \lambda^{2}(x, y)}
  &= \int\left(\int_{\interval{0}{\pi/2}_{x}}{\sin (x+y) \dd \lambda(y)}\right) \dd
    \lambda(x) \\
  &= \int_{0}^{\pi/2}\left(\int_{0}^{\pi/2}{\sin(x+y) \dd
    \lambda(y)}\right) \dd \lambda(x) \\
  &= \int_{0}^{\pi/2}{\cos(x) - \cos(x+\pi/2)\dd \lambda(x)} \\
  &= \sin(\pi/2) - \sin(0) - \sin(\pi) + \sin(\pi/2) \\
  &= 2.
\end{align*}
\subsection{Teil iii}
\begin{align*}
  \int_{0}^{1}\int_{0}^{1-z}\int_{0}^{1-y-z}{1}\dd x \dd y \dd z
  &= \int_{0}^{1} \int_{0}^{1-z} 1-y-z \dd y \dd z \\
  &= \int_{0}^{1}  (1-z)^{2} - \frac{(1-z)^{2}}{2} \dd z \\
  &= -\frac{(1-z)^{3}}{6} \bigg\vert^{1}_{0}\\
  &= \frac{1}{6}.
\end{align*}

\subsection{H 6.2}
\begin{Behauptung}
  Sei \(f\colon \interval{a}{b} \to \rinterval{0}{\infty}\) eine messbare und
  Riemann-integrierbare Funktion sowie
  \[M \coloneq \{(x, y) \in \interval{a}{b} \times \mathbb{R} \mid 0 \le y \le f(x)\}.\]
  Dann ist \(M\) eine Borel-Menge mit
  \begin{equation}
    \label{eq:1}
    \lambda^{2}(M) = \int_{a}^{b}{f(x) \dd x}.
  \end{equation}
\end{Behauptung}
\begin{proof}
  Wir zeigen, dass die Menge \(M\) eine Borel-Menge ist, indem wir
  zeigen, dass \(M\) eine abgeschlossene Menge ist.  Wir bemerken, dass
  \(f\) wegen ihrer Riemann-integrierbarkeit auf dem abgeschlossenen
  Interval \(\interval{a}{b}\) ihrer Maximum annehmmt.  D.h., es
  existiert \(x \in \interval{a}{b}\) sodass \(f(x) = \max
  f(\interval{a}{b}) = \sup f(\interval{a}{b})\).

  Wir zeigen, dass
  \[M^{c} = \mathbb{R}^{2} \setminus M = \{(x,y) \in \mathbb{R}^{2} \mid x < a \text{ oder } x > b
    \text{ oder } y > \max f(\interval{a}{b}) \text{ oder } y < 0\}\]
  eine offene Menge ist.  Sei \(p \in M^{c}\) beliebig.  Wählen wir
  \[0 < \varepsilon < \min d(p, q), \quad q \in M,\] dann ist
  \(U_{\varepsilon}(p) \subseteq M^{c}\) eine Umgebung von \(p\).  Weil \(p\) beliebig gewählt
  war, folgt die Behauptung dass \(M^{c}\) offen und \(M\) abgeschlossen.
  Damit ist \(M \in \mathcal{B}^{2}\).

  Wir zeigen, dass die Gleichung \eqref{eq:1} gilt.  Aus der
  Riemann-integrierbarkeit von \(f\) folgt, dass die Oberintegral und
  Unterintegral von \(f\) auf dem Intervall \(\interval{a}{b}\)
  existieren und übereinstimmen.  Aus der Definition von Ober- und
  Unterintegral folgt
\[    \overline{\int_{a}^{b}}{f(x)\dd x}
    = \underline{\int_{a}^{b}}{f(x) \dd x}\]
  und
  \[    \inf\{I(\varphi) \mid f \le \varphi,~ \varphi \text{ Treppenfunktion}\}
    = \sup\{I(\psi) \mid \psi \le f,~ \psi \text{ Treppenfunktion}\}\]
  wobei
  \[I(\varphi) = \sum_{i=1}^{n}{c_{i} \cdot (x_{i} - x_{i-1})}\]
  gilt.  Es existiert daher zwei Folgen von Treppenfunktionen
  \((\varphi_{i})\) und \((\psi_{i})\) sodass gilt
  \[\lim_{i \to \infty} \varphi_{i} = \varphi, \quad \lim_{i \to \infty} \psi_{i} = \psi, \quad \varphi_{i} < \varphi, \quad
    \psi_{i} > \psi, \quad \lim_{i \to \infty}{I(\varphi_{i} - \psi_{i})} = 0.\]
  Wir bemerken, dass die zwei Funktionenfolgen \((-\varphi_{i})\) und \((\psi_{i})\)
  monoton wachsende Folge messbarer Funktionen sind.  Damit gilt wegen
  Satz von Beppo Levi, dass
  \[\int_{M}(\lim_{i \to \infty}\psi_{i}) \dd \lambda^{2}(M) = \lim_{i \to \infty}I(\psi_{i}) =
    I(\psi) = \int_{a}^{b}{f(x) \dd x}. \]
  Damit gilt die Gleichung \eqref{eq:1}.
\end{proof}
\end{document}
