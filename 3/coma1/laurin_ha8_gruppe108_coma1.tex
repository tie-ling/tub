% 使用 chktex 检查 tex 文件中的语法错误
% settings for chktex
% chktex-file 3

\documentclass[draft,a5paper]{article}

% 让每个章节 subsection 在新的一页上开始
% 而不是紧接着上一章节
\AddToHook{cmd/subsection/before}{\clearpage}

% 隐藏默认的章节序号:实际作业中与这个冲突
% https://tex.stackexchange.com/a/30225
\setcounter{secnumdepth}{0}

% 设置页边距,上下左右
\usepackage{geometry}

\geometry{left=1cm,
 right=1cm,
 top=1cm,
 bottom=2cm}

% 让 TeX 支持德语
\usepackage[ngerman]{babel}

% 调整 Emacs 预览字体大小
% (setq preview-scale-function 1.5)

\usepackage{amsmath,mathtools,fontspec,amssymb,amsthm,parskip,interval,unicode-math}
\usepackage{xcharter-otf}


% 根据 AMS 建议,应为成对符号(比如绝对值)定义新命令
\providecommand{\abs}[1]{\left\lvert#1\right\rvert}
\providecommand{\norm}[1]{\left\lVert#1\right\rVert}
\providecommand{\ceil}[1]{\left\lceil#1\right\rceil}
\providecommand{\floor}[1]{\left\lfloor#1\right\rfloor}

% 定理定义,依赖于 amsthm
\theoremstyle{remark}
\newtheorem*{Behauptung}{Behauptung}
\newtheorem*{Lemma}{Lemma}
\newtheorem*{Satz}{Satz}
\newtheorem*{Definition}{Definition}

% 定义新函数,依赖于AMSmath
\DeclareMathOperator{\card}{card}

\newcommand{\wt}{\widetilde}
\newcommand{\dd}{\,\mathrm{d}}

% 标题与作者
\title{HA 8, CoMa 1, Laurin, Gruppe 108}
\author{Kuan 480169, Yu 478912, Guo 480788}

\begin{document}
\maketitle
\begin{center}
Ting Yu Kuan 480169, Shilong Yu 478912, Yuchen Guo 480788
\end{center}
\newpage
\subsection{1. Aufgabe}
\subsubsection{Teil a}
% ub 9 aufgabe 4
\begin{Behauptung}
  Für alle \(x \in \ointerval{0}{1}\) gilt
  \[f(x) \coloneq x \log_{2}(x) + (1-x)\log_{2}(1-x) \ge -1.\]
\end{Behauptung}
\begin{proof}
  Es gilt für die ersten Ableitung von \(f\) dass
  \begin{align*}
    f'(x) &= \log_{2}(x) + \frac{1}{\ln 2} - \log_{2}(1-x) -
            \frac{1}{\ln 2} \\
    &= \log_{2}(x) - \log_{2}(1-x).
  \end{align*}
  Wir berechnen \(f'(x) > 0\), erhalten wir dann \(x > \frac{1}{2}\).
  Wir berechnen \(f'(x) < 0\), erhalten wir dann \(x < \frac{1}{2}\).
  Deshalb besitzt \(f'(x)\) ein globales Minimum im Punkt \(x =
  \frac{1}{2}\).
  Wir erhalten \(f(\frac{1}{2}) = -1\), also das globales Minimum ist
  \(-1\) und die Behauptung gilt.
\end{proof}
\subsubsection{Teil b}
\begin{Behauptung}
  Jeder auf paarweisen Vergleichen basierende Algorithmus zum
  Sortieren einer \(n\)-elementigen total geordneten Mengen benötigt im
  Average-Case \(\Omega(n \log n)\) Vergleiche.
\end{Behauptung}
\begin{proof}
  Wir nehmen an, dass die Elemente des Arrays paarweise verschieden
  sind.  Wir betrachten einen innere Knote des Entscheidungsbaums.
  Jede innere Knoten des Baums entsprechen Vergleichen im Algorithmus
  und besitzt zwei Kinder.  Im Average-Case ist den Kinder gleich groß
  und jedes innere Knoten reduziert die Anzahl von Vergleichen um ein
  Halbe.
  \begin{align*}
    \log_2(n!) &= \sum_{i=1}^n \log_2(i) \\
               &\le \sum_{i = \lfloor n/2 \rfloor}^n \log_2(i) \\
               &\le \lceil n/2 \rceil \log_2(\lfloor n/2 \rfloor) \\
               &\le \frac{n \log(n-1)}{2} - 1 \\
               &\in \Omega(n\log(n))
  \end{align*}
\end{proof}
\subsection{2. Aufgabe}
\subsubsection{Teil a}
\begin{verbatim}
def insertionSort(A):
    n = len(A)
    for i in range(1, n):
        s = A[i]
        k = i
        while k > 0 and s < A[k-1]:
            A[k] = A[k-1]
            k -= 1
        A[k] = s
\end{verbatim}
\begin{Behauptung}
  Für den List \texttt{[8, 7, 6, 5, 4, 3, 2, 1]}
benötigt InsertionSort die größtmögliche Anzahl an
Vergleichen.
\end{Behauptung}
\begin{proof}
  InsertionSort benötigt für diesen List die größtmögliche
Anzahl an Vergleichen, denn die List hat die
Eigenschaft, dass die Zahl an der \(i\)-ten Stelle kleiner als
alle Zahlen an der \(0 \le j \le i-1\)-ten Stelle und
die innere \texttt{while}-Schleife benötigt jeweils
\(i-1\)-mal Vergleichen zu terminieren.
\end{proof}
\subsubsection{Teil b}
\begin{verbatim}
def mergeSort(A):
    if len(A) <= 1:
        return(A)
    else:
        return(merge(mergeSort(A[:len(A)//2]),
                     mergeSort(A[len(A)//2:])))
def merge(left, right):
    merged = []
    while left and right:
        if left[-1] < right[0]:
            return (left + right)
        if left[0] < right[0]:
            merged.append(left[0])
            left = left[1:]
        else:
            merged.append(right[0])
            right = right[1:]
    merged.extend(left + right)
    return merged
\end{verbatim}
\begin{Behauptung}
  Für den List \texttt{[5, 1, 7, 3, 6, 2, 8, 4]}
benötigt MergeSort die größtmögliche Anzahl an
Vergleichen.
\end{Behauptung}
\begin{proof}
  MergeSort benötigt für diesen List die größtmögliche Anzahl an
  Vergleichen, denn die linksseitige- und rechtsseitige-Arrays
  alternierende Elemente besitzt und daher jeden Paar von Element
  einmal Vergleich benötigt.
\end{proof}
\subsubsection{Teil c}
\begin{verbatim}
def quickSort(A):
    if len(A) <= 1:
        return A
    else:
        pivot = A[0]
        B = []
        C = []
        for i in A[1:]:
            if i <= pivot:
                B.append(i)
            else:
                C.append(i)
        return quickSort(B) + [pivot] + quickSort(C)
\end{verbatim}
\begin{Behauptung}
  Für den List \texttt{[1, 2, 3, 4, 5, 6, 7, 8]}
benötigt QuickSort die größtmögliche Anzahl an
Vergleichen mit \texttt{pivot = A[0]}.
\end{Behauptung}
\begin{proof}
  QuickSort benötigt für diesen List die größtmögliche
Anzahl an Vergleichen, denn die List ist bereits sortiert
und wir nehmen jedes Mal das erste Element der List als
Pivot-Element und der Algorithmus benötigt insgesamt
\begin{align*}
7+6+5+4+3+2+1 \text{ Mal Vergleichungen}
\end{align*}
um zu terminieren.  Die Länge der Liste reduziert jedes
Mal um Eins und Eins ist die kleinstmögliche Zahl, mit
der der Algorithmus noch terminieren kann.  Damit ist
die Anzahl der Vergleichen am größten.
\end{proof}
\subsection{3. Aufgabe}
\subsubsection{Teil a}
\begin{verbatim}
def find_min_max(A: list[int]):
    list_of_smaller_elements = []
    list_of_larger_elements = []
    for i in range(0, len(A) - 1, 2):
        if A[i] < A[i+1]:
            list_of_larger_elements.append(A[i+1])
            list_of_smaller_elements.append(A[i])
        else:
            list_of_larger_elements.append(A[i])
            list_of_smaller_elements.append(A[i+1])
    max_element = list_of_larger_elements[0]
    min_element = list_of_smaller_elements[0]
    if len(A) % 2 == 1:
        if A[-1] < min_element:
            list_of_smaller_elements.append(A[-1])
        else:
            list_of_larger_elements.append(A[-1])
    for i in list_of_larger_elements:
        if i > max_element:
            max_element = i
    for i in list_of_smaller_elements:
        if i < min_element:
            min_element = i
    return min_element, max_element
\end{verbatim}
\subsubsection{Teil b}
\begin{Behauptung}
  Der Algorithmus ist korrekt und die
  Anzahl der Vergleiche ist kleiner-gleich
  \[3 \left \lfloor \frac{n}{2} \right \rfloor.\]
\end{Behauptung}
\begin{proof}
    Im ersten Teil vergleichen wir paarweise die Elementen
  in der ganzen Liste, zu jedem Element \(x\) der Liste
  \texttt{list\_of\_smaller\_elements} existiert mindestens
  ein Element \(y\) der Liste
  \texttt{list\_of\_larger\_elements} mit der Eigenschaft
  \(x < y\).  Insgesamnt wird höchstens
  \(\left \lfloor \frac{n}{2} \right \rfloor + 1\)
  Vergleichen benötigt.

  Im nächsten Schritt wird jeweils das Minimum und das
  Maximum in der beiden Listen mittels einem naiven
  Algorithmus gefunden.  Dieses Schritt benötigt
  höchstens
  \[\left \lfloor \frac{n}{2} \right \rfloor - 1 + \left
    \lfloor \frac{n}{2} \right \rfloor\]
  Vergleichungen.
  Insgesamt wird höchstens
  \(3 \left \lfloor \frac{n}{2} \right \rfloor\) Schritten
  benötigt.

  Die Korrektheit ist klar, denn das Maximum und das
  Minimum jeweils größer gleich und kleiner gleich alle
  Elemente und deshalb in der zwei Teil-Liste enthalten sind.
\end{proof}
\subsection{4. Aufgabe}
\subsubsection{Teil i}
\begin{Behauptung}
  Der Algorithmus läuft auch bei einer Partitionierung in Gruppen der
  Größe \(7\) in Linearzeit.
\end{Behauptung}
\begin{proof}
  Wir betrachten die folgende Routine.  Partitioniere \(A\) in
  \(\floor{\abs{A}/7}\) Teilmengen mit \(7\) Elementen und eine Teilmenge
  mit der restlichen \((n \bmod 7)\) Elementen, Laufzeit \(O(\abs{A})\).
  Bestimme für jede dieser Teilmengen deren Median und bilde Menge
  \(A_{M}\) dieser Mediane, Laufzeit \(O(\abs{A})\).  Berechne mittels
  rekursivem Aufruf von Algorithmus den Median \(x\) von \(A_{M}\) und
  verwende \(x\) als Pivot-Element.

  Wir erhalten
  \begin{align*}
    \abs{A_{M}} &= \ceil{\frac{n}{7}} \\
    \abs{A_{> x}} &\ge 4 \left(\ceil{\frac{1}{2}\ceil{\frac{n}{7}}} -
                    2\right) \ge \frac{4n}{14} - 8 \\
    \implies &\abs{A_{> x}} \le \frac{10n}{14} + 8
  \end{align*}
  Folglich gilt für die monotone Laufzeitfunktion des rekursiven
  Algorithmus
  \[f(n) \le f\left(\ceil{\frac{n}{7}}\right) +
    f\left(\floor{\frac{10n}{14} + 8}\right) + c \cdot n\]
  für ein \(c > 0\) und alle \(n \in \mathbb{N}_{> 0}\).
  Wegen letzter Folie des Vorlesungs gilt \(f(n) \in O(n)\).
\end{proof}
\subsubsection{Teil ii}
Bei Verwendung von Dreiergruppen läuft der Algorithmus nicht mehr in
Linearzeit.  Denn in diesem Fall gilt
\[\abs{A_{< x}} \le \floor{\frac{5n}{6}+2} \not{<} n\]
und den Beweis am letzter Folie des Vorlesungs gilt nicht mehr.
\subsubsection{Teil iii}
Man kann den Quicksort Algorithmus so modifizieren, indem man den
Pivot mittels Median-Partitionierung findet.
% ub 9 aufgabe 1
% ub 9 aufgabe 3
\end{document}

