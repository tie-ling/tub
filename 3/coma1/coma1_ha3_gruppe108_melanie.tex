% 使用 chktex 检查 tex 文件中的语法错误
% settings for chktex
% chktex-file 3

% draft mode incompatible with tikz
\documentclass[a5paper]{article}

% 让每个章节 subsection 在新的一页上开始
% 而不是紧接着上一章节
\AddToHook{cmd/subsection/before}{\clearpage}

% 隐藏默认的章节序号:实际作业中与这个冲突
% https://tex.stackexchange.com/a/30225
\setcounter{secnumdepth}{0}

% 设置页边距,上下左右
\usepackage{geometry}
\geometry{a5paper,
 left=1cm,
 right=1cm,
 top=1cm,
 bottom=2cm
}

% 让 TeX 支持德语
\usepackage[ngerman]{babel}

% 调整 Emacs 预览字体大小
% (setq preview-scale-function 1.5)

% TeX Gyre Schola and unicode-math
% which in turn loads amsmath,mathtools,fontspec and friends
\usepackage{xcharter-otf}


% 加载数学相关的包
\usepackage{
  % 自然段间留空
  parskip,
  % 区间排版
  interval
}

% 中文支持,暂时不需要
% \usepackage[UTF8]{ctex}

% 根据 AMS 建议,应为成对符号(比如绝对值)定义新命令
\providecommand{\abs}[1]{\left\lvert#1\right\rvert}
\providecommand{\norm}[1]{\left\lVert#1\right\rVert}

\usepackage{amsthm}
% 定理定义,依赖于 amsthm
\theoremstyle{remark}
\newtheorem*{Behauptung}{Behauptung}
\newtheorem*{Lemma}{Lemma}
\newtheorem*{Satz}{Satz}
\newtheorem*{Definition}{Definition}


% 标题与作者
\title{HA 3, CoMa 1, Melanie, Gruppe 108}
\author{Kuan 480169, Yu 478912, Guo 480788}

% 画图工具
\usepackage{tikz,pgfplots}
\pgfplotsset{compat=1.18}


\begin{document}
\maketitle
\begin{center}
Ting Yu Kuan 480169, Shilong Yu 478912, Yuchen Guo 480788
\end{center}
\subsection{1. Aufgabe}
\subsubsection{Teil a}
\begin{Behauptung}
  Aus \(b_1 > b_2\) folgt \(n_1 > n_2\).
\end{Behauptung}
\begin{proof}
  Wir widerlegen diese Behauptung mit einem Beispiel.  Seien
  \[
    n_{1} = (1)_{3}, \quad n_{2} = (1)_{2}, \quad b_{1} = 3, \quad b_{2} = 2
  \]
  dann gilt
  \[
    b_{1} > b_{2}, \quad n_{1} = n_{2}.
  \]
  Widerspruch!
\end{proof}
\subsubsection{Teil b}
\begin{Behauptung}
  Aus \(n_1 > n_2\) folgt \(b_1 > b_2\).
\end{Behauptung}
\begin{proof}
  Es gilt
  \[n_1=\sum_{i=0}^{l-1}{z_ib_1^i}=\sum_{i=0}^{l-1}{ \left(
        \frac{b_1}{b_2} \right)^iz_i b_2^i}, \quad n_2=\sum_{i=0}^{l-1}{z_ib_2^i}.\]
  Setzen wir \(x \coloneq b_{1}/b_{2}\).
  Aus \(n_1>n_2\) folgt, dass
  \(x^i > 1\) gilt.
  Es gibt zwei Fälle.   Falls \(0 < x \le 1\), dann ist \(x^{i} \le 1\),
  Widerspruch!  Falls \(x > 1\) dann ist \(x^{i} > 1\) und die
  Voraussetzung ist erfüllt.  Wegen \(b_1, b_2 \in \mathbb{N}_{>0}\)
  gilt dann \(b_1>b_2\).
\end{proof}
\subsubsection{Teil c}
\begin{Behauptung}
  Aus \(b_1|b_2\) folgt \(n_1|n_2\).
\end{Behauptung}
\begin{proof}
  Wir widerlegen die Behauptung mit einem Beispiel.  Seien
  \[b_{1} = 3, \quad b_{2} = 15, \quad n_{1} = (1110)_{3}, \quad n_{2} =
    (1110)_{15}\]
  dann gilt
  \[n_{1} = (39)_{10}, \quad n_{2} = (3615)_{10}, \quad n_{2}/n_{1} = 92.69\]
\end{proof}
\subsubsection{Teil d}
\begin{Behauptung}
  Aus \(n_1|n_2\) folgt \(b_1|b_2\).
\end{Behauptung}
\begin{proof}
  Wir widerlegen diese Behauptung mit einem Beispiel.  Seien \(n_1=(0)_2\)
  und \(n_2=(0)_3\).  Es gilt \(n_1=n_2=(0)_{10}\) und daher \(n_1|n_2\).  Es
  gilt aber nicht \(2|3\).
\end{proof}
\subsubsection{Teil e}
\begin{Behauptung}
  Es gilt \(\frac{n_1}{n_2}=\frac{b_1}{b_2}\).
\end{Behauptung}
\begin{proof}
  Wir widerlegen diese Behauptung mit einem Beispiel.  Seien \(n_1=(1)_2\)
  und \(n_2=(1)_3\).  Es gilt \(\frac{n_1}{n_2}=1\neq \frac{3}{2}\).
\end{proof}
\subsubsection{Teil f}
\begin{Behauptung}
  Sei \(b_1 > b_2\) und \(m \in \mathbb{N}\).  Dann
  existiert ein \(N \in \mathbb{N}\), sodass folgendes gilt:  Ist \(l > N\)
  und \(z_{l-1}\neq 0\), dann ist \(n_1 > mn_2\).
\end{Behauptung}
\begin{proof}
  Es gilt
  $n_1=\sum_{i=0}^{l-1}{z_ib_1^i}=\sum_{i=0}^{l-1}{ \left( \frac{b_1}{b_2}
    \right)^iz_i b_2^i}$ und \(n_2=\sum_{i=0}^{l-1}{z_ib_2^i}\).  Wegen
  \(b_1>b_2\) gilt \(\frac{b_1}{b_2}>1\).  Seien \(N=2\) und \(l=3\).  Falls
  \(m = 1\).  Dann gilt
  \begin{align*}
    n_1&=z_2\left(\frac{b_1}{b_2} \right)^2 b_2^2 +
         z_1\left(\frac{b_1}{b_2} \right) b_2 + z_0\\
    n_2&=z_2 b_2^2 + z_1 b_2 + z_0\\
       &=mn_2
  \end{align*}
  Nun sei \(m \in \mathbb{N}\) beliebig.  Dann gilt wegen
  \(\frac{b_1}{b_2}>1\) dass \(n_1>mn_2\).
\end{proof}
\subsection{2. Aufgabe}
\begin{Behauptung}
  Für die Funktion
  \[f(n) \coloneq \sum_{i=1}^{n}{1/i}\]
  gilt die Beziehung \(f(n)
  \in \Theta(\log(n))\).
\end{Behauptung}
\begin{proof}
  Wir zeigen, dass die behauptete Bieziehung gilt, indem wir zeigen,
  dass die Beziehungen \(f(n) \in O(\log(n))\) und
  \(f(n) \in \Omega(\log(n))\) gelten.
  \begin{figure}[t]
    \centering
    \begin{tikzpicture}
      \begin{axis}[
        ymin=0,
        ymax=1,
        xmin=0,
        xmax=5,
        axis equal,
        axis on top=true,
        axis x line=middle,
        axis y line=middle,
        ]
        \addplot[samples=200,domain=0.2:10]{1/x};
        \fill [orange] (1,0) rectangle (2, 0.5);
        \fill [orange] (2,0) rectangle (3, 0.33);
        \fill [orange] (3,0) rectangle (4, 0.25);
        \fill [orange] (4,0) rectangle (5, 0.2);
        \draw (1,0) rectangle (2, 1);
        \draw (2,0) rectangle (3,0.5);
        \draw (3,0) rectangle (4,0.33);
        \draw (4,0) rectangle (5,0.25);
      \end{axis}
    \end{tikzpicture}
    \caption{Aufgabe Zwei}
  \end{figure}
  Wir zeigen zunächst, dass
  \begin{align*}
    f(n) - 1 \le \int_{1}^{n+1}{\frac{1}{x}dx} \le f(n) \tag*{\((\ast)\)}
  \end{align*}
  gilt.  Wir bemerken, dass
  \[ f(2) - 1 = \frac{1}{2} \cdot 1 \le \int_{1}^{2}{\frac{1}{x}dx} \le
    \int_{1}^{2}{\frac{1}{x}dx} + \int_{2}^{3}{\frac{1}{x}dx} = \int_{1}^{3}{\frac{1}{x}dx}\]
  gilt.  Wir erhalten dann
  \[ f(n) - 1 = \frac{1}{2} \cdot 1 + \cdots + \frac{1}{n} \cdot 1 \le
    \int_{1}^{2}{\frac{1}{x}dx} + \cdots + \int_{n}^{n+1}{\frac{1}{x}dx} = \int_{1}^{n+1}{\frac{1}{x}dx},\]
  die Unterschied entspricht die weiße Fläche unterhalb des Kurves und
  oberhalb des orange-farbige Rechtecks. Damit ist die linke Seite von
  \((\ast)\)   gezeigt.  Wir zeigen die Gültigkeit von der rechte Seite. Es
  gilt
  \[ \int_{1}^{2+1}\frac{1}{x}dx \le 1 \cdot 1 + \frac{1}{2} \cdot 1 = f(2).\]
  Daraus folgt, dass
  \[\int_{1}^{n+1}\frac{1}{x}dx \le 1 \cdot 1 + \cdots + \frac{1}{n} \cdot 1 = f(n).\]

  Nun sei \(n\) sehr groß.  Dann ist \(f(n)\) wegen Ungleichung
  \((\ast)\) nach oben und nach unten durch \(\int_{1}^{n+1}(1/x)dx\)
  beschränkt.  Damit folgt aus \(\int_{1}^{n+1}(1/x)dx = \log(x)\) dass
  \(f(n) \in \Theta(\log(n))\).
\end{proof}
\subsection{3. Aufgabe}
\subsubsection{Teil a}
\begin{proof}
  Sei der Code an der \(k\)-ten Stelle nicht erkennbar.

  Es gilt $\sum_{n=0}^6{a_{2n+1}}+3\sum_{n=1}^6{a_{2n}} \equiv 0
  \pmod{10}$.  Es gilt auch, \(1 \leq a_k \leq 9\).

  \begin{itemize}
  \item Falls \(k\) gerade.

    Dann gilt
\begin{align*}
  \sum_{n=0}^6{a_{2n+1}}+3\sum_{n=1}^{k-1}{a_{2n}}+3\sum_{n=k+1}^{6}{a_{2n}}+3a_k
  \equiv 0 \pmod{10}\\
  3a_k=10p-\left( \sum_{n=0}^6{a_{2n+1}}+3\sum_{n=1}^{k-1}{a_{2n}}+3\sum_{n=k+1}^{6}{a_{2n}} \right)
\end{align*}
mit \(p \in \mathbb{N}\).  Wegen \(3a_k\) muss die Rechtsseite der
Gleichung durch \(3\) teilbar sein, die mögliche Lösungen sind also
$\left\{ 3, 6, 9, 12, 15, 18, 21, 24, 27 \right\}$.  Wir bemerken,
dass alle mögliche Lösungen geteilt durch \(10\) einen eindeutigen Rest
besitzt, nämlich \(\left\{ 7, 4, 1, 8, 5, 2, 9, 6, 3 \right\}\).  Also,
die Werte von \(a_k\) und \(p\) ist durch den Rest von der größen Klammern
geteilt durch \(10\) eindeutig bestimmt.

  \item Falls \(k\) ungerade.

    Dann gilt
\begin{align*}
  \sum_{n=0}^{k-1}{a_{2n+1}}+\sum_{n=k+1}^6{a_{2n+1}}+3\sum_{n=1}^6{a_{2n}}+a_k
  \equiv 0  \pmod{10}\\
  a_k=10p-\left( \sum_{n=0}^{k-1}{a_{2n+1}}+\sum_{n=k+1}^6{a_{2n+1}}+3\sum_{n=1}^6{a_{2n}} \right).
\end{align*}
mit \(p \in \mathbb{N}\).  Wegen \(1 \leq a_k \leq 9\) ist dann \(p\)
eindeutig bestimmt.  Denn, für \(p-1\) oder \(p+1\) muss entweder $a_k >
9$ oder \(a_k<1\) gelten.
  \end{itemize}
\end{proof}
\subsubsection{Teil b und c}
\begin{proof}
  Sei der Code an der \(k\)-ten und \(k+1\) Stelle vertauscht.
  Es gilt $\sum_{n=0}^6{a_{2n+1}}+3\sum_{n=1}^6{a_{2n}} \equiv 0
  \pmod{10}$.  Es gilt auch, \(1 \leq a_k \leq 9\).

Jetzt werden \(a_k\) und \(a_{k+1}\) vertauscht.  Wir zeigen, dass
$3a_k+a_{k+1} \neq a_k+3a_{k+1}+10n$, also
$4\abs{a_k-a_{k+1}} \neq 10n$ für alle \(n \in \mathbb{Z}\)
gilt.

Nun betrachten wir die Fälle wenn
$\abs{a_k-a_{k+1}}$ gleich \(\left\{ 1,2,3,4,5,6,7,8 \right\}\) ist.

Sei \(\abs{a_k-a_{k+1}}=1\), dann gilt \(4\abs{a_k-a_{k+1}} =4 \neq 10n\).

Sei \(\abs{a_k-a_{k+1}}=2\), dann gilt \(4\abs{a_k-a_{k+1}} =8 \neq 10n\).

Sei \(\abs{a_k-a_{k+1}}=3\), dann gilt \(4\abs{a_k-a_{k+1}} =12 \neq 10n\).

Sei \(\abs{a_k-a_{k+1}}=4\), dann gilt \(4\abs{a_k-a_{k+1}} =16 \neq 10n\).

Sei \(\abs{a_k-a_{k+1}}=5\), dann gilt $4\abs{a_k-a_{k+1}} =20 =
10n$ mit \(n=2\).

Sei \(\abs{a_k-a_{k+1}}=6\), dann gilt \(4\abs{a_k-a_{k+1}} =24 \neq 10n\).

Sei \(\abs{a_k-a_{k+1}}=7\), dann gilt \(4\abs{a_k-a_{k+1}} =28 \neq 10n\).

Sei \(\abs{a_k-a_{k+1}}=8\), dann gilt \(4\abs{a_k-a_{k+1}} =32 \neq 10n\).

Damit haben wir gezeigt, dass der Code kann niemals korrekt sein, wenn
zwei ungleichen, aufeinanderfolgenden Ziffern vertauscht werden.
Außer deren Differenz ist gleich \(5\).
\end{proof}
\subsection{4. Aufgabe}
Seien \(x \coloneq (11011100)_{2}\) und \(y \coloneq (1011)_{2}\).
Wir berechnen die Summe \(x + y\).  Wegen Vorlesung ist die Algorithmus
\begin{verbatim}
Input: x = (x(k-1), ..., x(1), x(0))[b]
Input: y = (y(k-1), ..., y(1), y(0))[b]
Output: s = x + y = (s(k), ..., s(1), s(0))[b]
c <- 0
for j <- 0 to k - 1 do:
    s(j) <- (x(j) + y(j) + c) mod b
    c <- floor((x(j) + y(j) + c)/b)
s(k) <- c
output s
\end{verbatim}
sowie
\begin{verbatim}
c <- 0
j <- 0, k - 1 <- 7
s(0) <- (x(0) + y(0) + c) mod 2 = 1
c <- floor((x(0) + y(0) + c)/b) = floor(1/2) = 0
j <- 1, k - 1 = 7
s(1) <- (x(1) + y(1) + c) mod 2 = 1
c <- floor((x(1) + y(1) + c)/b) = floor(1/2) = 0
j <- 2, k - 1 = 7
s(2) <- (x(2) + y(2) + c) mod 2 = 1
c <- floor((x(2) + y(2) + c)/b) = floor(1/2) = 0
j <- 3, k - 1 = 7
s(3) <- (x(3) + y(3) + c) mod 2 = 0
c <- floor((x(3) + y(3) + c)/b) = floor(1) = 1
j <- 4, k - 1 = 7
s(4) <- (x(4) + y(4) + c) mod 2 = 0
c <- floor((x(4) + y(4) + c)/b) = floor(1) = 1
j <- 5, k - 1 = 7
s(5) <- (x(5) + y(5) + c) mod 2 = 1
c <- floor((x(5) + y(5) + c)/b) = floor(1) = 1
j <- 6, k - 1 = 7
s(6) <- (x(6) + y(6) + c) mod 2 = 1
c <- floor((x(6) + y(6) + c)/b) = floor(1) = 1
j <- 7, k - 1 = 7
s(7) <- (x(7) + y(7) + c) mod 2 = 1
c <- floor((x(7) + y(7) + c)/b) = floor(1) = 1
\end{verbatim}
Damit erhalten wir das Ergebnis \(x + y = (11100111)_{2}\).

Wir berechnen \(x-y\).  Das Zweierkomplement von \(y\) ist berechnen
durch: nimm \(2\)-adische Darstellung von \(y-1\) und bilden stellenweise
Differenz zu \(2-1\).  Damit erhalten wir 8 stellige 2
Komplementdarstellung \(-y = (00010101)_{2}\).  Damit erhalten wir
\(x - y = (11010001)_{2} \).

Wir berechnen \(x \cdot y\).  Da \(y = (11)_{10}\) gilt, addieren wir \(x\) elf
Mal.
\begin{verbatim}
11011100
+ 11011100 = 11011100
+ 11011100 = 110111000
+ 11011100 = 1010010100
+ 11011100 = 1101110000
+ 11011100 = 10001001100
+ 11011100 = 10100101000
+ 11011100 = 11000000100
+ 11011100 = 11011100000
+ 11011100 = 11110111100
+ 11011100 = 100010011000
+ 11011100 = 100101110100
\end{verbatim}
Wir berechnen \(x/y\).
\begin{verbatim}
11011100
- 1011 = 11010001
- 1011 = 11000110
- 1011 = 10111011
- 1011 = 10110000
- 1011 = 10100101
- 1011 = 10011010
- 1011 = 10001111
- 1011 = 10000100
- 1011 = 1111001
- 1011 = 1101110
- 1011 = 1100011
- 1011 = 1011000
- 1011 = 1001101
- 1011 = 1000010
- 1011 = 110111
- 1011 = 101100
- 1011 = 100001
- 1011 = 10110
- 1011 = 1011
- 1011 = 0
\end{verbatim}
Also \(x/y = 20\) und der Rest ist Null.
\end{document}
