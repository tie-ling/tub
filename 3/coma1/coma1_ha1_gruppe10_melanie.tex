% 使用 chktex 检查 tex 文件中的语法错误
% settings for chktex
% chktex-file 3

% draft mode incompatible with tikz
\documentclass[a5paper]{article}

% 让每个章节 subsection 在新的一页上开始
% 而不是紧接着上一章节
\AddToHook{cmd/subsection/before}{\clearpage}

% 隐藏默认的章节序号:实际作业中与这个冲突
% https://tex.stackexchange.com/a/30225
\setcounter{secnumdepth}{0}

% 设置页边距,上下左右
\usepackage{geometry}
\geometry{a5paper,
 left=1cm,
 right=1cm,
 top=1cm,
 bottom=2cm
}

% 让 TeX 支持德语
\usepackage[ngerman]{babel}

% 调整 Emacs 预览字体大小
% (setq preview-scale-function 1.5)

% TeX Gyre Schola and unicode-math
% which in turn loads amsmath,mathtools,fontspec and friends
\usepackage{xcharter-otf}


% 加载数学相关的包
\usepackage{
  % 自然段间留空
  parskip,
  % 区间排版
  interval
}

% 中文支持,暂时不需要
% \usepackage[UTF8]{ctex}

% 根据 AMS 建议,应为成对符号(比如绝对值)定义新命令
\providecommand{\abs}[1]{\left\lvert#1\right\rvert}
\providecommand{\norm}[1]{\left\lVert#1\right\rVert}

\usepackage{amsthm}
% 定理定义,依赖于 amsthm
\theoremstyle{remark}
\newtheorem*{Behauptung}{Behauptung}
\newtheorem*{Lemma}{Lemma}
\newtheorem*{Satz}{Satz}
\newtheorem*{Definition}{Definition}


% 标题与作者
\title{HA 1, CoMa 1, Melanie, Gruppe 10}
\author{Zhang 484981, Wernicke 492703, Guo 480788}

\begin{document}
\maketitle
\begin{center}
  Meng Zhang 484981, Moritz Karl Wernicke 492703, Yuchen Guo 480788
\end{center}

\newpage

\subsection*{1. Aufgabe}
\subsubsection*{1. Aufgabe, Teil a}
\begin{enumerate}
\item \(R = \{(a, b) \in \mathbb{N} \times \mathbb{N} \mid a \le b\}\) ist keine Äquivalenzrelation.
  Denn aus \(2 \le 3\) folgt nicht \(3 \le 2\).  Die ist nicht symmetrisch.
\item \(R = \{(a, b) \in \mathbb{N} \times \mathbb{N} \mid a \ne b\}\) ist keine Äquivalenzrelation.
  Denn es gilt nicht \(a \ne a\).  Die ist nicht reflexiv.
\item
  \(R = \{(a, b) \in \mathbb{N} \times \mathbb{N} \mid \abs{a - b} > d\}\), für ein festes
  \(d > 0\) ist keine Äquivalenzrelation.  Denn
  \( a - a = 0 > d > 0 \) gilt nicht.  Die ist nicht reflexiv.
\item \(R = \{(a, b) \in \mathbb{N} \times \mathbb{N} \mid f(a) = f(b)\}\), für eine feste Abbildung
  \(f\colon \mathbb{R} \to \mathbb{R}\) ist eine Äquivalenzrelation.  Denn für alle \(a \in A\) gilt
  \(f(a) = f(a)\).  Dies ist auch symmetrisch, denn für alle \((a, b) \in
  R\) gilt \(f(a) = f(b) = f(a)\).  Dies ist auch transitiv.  Denn aus
  \(f(a) = f(b)\) und \(f(b) = f(c)\) folgt \(f(a) = f(c)\).
\end{enumerate}
\subsubsection*{1. Aufgabe, Teil b}
Sei \(A = \{-4, -3, -2, -1, 0, 1, 2, 3, 4\}, R \subseteq A \times A\) und \[ R =
  \{(a, b) \in A \times A \mid a^{4} = b^{4}\}.\]
Dann gilt,
\begin{align*}
  [4] = \{4, -4\}, \quad [3] = \{3, -3\}, \quad [-2] = \{2, -2\}, \quad [1] = \{1,
  -1\}, \quad [0] = \{0\}.
\end{align*}
\subsection*{2. Aufgabe}
\begin{enumerate}
\item \(\{(a, b) \in \mathbb{Z} \times \mathbb{Z} \mid b = \abs{a}\}\) ist eine Abbildung von \(\mathbb{Z}\)
  nach \(\mathbb{Z}\), denn für alle \(a \in \mathbb{Z}\) existiert genau eine eindeutig
  bestimmte Zahl \(b \in \mathbb{Z}\) sodass gilt \(b = \abs{a}\).  Die ist aber
  weder injektiv wegen \(\abs{1} = \abs{-1} = 1\) noch surjektiv wegen
  \(\forall a \in \mathbb{Z}, \abs{a} \ge 0\).
\item \(\{(a, b) \in \mathbb{Z} \times \mathbb{Z} \mid a = \abs{b}\}\) ist keine Abbildung.  Denn
  für jeder Urbild \(a \ne 0\) existiert zwei Bilder \(b, -b\).
\item \(\{(a, b) \in \mathbb{Z} \times \mathbb{Z} \mid b = a/2\}\) ist keine Abbildung.  Denn für
  alle ungerade \(a\) existiert kein Bild in \(\mathbb{Z}\).
\item \(\{(a, b) \in \mathbb{Z} \times \mathbb{Z} \mid (b-a) \bmod 7 = 0 \wedge -a-3 \le b \le -a+3\}\) ist
  eine Abbildung.
  \begin{proof}
    Wir zeigen zuerst, dass es zu jedem \(a \in \mathbb{Z}\) eine \(b \in
    \mathbb{Z}\) existiert, die obigen Bedingungen erfüllt.  Sei \(c \coloneq
    \min(\abs{a \bmod 7}, (7 - \abs{a \bmod 7}))\).  Dann ist \(0 \le c \le 3\),
    \(-b \coloneq a - c\) und \(-a - 3 \le b = -a + c \le -a + 3\).  Damit ist die
    Existenz von \(b\) gezeigt.

    Wir zeigen, dass solche \(b\) eindeutig bestimmt ist.  Sei \(a \in \mathbb{Z}\)
    beliebig und \(p, q \in \mathbb{Z}, p \ne q\) mit \(b-a = 7p\) und \(b-a = 7q\).
    Dann gilt insbesondere
    \[b = 7p + a, \quad b= 7q + a \]
    sowie
    \[\abs{7p + 2a} \le 3, \quad \abs{7q + 2a} \le 3.\]
    Daraus folgt, dass
    \[\abs{p - q} \le 6/7\]
    dies ist aber im Widerspruch zur \(\abs{p - q} \ge 1\) wegen \(p, q \in
    \mathbb{Z}, p \ne q\).  Damit ist die Eindeutigkeit gezeigt.

    [In,Sur,Bi]jektivität???
  \end{proof}
\end{enumerate}
\subsection*{3. Aufgabe}
Seien \(X, Y, Z\) Mengen und \(f\colon X \to Y, g \colon Y \to Z\) Abbildungen.
\begin{enumerate}
\item Behauptung.  Aus der Injektivität von \(f, g\) folgt diesselben
  von \(g \circ f\).
  \begin{proof}
    Wegen der Injektivität von \(f\) folgt \(f(x_{1}) \ne f(x_{2})\) für
    alle \(x_{1}, x_{2} \in X, x_{1} \ne x_{2}\).  Wegen der Injektivität
    von \(g\) folgt \(g(f(x_{1}) \ne g(f(x_{2})))\).  Damit ist \((g \circ f)
    (x_{1}) \ne (g \circ f)(x_{2}) \) für alle \(x_{1} \ne x_{2}\) und die
    Injektivität ist bewiesen.
  \end{proof}
\item Behauptung.  Ist \(g \circ f\) bijektiv.  Dann ist \(g\) surjektiv und
  \(f\) injektiv.
  \begin{proof}
    Weil \(g \circ f\) bijektiv ist, existiert zu jedem \(z \in Z\) ein
    eindeutig bestimmte \(x \in X\) mit \((g \circ f)(x) = g(f(x))= z\).
    Insbesondere existiert zu jedem \(z \in Z\) ein \(f(x) \in Y\) sodass
    \(g(f(x)) = z\).  Damit ist \(g\) surjektiv.

    Weil \(g \circ f\) bijektiv ist, gilt für alle \(x_{1}, x_{2} \in X, x_{1}
    \ne x_{2}\) dass \((g \circ f)(x_{1}) \ne (g \circ f)(x_{2})\).  Daraus folgt,
    dass \(g(f(x_{1})) \ne g(f(x_{2}))\) und \(f(x_{1}) \ne f(x)_{2}\).  Damit
    ist \(f\) injektiv.
  \end{proof}
\item Aus der Surjektivität von \(g \circ f\) und \(g\) folgt nicht die
  Surjektivität von \(f\).

  Denn sei \(f\colon X \to Y' \subsetneq Y\) mit \(g(Y') = Z\).  Dann ist \(g \circ f\) und \(g\)
  surjektiv, aber \(f\) nicht surjektiv.
\end{enumerate}
\subsection*{4. Aufgabe}
Sei die Funktion \(f\colon \mathbb{N} \to \mathbb{N}\) mit
\begin{align*}
  f(x) =
  \begin{cases}
    3(x - 1), & \text{falls } x \bmod 2 \ne 0, \quad x \bmod 6 \ne 0; \\
    x/2, & \text{falls } x \bmod 2 = 0,\quad x \bmod 6 \ne 0; \\
    x+3, & \text{falls } x \bmod 2 = 0, \quad x \bmod 6 = 0. \\
  \end{cases}
\end{align*}
\begin{proof}
  Wir betrachten Injektivität.  Es gibt drei Fälle.
  \begin{enumerate}
  \item Angenommen,  \(a \ne b\) und \(f(a) = 3(a-1) = f(b) = b/2\).

    Dann folgt \(b = 6a - 6\) im Widerspruch zur Voraussetzung dass \(b
    \bmod 6 \ne 0\).

  \item Angenommen,  \(a \ne b\) und \(f(a) = 3(a-1) = f(b) = b+3\).

    Dann folgt \(b = 3a - 6\).  Wegen Voraussetzung dass \(a \bmod 2 \ne 0\)
    und \(a \bmod 6 \ne 0\) gilt \(3a - 6 \bmod 6 \ne 0\).  Widerspruch!

  \item Angenommen,  \(a \ne b\) und \(f(a) = a/2 = f(b) = b+3\).

    Dann folgt \(a = 2b + 6\).  Aus \(b \bmod 6 = 0\) folgt \(a \bmod 6 = 0\)
    im Widerspruch zur Voraussetzung.
  \end{enumerate}
  Damit ist \(f\) injektiv.
\end{proof}
\end{document}
