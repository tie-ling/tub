% 使用 chktex 检查 tex 文件中的语法错误
% settings for chktex
% chktex-file 3

\documentclass[fleqn,draft,a4paper,11pt]{article}

% 让每个章节 subsection 在新的一页上开始
% 而不是紧接着上一章节
\AddToHook{cmd/subsection/before}{\clearpage}

% 隐藏默认的章节序号:实际作业中与这个冲突
% https://tex.stackexchange.com/a/30225
\setcounter{secnumdepth}{0}

% 设置页边距,上下左右
\usepackage{geometry}

\geometry{margin=3cm}

% 让 TeX 支持德语
\usepackage[ngerman]{babel}

% 调整 Emacs 预览字体大小
% (setq preview-scale-function 1.5)

\usepackage{amsmath,mathtools,fontspec,amsthm,parskip,interval,unicode-math}
\usepackage{xcharter-otf}


% 根据 AMS 建议,应为成对符号(比如绝对值)定义新命令
\providecommand{\skp}[1]{\langle#1\rangle}
\providecommand{\abs}[1]{\left\lvert#1\right\rvert}
\providecommand{\norm}[1]{\left\lVert#1\right\rVert}
\providecommand{\ceil}[1]{\left\lceil#1\right\rceil}
\providecommand{\floor}[1]{\left\lfloor#1\right\rfloor}

% 定理定义,依赖于 amsthm
\theoremstyle{remark}
\newtheorem*{Behauptung}{Behauptung}
\newtheorem*{Lemma}{Lemma}
\newtheorem*{Satz}{Satz}
\newtheorem*{Definition}{Definition}

% 定义新函数,依赖于AMSmath
\DeclareMathOperator{\card}{card}
\DeclareMathOperator{\rg}{rg}

\providecommand{\R}[1]{\mathrm{R#1}}

\newcommand{\wt}{\widetilde}
\newcommand{\dd}{\,\mathrm{d}}

% 标题与作者
\title{HA 11, CoMa 1, Melanie, Gruppe 108}
\author{Kuan 480169, Yu 478912, Guo 480788}

\begin{document}
\maketitle
\begin{center}
Ting Yu Kuan 480169, Shilong Yu 478912, Yuchen Guo 480788
\end{center}
\newpage
\subsection{1. Aufgabe}
\subsubsection{Teil a}
Die Menge der Kreise eines Matroids hat die Eigenschaft Eins bis Drei.

\begin{proof}
  Eigenschaft Eins folgt direkt aus Definition des
  Unabhängigkeitssystem \(\emptyset \in \mathcal{F}\).

  Eigenschaft Zwei folgt aus der Minimalität von \(C_{1}\) und \(C_{2}\).

  Eigenschaft Drei.  Angenommen, es gibt keinen Kreis in \((C_{1} \cup
  C_{2}) \setminus \{e\}\).  Dann muss \((C_{1} \cup C_{2} \setminus \{e\})\) eine
  unabhängige Menge sein.  Sei \(a\) ein beliebiges Element aus der nach
  (2) nichtleeren Menge \(C_{2} \setminus C_{1}\).  Da \(C_{2}\) als Kreis eine
  minimale abhängige Menge ist, muss \(C_{2} \setminus \{a\}\) unabhängig sein.
  Sei nun \(A \in \mathcal{F}\) maximal mit \(C_{2} \setminus \{a\} \subseteq A \subseteq C_{1} \cup C_{2}\).
  Dann ist offenbar \(\{a\} \notin A\), da sonst \(C_{2} \subseteq A\) sei, im
  Widerspruch zur Unabhängigkeit von \(A\).  Weil \(C_{1}\) abhängig ist,
  und somit \(C_{1}\) keine Teilmenge von \(A\) ist, gibt es ein \(b \in
  C_{1} \setminus A\), das wegen \(\{a\} \in C_{2} \setminus C_{1}\) von \(\{a\}\)
  verschieden sein muss.  Dann folgt, dass
  \[\abs{A} \le \abs{(C_{1} \cup C_{2}) \setminus \{a, b\}} = \abs{C_{1} \cup C_{2}}
    - 2 < \abs{(C_{1} \cup C_{2}) \setminus \{e\}}.\]
  Nach der Definition von Matroide gilt: existiert ein \(x \in (C_{1} \cup
  C_{2}) \setminus \{e\}\) mit \(A \cup \{x\} \in \mathcal{F}\), und \(C_{2} \setminus \{a\} \subseteq A \cup \{x\}
  \subseteq C_{1} \cup C_{2}\) im Widerspruch zu \(A\) maximal.
\end{proof}
\subsection{2. Aufgabe}
Wir nummieren die Spalten von der Matrix
\[ A =
  \begin{bmatrix}
    1 & 2 & 3 & 0 & 3 \\
    0 & 0 & 1 & 1 & 1 \\
    0 & 0 & 0 & 1 & 1
  \end{bmatrix}
\]
mit \(\{a,b,c,d,e\}\) durch.  Dann sind die Basen
\[\{a,c,d\}, \{b,c,d\}, \{c,d,e\}, \{a,c,e\}, \{b,c,e\}\]
und die Kreise
\[\{a,b\},\{a,d,e\},\{b,d,e\}\]
\end{document}

