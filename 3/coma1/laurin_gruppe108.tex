% 使用 chktex 检查 tex 文件中的语法错误
% settings for chktex
% chktex-file 3

% draft mode incompatible with tikz
\documentclass[a5paper]{article}

% 隐藏默认的章节序号:实际作业中与这个冲突
% https://tex.stackexchange.com/a/30225
\setcounter{secnumdepth}{0}

% 设置页边距,上下左右
\usepackage{geometry}
\geometry{a5paper,
 left=1cm,
 right=1cm,
 top=1cm,
 bottom=2cm
}

% 让 TeX 支持德语
\usepackage[ngerman]{babel}

% 调整 Emacs 预览字体大小
% (setq preview-scale-function 1.5)

% TeX Gyre Schola and unicode-math
% which in turn loads amsmath,mathtools,fontspec and friends
\usepackage{xcharter-otf}


% 加载数学相关的包
\usepackage{
  % 自然段间留空
  parskip,
  % 区间排版
  interval
}

% 中文支持,暂时不需要
% \usepackage[UTF8]{ctex}

% 根据 AMS 建议,应为成对符号(比如绝对值)定义新命令
\providecommand{\abs}[1]{\left\lvert#1\right\rvert}
\providecommand{\norm}[1]{\left\lVert#1\right\rVert}

\usepackage{amsthm}
% 定理定义,依赖于 amsthm
\theoremstyle{remark}
\newtheorem*{Behauptung}{Behauptung}
\newtheorem*{Lemma}{Lemma}
\newtheorem*{Satz}{Satz}
\newtheorem*{Definition}{Definition}

% 标题与作者
\title{HA 5, CoMa 1, Laurin, Gruppe 108}
\author{Kuan 480169, Yu 478912, Guo 480788}

% 画图工具
\usepackage{tikz,pgfplots}
\pgfplotsset{compat=1.18}
\usetikzlibrary{shapes.geometric}
 \usepackage{graphicx}

\begin{document}
\maketitle
\begin{center}
Ting Yu Kuan 480169, Shilong Yu 478912, Yuchen Guo 480788
\end{center}
\newpage
\subsection{1. Aufgabe}
\begin{Behauptung}
  Die Graphen \(G_{1}\) und \(G_{2}\) sind isomorph.  Die Graphen \(G_{3}\)
  und \(G_{5}\) sind isomorph.  Der Graph \(G_{4}\) ist nicht isomorph zu
  alle anderen Graphen.
\end{Behauptung}
\begin{proof}
Wir zeigen zuerst, dass \(G_{4}\) nicht isomorph zu alle anderen Graphen
ist.  Denn, es gilt zu jedem \(u \in V_{i}\), \(i \in \{1, 2, 3, 5\}\) dass
\(\abs{\delta(u)} = 4\).  Es existiert aber eine Knote \(v \in V_{4}\), dass
\(\abs{\delta(v)} = 3\) gilt.  Damit existiert keine bijektive Abbildung
\(\varphi\colon V_{4} \to V_{i}, i \in \{1, 2, 3, 5\}\).

Die Graphen \(G_{1}\) und \(G_{4}\) sind isomorph.  Die gesuchte Bijektion
ist wie folgt definiert:
\begin{figure}[h]
  \centering
  \includegraphics[width=0.4\textwidth]{./g35.png}
\end{figure}

Wir zeigen, dass \(\varphi\) eine Bijektion mit \[\{u, v\} \in E_{1} \Leftrightarrow \{\varphi(u),
  \varphi(v)\} \in E_{2}\]
ist.  Wir nehmen der Knote (2) als Beispiel.  Es gilt
\[\{2,1\}, \{2,7\}, \{2,4\}, \{2, 3\} \in E_{1} \quad \text{und} \quad
  \{\varphi2,\varphi1\}, \{\varphi2,\varphi7\}, \{\varphi2,\varphi4\}, \{\varphi2, \varphi3\} \in E_{2}\]
die andere Fälle sind analog.

Die Graphen \(G_{3}\) und \(G_{5}\) sind isomorph.  Die gesuchte Bijektion
ist wie folgt definiert:
\begin{figure}[h]
  \centering
  \includegraphics[width=0.6\textwidth]{./g12.png}
\end{figure}

Wir können analog nach oben zeigen, dass \(\varphi\) eine Bijektion mit \[\{u, v\} \in E_{1} \Leftrightarrow \{\varphi(u),
  \varphi(v)\} \in E_{2}\]
ist.
\end{proof}
\subsection{2. Aufgabe}
\begin{Behauptung}
  Es existiert ein Graph \(G\) mit \(\abs{V} = 5\) sodass \(\overline{G}\)
  zu \(G\) isomorph ist.
\end{Behauptung}
\begin{proof}
  Sei \(\varphi\) wie folgt definiert:
\begin{figure}[h]
  \centering
  \includegraphics[width=0.6\textwidth]{./ha5auf2.png}
\end{figure}

  Dann ist die zwei Graphen isomorph und komplementär.
\end{proof}
\begin{Behauptung}
  Ist \(G = (V, E)\) mit \(\abs{V}> 2\) nicht zusammenhängend, dannist der
  Komplementärgraph \(\overline{G}\) zusammenhängend.
\end{Behauptung}
\begin{proof}
  Seien \(u, v \in V(G)\) beliebig.  Wir zeigen, dass es einen
  \(u\)-\(v\)-Weg in \(\overline{G}\) existiert. Es gibt zwei Fälle.
  \begin{itemize}
  \item Falls \(u, v\) nicht adjazent in \(G\) ist, dann ist \(u, v\)
    adjazent in \(\overline{G}\) und es existiert deswegen einen
    \(u\)-\(v\)-Weg.
  \item Falls \(u, v\) adjazent in \(G\) ist.  Dann existiert mindestens
    einen Knoten \(w \in V(G)\) sodass es kein \(u\)-\(w\)-Weg und kein
    \(v\)-\(w\)-Weg gibt.  (Da \(G\) unzusammenhangend ist.) Dann ist
    insbesondere die Paare \(u\)-\(w\) und \(v\)-\(w\) nicht adjazent.  Damit
    sind die Paare adjazent in \(\overline{G}\) und es existiert ein
    \(u\)-\(v\)-Weg in \(\overline{G}\).
  \end{itemize}
  Weil \(u, v \in V(G)\)  beliebig war, folgt die Behauptung.
\end{proof}
\end{document}
