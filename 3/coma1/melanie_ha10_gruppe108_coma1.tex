% 使用 chktex 检查 tex 文件中的语法错误
% settings for chktex
% chktex-file 3

\documentclass[fleqn,draft,a5paper]{article}

% 让每个章节 subsection 在新的一页上开始
% 而不是紧接着上一章节
\AddToHook{cmd/subsection/before}{\clearpage}

% 隐藏默认的章节序号:实际作业中与这个冲突
% https://tex.stackexchange.com/a/30225
\setcounter{secnumdepth}{0}

% 设置页边距,上下左右
\usepackage{geometry}

\geometry{left=1cm,
 right=1cm,
 top=1cm,
 bottom=2cm}

% 让 TeX 支持德语
\usepackage[ngerman]{babel}

% 调整 Emacs 预览字体大小
% (setq preview-scale-function 1.5)

\usepackage{amsmath,mathtools,fontspec,amsthm,parskip,interval,unicode-math}
\setmainfont{texgyreschola}%
 [
  Extension = .otf ,
  UprightFont = *-regular,
  ItalicFont = *-italic,
  BoldFont = *-bold,
  BoldItalicFont = *-bolditalic,
  Ligatures=TeX,
 ]
\setmathfont{texgyreschola-math.otf}

% 根据 AMS 建议,应为成对符号(比如绝对值)定义新命令
\providecommand{\skp}[1]{\langle#1\rangle}
\providecommand{\abs}[1]{\left\lvert#1\right\rvert}
\providecommand{\norm}[1]{\left\lVert#1\right\rVert}
\providecommand{\ceil}[1]{\left\lceil#1\right\rceil}
\providecommand{\floor}[1]{\left\lfloor#1\right\rfloor}

% 定理定义,依赖于 amsthm
\theoremstyle{remark}
\newtheorem*{Behauptung}{Behauptung}
\newtheorem*{Lemma}{Lemma}
\newtheorem*{Satz}{Satz}
\newtheorem*{Definition}{Definition}

% 定义新函数,依赖于AMSmath
\DeclareMathOperator{\card}{card}
\DeclareMathOperator{\rg}{rg}

\providecommand{\R}[1]{\mathrm{R#1}}

\newcommand{\wt}{\widetilde}
\newcommand{\dd}{\,\mathrm{d}}

% 标题与作者
\title{HA 10, CoMa 1, Melanie, Gruppe 108}
\author{Kuan 480169, Yu 478912, Guo 480788}

\begin{document}
\maketitle
\begin{center}
Ting Yu Kuan 480169, Shilong Yu 478912, Yuchen Guo 480788
\end{center}
\newpage
\subsection{1. Aufgabe}
Sei der einfacher Graph \(A\) eine Aboreszenz.  Dann ist der Graph
\(A\) ein zusammenhängendes Branching. D.h., der zu Grunde liegende
ungerichtete Graph hat kein Kreis, ist zusammenhängend und es gilt
\(\abs{\delta^{-}(v)} \le 1\) für alle Knoten \(v\).

Sei \(I\) die Inzidenzmatrix von \(A\).
\subsubsection{Teil a}
\begin{Behauptung}
  Die Spalten von \(I\) sind linear unabhängig.
\end{Behauptung}
\begin{proof}
  Induktion über die Anzahl von Knoten \(n\).

  Induktionsanfang. Falls \(n = 2\), dann gilt
  die Behauptung offenbar.

  Induktionsvoraussetzung. Wir nehmen an, dass die Behauptung für ein
  festes \(n\) gelte.

  Induktionsschritt.  Wir fügen einen Knote \(w\) hinzu.  O.B.d.A. sei
  \(w\) einen Blatt und seien die Spalten von \(I = (p_{1}, \ldots, p_{n+1})\),
  wobei \(p_{n+1}\) die Kante von \(w\) ist.  Wir betrachten die lineare
  Kombination von die Spalten.
  \[
    a_{1}p_{1} + \ldots + a_{n+1}p_{n+1} = 0.
  \]
  Falls \(a_{n+1} = 0\).  Dann ist wegen Induktionsvoraussetzung dass
  \(a_{1}, \ldots, a_{n} = 0\).  Falls \(a_{n+1} \ne 0\), dann gilt
  \[
    -\frac{a_{1}}{a_{n+1}}p_{1} + \ldots + -\frac{a_{n}}{a_{n+1}}p_{n} = p_{n+1}
  \]
  im Widerspruch zur Voraussetzung.  Damit folgt die Behauptung.
\end{proof}
\subsubsection{Teil b}
Fehlt.

\subsection{2. Aufgabe}
\subsubsection{Teil a}
\begin{align*}
  & &\begin{bmatrix}
    2 & 1 & 1 \\
    6 & 3 & 1 \\
    8 & 1 & 0
  \end{bmatrix} \\
  &\R{3} \leftarrow \R{3} + (-4) \R{1} &
  \begin{bmatrix}
    2 & 1 & 1 \\
    6 & 3 & 1 \\
    0 & -3 & -4
  \end{bmatrix} \\
  &\R{2} \leftarrow \R{2} + (-3) \R{1} &
  \begin{bmatrix}
    2 & 1 & 1 \\
    0 & 0 & -2 \\
    0 & -3 & -4
  \end{bmatrix} \\
  &\R{1} \leftarrow \frac12 \R{1} &
  \begin{bmatrix}
    1 & \frac12 & \frac12 \\
    0 & 0 & -2 \\
    0 & -3 & -4
  \end{bmatrix} \\
&  \R{2} \leftrightarrow \R{3} &
  \begin{bmatrix}
    1 & \frac12 & \frac12 \\
    0 & -3 & -4 \\
    0 & 0 & -2
  \end{bmatrix} \\
 & \R{3} \leftarrow -\frac12 \R{3} & 
  \begin{bmatrix}
    1 & \frac12 & \frac12 \\
    0 & -3 & -4 \\
    0 & 0 & 1
  \end{bmatrix} \\
 & \R{2} \leftarrow \R{2} + 4 \R{3} & 
  \begin{bmatrix}
    1 & \frac12 & \frac12 \\
    0 & -3 & 0 \\
    0 & 0 & 1
  \end{bmatrix} \\
 & \R{2} \leftarrow -\frac13 \R{2} & 
  \begin{bmatrix}
    1 & \frac12 & \frac12 \\
    0 & 1 & 0 \\
    0 & 0 & 1
  \end{bmatrix} \\
 & \R{1} \leftarrow \R{1} + -\frac12 \R{2} &
  \begin{bmatrix}
    1 & 0 & \frac12 \\
    0 & 1 & 0 \\
    0 & 0 & 1
  \end{bmatrix} \\
 & \R{1} \leftarrow \R{1} + - \frac12 \R{3} &
  \begin{bmatrix}
    1 & 0 & 0 \\ 0 & 1 & 0 \\ 0 & 0 & 1
  \end{bmatrix}
\end{align*}
Dann gilt
\begin{align*}
  L =
  \begin{bmatrix}
    1 & 0 & 0 \\
    \frac34 & 1 & 0 \\
    \frac14 & \frac13 & 1
  \end{bmatrix}, \quad
  U =
  \begin{bmatrix}
    8 & 1 & 0 \\
    0 & \frac94 & 1 \\
    0 & 0 & \frac23
  \end{bmatrix}, \quad
  P =
  \begin{bmatrix}
    0 & 0 & 1 \\
    0 & 1 & 0 \\
    1 & 0 & 0
  \end{bmatrix}, \quad A = PLU.
\end{align*}
\subsubsection{Teil b}
Wir berechnen \(A x = PLUx =
\begin{bmatrix}
  1 \\ 5 \\ 7
\end{bmatrix}
\).  Sei \(y = Ux\).  Dann gilt
\begin{align*}
  L y =
  P^{-1} \cdot
  \begin{bmatrix}
    1 \\ 5 \\ 7
  \end{bmatrix} =
  \begin{bmatrix}
    7 \\ 5 \\ 1
  \end{bmatrix}, \quad
  y =
  \begin{bmatrix}
    7 \\ -1/4 \\ -2/3
  \end{bmatrix}, \quad
  Ux = y, \quad
  x =
  \begin{bmatrix}
    5/6 \\ 1/3 \\ -1
  \end{bmatrix}.
\end{align*}
Wir berechnen \(A x = PLUx =
\begin{bmatrix}
  5 \\ 1 \\ 2
\end{bmatrix}
\).  Sei \(y = Ux\).  Dann gilt
\begin{align*}
  L y =
  P^{-1} \cdot
  \begin{bmatrix}
    5 \\ 1 \\ 2
  \end{bmatrix} =
  \begin{bmatrix}
    2 \\ 1 \\ 5
  \end{bmatrix}, \quad
  y =
  \begin{bmatrix}
    2 \\ -1/2 \\ 14/3
  \end{bmatrix}, \quad
  Ux = y, \quad
  x =
  \begin{bmatrix}
    2/3 \\ -10/3 \\ 7
  \end{bmatrix}.
\end{align*}
\subsection{3. Aufgabe}
\begin{multline*}
  \begin{bmatrix}
    1 & 0 & 2 \\
    3 & 4 & 0 \\
    0 & 5 & 6
  \end{bmatrix}
  \cdot
  \begin{bmatrix}
    a \\ b \\ c
  \end{bmatrix}
  =
  \begin{bmatrix}
    10 \\ 18 \\ 12
  \end{bmatrix} \to
  \begin{cases}
    a + 2c = 10 \\
    3a + 4b = 18 \\
    5b + 6c = 12
  \end{cases} \to
  \begin{cases}
    a = 6 \\ b = 0 \\ c = 2
  \end{cases}
  \\
  \begin{bmatrix}
    1 & 0 & 2 \\
    3 & 4 & 0 \\
    0 & 5 & 6
  \end{bmatrix}
  \cdot
  \begin{bmatrix}
    a \\ b \\ c
  \end{bmatrix}
  =
  \begin{bmatrix}
    7 \\ 31 \\ 26
  \end{bmatrix} \to
  \begin{cases}
    a + 2c = 7 \\
    3a + 4b = 31 \\
    5b + 6c = 26
  \end{cases} \to
  \begin{cases}
    a = 5 \\ b = 4 \\ c = 1
  \end{cases}
  \\
  \begin{bmatrix}
    1 & 0 & 2 \\
    3 & 4 & 0 \\
    0 & 5 & 6
  \end{bmatrix}
  \cdot
  \begin{bmatrix}
    a \\ b \\ c
  \end{bmatrix}
  =
  \begin{bmatrix}
    0 \\ 12 \\ 15
  \end{bmatrix} \to
  \begin{cases}
    a + 2c = 0 \\
    3a + 4b = 12 \\
    5b + 6c = 15
  \end{cases} \to
  \begin{cases}
    a = 0 \\ b = 3 \\ c = 0
  \end{cases}  \\
\end{multline*}
Wir erhalten dann
\begin{align*}
  X =
  \begin{bmatrix}
    6 & 5 & 0 \\ 0 & 4 & 3 \\ 2 & 1 & 0
  \end{bmatrix}.
\end{align*}

\subsection{4. Aufgabe}
\begin{Behauptung}
  Sei \(D=(V,E)\) ein einfacher, gerichteter Graph und \(A\) seine
  Adjazenzmatrix.  Graph \(D\) ist genau dann azyklisch, falls eine
  Permutationsmatrix \(P\) existiert sodass \(P^{T}AP\) eine untere
  Dreiecksmatrix ist.
\end{Behauptung}
\begin{proof}
  Graph \(D\) ist genau dann azyklisch, falls es eine topologischer
  Sortierung von den Knoten existert, sodass
  \[\pi(j) \not{\le} \pi(i) \quad \text{für } i < j\]
  gilt.

  Da für alle Einträge oberhalb der Diagonal \(i < j\) gilt, sind solche
  Einträge genau dann gleich Null, wenn topologischer Sortierung
  existiert.  Damit folgt die Behauptung.
\end{proof}
\end{document}

