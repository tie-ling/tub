% 使用 chktex 检查 tex 文件中的语法错误
% settings for chktex
% chktex-file 3

% draft mode incompatible with tikz
\documentclass[a5paper]{article}

% 让每个章节 subsection 在新的一页上开始
% 而不是紧接着上一章节
\AddToHook{cmd/subsection/before}{\clearpage}

% 隐藏默认的章节序号:实际作业中与这个冲突
% https://tex.stackexchange.com/a/30225
\setcounter{secnumdepth}{0}

% 设置页边距,上下左右
\usepackage{geometry}
\geometry{a5paper,
 left=1cm,
 right=1cm,
 top=1cm,
 bottom=2cm
}

% 让 TeX 支持德语
\usepackage[ngerman]{babel}

% 调整 Emacs 预览字体大小
% (setq preview-scale-function 1.5)

% TeX Gyre Schola and unicode-math
% which in turn loads amsmath,mathtools,fontspec and friends
\usepackage{xcharter-otf}


% 加载数学相关的包
\usepackage{
  % 自然段间留空
  parskip,
  % 区间排版
  interval
}

% 中文支持,暂时不需要
% \usepackage[UTF8]{ctex}

% 根据 AMS 建议,应为成对符号(比如绝对值)定义新命令
\providecommand{\abs}[1]{\left\lvert#1\right\rvert}
\providecommand{\norm}[1]{\left\lVert#1\right\rVert}

\usepackage{amsthm}
% 定理定义,依赖于 amsthm
\theoremstyle{remark}
\newtheorem*{Behauptung}{Behauptung}
\newtheorem*{Lemma}{Lemma}
\newtheorem*{Satz}{Satz}
\newtheorem*{Definition}{Definition}

% 标题与作者
\title{HA 4, CoMa 1, Laurin, Gruppe 108}
\author{Kuan 480169, Yu 478912, Guo 480788}

% 画图工具
\usepackage{tikz,pgfplots}
\pgfplotsset{compat=1.18}


\begin{document}
\maketitle
\begin{center}
Ting Yu Kuan 480169, Shilong Yu 478912, Yuchen Guo 480788
\end{center}
\subsection{3. Aufgabe}
\subsubsection{Teil a}
\begin{proof}
  Wir beweisen diese Aussage mittels Kontraposition.

  Es gelte die folgende Aussage:
  \begin{quote}
    Es gibt nur \(0\) oder \(1\) oder \(2\) Personen die sich alle
    gegenseitig die Hand geschüttelt haben, \textbf{und},
    es gibt nur \(0\) oder \(1\) oder \(2\) Personen, die sich alle
    gegenseitig nicht die Hand geschüttelt haben.
  \end{quote}

  Der Fall mit nur \(1\) Person macht kein Sinn, deshalb erhalten wir
  die folgende Aussage:
  \begin{quote}
    Es gibt nur \(0\) oder \(2\) (ein Paar) Personen die sich alle
    gegenseitig die Hand geschüttelt haben, \textbf{und}, es gibt nur
    \(0\) oder \(2\) (ein Paar) Personen, die sich alle gegenseitig nicht
    die Hand geschüttelt haben.
  \end{quote}

  Es gibt nun vier Fällen zu betrachten, nämlich, \(0\) und \(0\); \(0\) und
  \(2\); \(2\) und \(0\); \(2\) und \(2\).  In allen vier Fällen ist im
  Widerspruch zur Voraussetzung die Anzahl der Coma-Betreuer weniger
  als \(6\).
\end{proof}
\subsubsection{Teil b}
\begin{proof}
  Wir beweisen diese Aussage mittels Kontraposition.

  Wir durchnummerieren die Betreuern mit natürliche Zahlen von \(1\) bis
  \(n\) und sei die Menge der Anzahl von Handschütteln
\begin{align*}
M\coloneq \left\{ a_1, a_2, \ldots, a_n \right\}.
\end{align*}

Angenommen, es gilt für alle \(p,q \in \mathbb{N}_{\leq n}\) mit $p \neq
q$ dass \(a_p \neq a_q\).

O.B.d.A sei auch \(a_1 < a_2 < \ldots < a_n\).  Es gilt nun $a_n \leq n
- 1$, denn ein Person kann höchstens alle andere \(n-1\) Personen die
Hand schütteln.  Wegen Transitivität gilt \(a_{n-1} < n - 1\).  Wegen
\(a_{n-1} \in \mathbb{N}\) gilt \(a_{n-1} \leq n-2\), und so weiter.
Damit erhalten wir \(a_1 \leq 0\), also \(a_1 = 0\).

Wir bemerken, dass die Menge der Anzahl von Handschütteln ist dann
durch die im oben genannten Beschränkungen eindeutig bestimmt, also
wegen \(a_1=0\) und \(a_1 < a_2\) und \(a_2 \leq 1\) muss \(a_2 = 1\) sein.
Also die Menge ist \(\left\{ 0, 1, 2, 3, \ldots, n-1 \right\}\).  Aber
der letzter Betreuer hat eine Anzahl von \(n-1\), also alle andere
Personen die Hand geschüttelt, im Widerspruch zur \(a_1=0\).
\end{proof}
\subsection{4. Aufgabe}
\subsubsection{Teil a}
\begin{Behauptung}
  Es gilt \(\chi(G) \le \abs{V}\).
\end{Behauptung}
\begin{proof}
  O.B.d.A betrachten wir den Fall, wo zu beliebige Paar
  \((v, w) \in V \times V, v \ne w\) eine Kante \(e = \{v, w\} \in E\) existiert.
  Weil die Abbildung \(c\colon V \to \{1, \ldots, k\}\) für alle \(\{v, w\} \in E\) die
  Eigenschaft \(c(v) \ne c(w)\) besitzt, folgt aus Voraussetzung, dass
  \[\forall v, w \in V, v \ne w\colon c(v) \ne c(w)\]
  damit gilt in diesem Fall
  \[\chi(G) = \abs{V}.\]
  In alle anderen Fälle muss nicht \(c(v) \ne c(w)\) zwingend für
  \(\forall v, w \in V\) gelten, daraus folgt, dass
  \[\chi(G) \le \abs{V}.\]
\end{proof}

\subsubsection{Teil b}
Analog zum [Teil a], im Fall
\[\forall(v, w) \in V \times V, v \ne w\colon \exists e = \{v, w\} \in E\]
gilt
\[\abs{V} = \chi(G) = n.\]

\subsubsection{Teil c}

Sei \(n \in \mathbb{N}_{> 1}\) und definieren wir
\[C_{n} \coloneq (V_{n}, E_{n}),\quad  V_{n} \coloneq \{1, \ldots, n\},\quad  E_{n} \coloneq \{\{a, b\} \in
  V_{n} \times V_{n}\mid b = (a+1) + n\mathbb{Z} \}.\]

Dann gilt wegen \(1 \le a, b \le n\) dass
\[
  b = \begin{cases}
    a + 1, & 1 \le a \le n, \\
    1, & a = n.
  \end{cases}
\]
Wir unterscheiden zwei Fälle.  Falls \(n\) gerade ist, dann gilt \(k =
2\).  Beweis durch Induktion.  Falls \(n = 2\), dann existiert zwei Kante
\((1, 2), (2, 1)\).  Offensichtlich gilt
\[
  k = 2, \quad c(1) = 1, \quad c(2) = 2.
\]  Wählen wir eine
gerade \(n\)  fest.  Wir betrachten \(n + 2\).  Durch Hinzufügen von \((n,
n+1), (n+1, n+2), (n+2, 1)\) und Entfernen von \((n, 1)\) erhalten wir
\[k = 2,~ c(n) = 2,~ c(n+1) = 1,~ c(n+2) = 2,~ c(1) = 1.\]

Falls \(n\) ungerade ist, dann gilt \(k = 3\).
Beweis durch Induktion.  Falls \(n = 3\), dann existiert zwei Kante
\((1, 2), (2, 3), (3, 1)\).  Offensichtlich gilt
\[
  k = 3, \quad c(1) = 1, \quad c(2) = 2, \quad c(3) =3
\]  Wählen wir eine
ungerade \(n\)  fest.  Wir betrachten \(n + 3\).  Durch Hinzufügen von \((n,
n+1), (n+1, n+2), (n+2, n+3), (n+3, 1)\) und Entfernen von \((n, 1)\) erhalten wir
\[k = 3,~ c(n) = 3,~ c(n+1) = 1,~ c(n+2) = 2,~ c(n+3) = 3,~ c(1)=1.\]

\end{document}
