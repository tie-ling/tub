% 使用 chktex 检查 tex 文件中的语法错误
% settings for chktex
% chktex-file 3

\documentclass[draft,a5paper]{article}

% 让每个章节 subsection 在新的一页上开始
% 而不是紧接着上一章节
\AddToHook{cmd/subsection/before}{\clearpage}

% 隐藏默认的章节序号:实际作业中与这个冲突
% https://tex.stackexchange.com/a/30225
\setcounter{secnumdepth}{0}

% 设置页边距,上下左右
\usepackage{geometry}
\geometry{a5paper,
 left=1cm,
 right=1cm,
 top=1cm,
 bottom=2cm
}

% 让 TeX 支持德语
\usepackage[ngerman]{babel}

% 调整 Emacs 预览字体大小
% (setq preview-scale-function 1.5)

% TeX Gyre Schola and unicode-math
% which in turn loads amsmath,mathtools,fontspec and friends
\usepackage{xcharter-otf}


% 加载数学相关的包
\usepackage{
  % 自然段间留空
  parskip,
  % 区间排版
  interval
}

% 中文支持,暂时不需要
% \usepackage[UTF8]{ctex}

% 根据 AMS 建议,应为成对符号(比如绝对值)定义新命令
\providecommand{\abs}[1]{\left\lvert#1\right\rvert}
\providecommand{\norm}[1]{\left\lVert#1\right\rVert}

\usepackage{amsthm}
% 定理定义,依赖于 amsthm
\theoremstyle{remark}
\newtheorem*{Behauptung}{Behauptung}
\newtheorem*{Lemma}{Lemma}
\newtheorem*{Satz}{Satz}
\newtheorem*{Definition}{Definition}

\title{HA 2, CoMa 1, Melanie, Gruppe 10}
\author{Erkan 381318, Wernicke 492703, Guo 480788}

\begin{document}
\maketitle
\begin{center}
Tolga Erkan 381318,  Moritz Karl Wernicke 492703, Yuchen Guo 480788
\end{center}
\subsection{1. Aufgabe}
\begin{enumerate}
\item Behauptung.  \(5^{n} - 2^{n}\) ist für alle natürlichen Zahlen \(n\)
  durch \(3\) teilbar.
  \begin{proof}
    Wir beweisen die Behauptung mittels vollständigen Induktion.

    Induktionsanfang.  \(n=0\).  Dann ist \(5^{0} - 1^{0} = 1 - 1 = 0\)
    durch \(3\) teilbar.

    Induktionsvorausssetzung.  Angenommen, die Behauptung gelte für
    ein fest gewähltes \(n \in \mathbb{N}\).

    Induktionsschritt.  \(n \to n+1\).  Dann gilt
    \[5^{n+1} - 2^{n+1} = 5^{n} \cdot 5 - 2^{n} \cdot 2 = 5^{n} \cdot 5 - 2^{n}\cdot
      (5-3) = 5(5^{n} - 2^{n}) + 2^{n} \cdot 3\]
    indem wegen Induktionsvoraussetzung \(5^{n} - 2^{n}\) durch \(3\)
    teilbar ist.  Daraus folgt, dass \(5^{n+1} - 2^{n+1}\) durch \(3\)
    teilbar ist.  Damit gilt die Behauptung.
  \end{proof}
\item Es gilt
  \[5 \log_{2}(n) \prec 3\sqrt{n} \prec 2^{n}\log_{2}(n) \prec 3^{n} \prec n! \]
\item Behauptung.  Für alle natürlichen Zahlen \(m \ge 2\) und \(n \ge 1\)
  gilt \(F_{m+n} = F_{m-1}F_{n} + F_{m}F_{n+1}\).
  \begin{proof}
    durch vollständigen Induktion über \(n\).

    Induktionsanfang.  Sei \(n = 1\).  Dann gilt
    \[F_{m+1} = F_{m-1} + F_{m} = F_{m-1} F_{1} + F_{m} F_{2}.\]
    Sei \(n = 2\).  Dann gilt
    \[F_{m+2} = F_{m+1} + F_{m} = F_{m-1} + F_{m} + F_{m} = F_{m-1}
      F_{2} + F_{m} F_{3}.\]

    Induktionsvoraussetzung. Angenommen, die Behauptung gelte für
    ein fest gewähltes \(n \in \mathbb{N}\) und \(n+1\).

    Induktionsschritt.  Wir zeigen, dass für \(n+2\) gilt
    \[F_{m+n+2} = F_{m-1}F_{n+2} + F_{m}F_{n+3}.\]
    Dies gilt, wegen
    \begin{align*}
      F_{m+n+2} &= F_{m+n} + F_{m+n+1} \\
                &= (F_{m-1}F_{n} + F_{m}F_{n+1}) + (F_{m-1}F_{n+1} +
                  F_{m}F_{n+2}) \\
                &= F_{m-1}(F_{n} + F_{n+1}) + F_{m}(F_{n+1}+F_{n+2}) \\
                &= F_{m-1}F_{n+2} + F_{m}F_{n+3}.
    \end{align*}
  \end{proof}
\end{enumerate}
\subsection{2. Aufgabe}
\begin{enumerate}
\item Behauptung.  Sei \(f, g \in O(h)\) und \(a, b \in \mathbb{R}_{>0}\).  Dann folgt
  \(af + bg \in O(h)\).

  \begin{proof}
    Wegen Definition enhält die Menge \(O(h)\) alle
    \(f\colon \mathbb{N} \to \mathbb{R}\), falls eine \(\alpha > 0\) existiert sodass
    \(0 \le f(n) \le \alpha h(n)\) für alle \(n \ge n_{0}, n \in \mathbb{N}\).

    Sei \(\alpha > 0\) und \(n_{0} \in \mathbb{N}\) sodass \(0 \le f(n) \le \alpha h(n_{0})\) für
    alle \(n \ge n_{0}\).  Sei \(\beta > 0\) und \(n_{1} \in \mathbb{N}\) sodass \(0 \le g(n)
    \le \beta h(n_{1})\) für alle \(n \ge n_{1}\).  Sei \(\gamma \coloneq \alpha \cdot a + \beta \cdot b\) und
    \(n_{3} \coloneq \max{(n_{0}, n_{1})}\).  Dann folgt
    \[0 \le af(n) + bg(n) \le \gamma h(n); \quad n \ge n_{3}.\]
    Damit gilt die Behauptung.
  \end{proof}

\item Behauptung.  \(f \in O(g)\) genau dann, falls \(g \in \Omega(f)\).
  \begin{proof}
    Hinrichtung.  Angenommen, \(f \in O(g)\).  Dann existiert \(\alpha > 0\) und
    \(n_{0} \in \mathbb{N}\) sodass \(0 \le f(n) \le \alpha g(n)\) für alle \(n \ge n_{0}\).  Dann
    gilt
    \[ 0 \le \frac{1}{\alpha}f(n) \le g(n); \quad n \ge n_{0}.\]
    Daraus folgt, dass \(g \in \Omega(f)\).

    Rückrichtung.  Angenommen, \(g \in \Omega(f)\).  Dann existiert \(\beta > 0\) und
    \(n_{0} \in \mathbb{N}\) sodass \(0 \le \beta f(n) \le g(n)\) für alle \(n \ge n_{0}\).  Dann
    gilt
    \[ 0 \le f(n) \le \frac{1}{\beta}g(n); \quad n \ge n_{0}.\]
    Daraus folgt, dass \(f \in O(g)\).
  \end{proof}
\item Behauptung.  Aus \(f \in O(g)\) und \(g \in O(h)\) folgt \(f \in O(h)\).
  \begin{proof}
    Wegen Voraussetzung existiert \(\alpha, \beta >0\) und \(n_{0}, n_{1} \in \mathbb{N}\)
    sodass gilt
    \[0 \le f(n) \le \alpha g(n), \quad 0 \le g(m) \le\beta h(m), \quad n \ge n_{0}, m \ge n_{1}.\]
    Dann gilt
    \[0 \le f(n) \le \alpha \beta h(n), \quad n \ge \max(n_{0}, n_{1}).\]
    Daraus folgt, dass \(f\in O(h)\) gilt.
  \end{proof}
\item Behauptung.  Aus \(f \in \Omega(g)\) und \(g \in \Omega(h)\) folgt \(f \in \Omega(h)\).
  \begin{proof}
    Wegen Voraussetzung existiert \(\alpha, \beta >0\) und \(n_{0}, n_{1} \in \mathbb{N}\)
    sodass gilt
    \[0 \le \alpha g(n) \le f(n), \quad 0 \le \beta h(m) \le g(m), \quad n \ge n_{0}, m \ge n_{1}.\]
    Dann gilt
    \[0 \le \alpha \beta h(n) \le f(n), \quad n \ge \max(n_{0}, n_{1}).\]
    Daraus folgt, dass \(f\in \Omega(h)\) gilt.
  \end{proof}
\end{enumerate}
\subsection{3. Aufgabe}
\subsubsection{3. Aufgabe, Teil a}
\begin{Behauptung}
  Man kann die Ebene mit \(n \in \mathbb{N}\) Geraden in höchstens \(\frac{n^{2} + n
    + 2}{2}\) Gebiete zerlegen.
\end{Behauptung}
\begin{proof}
  Wir bemühen einen Beweis mittels vollständigen Induktion.
  \begin{itemize}
  \item   Induktionsanfang \(n = 0\).  Wenn es keine Gerade gibt, bleibt die
    Ebene ganz.  Also es gibt genau ein Gebiet.  Andereseits gilt
    \(\frac{0^2+0+2}{2}=1\).  Daher gilt die Aussage für \(n = 0\).


  \item Induktionsvoraussetzung.  Es gelte die Aussage, dass man mit
    \(n \in \mathbb{N}\) Geraden die Ebene in höchstens
    \(\frac{n^2+n+2}{2}\) Gebiete zerlegen kann.

  \item   Induktionsschritt \(n \rightarrow n+1\).  Ohne Beschränkung der Allgemeinheit
    betrachten wir den Fall, wo die Ebene mit \(n\) Geraden genau in
    \(\frac{n^2+n+2}{2}\) Gebiete geteilt wird.

  \item   Nun fügen wir eine \((n+1)\)-ten Gerade hinzu, sodass die \((n+1)\)-ten
    Gerade alle \(n\) Geraden schneidet, und alle Schnittpunkte der
    \((n+1)\)-ten Gerade von bereits existierenden Schnittpunkte
    verschieden sind.  Weil jedes Gebiet konvex ist, kann am meisten
    \(n+1\) neue Gebiete von der Teilung durch \((n+1)\)-ten Gerade
    entstehen.  Damit gilt,
    \[f(n+1)=f(n)+n+1=\frac{n^2+n+2}{2}+n+1=\frac{(n+1)(n+2)+2}{2}.\]
  \end{itemize}
\end{proof}
\subsubsection{3. Aufgabe, Teil b}
\begin{Behauptung}
  Es gibt unendlich viele Primzahlen.
\end{Behauptung}
\begin{proof}
  durch Widerspruch.  Angenommen, es gibt endlich viele Primzahlen:
  \[M= \left\{ p_1, p_2, p_3, \ldots, p_n \right\}.\]  Wir betrachten die
  Zahl \(a=p_1\cdot p_2 \cdot p_3 \cdots \cdot p_n+1\).  Weil \(a\) größer als alle
  Primzahlen ist, kann \(a\) nicht prim sein.  Also, \(p \notin M\).  Es gilt
  aber, dass \(a\) geteilt durch alle Primzahlen \(p \in M\) immer einen
  Rest von \(1\) hat.  Es gilt, \(a\) ist eine Primzahl.  Diese ist im
  Widerspruch zur Voraussetzung, dass die Menge \(M\) alle Primzahl
  enthält.
\end{proof}
\newpage
\subsection{4. Aufgabe}
\subsubsection{4. Aufgabe, Teil a}
\begin{Behauptung}
  Für natürliche Zahlen \(b \ge 2\) und \(k \ge 1\) gilt
  \[\sum_{i=0}^{k-1}{b^{i}} = \frac{b^{k}-1}{b-1}.\]
\end{Behauptung}
\begin{proof}
  Denn multiplizieren wir die linke Seite mit \((b-1)\) dann erhalten
  wir
  \[b - 1 + b^{2} - b + \cdots + b^{k} - b^{k-1} = b^{k} - 1.\]
\end{proof}
\subsubsection{4. Aufgabe, Teil b}
Sei \(a \in \mathbb{N}\) gegeben.  Dann ist \(a\) in der \(b\)-adischen Darstellung
durch
\begin{align*}
  a &= a_{0}b^{0} + a_{1}b^{1} + \cdots + a_{n}b^{n} \\
    &= a_{0}(b^{0} - 1) +  a_{0} + a^{1}(b^{1} - 1) + a^{1} + \cdots +
      a_{n}(b^{n} - 1) + a_{n} \\
    &= (a_{0} + \cdots + a_{n}) + a_{0}(b^{0} - 1) + \cdots + a_{n}(b^{n} - 1) \\
  &= (a_{0} + \cdots + a_{n}) + a_{0}(b^{0})(b-1) + \cdots + a_{n}(b^{0} +
    \cdots + b^{n-1})(b-1) \tag*{(*)}
\end{align*}
gegeben.  Also ist \(a\) genau dann durch \((b-1)\) teilbar, falls ihre
Quersumme \((a_{0}+ \cdots +a_{n})\) durch \((b-1)\) teilbar ist, wie man in
der Gleichung (*) sofort sehen kann.
\end{document}
