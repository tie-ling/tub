% 使用 chktex 检查 tex 文件中的语法错误
% settings for chktex
% chktex-file 3

% draft mode incompatible with tikz
\documentclass[a5paper]{article}

% 让每个章节 subsection 在新的一页上开始
% 而不是紧接着上一章节
\AddToHook{cmd/subsection/before}{\clearpage}

% 隐藏默认的章节序号:实际作业中与这个冲突
% https://tex.stackexchange.com/a/30225
\setcounter{secnumdepth}{0}

% 设置页边距,上下左右
\usepackage{geometry}
\geometry{a5paper,
 left=1cm,
 right=1cm,
 top=1cm,
 bottom=2cm
}

% 让 TeX 支持德语
\usepackage[ngerman]{babel}

% 调整 Emacs 预览字体大小
% (setq preview-scale-function 1.5)

% TeX Gyre Schola and unicode-math
% which in turn loads amsmath,mathtools,fontspec and friends
\usepackage{xcharter-otf}

% 加载数学相关的包
\usepackage{
  % 自然段间留空
  parskip,
  % 区间排版
  interval
}

% 中文支持,暂时不需要
% \usepackage[UTF8]{ctex}

% 根据 AMS 建议,应为成对符号(比如绝对值)定义新命令
\providecommand{\abs}[1]{\left\lvert#1\right\rvert}
\providecommand{\norm}[1]{\left\lVert#1\right\rVert}
\newcommand{\mt}{\texttt}
\newcommand{\npreceq}{\npreccurlyeq}

\usepackage{amsthm}
% 定理定义,依赖于 amsthm
\theoremstyle{remark}
\newtheorem*{Behauptung}{Behauptung}
\newtheorem*{Lemma}{Lemma}
\newtheorem*{Satz}{Satz}
\newtheorem*{Definition}{Definition}

% 标题与作者
\title{HA 7, CoMa 1, Laurin, Gruppe 108}
\author{Kuan 480169, Yu 478912, Guo 480788}

% 画图工具
\usepackage{tikz,pgfplots}
\pgfplotsset{compat=1.18}
\usetikzlibrary{shapes.geometric}
 \usepackage{graphicx}

\begin{document}
\maketitle
\begin{center}
Ting Yu Kuan 480169, Shilong Yu 478912, Yuchen Guo 480788
\end{center}
\newpage
\subsection{1. Aufgabe}
\subsubsection{Teil i}
\begin{proof}
Es gibt insgesamt vier Aussagen zu beweisen.

\begin{itemize}
\item Ein größtes Element \(p\) von \(S\) ist ein maximales
  Element in \(S\).

  Weil \(p\) ein größtes Element von \(S\) ist, gilt für
  alle \(y \in S\) dass \(y \preceq p\).  Insbesondere, es
  gilt für alle \(y \in S\) mit \(y \succeq p\) dass
  \(y \preceq p\).  Wegen Antisymmetrie folgt dann
  \(y = p\).  Die Definition eines maximalen Elements ist
  erfüllt.
\item Es gibt kein anderes maximale Element in \(S\).

  Sei \(q\) ein maximales Element in \(S\).  Weil \(p\) ein
  größtes Element von \(S\) ist, gilt \(q \preceq p\).  Weil
  \(q\) ein maximales Element in \(S\) ist und
  \(q \preceq p\) gilt, gilt \(q \succeq p\).  Wegen
  Antisymmetrie folgt dann \(p = q\).  Das größten Element
  ist dann eindeutig bestimmt.
\item Ein kleinstes Element von \(S\) ist ein
  minimales Element in \(S\).

  Weil \(p\) ein kleinstes Element von \(S\) ist, gilt für
  alle \(y \in S\) dass \(y \succeq p\).  Insbesondere, es
  gilt für alle \(y \in S\) mit \(y \preceq p\) dass
  \(y \succeq p\).  Wegen Antisymmetrie folgt dann
  \(y = p\).  Die Definition eines minimalen Elements ist
  erfüllt.
\item Es gibt kein anderes minimale Element in \(S\).

  Sei \(q\) ein minimales Element in \(S\).  Weil \(p\) ein
  kleinstes Element von \(S\) ist, gilt \(q \succeq p\).
  Weil \(q\) ein minimales Element in \(S\) ist und
  \(q \succeq p\) gilt, gilt \(q \preceq p\).  Wegen
  Antisymmetrie folgt dann \(p = q\).  Das kleinsten
  Element ist dann eindeutig bestimmt. \qedhere
\end{itemize}
\end{proof}
\subsubsection{Teil ii}
\begin{proof}
  durch Widerspruch.  Wir betrachten die Menge \(M\) aller minimalen
  Elementen in \(S\).  Zuerst ist diese Menge nicht leer, denn in dem
  schlechtesten Fall dass \(x \npreceq y\) für alle \(x \ne y\) in \(S\) ist jedes
  Element selbst ein minimales Element.  Wegen \(S\) endlich ist \(M\)
  auch endlich.

  Angenommen, es gibt kein minimales Element \(x \in M\)  sodass mit \(x
  \preceq y\).  Daraus folgt, dass für alle \(s \in \{s \in S \mid s \preceq y\}\) und alle
  \(x \in M\) gilt \(x \npreceq s\), denn andersfalls folgt aus Transitivität dass
  \(x \preceq y\) im Widerspruch zur Voraussetzung.

  Es gilt jetzt, für alle \(x \in M\) und für alle \(s \in \{s \in S \mid s
  \preceq y\}\) dass \(x \npreceq s\).  Wir bemerken, dass die Menge \(\{s \in S \mid s
  \preceq y\}\) eine partielle Ordnung enthält und daher ein minimales
  Element \(s_{m}\) besitzt.  Wegen \(x \npreceq s_{m}\) für alle \(x \in M\) ist
  \(s_{m}\) minimal in \(S\) und soll \(s_{m} \in M\) gelten, im Widerspruch
  zur Voraussetzung.
\end{proof}
\subsubsection{Teil iii}
  Wir zeigen, dass es eine Bijektion
  \[\pi\colon \{1, 2, \ldots, \abs{S}\} \to S\]
  existiert mit \(\pi(j) \npreceq \pi(i)\) für alle \(i < j\).

  Wir konstruieren eine Bijektion mittels einem rekursiven
  Algorithmus.
\begin{verbatim}
Pseudo-Python-Code

def relation(a, b):
    if a ?< b:
        return True
    return False

def toposort(S):
    if len(S) <= 1:
        return S

    x = S[0]
    left = list()
    right = list()

    for element in S:
        if relation(x, element):
            left.append(element)
        else:
            right.append(element)

    return toposort(left) + [x] + toposort(right)
\end{verbatim}
  \begin{Behauptung}
    Der Algorithmus ist korrekt und terminiert.
  \end{Behauptung}
  \begin{proof}
    Wir zeigen die Korrektheit der Algorithmus.  Wir nummieren alle
    Elemente in \mt{Output}  mit \(\{x_{1}, \ldots, x_{\abs{S}}\}\) durch.

    Sei  \(i < j\) beliebige Indizes des \mt{Output}.  Wir
    zeigen, dass \(\pi(i) \npreceq \pi(j)\) gilt.  Wegen Definition des Programms
    ist \(\pi(i)\) in \mt{toposort(left)} enthält.  Dieser List enthält
    wegen Definition alle \(j\) sodass \(\pi(j) \preceq \pi(i)\).  Damit ist die
    Bedingung von Topologische Sortierung erfüllt.

    Der Programm terminiert, denn jedes Aufruf reduziert die Länge des
    Lists um Eins.
  \end{proof}

\subsection{2. Aufgabe}

\subsubsection{Teil i}
Die Zahl Eins ist ein kleinste Element, denn
\[\forall n \in \mathbb{N}\colon 1 \vert n.\]
Wegen [1. Aufgabe] ist das kleinste Element eindeutig bestimmt und
gleichzeitig das einzige minimale Element.

Es gibt kein maximales Element.  Denn sei \(i \in \mathbb{N}\) ein minimales
Element, das heißt, alle durch \(i\) teilbare natürliche Zahlen ist nur
\(i\) selbst.  Dies ist nicht wahr, denn \(i\mathbb{N}\) ist durch \(i\) teilbar.
Damit existiert auch kein größes Element.
\subsubsection{Teil ii}
Die leere Menge ist ein kleinste Element, denn alle Mengen
sie enthält.  Die ist auch das minimalen Element und
eindeutig, wegen [1. Aufgabe].

Es gilt \(\{1, 2, 3, 4\} \notin S\).  Daraus folgt, dass es kein größtes
Element gibt.   Hingegen sind
\[ T \coloneq \{\{1, 3\}, \{1, 4\}, \{2\}\}\]
maximal.  Denn für alle \(M \in S\) mit \(N \preceq M\) und \(N \in T\) gilt \(M = N\).
\subsubsection{Teil iii}
Sei \(S = \mathcal{P}(\mathbb{N})\), also die Potenzmenge von \(\mathbb{N}\).  Dann ist \(\emptyset\) das
kleinste Element und \(\mathbb{N}\) das größte Element.  Eindeutigkeit und
Gleichheit mit minimales und maximales Element folgt aus [1. Aufgabe].
\subsubsection{Teil iv}
Es existiert kein größtes oder kleinstes oder maximales oder minimales
Element.  Denn sei \((a, b) \in \mathbb{R}^{2}\) beliebig.  Dann ist
\[ (a-1, b-1) \preceq (a,b) \preceq (a+1, b+1).\]

\subsubsection{Teil v}
Sei \(S = B(0, 1)\).  Dann sind
\[\{(-1/\sqrt2,-1/\sqrt2), (-1,0),(0,-1)\}\]
minimal und
\[\{(1/\sqrt2,1/\sqrt2), (1,0),(0,1)\}\]
maximal.  Beweis.  Sei \((a, b) \in S\).  Falls \(b = 0\), dann gilt
\[(-1, 0)\preceq(a, b)\preceq(1, 0).\]
Falls \(a = 0\) ist der Beweis analog.  Falls \(a, b \ne 0\), dann gilt
\[(-1/\sqrt2,-1/\sqrt2) \preceq (a, b) \preceq (1/\sqrt2,1/\sqrt2).\]
Es existiert kein größtes oder kleinstes Element, denn falls
sie existiere, seien sie eindeutig bestimmt und gleich das einzige
maxi oder mini Element.
\subsubsection{Teil vi}
Die minimalen Elemente ist
\[\{(-1, 0)\}\]
denn für alle \((a, b) \in S\) gilt \(-1 \le a\).

\subsection{3. Aufgabe}
\subsubsection{Teil i}
\begin{verbatim}
######## Aufruf 0
def findMissingNumber([6, 4], 4):
    if len(A) == 0:              # len(A) = 4 != 0
        return a
    else:
        m = a + len(A) // 2      # m = 5 = 4 + 1
        l = []
        L = []
        for i in A:              # i = 6
            if i > m:            # i > 5
                L.append(i)      # L = [6]
            else:
                l.append(i)
        for i in A:              # i = 4
            if i > m:            # i !> 5
                L.append(i)
            else:
                l.append(i)      # l = [4]
        if len(L) >= len(l):     # len([6]) == len([4])
            return findMissingNumber(l, a)
                                 # return findMissingNumber([4], 4)
        else:
            return findMissingNumber(L, m+1)

######## Aufruf 1
def findMissingNumber([4], 4):
    if len(A) == 0:              # len(A) = 1 != 0
        return a
    else:
        m = a + len(A) // 2      # m = 4 + 0
        l = []
        L = []
        for i in A:              # i = 4
            if i > m:            # i !> 4
                L.append(i)
            else:
                l.append(i)      # l = [4]
        if len(L) >= len(l):     # len(L) == 0 !> len(l) == 1
            return findMissingNumber(l, a)

        else:
            return findMissingNumber(L, m+1)
                                 # return findMissingNumber([], 5)

######## Aufruf 2
def findMissingNumber([], 5):
    if len(A) == 0:              # len(A) = 0 == 0
        return 5
\end{verbatim}
\subsubsection{Teil ii}
\begin{verbatim}
def findMissingNumber(A, a):
    if len(A) == 0:
        return a
    else:
        m = a + len(A) // 2
        l = []
        L = []
        for i in A:
            if i > m:
                L.append(i)
            else:
                l.append(i)
        if len(L) >= len(l):
            return findMissingNumber(l, a)
        else:
            return findMissingNumber(L, m+1)
\end{verbatim}
\begin{Behauptung}
  Der Algorithmus ist korrekt.
\end{Behauptung}
\begin{proof}
  Der Algorithmus ist ähnlich zur binären Suche.  In jeder Aufruf
  teilen wir den Input-List \mt{A} nach \mt{floor}-Mittelpunkt \mt{m} in zwei
  Listen: links bezeichnen wir mit \mt{l} und rechts mit \mt{L}, wobei
  für alle \mt{i} \(\in\) \mt{L} gilt \mt{i} \(>\) \mt{m} und für alle
  \mt{i} \(\in\) \mt{l} gilt \mt{i} \(\le\) \mt{m}.

  Falls \mt{b} \(-\) \mt{a} gerade ist. Dann gilt
\begin{verbatim}
  m = len(A) // 2 + a = (b - a)/2 + a.

  Input A Mit fehlenden x, aufsteigend sortiert:
  [a, a+1, a+2, ..., m-1, m, m+1, ..., b]
   <-------- n -------->     <--- n --->
          Liste l             Liste L  aber in beliebige Reihenfolge
\end{verbatim}
  Falls links die gesuchte Zahl \mt{x} enthält, dann gilt
\begin{verbatim}
len(l) = n - 1 < n = len(L)
\end{verbatim}
  In diesem Fall ist der nächster Aufruf \mt{findMissingNumber(l, a)}.

  Falls rechts die gesuchte Zahl \mt{x} enthält, dann gilt
\begin{verbatim}
len(l) = n > n - 1 = len(L)
\end{verbatim}
  In diesem Fall ist der nächster Aufruf \mt{findMissingNumber(L, m+1)}.

  Falls \mt{b} \(-\) \mt{a} ungerade ist. Dann gilt
\begin{verbatim}
  m = len(A) // 2 + a = (b - a - 1)/2 + a.

  Input A Mit fehlenden x, aufsteigend sortiert:
  [a, a+1, a+2, ..., m, m+1, ..., b]
   <------- n -------|---- n ----->
        Liste l         Liste L  aber in beliebige Reihenfolge
\end{verbatim}
  Falls links die gesuchte Zahl \mt{x} enthält, dann gilt
\begin{verbatim}
len(l) = n - 1 < n = len(L)
\end{verbatim}
  In diesem Fall ist der nächster Aufruf \mt{findMissingNumber(l, a)}.

  Falls rechts die gesuchte Zahl \mt{x} enthält, dann gilt
\begin{verbatim}
len(l) = n > n - 1 = len(L)
\end{verbatim}
  In diesem Fall ist der nächster Aufruf \mt{findMissingNumber(L,
    m+1)}.

  Der Algorithmus terminiert, falls \mt{A} leer ist.  Denn jede
  rekursive Aufruf reduziert die Länge von Input-Liste \mt{A} um ein Halbe.
\end{proof}
\subsubsection{Teil iii}
Jede rekursive Aufruf reduziert die
Länge von Input-Liste \mt{A} um ein Halbe, und jede rekursive Aufruf
enthält eine \mt{for}-Schleife, die die Länge der Input \mt{A}
durchläuft.  Damit gilt
\begin{align*}
  n + \left \lfloor \frac{n}{2}  \right \rfloor + \left \lfloor
  \frac{n}{4} \right \rfloor + \left \lfloor \frac{n}{8}
  \right \rfloor + \ldots + 0 \in \Theta(n).
\end{align*}
\subsection{4. Aufgabe}
\subsubsection{Teil i}
Wir modifizieren den Algorithmus Insertion-Sort, indem wir lineare
Vergleiche mit binären Suche ersetzen.  Damit erhalten wir \(\Theta(n\log
n)\) Vergleichungen.

\subsubsection{Teil ii}
Für das \(p\)-ten Element, wir müssen \(\log p\) Vergleichungen
und im schlimmsten Fall \(p\) Zuweisungsoperationen durchführen.

Summieren wir über \(p\),  wir erhalten
\[ \sum_{p = 1}^{n} (p + \log p) = \frac{n(n+1)}{2} + \log (n!) \in O(n^2). \]
\subsubsection{Teil iii}
\begin{verbatim}
Anfangszustand:
Kr7,Ka7,KrK,H10,H7,KrB,KrD,P10,Kr9,HD
\end{verbatim}
\end{document}

