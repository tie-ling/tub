% 使用 chktex 检查 tex 文件中的语法错误
% settings for chktex
% chktex-file 3

% draft mode incompatible with tikz
\documentclass[a5paper]{article}

% 让每个章节 subsection 在新的一页上开始
% 而不是紧接着上一章节
\AddToHook{cmd/subsection/before}{\clearpage}

% 隐藏默认的章节序号:实际作业中与这个冲突
% https://tex.stackexchange.com/a/30225
\setcounter{secnumdepth}{0}

% 设置页边距,上下左右
\usepackage{geometry}
\geometry{a5paper,
 left=1cm,
 right=1cm,
 top=1cm,
 bottom=2cm
}

% 让 TeX 支持德语
\usepackage[ngerman]{babel}

% 调整 Emacs 预览字体大小
% (setq preview-scale-function 1.5)

% TeX Gyre Schola and unicode-math
% which in turn loads amsmath,mathtools,fontspec and friends
\usepackage{xcharter-otf}


% 加载数学相关的包
\usepackage{
  % 自然段间留空
  parskip,
  % 区间排版
  interval
}

% 中文支持,暂时不需要
% \usepackage[UTF8]{ctex}

% 根据 AMS 建议,应为成对符号(比如绝对值)定义新命令
\providecommand{\abs}[1]{\left\lvert#1\right\rvert}
\providecommand{\norm}[1]{\left\lVert#1\right\rVert}

\usepackage{amsthm}
% 定理定义,依赖于 amsthm
\theoremstyle{remark}
\newtheorem*{Behauptung}{Behauptung}
\newtheorem*{Lemma}{Lemma}
\newtheorem*{Satz}{Satz}
\newtheorem*{Definition}{Definition}

% 标题与作者
\title{HA 6, CoMa 1, Laurin, Gruppe 108}
\author{Kuan 480169, Yu 478912, Guo 480788}

% 画图工具
\usepackage{tikz,pgfplots}
\pgfplotsset{compat=1.18}
\usetikzlibrary{shapes.geometric}
 \usepackage{graphicx}

\begin{document}
\maketitle
\begin{center}
Ting Yu Kuan 480169, Shilong Yu 478912, Yuchen Guo 480788
\end{center}
\newpage
\subsection{1. Aufgabe}
\subsubsection{Teil i}
\begin{Behauptung}
Die Aussage gilt für jeden Baum auf \(n\) Knoten
\(\implies\) Die Aussage gilt für jeden einfachen, ungerichteten,
zusammenhängenden Graphen auf \(n\) Knoten.
\end{Behauptung}
\begin{proof}
  Aus der Definition von Baum folgt, dass ein Baum ein ungerichteter,
  zusammenhängender Graph ohne Kreise ist.

  Nun fügen wir Kanten hinzu, sodass dieser ungerichteter,
  zusammenhängender Graph nicht mehr ein Baum ist.  Wir haben aber
  vorausgesetzt, dass es einen Knoten in dem Baum gibt, dessen Abstand
  von allen anderen Knoten höchstens \(\frac{n}{2}\) ist.  Der Abstand
  kann nur durch das Hinzufügen von Kanten verkürzt werden.  Daraus
  folgt, dass die Aussage für jeden einfachen, ungerichteten,
  zusammenhängenden Graphen auf \(n\) Knoten gilt.
\end{proof}
\subsubsection{Teil ii}
\begin{Behauptung}
  Es gibt in jedem einfachen, ungerichteten,
zusammenhängenden Graphen \(G = (V, E, \Psi)\) ohne Kreise (Baum) mit \(n\) Knoten einen
Knoten, dessen Abstand von allen anderen Knoten höchstens
\(\frac{n}{2}\) ist.
\end{Behauptung}
\begin{proof}
    Wir bemühen uns um einen Widerspruchsbeweis.

  Falls \(n\) gerade.  Angenommen, der Abstand zwischen allen Knoten im
  Baum \(G\) ist mindestens \(\frac{n}{2}+1\).  Es gibt insgesamt $\left|
    V(G) \right|-1=n-1$ Kanten. Im Fall \(n=2\) dann gibt es nur \(2\)
  Knoten und \(1\) Kante.  Der Abstand zwischen allen Knoten ist \(1\) im
  Widerspruch zur Annahme \(\geq 2\).

  Falls \(n\) ungerade.  Angenommen, der Abstand zwischen allen Knoten
  im Baum \(G\) ist mindestens \(\lceil\frac{n}{2}\rceil\). Es gibt insgesamt $\left|
    V(G) \right|-1=n-1$ Kanten. Im Fall \(n=3\) dann gibt es nur \(3\)
  Knoten und \(2\) Kante.  Der Abstand zwischen den mittleren Punkt und
  den beiden Punkt am Rand ist \(1\) im Widerspruch zur Annahme \(\geq 2\).
\end{proof}
\subsection{2. Aufgabe}
Fehlt.

\subsection{3. Aufgabe}

\subsubsection{Teil i}

\begin{figure}[h]
\centering
\begin{tikzpicture}
  \begin{scope}[every node/.style={circle,thick,draw}]
    \node (1) at (0,1) {1};
    \node (2) at (0,0) {2};
    \node (3) at (0,-1) {3};
    \node (4) at (2,1) {4};
    \node (5) at (2,0) {5};
    \node (6) at (2,-1) {6};
\end{scope}
\begin{scope}
  \path [-] (1) edge[bend right=60] (2);
  \path [-] (2) edge[bend right=60] (3);
  \path [-] (3) edge (4);
  \path [-] (2) edge (5);
  \path [-] (4) edge[bend left=60] (5);
  \path [-] (5) edge[bend left=60] (6);
  \path [-] (4) edge[bend left=90] (6);
\end{scope}
\end{tikzpicture}
\caption{Original.}
\end{figure}

\begin{figure}[h]
\centering
\begin{tikzpicture}
  \begin{scope}[every node/.style={circle,thick,draw}]
    \node (1) at (0,1) {1};
    \node (2) at (0,0) {2};
    \node (3) at (0,-1) {3};
    \node (4) at (2,1) {4};
    \node (6) at (2,-1) {6};
\end{scope}
\begin{scope}
  \path [-] (1) edge[bend right=60] (2);
  \path [-] (2) edge[bend right=60] (3);
  \path [-] (3) edge (4);
  \path [-] (2) edge (4);
  \path [-] (4) edge[bend left=60] (4);
  \path [-] (4) edge[bend left=60] (6);
\end{scope}
\end{tikzpicture}
\caption{Kontraktion.}
\end{figure}

\[
  \begin{pmatrix}
0 & 1 & 0 & 0 & 0 \\
    1 & 0 & 1 & 1 & 0\\

    0 & 1 & 0 & 1 & 0\\
    0 & 1 & 1 & 0 & 1\\
    0 & 0 & 0 & 1 & 0
  \end{pmatrix}
\]
\subsubsection{Teil ii}
\begin{Behauptung}
  \(G\) zusammenhängend \(\implies\) \(G_{/e}\) zusammenhängend.
\end{Behauptung}
\begin{proof}
    Angenommen, \(G_{/e}\) ist nicht zusammenhängend.  Dann existiert
  mindestens ein \(v \in E_{/e}\) mit \(\delta(v) = \emptyset\).
  \begin{itemize}
  \item   Falls \(v \notin \Psi(e)\), dann ist \(G\) nicht
    zusammenhängend.
  \item   Falls \(v \in \Psi(e)\), dann gilt
  $(\Psi(e_1) \cup \Psi(e_2)) \setminus \left\{ e \right\} =
  \emptyset$.  Das heißt, die Knoten \(e_1, e_2\) besitzt keine Kante
  anders als \(e\).  Es gibt kein Weg zwischen \(e_1, e_2\) und alle
  andere Knoten in der Knotenmenge.  Dann ist \(G\) nicht
  zusammenhängend.
  \end{itemize}
\end{proof}
\begin{Behauptung}
  \(G_{/e}\) zusammenhängend \(\implies\) \(G\) zusammenhängend.
\end{Behauptung}
\begin{proof}
    Angenommen, \(G\) ist nicht zusammenhängend.  Es existiert ein $a, b
  \in V$ sodass es kein a-b Weg gibt.  Nach Kontraktion gibt es immer
  keinen a-b Weg.  Deshalb ist \(G_{/e}\) nicht zusammenhängend.
\end{proof}
\subsubsection{Teil iii}
\begin{Behauptung}
  \(G\) zusammenhängend mit \(n\) Kanten \(\implies\) nach \(n\)-ten
Kantenkontraktion gibt es nur eine Knoten.
\end{Behauptung}
\begin{proof}
    Dann gibt es für alle Knoten \(a, b \in V\) ein a-b Weg.  Daraus folgt
  die Behauptung.
\end{proof}
\begin{Behauptung}
  Nach \(n\)-ten Kantenkontraktion gibt es nur eine Knoten. \(\implies\) \(G\)
zusammenhängend mit \(n\) Kanten.
\end{Behauptung}
\begin{proof}
    Angenommen, \(G\) ist zusammenhängend aber lässt sich nicht auf eine
  Knoten kontraktieren.  Dann ist \(G\) im Widerspruch zur Voraussetzung
  nicht zusammenhängend.
\end{proof}

\subsection{4. Aufgabe}
\begin{verbatim}
def MinMaxSort(A, sta=0, end=-1):
    if end < 0:
        end += len(A)
    if sta >= end:
        return A
    mini = maxi = A[sta]
    mi = ma = sta

    for i in range(sta+1, end+1):
        if A[i] > maxi:
            maxi, ma = A[i], i
        if A[i] < mini:
            mini, mi = A[i], i

    A[mi], A[sta] = A[sta], A[mi]

    if (ma != sta):
        A[ma], A[end] = A[end], A[ma]
    else:
        A[mi], A[end] = A[end], A[mi]
    return MinMaxSort(A, sta+1, end-1)
\end{verbatim}
\subsubsection{Teil i}
\begin{Behauptung}
  Die Funktion ist korrekt.
\end{Behauptung}
\begin{proof}
    Wir zeigen, dass diese Funktion korrekt ist, mittels Induktion.

  Induktionsanfang.  Eingabe ist List \texttt{A}.  Beim
  ersten Aufruf dieser Funktion gilt \texttt{sta=0} und
  \texttt{end=len(A)-1}.  Innerhalb der \texttt{for}-Schleife wird
  jeweils
  \begin{itemize}
  \item das größte Element der Variable \texttt{maxi},
  \item der Index des größte Element der Variable \texttt{ma},
  \item das kleinste Element der Variable \texttt{mini},
  \item der Index des kleinste Element der Variable \texttt{mi},
  \end{itemize}
  zugewiesen.  Anschließend werden das Element am Anfang dieses Lists
  mit dem kleinsten Element dieses ganzen Lists und das Element am
  Ende dieses Lists mit dem größten Element dieses ganzen Lists
  getauscht.  So dass es gilt
  \begin{itemize}
  \item \texttt{L[0] \(\leq\) L[i]} für alle \texttt{i} \(\leq\)
    \texttt{len[L] - 1}, und
  \item \texttt{L[-1] \(\geq\) L[i]} für alle \texttt{i} \(\leq\)
    \texttt{len[L] - 1}.
  \end{itemize}

  Induktionsannahme.  Beim \(n\)-ten rekursiven Aufruf dieser
  Funktion, also \texttt{MinMaxSort (L, n, len(L)-n)} gilt, dass
  jeweils die ersten und die letzten \(n\) Elemente dieses Lists korrekt
  sortiert sind.

  Induktionsschritt.  Als Eingabe \texttt{(L, n, len(L)-n)}
  \(\rightarrow\) \texttt{(L, n+1, len(L)-n-1)}.  Dann gilt wegen
  Induktionsannahme dass
  \begin{itemize}
  \item \texttt{L[n+1]} \(\geq\) \texttt{L[i]} für alle
  \(0\leq\) \texttt{i} \(\leq\) \texttt{n},
\item \texttt{L[len(L)-n-1]} \(\leq\) \texttt{L[i]} für alle
  \texttt{len(L)-n} \(\leq\) \texttt{i} \(\leq\) \texttt{len(L)}.
\end{itemize}

Analog zum ersten Aufruf gilt auch \texttt{L[len(L)-n-1]}
größer-gleich und \texttt{L[n+1]} kleiner-gleich alle Elemente
inzwischen.  Deshalb werden die Elemente korrekt sortiert.
\end{proof}
\subsubsection{Teil ii}
\begin{Behauptung}
  Sei Laufzeitfunktion dieses Funktions \(g\).  Es gilt \(g \in \Theta(n^2)\).
\end{Behauptung}
\begin{proof}
    Wir bemerken, dass nur die \texttt{for}-Schleife und deren rekursive
  Aufrufe eine nicht konstante Laufzeit haben.

  Die Funktion terminiert genau dann, wenn alle Elemente sortiert
  sind. O.B.d.A, sei der Lange des Lists gerade, dann benötigt die
  Funktion
\begin{align*}
\underbrace{n + (n - 2) + (n - 4) + (n - 6) + \ldots + 0}_{n/2\text{ Elemente}}
\end{align*}
Schritte zu terminieren.  Die Laufzeitfunktion ist dann
\begin{align*}
\frac{n^2}{2}-n^2-n.
\end{align*}

Damit folgt  \(g \in \Theta(n^2)\).
\end{proof}
\end{document}

