% 使用 chktex 检查 tex 文件中的语法错误
% settings for chktex
% chktex-file 3

\documentclass[fleqn,draft,a5paper]{article}

% 让每个章节 subsection 在新的一页上开始
% 而不是紧接着上一章节
\AddToHook{cmd/subsection/before}{\clearpage}

% 隐藏默认的章节序号:实际作业中与这个冲突
% https://tex.stackexchange.com/a/30225
\setcounter{secnumdepth}{0}

% 设置页边距,上下左右
\usepackage{geometry}

\geometry{left=1cm,
 right=1cm,
 top=1cm,
 bottom=2cm}

% 让 TeX 支持德语
\usepackage[ngerman]{babel}

% 调整 Emacs 预览字体大小
% (setq preview-scale-function 1.5)

\usepackage{amsmath,mathtools,fontspec,amsthm,parskip,interval,unicode-math}
\usepackage{xcharter-otf}


% 根据 AMS 建议,应为成对符号(比如绝对值)定义新命令
\providecommand{\skp}[1]{\langle#1\rangle}
\providecommand{\abs}[1]{\left\lvert#1\right\rvert}
\providecommand{\norm}[1]{\left\lVert#1\right\rVert}
\providecommand{\ceil}[1]{\left\lceil#1\right\rceil}
\providecommand{\floor}[1]{\left\lfloor#1\right\rfloor}

% 定理定义,依赖于 amsthm
\theoremstyle{remark}
\newtheorem*{Behauptung}{Behauptung}
\newtheorem*{Lemma}{Lemma}
\newtheorem*{Satz}{Satz}
\newtheorem*{Definition}{Definition}

% 定义新函数,依赖于AMSmath
\DeclareMathOperator{\card}{card}
\DeclareMathOperator{\rg}{rg}
\DeclareMathOperator{\sspan}{span}

\newcommand{\wt}{\widetilde}
\newcommand{\dd}{\,\mathrm{d}}

% 标题与作者
\title{HA 9, CoMa 1, Laurin, Gruppe 108}
\author{Kuan 480169, Yu 478912, Guo 480788}

\begin{document}
\maketitle
\begin{center}
Ting Yu Kuan 480169, Shilong Yu 478912, Yuchen Guo 480788
\end{center}
\newpage
\subsection{1. Aufgabe}
\subsubsection{Teil a}
Seien
\begin{align*}
  \begin{bmatrix}
    3 & -4 & 8 \\
    4 & 3 & -6 \\
    0 & 2 & 25
  \end{bmatrix},
  \quad
  b =
  \begin{bmatrix}
    3 \\ 4 \\ 1
  \end{bmatrix}.
\end{align*}
Sei \(Q \in \mathbb{R}^{3 \times 3}\) definiernt durch die Spalten \(u_{1}, u_{2},
u_{3}\).  Wir bestimmen \(Q\) mit
\begin{align*}
  b_{i} = a_{i} - \sum_{j=1}^{i-1}{\skp{u_{j}, a_{i}}u_{j}}, \quad u_{i} \coloneq \frac{b_{i}}{\skp{b_{i}}}.
\end{align*}
Dann gilt
\begin{multline*}
  b_{1} = a_{1} - \sum_{j=1}^{0}\skp{u_{j}, a_{1}} \cdot u_{j} = a_{1} =
  \begin{bmatrix}
    3 \\ 4 \\ 0
  \end{bmatrix}. \\
  u_{1} \coloneq \frac{b_{1}}{\skp{b_{1}}} =
  \begin{bmatrix}
    \frac35 \\ \frac45 \\ 0
  \end{bmatrix}
  \text{ wobei } \skp{b_{1}} = \sqrt{3^{2} + 4^{2}} = 5. \\
  b_{2} = a_{2} - \sum_{j=1}^{1}\skp{u_{j}, a_{2}}u_{j} = a_{2} -
  \skp{u_{1}, a_{2}} u_{1} = a_{2} - 0 \cdot u_{1} =
  \begin{bmatrix}
    -4 \\ 3\\ 2
  \end{bmatrix}. \\
  u_{2} = \frac{b_{2}}{\skp{b_{2}}} =
  \frac{1}{\sqrt{29}}
  \begin{bmatrix}
    -4 \\ 3 \\ 2
  \end{bmatrix}. \\
  b_{3} = a_{3} - \skp{u_{1}, a_{3}}u_{1} - \skp{u_{2}, a_{3}}u_{2} =
  a_{3} - 0 =
  \begin{bmatrix}
    8 \\ -6 \\ 25
  \end{bmatrix}. \\
  u_{3} = \frac{b_{3}}{\skp{b_{3}}} =
  \begin{bmatrix}
    8 \\ -6 \\ 25
  \end{bmatrix} \cdot \frac{1}{5\sqrt{29}} \text{ wobei } \skp{b_{3}} =
    \sqrt{64 + 36 + 625} = 5 \sqrt{29}. \\
  \end{multline*}
  Dann gilt
  \begin{align*}
    Q =
    \begin{bmatrix}
      \frac35 & - \frac{4}{\sqrt{29}} & \frac{8}{5\sqrt{29}} \\
      \frac45 & \frac{3}{\sqrt{29}} & - \frac{6}{5\sqrt{29}} \\
      0 & \frac{2}{\sqrt{29}} & \frac{25}{5\sqrt{29}}
    \end{bmatrix},
    \quad
    Q^{T} =
    \begin{bmatrix}
      \frac35 & \frac45 & 0 \\
      -\frac{4}{\sqrt{29}} & \frac{3}{\sqrt{29}} & \frac{2}{\sqrt{29}}
      \\
      \frac{8}{5\sqrt{29}} & -\frac{6}{5\sqrt{29}} & \frac{25}{5\sqrt{29}}.
    \end{bmatrix}
  \end{align*}
  Dann gilt
  \begin{multline*}
    Q^{T} \cdot Q =     \begin{bmatrix}
      \frac35 & \frac45 & 0 \\
      -\frac{4}{\sqrt{29}} & \frac{3}{\sqrt{29}} & \frac{2}{\sqrt{29}}
      \\
      \frac{8}{5\sqrt{29}} & -\frac{6}{5\sqrt{29}} & \frac{25}{5\sqrt{29}}.
    \end{bmatrix} \cdot     \begin{bmatrix}
      \frac35 & - \frac{4}{\sqrt{29}} & \frac{8}{5\sqrt{29}} \\
      \frac45 & \frac{3}{\sqrt{29}} & - \frac{6}{5\sqrt{29}} \\
      0 & \frac{2}{\sqrt{29}} & \frac{25}{5\sqrt{29}}
    \end{bmatrix}
    =
    \begin{bmatrix}
      1 & 0 & 0 \\ 0 & 1 & 0 \\ 0 & 0 & 1
    \end{bmatrix}
    = I_{3}.
  \end{multline*}
  \subsubsection{Teil b}
  Wir berechnen \(R\).
  \begin{multline*}
    R[1,1] = \skp{u_{1}, a_{1}} = \frac{9}{5} + \frac{16}{5} = 5. \\
    R[1,2] = R[1,3] = R[2,1] = R[3,1] = R[3,2] = 0. \\
    R[2,2] = \skp{u_{2}, a_{2}} = \frac{1}{\sqrt{29}}
    \begin{bmatrix}
      -4 & 3 & 2
    \end{bmatrix}\cdot
    \begin{bmatrix}
      -4 \\ 3 \\ 2
    \end{bmatrix} = \sqrt{29}. \\
    R[3,3] = \skp{u_{3}, a_{3}} = 5\sqrt{29}. \\
  \end{multline*}
  Also gilt
  \begin{multline*}
    R =
    \begin{bmatrix}
      5 & 0 & 0 \\ 0 & \sqrt{29} & 0 \\ 0 & 0 & 5\sqrt{29}
    \end{bmatrix}. \\
    Q \cdot R =     \begin{bmatrix}
      \frac35 & - \frac{4}{\sqrt{29}} & \frac{8}{5\sqrt{29}} \\
      \frac45 & \frac{3}{\sqrt{29}} & - \frac{6}{5\sqrt{29}} \\
      0 & \frac{2}{\sqrt{29}} & \frac{25}{5\sqrt{29}}
    \end{bmatrix} \cdot R =
    \begin{bmatrix}
    3 & -4 & 8 \\
    4 & 3 & -6 \\
    0 & 2 & 25
    \end{bmatrix} = A. \\
  \end{multline*}
  \subsubsection{Teil c}
  Wir lösen das lineare Gleichungssystem \(A x = b\) mithilfe der
  \(QR\)-Zerlegung.
  Wegen \(A x = QRx = b\) setzen wir \(Rx = y \in \mathbb{R}^{3 \times 1}\).  Wir
  berechnen \(y\) mittels
  \begin{align*}
     y = Q^{T} b =  \begin{bmatrix}
      \frac35 & \frac45 & 0 \\
      -\frac{4}{\sqrt{29}} & \frac{3}{\sqrt{29}} & \frac{2}{\sqrt{29}}
      \\
      \frac{8}{5\sqrt{29}} & -\frac{6}{5\sqrt{29}} & \frac{25}{5\sqrt{29}}.
    \end{bmatrix} \cdot
    \begin{bmatrix}
      3 \\ 4 \\ 1
    \end{bmatrix}
    =
    \begin{bmatrix}
      5 \\ \frac{2}{\sqrt{29}} \\ \frac{5}{\sqrt{29}}.
    \end{bmatrix}
  \end{align*}
  Dann gilt \(R x = y \) mit
  \begin{align*}
    \begin{bmatrix}
      5 & 0 & 0 \\ 0 & \sqrt{29} & 0 \\ 0 & 0& 5\sqrt{29}
    \end{bmatrix}\cdot
    \begin{bmatrix}
      x_{1} \\ x_{2} \\ x_{3}
    \end{bmatrix} =
    \begin{bmatrix}
      5  \\ \frac{2}{\sqrt{29}} \\ \frac{5}{\sqrt{29}}
    \end{bmatrix}. \quad
        x =
    \begin{bmatrix}
      1 \\ \frac{2}{29} \\ \frac{1}{29}
    \end{bmatrix}
  \end{align*}
  \subsection{2. Aufgabe}
  \subsubsection{Teil a}
  Wir berechnen die \(LU\)-Zerlegung von \(A\) mittels
  Gauß-Jordan-Elimination.
  \begin{align*}
  \begin{bmatrix}
    1 & 2 & 3 \\ 4 & 6 & 6 \\ 4 & 2 & 2
  \end{bmatrix}
    \to
    \begin{bmatrix}
      1 & 2 & 3 \\ 0 & -2 & -6 \\ 0 & -6 & -10
    \end{bmatrix}
    \to
    \begin{bmatrix}
      1 & 2 & 3 \\ 0 & -2 & -6 \\ 0 & 0 & 8
    \end{bmatrix}
  \end{align*}
  also die oberen Matrix ist
  \begin{align*}
    U = 
    \begin{bmatrix}
      1 & 2 & 3 \\ 0 & -2 & -6 \\ 0 & 0 & 8
    \end{bmatrix}
  \end{align*}
  normierte untere Dreieicksmatrix
  \begin{align*}
    L =
    \begin{bmatrix}
      1 & 0 & 0 \\ 4 & 1 & 0 \\ 4 & 3 & 1
    \end{bmatrix}.
  \end{align*}
  \subsubsection{Teil b}
  Wir verwenden die \(LU\)-Zerlegung, um die Lösung der linearen
  Gleichungssysteme
  \begin{align*}
    Ax = b =
    \begin{bmatrix}
      8 \\ 16 \\ 24
    \end{bmatrix},
    \quad
    Ax = c =
    \begin{bmatrix}
      3 \\ -2 \\ 2
    \end{bmatrix}
  \end{align*}
  zu berechnen.

  Wegen \(Ax = LUx = b\) sei \(y = Ux =
  \begin{bmatrix}
    y_{1} \\ y _{2} \\ y_{3}
  \end{bmatrix}
  \).  Dann gilt
  \begin{align*}
    Ly = b =
    \begin{bmatrix}
      1 & 0 & 0 \\ 4&1&0 \\ 4&3&1
    \end{bmatrix}\cdot
    \begin{bmatrix}
      y_{1} \\ y_{2} \\ y_{3}
    \end{bmatrix}
    =
    \begin{bmatrix}
      8 \\ 16 \\ 24
    \end{bmatrix}
  \end{align*}
  erhalten wir
  \begin{align*}
    y =
    \begin{bmatrix}
      8 \\ -16 \\ 40
    \end{bmatrix}.
  \end{align*}
  Also
  \begin{align*}
    y = Ux =
    \begin{bmatrix}
      8 \\ -16 \\ 40
    \end{bmatrix}
    =
    \begin{bmatrix}
      1 & 2 & 3 \\ 0&-2&-6\\0&0&8
    \end{bmatrix}
    \begin{bmatrix}
      x_{1}\\ x_{2} \\ x_{3}
    \end{bmatrix}
  \end{align*}
  erhalten wir \(x =
  \begin{bmatrix}
    7 \\ -7 \\ 5
  \end{bmatrix}.
  \)  Sei
  \begin{align*}
    Ax = c =
    \begin{bmatrix}
      3 \\ -2 \\ 2
    \end{bmatrix}
    = LUx, \quad y = Ux =
    \begin{bmatrix}
      y_{1} \\ y_{2} \\ y_{3}
    \end{bmatrix}
  \end{align*} dann gilt
  \begin{align*}
    Ly = c =
    \begin{bmatrix}
      1 & 0 & 0 \\ 4&1&0 \\ 4&3&1
    \end{bmatrix}
    \begin{bmatrix}
      y_{1} \\ y_{2} \\ y_{3}
    \end{bmatrix}
    =
    \begin{bmatrix}
      3 \\ -2 \\ 2
    \end{bmatrix}
  \end{align*}
  erhalten wir
  \begin{align*}
    y =
    \begin{bmatrix}
      3 \\ -14 \\ 32
    \end{bmatrix}.
  \end{align*}
  Dann gilt
  \begin{align*}
    Ux =
    \begin{bmatrix}
      1&2&3\\0&-2&-6\\0&0&8
    \end{bmatrix}
    \begin{bmatrix}
      x_{1} \\ x_{2} \\ x_{3}
    \end{bmatrix}
    =
    \begin{bmatrix}
      3 \\ -14 \\ 32
    \end{bmatrix}.
  \end{align*}
  Erhalten wir
  \begin{align*}
    x =
    \begin{bmatrix}
      1 \\ -5 \\ 4
    \end{bmatrix}.
  \end{align*}
  \subsection{3. Aufgabe}
  Sei
  \begin{align*}
    \begin{bmatrix}
      9 & 2 & 4 \\ 3 & 0 & -3 \\ \lambda & 2 & 9
    \end{bmatrix}
    \begin{bmatrix}
      x_{1} \\ x_{2} \\ x_{3}
    \end{bmatrix}
    =
    \begin{bmatrix}
      \mu \\ 8 \\ 6
    \end{bmatrix}
  \end{align*}
  also
  \begin{gather*}
    \left[
    \begin{array}{ccc|c}
      9 & 2  & 4 & \mu \\
      3 & 0 & -3 & 8 \\
      \lambda & 2 & 9 & 6
    \end{array}
    \right]
    \to
    \left[
    \begin{array}{ccc|c}
      3 & 0 & -3 & 8 \\
      0 & 2 & 13 & \mu - 24 \\
      0 & 0 & \lambda -4 & 30 - \frac83\lambda - \mu
    \end{array}\right].
\end{gather*}
Daraus folgt
\begin{align*}
  x_{1} &= \frac{58-3\mu}{3\lambda-12} \\
  x_{2} &= \frac{3\mu\lambda+27\mu+32\lambda-882}{6\lambda-24} \\
  x_{3} &= \frac{90-3\mu-8\lambda}{3\lambda-12}
\end{align*}
Die Lösungsmenge des linearen Gleichungssystems ist
\begin{align*}
  \left\{\left(\frac{58-3\mu}{3\lambda-12}, \frac{3\mu\lambda+27\mu+32\lambda-882}{6\lambda-24}, \frac{90-3\mu-8\lambda}{3\lambda-12}\right)\right\}.
\end{align*}
Falls \(\lambda = 4\) oder \(\mu=\frac{58}{3}\) ist, besitzt die System keine
Lösung.
\subsection{4. Aufgabe}
Seien \(A \in \mathbb{R}^{m\times n}\) und für \(J \subseteq \{1, \ldots, n\}\) sei \(A_{\cdot,J} \in \mathbb{R}^{m \times
  \abs{J}}\) die Teilmatrix mit alle Spalten, deren Indizes in \(J\)
enthält.
\begin{Behauptung}
  Für alle \(J_{1}, J_{2} \subseteq \{1, \ldots, n\}\) gilt
  \[\rg(A_{\cdot,J_{1} \cup J_{2}}) + \rg(A_{\cdot,J_{1} \cap J_{2}}) \le
    \rg(A_{\cdot,J_{1}}) + \rg(A_{\cdot,J_{2}}).\]
\end{Behauptung}
\begin{proof}
  Wir bemerken zuerst, dass die Zeilenrang und die Spaltenrang
  übereinstimmen.  Seien die von Spalten in \(J_{1}\) aufgespannte
  Unterraum \(W_{1}\) und die von Spalten in \(J_{2}\) aufgespannte
  Unterraum \(W_{2}\).  Wegen Dimensionsformel gilt
  \begin{align*}
    \dim \sspan{(W_{1} \cup W_{2})} = \dim W_{1} + \dim W_{2} - \dim(W_{1} \cap W_{2}).
  \end{align*}
  Wegen \(A_{J_{1}\cup J_{2}} \in \sspan(W_{1} \cup W_{2})\) und \(A_{J_{1}\cap
    J_{2}} \in W_{1} \cap W_{2}\) folgt, dass
  \begin{align*}
    \rg A_{J_{1}\cup J_{2}} \le \dim \sspan(W_{1} \cup W_{2}), \quad \rg A_{J_{1}\cap
    J_{2}} \le  \dim W_{1} \cap W_{2}.
  \end{align*}
  Summieren wir die beide Seite, erhalten wir
  \begin{align*}
    \rg A_{J_{1}\cup J_{2}} + \rg A_{J_{1}\cap J_{2}}
    &\le \sspan(W_{1} \cup W_{2}) +  \dim W_{1} \cap W_{2} \\
    &= \dim W_{1} + \dim W_{2} = \rg(A_{\cdot,J_{1}}) + \rg(A_{\cdot,J_{2}})
  \end{align*}
  wie behauptet.
\end{proof}
\end{document}

