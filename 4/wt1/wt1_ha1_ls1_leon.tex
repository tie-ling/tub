% 使用 chktex 检查 tex 文件中的语法错误
% settings for chktex
% chktex-file 3

\documentclass[fleqn,draft,a5paper]{article}

% 让每个章节 subsection 在新的一页上开始
% 而不是紧接着上一章节
\AddToHook{cmd/subsection/before}{\clearpage}

% 隐藏默认的章节序号:实际作业中与这个冲突
% https://tex.stackexchange.com/a/30225
\setcounter{secnumdepth}{0}

% 设置页边距,上下左右
\usepackage{geometry}
\geometry{left=1cm,
 right=1cm,
 top=1cm,
 bottom=2cm}

% 让 TeX 支持德语
\usepackage[ngerman]{babel}
\usepackage{amsmath,mathtools,fontspec,amssymb,amsthm,parskip,interval,mathtools,fontspec}
\setmainfont{TeX Gyre Termes}

% 根据 AMS 建议,应为成对符号(比如绝对值)定义新命令
\providecommand{\skp}[1]{\langle#1\rangle}
\providecommand{\abs}[1]{\left\lvert#1\right\rvert}
\providecommand{\norm}[1]{\left\lVert#1\right\rVert}
\providecommand{\ceil}[1]{\left\lceil#1\right\rceil}
\providecommand{\floor}[1]{\left\lfloor#1\right\rfloor}

% 定理定义,依赖于 amsthm
\theoremstyle{remark}
\newtheorem*{Behauptung}{Behauptung}
\newtheorem*{Lemma}{Lemma}
\newtheorem*{Satz}{Satz}
\newtheorem*{Definition}{Definition}
% (TeX-run-style-hooks "amsmath,mathtools,fontspec")
% 定义新函数,依赖于AMSmath
\DeclareMathOperator{\card}{card}
\DeclareMathOperator{\rg}{rg}

\providecommand{\R}[1]{\mathrm{R#1}}

\newcommand{\wt}{\widetilde}
\newcommand{\dd}{\,\mathrm{d}}

% 标题与作者
\title{HA 1, WT 1, Leon, LS1}
\author{Liu 503531, Zhang 484981, Guo 480788}

\begin{document}
\maketitle
\begin{center}
  Shengjie Liu 503531, Meng Zhang 484981, Yuchen Guo 480788
\end{center}
\newpage
\subsection{1. Aufgabe}
\(A\) bezeichnet das Ereignis, dass der Stromfluss nicht unterbrochen
wird.
\begin{itemize}
\item \(A = A_{1} \cap A_{2} \cap A_{3} \cap A_{4}\)
\item \(A = A_{1} \cup A_{2} \cup A_{3} \cup A_{4}\)
\item \(A = A_{1} \cap (A_{2} \cup A_{3} \cup A_{4})\)
\item \(A = (A_{1} \cup A_{2}) \cap (A_{3} \cup A_{4})\)
\end{itemize}
\subsection{2. Aufgabe}
\begin{Behauptung}
  Für \(n \in \mathbb{N} \setminus \{0\}\) seien \((A_{i})_{i \le n}\) mit \(\abs{A_{i}} < \infty\).
  Dann gilt
  \[\abs{\bigcup_{j=1}^{n}{A_{j}}}=\sum_{j=1}^{n}{(-1)^{j+1}} \sum_{1 \le i_{1} \le
      \cdots \le i_{j} \le n}{\abs{A_{i_{1}} \cap \cdots \cap A_{i_{j}}}}.\]
\end{Behauptung}
\begin{proof}
  Wir definieren ein Wahrscheinlichkeitsraum \((\Omega, \mathcal{A}, \mathbb{P})\) mit
  \[\Omega \coloneq \bigcup_{j=1}^{n}{A_{j}}, \quad \mathcal{A} \coloneq \mathcal{P}(\Omega), \quad \mathbb{P}(A) \coloneq
    \frac{\abs{A}}{\abs{\Omega}}.\]
  Weil \(n\) endlich und alle \(A_{j}\) endlich sind, sind das Ergebnisraum
  \(\Omega\) und die \(\sigma\)-Algebra \(\mathcal{A}\) auch endlich.
  Die Abbildung \(\mathbb{P}\colon \mathcal{A} \to \interval{0}{1}, A \mapsto \frac{\abs{A}}{\abs{\Omega}}\)
  ist ein Wahrscheinlichkeitsmaß, denn sie ist normiert und
  \(\sigma\)-additiv.

 Aus der Siebformel von Sylvester folgt, dass
  \[\mathbb{P}\left(\bigcup_{j=1}^{n}{A_{j}}\right) = \sum_{j=1}^{n}{(-1)^{j+1}} \sum_{1 \le i_{1} \le
      \cdots \le i_{j} \le n}{\mathbb{P}{(A_{i_{1}} \cap \cdots \cap A_{i_{j}})}}.\]
  Es existiert eine bijektive Abbildung
  \[f\colon \interval{0}{1} \to \mathbb{R}_{\le \abs{\Omega}}, \quad \mathbb{P}(A) \mapsto \abs{A} = \mathbb{P}(A) \cdot
    \abs{\Omega}.\]
  Damit folgt die Behauptung aus der Komposition \(f \circ \mathbb{P}\), das heißt,
  wir bekommen das Formel in der Behauptung, indem wir die Siebformel
  von Sylvester mit \(\abs{\Omega}\) multipliziert.
\end{proof}
\subsection{3. Aufgabe}
Seien \((\Omega, \mathcal{A}, \mathbb{P})\) ein Wahrscheinlichkeitsraum und \((A_{n})_{n \in \mathbb{N}} \subseteq
\mathcal{A}\) eine Folge von Ereignissen.
\begin{Behauptung}
  Aus \(A_{n} \uparrow A\) folgt \(\lim_{n \to \infty} \mathbb{P}(A_{n}) = \mathbb{P} \left(\bigcup_{n \in \mathbb{N}}{A_{n}}\right)\).
\end{Behauptung}
\begin{proof}
  Wegen Voraussetzung ist \(\mathbb{P}\) ein Wahrscheinlichkeitsmaß, insbesondere
  ist \(\mathbb{P}\) \(\sigma\)-additiv.

  Wir definieren \(A \coloneq \bigcup_{n \in \mathbb{N}}{A_{n}}\) und
  \(A_{0} = \emptyset\). Wegen \(A_{i-1} \subseteq A_{i}\) und
  \(\sigma\)-Additivität, es gilt
  \[\mathbb{P}(A) = \sum_{i=1}^{\infty}{\mathbb{P}(A_{i} \setminus A_{i-1})}.\]
  Wir betrachten die partielle Summe von der Reihe, dann erhalten wir
  \[\sum_{i=1}^{\infty}{\mathbb{P}(A_{i} \setminus A_{i-1})} = \lim_{n \to \infty}{\sum_{i=1}^{n}{\mathbb{P}(A_{i}
        \setminus A_{i-1})}} = \lim_{n \to \infty}{\mathbb{P}(A_{n})}.\]
  Damit gilt die Behauptung.
\end{proof}
\begin{Behauptung}
  Falls \(A_{n} \downarrow A\), dann gilt \(\lim_{n \to \infty}{\mathbb{P}(A_{n})} = \mathbb{P}\left(\bigcap_{n \in
    \mathbb{N}} A_{n}\right)\).
\end{Behauptung}
\begin{proof}
  Weil \((\Omega, \mathcal{A}, \mathbb{P})\) ein Wahrscheinlichkeitsraum ist, ist
  \(\mathbb{P}\) normiert.  Daraus folgt, dass \(\mathbb{P}(\Omega) = 1\) und
  \(\mathbb{P}(A_{n}) < \infty\) für alle \(n \in \mathbb{N}\).  Setze
  \(B_{n} \coloneq A_{n} \setminus A \in \mathcal{A}\) für jedes
  \(n \in \mathbb{N}\).  Dann gilt \(B_{n} \downarrow \emptyset\).  Es gilt dann
  \[\mathbb{P}(A_{n}) - \mathbb{P}(A) = \mathbb{P}(B_{n})\]
  und
  \[\lim_{n \to \infty}{\mathbb{P}(B_{n})} = 0.\]
  Daraus folgt, dass
  \[\lim_{n \to \infty}{\mathbb{P}(A_{n}) - \mathbb{P}(A)} = 0\]
  und daher
  \[\lim_{n \to \infty}{\mathbb{P}(A_{n})} = \mathbb{P}(A) = \mathbb{P}\left(\bigcap_{n \in
    \mathbb{N}} A_{n}\right).\]
\end{proof}
\subsection{4. Aufgabe}
\subsubsection{Teil a}
Wir wählen das folgende diskrete Wahrscheinlichkeitsraum.
\[ \Omega = \{(6,5), (6,1), (2,5), (2,1)\}, \quad \abs{\Omega} = 4, \quad \mathcal{A} = \mathcal{P}(\Omega) \]
und
\[p((6,5)) = p((6,1)) =\frac{2\times3}{36}, \quad p((2,5)) = p((2,1)) = 
  \frac{4\times3}{36} \]
und daher
\[\mathbb{P}(\{\text{Mika siegt}\}) = \mathbb{P}(\{(6,5),(6,1),(2,1)\})=
  \frac{24}{36}.\]
\subsubsection{Teil b}
Wir bemerken, dass es nur eine Möglichkeit gibt, um die Bedingung,
Charlie in jedem Fall die besseren Gewinnchancen hat.
\begin{center}
\begin{tabular}{c c c| l}
  Mika & & Charlie & \\
  \hline
  W1 & < & W2 & geht nicht wegen Teil a \\
  W1 & < & W3 & \\
  W3 & < & W1 & geht nicht wegen W1 < W3 \\
  W3 & < & W2 & \\
  W2 & < & W3 & geht nicht wegen W3 < W2 \\
  W2 & < & W1
\end{tabular}  
\end{center}
Das heißt, die Bedingung dass Charlie in jedem Fall die besseren
Gewinnchancen hat, ist genau dann erfüllt, falls für W3 dass \(W_{1} <
W_{3}\) und \(W_{3} < W_{2}\); wobei \(W_{i}\) die relative Gewinnchancen
bezeichnet.

Für uns sind nur die Augenzahlen \(\{7,4,0\}\) interessant, denn die
Zahl \(7\) ist größer als alle vorhandende Augenzahlen, die Zahl \(0\) ist
kleiner als alle vorhandende Augenzahlen, und die Zahl \(4\) liegt
zwischen alle vorhandende Augenzahlen.  Wir müssen nur die Anzahl von
\(\{7,4,0\}\) im \(W_{3}\) so bestimmen, sodass die oberen Bedingungen
erfüllt sind.

Wir bezeichnen die Anzahl von \(\{7,4,0\}\) im \(W_{3}\) jeweils mit
\(a,b,c\).  Daraus folgt, dass
\begin{align*}
  &\Omega_{W_{1},W_{3}} = \{(7,6),(7,2),(4,6),(4,2),(0,6),(0,2)\}, \\
  &\Omega_{W_{2},W_{3}} = \{(7,5),(7,1),(4,5),(4,1),(0,5),(0,1)\}
\end{align*}
dann ist
\[\mathbb{P}(\{W_{3} \text{ siegt im } W_{1}, W_{3} \text{ Paar}\}) =
\mathbb{P}(\{(7,6),(7,2),(4,2)\}) = \frac{a}{6}+\frac{b}{6}\cdot\frac{4}{6} > \frac{1}{2}\]
und
\[\mathbb{P}(\{W_{2} \text{ siegt im } W_{2}, W_{3} \text{ Paar}\}) =
  \mathbb{P}(\{(4,5),(0,5),(0,1)\}) = \frac{b}{12}+\frac{c}{6} > \frac{1}{2}.\]
Daraus erhalten wir die Ungleichungen
\begin{align*}
  \begin{cases}
    3a+2b > 9 \\
    b+2c >6 \\
    a+b+c =6
  \end{cases}
  \to
  \begin{cases}
    c < 3 \\
    b > 0 \\
    6-a > c
  \end{cases}
\end{align*}

Eine Lösung ist \(a=3,b=1,c=2\). Damit hat Charlie mit
\(W_{3} = \{7,7,7,4,0,0\}\) in jedem Fall die besseren Gewinnchancen.
\end{document}

