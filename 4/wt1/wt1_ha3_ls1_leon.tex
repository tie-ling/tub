% 使用 chktex 检查 tex 文件中的语法错误
% settings for chktex
% chktex-file 3

\documentclass[fleqn,draft,a5paper]{article}

% 隐藏默认的章节序号:实际作业中与这个冲突
% https://tex.stackexchange.com/a/30225
\setcounter{secnumdepth}{0}

% 设置页边距,上下左右
\usepackage{geometry}
\geometry{left=1cm,
  right=1cm,
  top=1cm,
  bottom=2cm}

% (TeX-run-style-hooks "amsmath,mathtools,fontspec" "amsthm")

% 让 TeX 支持德语
\usepackage[ngerman]{babel}
\usepackage{amsmath,amssymb,mathtools,amsthm,parskip,interval,fontspec}
\setmainfont{TeX Gyre Termes}

% 根据 AMS 建议,应为成对符号(比如绝对值)定义新命令
\providecommand{\skp}[1]{\langle#1\rangle}
\providecommand{\abs}[1]{\left\lvert#1\right\rvert}
\providecommand{\norm}[1]{\left\lVert#1\right\rVert}
\providecommand{\ceil}[1]{\left\lceil#1\right\rceil}
\providecommand{\floor}[1]{\left\lfloor#1\right\rfloor}

% 定理定义,依赖于 amsthm
\theoremstyle{remark}
\newtheorem*{Behauptung}{Behauptung}
\newtheorem*{Lemma}{Lemma}
\newtheorem*{Satz}{Satz}
\newtheorem*{Definition}{Definition}

% 定义新函数,依赖于AMSmath
\DeclareMathOperator{\card}{card}
\DeclareMathOperator{\Rang}{Rang}

\providecommand{\R}[1]{\mathrm{R#1}}
\newcommand{\cP}{\mathbb{P}}
\newcommand{\wt}{\widetilde}
\newcommand{\dd}{\,\mathrm{d}}
\title{WT1, HA3, Gruppe LS1, Leon}
\author{Liu 503531, Zhang 484981, Guo 480788}

\begin{document}
\maketitle

\subsection{Aufgabe 1}
Aus der Aufgabe folgt, dass $3 \%$ der Klausuren sind
Betrugsversuche
\begin{align*}
  P(K) = 0.03.
\end{align*}
Das Software zeigt bei $95 \%$ der
Betrugsversuche ein positives Ergebnis
\begin{align*}
  P(\texttt{Test positiv} \mid K) = 0.95.
\end{align*}
Das besagter Software zeigt auch bei $a \cdot 100\%$ der Klausuren, die kein
Betrugsversuche sind, ein positives Ergebnis
\begin{align*}
  P(\texttt{Test positiv} \mid K^{c}) = a.
\end{align*}
Wir berechnen die bedingte Wahrscheinlichkeit mit Formel von Bayes
\begin{align*}
  &\frac{P(\texttt{Test positiv} \mid K)P(K)}{P(\texttt{Test positiv} \mid K)P(K)+P(\texttt{Test
    positiv} \mid K^{c})P(K^{c})} \\
  &= \frac{0.95 \cdot 0.03}{0.95 \cdot 0.03 + a \cdot 0.97}
\end{align*}
\subsection{Aufgabe 2}
\begin{Behauptung}
  Sei $\Omega = \{a\}$.  Dann ist $\Omega$ abzählbar, und für jede
  Zähldichte $p$ auf $\Omega$ gilt: jede Familie von Ereignissen ist
  unabhängig bezüglich des von $p$ induzierten W-Maßes.
\end{Behauptung}
\begin{proof}
  Zuerst ist $\Omega$ endlich, also abzählbar. Sei $p$ eine beliebige
  Zähldichte auf $\Omega$.  Dann ist $p(a) = 1$ und der von $p$
  induzierten W-Maß ist definiert durch Satz 1.11.

  Sei $(A_{i})_{i \in I}$ eine Familie von Ereignissen in $\Omega$.  Für
  jede nichtleere endliche Teilmenge $J \subseteq I$ ist
  \begin{align*}
    P\left(\bigcap_{j \in J} A_{j}\right) = \prod_{j \in J}P(A_{j})
  \end{align*}
  denn es gibt nur zwei Möglichkeiten: falls existiert $k \in J$ mit
  $A_{k} = \emptyset$, dann ist die Gleichung gleich Null.  Falls keine $k
  \in J$ mit $A_{k} = \emptyset$ existiert, dann ist $A_{j} = \{a\}$ für
  alle $j \in J$ und die Gleichung ist gleich Eins.
\end{proof}
Sei $\Omega = \mathbb{N}_{0}$ und $p\colon \Omega \to \left[0, 1\right]$ eine
Zähldichte.  Wir wählen $p$ derart, dass $p(\omega) > 0$ für jede
gerade Zahl $\omega \in \Omega$ und $A, B $ unabhängig, falls $A \subseteq \Omega$ nur
gerade Zahlen und $B \subseteq \Omega$ nur ungerade Zahlen enthält.
\begin{Behauptung}
  Wir wählen
  \begin{align*}
    p\colon \Omega \to \left[0, 1\right], \omega \mapsto
    \begin{cases}
      \frac{1}{k(k+1)}, & \text{falls $\omega > 0$ gerade;} \\
      0, & \text{falls $\omega > 0$ ungerade;} \\
      1 - \sum_{k \in 2 \mathbb{N}}{\frac{1}{k(k+1)}}, & \text{falls $\omega = 0$.}
    \end{cases}
  \end{align*}
  Dann erfüllt $p\colon \Omega = \mathbb{N}_{0} \to \left[0, 1\right]$ die genannte Bedingungen.
\end{Behauptung}
\begin{proof}
  Zuerst ist konvergiert die Reihe
  \begin{align*}
    \sum_{\omega \in \mathbb{N} \setminus \{0\}}\frac{1}{k(k+1)}
  \end{align*}
  absolut mit Grenzwert $1$ wegen HA2, Aufgabe 4.
  Deshalb konvergiert ein Teil davon
  \begin{align*}
    \sum_{\omega \in 2 \mathbb{N} \setminus \{0\}}\frac{1}{k(k+1)}
  \end{align*}
  auch absolut mit einen Grenzwert $0 < a < 1$.  Wir setzen
  $p(0) = 1 - a$ und $p(\omega)=0$ für alle $\omega$ ungerade, dann ist
  $p$ normiert und $\sigma$-additiv.

  Sei $A, B \subseteq \Omega$ beliebig sodass $A$ nur gerade Zahlen und
  $B$ nur ungerade Zahlen enthält.  Dann gilt
  \begin{align*}
    P(A \cap B) = P(\emptyset) = 0 = P(A) \cdot P(B) = P(A) \cdot 0 = 0
  \end{align*}
  wie behauptet.
\end{proof}
\subsection{Aufgabe 3}
\begin{Behauptung}
Es sei $(\Omega, p)$ ein diskreter Wahrscheinlichkeitsraum mit induzierten
W-Maß $\cP$.  Die Ereignisse $A_{1}, \ldots, A_{n}$ mit $\cP(A_{1} \cap \ldots \cap A_{n})
>0 $ sind genau dann unabhängig, falls für jedes $k \in \{2, \ldots, n\}$,
jede Auswahl $1 \le i_{1} < \ldots < i_{k} \le n$, und alle $j \in \{1, \ldots, k\}$
dass
\[ \cP(A_{i_{j}}\mid \bigcap_{\substack{m=1 \\ m \ne j}}^{k}A_{i_{m}}) =
    \cP(A_{i_{j}})
\]
gilt.  
\end{Behauptung}
\begin{proof}
  Hinrichtung.  Es folgt aus Definition von der bedingte
  Wahrscheinlichkeit von $A$ unter der Bedingung $B$ dass
  \begin{align*}
    \cP(A_{i_{j}}\mid \bigcap_{\substack{m=1 \\ m \ne j}}^{k}A_{i_{m}}) =
    \cP(A_{i_{j}}) 
    = \frac{\cP(\bigcap_{m=1}^{k}A_{i_{m}})}{\cP(\bigcap_{\substack{m=1 \\ m \ne j}}^{k}A_{i_{m}})}
  \end{align*}
  Damit gilt
  \[\cP(A_{i_{j}})  \cdot \cP(\bigcap_{\substack{m=1 \\ m \ne j}}^{k}A_{i_{m}}) =
    \cP(\bigcap_{m=1}^{k}A_{i_{m}}). \tag{1}\]
  Korollar.  Es gilt \[
    \cP(\bigcap_{\substack{m=1 \\ m \ne j}}^{k}A_{i_{m}}) = \prod_{\substack{m=1 \\
      m \ne j}}^{k} \cP(A_{i_{m}}).
\]
Beweis.  Induktion über $k$.
Induktionsanfang. Wir betrachten den Fall $k = 2$.  Dann ist für alle $1
  \le i_{1}  < i_{2} \le n$ und für alle $j \in \{1, 2\}$ dass
  \[\cP(A_{i_{1}} \mid A_{i_{2}}) = \cP(A_{i_{1}}) = \frac{\cP(A_{i_{1}}
      \cap A_{i_{2}})}{\cP(A_{i_{2}})}, \quad \cP(A_{i_{2}} \mid
    A_{i_{1}}) = \cP(A_{i_{2}}) = \frac{\cP(A_{i_{1}}
      \cap A_{i_{2}})}{\cP(A_{i_{1}})}.\]
  Dann folgt $\cP(A_{i_{1}} \cap A_{i_{2}}) = \cP(A_{i_{1}})\cP(A_{i_{2}})$.

  Induktionsschritt. Angenommen, das Korollar gelte für ein $k \coloneq a$, wir
  betrachten $k \coloneq a+1$.  Dies gilt wegen (1).  Damit ist das Korollar
  bewiesen.

  Damit ist die Bedingung $P(\bigcap_{j \in J} A_{j}) = \prod_{j \in J}P(A_{j})$
  erfüllt und die Ereignisse sind unabhängig.

  Rückrichtung.
\end{proof}
\subsection{Aufgabe 4}
\begin{Behauptung}
  Seien $(\Omega, \mathcal{A}, \cP)$ ein W-Raum und $C_{1}, C_{2}, \ldots \in \mathcal{A}$ höchstens
  abzählbar unendlich viele paarweise disjunkte Ereignisse mit
  positiven Wahrscheinlichkeiten und $C \coloneq \bigcup_{k \ge 1}C_{k}$.  Weiter
  sei $A \in \mathcal{A}$ gegeben.  Falls $\cP(A \mid C_{k})$ nicht von $k$ abhängt,
  gilt stets $\cP(A \mid C) = \cP(A \mid C_{k})$.
\end{Behauptung}
\end{document}
