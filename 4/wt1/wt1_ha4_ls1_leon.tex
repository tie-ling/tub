% 使用 chktex 检查 tex 文件中的语法错误
% settings for chktex
% chktex-file 3

\documentclass[fleqn,draft,a5paper,12pt]{article}

% 隐藏默认的章节序号:实际作业中与这个冲突
% https://tex.stackexchange.com/a/30225
\setcounter{secnumdepth}{0}

% 设置页边距,上下左右
\usepackage{geometry}
\geometry{left=1cm,
  right=1cm,
  top=1cm,
  bottom=2cm}

% (TeX-run-style-hooks "amsmath,mathtools,fontspec" "amsthm")

% 让 TeX 支持德语
\usepackage[ngerman]{babel}
\usepackage{amsmath,unicode-math,amsthm,parskip,interval}
\setmainfont{TeX Gyre Termes}
\setmathfont{TeX Gyre Termes Math}
\setmathfont[range={\mathcal}]{latinmodern-math.otf}

% 根据 AMS 建议,应为成对符号(比如绝对值)定义新命令
\providecommand{\skp}[1]{\langle#1\rangle}
\providecommand{\abs}[1]{\left\lvert#1\right\rvert}
\providecommand{\norm}[1]{\left\lVert#1\right\rVert}
\providecommand{\ceil}[1]{\left\lceil#1\right\rceil}
\providecommand{\floor}[1]{\left\lfloor#1\right\rfloor}

% 定理定义,依赖于 amsthm
\theoremstyle{remark}
\newtheorem*{Behauptung}{Behauptung}
\newtheorem*{Lemma}{Lemma}
\newtheorem*{Satz}{Satz}
\newtheorem*{Definition}{Definition}

% 定义新函数,依赖于AMSmath
\DeclareMathOperator{\card}{card}
\DeclareMathOperator{\Rang}{Rang}

\providecommand{\R}[1]{\mathrm{R#1}}
\newcommand{\cP}{\mathbb{P}}
\newcommand{\wt}{\widetilde}
\newcommand{\dd}{\,\mathrm{d}}
\title{WT1, HA4, Gruppe LS1, Leon}
\author{Liu 503531, Zhang 484981, Guo 480788}

\begin{document}
\maketitle

\subsection{Aufgabe 1}
Seien \((\Omega, \mathcal{A}, P)\) ein W-Raum und \((E, \mathcal{E})\) ein messbarer Raum.

Eine Abbildung \(X\colon \Omega \to E\) heißt \(E\)-wertige Zufallsvariable, falls für
alle \(B \in \mathcal{E}\) dass
\(\{X \in B\} = \{\omega \in \Omega \mid X(\omega) \in B\}\in \mathcal{A}\).

Ist \((E, \mathcal{E}) = (\mathbb{R}, \mathcal{B}(\mathbb{R}))\), so heißt \(X\) reellwertige Zufallsvariable.

Ist \(X\) eine reelle Zufallsvariable, so heißt
\[F_{X}\colon \mathbb{R} \to \mathbb{R}, \quad F_{X}(t) \coloneq P(X \le t)\]
die Verteilungsfunktion von \(X\).

Die Dichte \(f_{X}\colon \mathbb{R} \to \mathbb{R}\) von Verteilungsfunktion \(F_{X}\) ist
\[F_{X}(t) = \int_{- \infty}^{t}f_{X}(x) \dd x.\]

Das Bildmaß \(\cP \circ X^{-1}\) ist ein W-Maß auf \((E, \mathcal{E})\) und heißt
Verteilung von \(X\).
\subsubsection{Teil i}
Sei \(Y=aX+b\) mit \(a, b \in \mathbb{R}\) und \(a \ne 0\).  Dann folgt aus Definition
dass
\[
  F_{X}(t) = \cP (X \le t) = \int_{-\infty}^{t}f_{X}(x) \dd x
\]
und
\[
  F_{Y}(t) = \cP(Y \le t) = \int_{-\infty}^{t}f_{Y}(x) \dd x
\]
Daher
\[
  \text{falls } a > 0, \quad \cP(Y \le t) = \cP\left(X \le
    \frac{t-b}{a}\right) = \int_{-\infty}^{\frac{t-b}{a}}f_{X}(x) \dd x =
  \int_{-\infty}^{t}f_{Y}(x) \dd x.
\]
und
\[
  \text{falls } a < 0, \quad \cP(Y \le t) = \cP\left(X \le
    \frac{t-b}{-a}\right) = \int_{-\infty}^{\frac{t-b}{-a}}f_{X}(x) \dd x =
  \int_{-\infty}^{t}f_{Y}(x) \dd x.
\]
Es muss \(a \ne 0\) sein, denn sonst wäre \(Y\) konstant und es gilt \(\{Y
\le t\} = \emptyset\) für alle \(t < b\).
\subsubsection{Teil ii}
\(Y \coloneq \abs{X}\).  Dann gilt wegen Definition dass
\[\cP(Y \le t) = \cP(-t \le X \le t) = \int_{-t}^{t}f_{X}(x) \dd x =
  \int_{-\infty}^{t}f_{Y}(x) \dd x. \]
\subsubsection{Teil iii}
\(Y \coloneq X^{k}\) mit \(k \in \mathbb{N} \setminus \{0\}\). Dann gilt wegen Definition dass
\[\cP(Y \le t) = \cP(X^{k} \le t) = ? \]
\subsection{Aufgabe 2}
\begin{Behauptung}
  Sei \(X\) eine reele Zufallsgröße mit Dichte \(f_{X}\). Es gilt genau
  dann \(X \sim (-X)\), wenn für alle \(t \in \mathbb{R}\) dass
  \(f_{X}(t) = f_{X}(-t)\) gilt.
\end{Behauptung}
\begin{proof}
  Hinrichtung.  Angenommen, 
\end{proof}
\end{document}
