\setuppapersize[A5]
\mainlanguage[de]
\setupbodyfont[times,9pt]
\define[1]\abs{\lvert#1\rvert}
\defineenumeration[Satz]
                  [text=Satz.,number=no,headstyle=italic,title=no,style=normal]
\defineenumeration[Def]
                  [text=Definition.,number=no,headstyle=italic,title=no,style=normal]
\definemathmatrix
[pmatrix]
[left={\left(},right={\right)}]
\starttext
\startDef
  Die Gesamtheit aller möglichen Ausgänge heißt Ergebnisraum $\Omega$.  Ein
  Element $\omega \in \Omega $ heißt Ergebnis.
  Teilmengen $A \subseteq \Omega$ heißen Ereignisse.  Die Gesamtheit aller
  Ereignisse ist die Potenzmenge $\mathcal{P}(\Omega)$.  Unter den Ereignissen sind
  das sichere Ereignis $\Omega$, das unmögliche Ereignis $\emptyset$, die
  Elementarereignisse $\{\omega\}$ für $\omega \in \Omega$.
\stopDef
\startDef
  Es sei $\Omega \ne \emptyset$.  Ein Mengensystem $\mathcal{A} \subseteq \mathcal{P}(\Omega)$ heißt $\sigma$-Algebra in
  $\Omega$, falls gilt: $\Omega \in \mathcal{A}$, komplementärstabil aus $A \in \mathcal{A}$ folgt
  $A^{c} \in \mathcal{A}$, vereinigungsstabil bzgl. abzählbar viele Vereinigungen.
\stopDef
\startDef
  $(\Omega, \mathcal{A})$ ist messbarer Raum. $A \subseteq \Omega$ mit $A \in \mathcal{A}$ ist $\mathcal{A}$-messbar.
\stopDef
\startSatz
  Sei $C \subseteq \mathcal{P}(\Omega)$ beliebig.  Dann ist der Durchschnitt aller
  $\sigma$-Algebra $A_{i}$, der $C$ enthält, eine $\sigma$-Algebra, die von $C$
  erzeugte $\sigma$-Algebra.
\stopSatz
\startDef
  Sei $(\Omega, \mathcal{A})$ ein messbarer Raum.  Eine Abbildung $P\colon \mathcal{A} \to \left[0,
    1\right]$ heißt Wahrscheinlichkeitsmaß auf $\mathcal{A}$, falls gilt:
  normiert $P(\Omega)=1$; $\sigma$-additiv: für jede Folge paarweiser disjunkter
  Ereignisse in $\mathcal{A}$ gilt $P(\bigcup A_{n})=\sum P(A_{n})$.
\stopDef
\startDef
  Sei $\Omega \ne \emptyset$, $\mathcal{A}$ eine $\sigma$-Algebra in
  $\Omega$ und $P$ ein Wahrscheinlichkeitsmaß auf $\mathcal{A}$.  Dann heißt das
  Tripel $(\Omega, \mathcal{A}, P)$ Wahrscheinlichkeitsraum.
\stopDef
\startSatz
  Sei $(\Omega, \mathcal{A}, P)$ ein Waahrscheinlichkeitsraum.  Dann gilt für $A_{1},
  \ldots, A_{n} \in \mathcal{A}$ dass
  \startformula
    P(A_{1} \cup \ldots \cup A_{n}) = \sum_{j=1}^{n}{(-1)^{j-1} \sum_{1\le i_{1}< \cdots <
        i_{j} \le n} P(A_{i_{1}} \cap \ldots \cap A_{i_{j}}).}
  \stopformula
\stopSatz
\startDef
  Der Ergebnisraum $\Omega$ heißt diskret, falls die Menge $\Omega$ höchstens
  abzählbar unendlich ist.
\stopDef
\startDef
  Sei $\Omega \ne \emptyset$ höchstens abzählbar.  Eine Funktion $p\colon \Omega \to \left[0,
    1\right]$ mit $\sum_{\omega \in \Omega}p(\omega)=1$ heißt Zähldichte oder Wahrscheinlichkeitsfunktion.
\stopDef
\startSatz
  Sei $\Omega$ abzahelbar und $p$ eine Zähldichte.  Dann wird durch $P(A)
  \colonequals \sum_{\omega \in A}p(\omega), A \subseteq \Omega$ ein Wahrscheinlichkeitsmaß definiert.
\stopSatz
\startDef
  Laplacescher Wahrscheinlichkeitsraum.  Ist $\Omega$ eine endliche Menge,
  so definiert $p(\omega) \colonequals \frac{1}{\abs{\Omega}}, \omega \in \Omega$ eine
  Zähldichte auf $\Omega$.  Für eine beliebige Ereignis $A$ folgt, $P(A) =
  \frac{\abs{A}}{\abs{\Omega}}$.  Dann heißt $P(A)$
  Laplace-Wahrscheinlichkeit von $A$. Weil jedes Elementarereignis
  gleichwahrscheinlich ist, spricht man von $P$ auch als der
  Gleichverteilung auf $\Omega$.
\stopDef

\startSatz
  Es ist $\frac{n!}{(n-k)! k!} \equalscolon
  \startpmatrix
    \NC n \NR \NC k \NR
  \stoppmatrix
  $.  Dann ist
Mit Zurücklegen, in Reihenfolge $n^{k}$, Mit Zurücklegen, ohne
Reihenfolge $
\startpmatrix
  \NC n+k-1 \NR \NC k \NR
\stoppmatrix
$, ohne Zurücklegen, in Reihenfolge $
  \frac{n!}{(n-k)!}
  $ , ohne Zurücklegen ohne Reihenfolge
  $
  \startpmatrix
    \NC n \NR \NC k \NR
  \stoppmatrix
  $
\stopSatz
\stoptext
