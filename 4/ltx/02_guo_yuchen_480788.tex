% Datum 29.04.2024
% Vorname Yuchen
% Nachname Guo
% Matrikelnummer 480788
% E-Mail-Adresse gyuchen86@gmail.com

% Beginn der Präambel
\documentclass{article}
\usepackage[utf8]{inputenc}
\usepackage[T1]{fontenc}
\usepackage[ngerman]{babel}
\usepackage{blindtext}
\usepackage{array}
\title{Bärenstarker Titel}
\author{Yuchen Guo, Eisbär}
\date{\today}
% Beginn der eigentlichen Text-Umgebung
\begin{document}
\maketitle
\tableofcontents
\newpage

\begin{description}

\item[foo]
  bar

\item[baz]

  bang

\end{description}
\begin{itemize}
\item erst
\item zweit
\item
  \begin{itemize}
  \item erst 
  \item zweit
  \item
    \begin{itemize}
    \item erst
    \item zweit
    \end{itemize}
  \end{itemize}
\end{itemize}
\begin{enumerate}
\item erst
\item zweit
\item
  \begin{enumerate}
  \item erst
  \item zweit
  \item
    \begin{enumerate}
    \item erst
    \item zweit
    \item
      \begin{enumerate}
      \item erst
      \item zweit
      \end{enumerate}
    \end{enumerate}
  \end{enumerate}
\end{enumerate}
\begin{itemize}
\item erst
\item
  \begin{enumerate}
  \item erst
  \item
    \begin{itemize}
    \item erst
    \item
      \begin{enumerate}
      \item zweit
      \item
        \begin{itemize}
        \item erst
        \item
          \begin{enumerate}
          \item erst
          \end{enumerate}
        \end{itemize}
      \end{enumerate}
    \end{itemize}
  \end{enumerate}
\end{itemize}

\begin{tabular}{|l|c|c|c|r|}
  a & \multicolumn{3}{c}{Multi-column}  & e \\
  \hline
  a & b & c & d & e \\
  a & b & c & d & e \\
  a & b & c & d & e
\end{tabular}

\begin{tabular}{*{6}{r}*{6}{l}}
  a &a &a &a&a&a&a&a&a&a&a&a \\
  \hline
  a &a &a &a&a&a&a&a&a&a&a&a \\
  a &a &a &a&a&a&a&a&a&a&a&a \\
\end{tabular}

\begin{tabular}{c c m{5cm}}
  a & a & There is an illegal character in the argument of an
          array or tabular environment, or in the second argument of a multicolumn command. \\
  a & a & There is an illegal character in the argument of an
          array or tabular environment, or in the second argument of a multicolumn command. \\
  a & a & There is an illegal character in the argument of an
          array or tabular environment, or in the second argument of a multicolumn command. \\
  a & a & There is an illegal character in the argument of an
          array or tabular environment, or in the second argument of a multicolumn command. \\
\end{tabular}

\blindtext

Beim Körperbau \textbf{unterscheiden} \textsc{sich} \emph{Eisbären}
von \textsf{anderen} \textit{Bärenarten} durch einen
\textbf{\textit{langen}} \textsl{\textit{Hals}} und
\textsf{\textit{einen}} relativ \texttt{\textit{kleinen}}, {\Huge
  flacheren} {\tiny Kopf}. Im {\LARGE Gegensatz} zu den nahe
verwandten Braunbären fehlt ihnen der Muskelberg am Nacken. Die Augen
sind verhältnismäßig klein. Die Ohrmuscheln sind nach vorn
aufgerichtet und rund geformt. Wie die meisten Bären besitzen Eisbären
42 Zähne, und wie alle Bären sind sie Sohlengänger. Ihre Vorderbeine
sind lang und kräftig; die großen Vordertatzen sind paddelförmig
ausgebildet und mit Schwimmhäuten versehen, was ein schnelles
Schwimmen ermöglicht. \textbf{\Huge Auf den muskulösen} Hinterbeinen
\texttt{\tiny können sich die Eisbären} zu \textit{\LARGE maximaler Höhe erheben} (etwa
bei \textsc{\huge Kämpfen oder}); die Hintertatzen
dienen beim Schwimmen als Steuerruder. Die Fußsohlen sind dicht
behaart, was dem Kälteschutz dient und auch das Ausrutschen auf dem
Eis verhindert. 

\section{Knut}
\blindtext

Beim Körperbau \textbf{unterscheiden} \textsc{sich} \emph{Eisbären}
von \textsf{anderen} \textit{Bärenarten} durch einen
\textbf{\textit{langen}} \textsl{\textit{Hals}} und
\textsf{\textit{einen}} relativ \texttt{\textit{kleinen}}, {\Huge
  flacheren} {\tiny Kopf}. Im {\LARGE Gegensatz} zu den nahe
verwandten Braunbären fehlt ihnen der Muskelberg am Nacken. Die Augen
sind verhältnismäßig klein. Die Ohrmuscheln sind nach vorn
aufgerichtet und rund geformt. Wie die meisten Bären besitzen Eisbären
42 Zähne, und wie alle Bären sind sie Sohlengänger. Ihre Vorderbeine
sind lang und kräftig; die großen Vordertatzen sind paddelförmig
ausgebildet und mit Schwimmhäuten versehen, was ein schnelles
Schwimmen ermöglicht. \textbf{\Huge Auf den muskulösen} Hinterbeinen
\texttt{\tiny können sich die Eisbären} zu \textit{\LARGE maximaler Höhe erheben} (etwa
bei \textsc{\huge Kämpfen oder}); die Hintertatzen
dienen beim Schwimmen als Steuerruder. Die Fußsohlen sind dicht
behaart, was dem Kälteschutz dient und auch das Ausrutschen auf dem
Eis verhindert. 
\subsection{Knuti}
\blindtext

Beim Körperbau \textbf{unterscheiden} \textsc{sich} \emph{Eisbären}
von \textsf{anderen} \textit{Bärenarten} durch einen
\textbf{\textit{langen}} \textsl{\textit{Hals}} und
\textsf{\textit{einen}} relativ \texttt{\textit{kleinen}}, {\Huge
  flacheren} {\tiny Kopf}. Im {\LARGE Gegensatz} zu den nahe
verwandten Braunbären fehlt ihnen der Muskelberg am Nacken. Die Augen
sind verhältnismäßig klein. Die Ohrmuscheln sind nach vorn
aufgerichtet und rund geformt. Wie die meisten Bären besitzen Eisbären
42 Zähne, und wie alle Bären sind sie Sohlengänger. Ihre Vorderbeine
sind lang und kräftig; die großen Vordertatzen sind paddelförmig
ausgebildet und mit Schwimmhäuten versehen, was ein schnelles
Schwimmen ermöglicht. \textbf{\Huge Auf den muskulösen} Hinterbeinen
\texttt{\tiny können sich die Eisbären} zu \textit{\LARGE maximaler Höhe erheben} (etwa
bei \textsc{\huge Kämpfen oder}); die Hintertatzen
dienen beim Schwimmen als Steuerruder. Die Fußsohlen sind dicht
behaart, was dem Kälteschutz dient und auch das Ausrutschen auf dem
Eis verhindert. 
\subsubsection{Knuut}
\blindtext

Beim Körperbau \textbf{unterscheiden} \textsc{sich} \emph{Eisbären}
von \textsf{anderen} \textit{Bärenarten} durch einen
\textbf{\textit{langen}} \textsl{\textit{Hals}} und
\textsf{\textit{einen}} relativ \texttt{\textit{kleinen}}, {\Huge
  flacheren} {\tiny Kopf}. Im {\LARGE Gegensatz} zu den nahe
verwandten Braunbären fehlt ihnen der Muskelberg am Nacken. Die Augen
sind verhältnismäßig klein. Die Ohrmuscheln sind nach vorn
aufgerichtet und rund geformt. Wie die meisten Bären besitzen Eisbären
42 Zähne, und wie alle Bären sind sie Sohlengänger. Ihre Vorderbeine
sind lang und kräftig; die großen Vordertatzen sind paddelförmig
ausgebildet und mit Schwimmhäuten versehen, was ein schnelles
Schwimmen ermöglicht. \textbf{\Huge Auf den muskulösen} Hinterbeinen
\texttt{\tiny können sich die Eisbären} zu \textit{\LARGE maximaler Höhe erheben} (etwa
bei \textsc{\huge Kämpfen oder}); die Hintertatzen
dienen beim Schwimmen als Steuerruder. Die Fußsohlen sind dicht
behaart, was dem Kälteschutz dient und auch das Ausrutschen auf dem
Eis verhindert. 
\end{document}
