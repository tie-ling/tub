% Datum 29.04.2024
% Vorname Yuchen
% Nachname Guo
% Matrikelnummer 480788
% E-Mail-Adresse gyuchen86@gmail.com

% Beginn der Präambel
\documentclass{article}
\usepackage[utf8]{inputenc}
\usepackage[T1]{fontenc}
\usepackage[ngerman]{babel}
\usepackage{array}
\usepackage{amsmath,amssymb}
\usepackage{accents}
\title{Bärenstarker Titel}
\author{Yuchen Guo, Eisbär}
\date{\today}
\newcommand{\R}{\mathbb{R}}
\newcommand{\N}{\mathbb{N}}
\newcommand{\Z}{\mathbb{Z}}
\newcommand{\C}{\mathbb{C}}
\newcommand{\p}{\partial}
\newcommand\borel[1]{\mathcal{#1}}
\newcommand\caratheo[2]{\left[#1,#2\right]}
\newcommand\bb[1]{\boldsymbol{#1}}

% Beginn der eigentlichen Text-Umgebung
\begin{document}
\maketitle
\tableofcontents
\(\N \subseteq \Z \subseteq \R \subseteq \C\). \(\borel{C}, \caratheo{a}{b}\).
\begin{itemize}
\item Es gilt
  \begin{align*}
    M_{1} &= \{(x, y) \in \mathbb{R}^{2} \mid |x| + |y| = 1\} \\
    \mathring{M}_{1} & = \emptyset \\
    \partial M_{1} &= \{(x, y) \in \mathbb{R}^{2} \mid |x| + |y| = 1\} = M_{1}
  \end{align*}
\item Mit Hilfe der Rechenregeln für Summen (Linearität) erhalten wir:
  \begin{align*}
    \nabla(f+g)(x)
    &=
      \begin{pmatrix}
        \partial_{x_{1}}(f(x)+g(x)) \\
        \ldots \\
        \partial_{x_{n}}(f(x)+g(x))
      \end{pmatrix} \\
    &=    \begin{pmatrix}
      \partial_{x_{1}}f(x)+\partial_{x_{1}}g(x) \\
      \ldots \\
      \partial_{x_{n}}f(x)+\partial_{x_{n}}g(x)
    \end{pmatrix} \\
    &= \nabla f(x) + \nabla g(x)
  \end{align*}
\item Wir betrachten das folgende lineare Gleichungssystem:
  \[
    \begin{pmatrix}
      1 & 1 & 1 \\ 2 & -2 & -1 \\ 4 & 4 & 1
    \end{pmatrix}
    \begin{pmatrix}
      c_{1} \\ c_{2} \\ c_{3}
    \end{pmatrix}
    =
    \begin{pmatrix}
      2 \\ -1 \\ 0
    \end{pmatrix}
  \]
  Wir lösen es mit Hilfe des Gauss-Algorithmus:
  \begin{align*}
    \begin{matrix}
      1 & 1 & 1 & | & 2 \\
      2 & -2 & -1 & | & -1 \\
      4 & 4 & 1 & | & 0
    \end{matrix}
    \rightsquigarrow
    &\begin{matrix}
      1 & 1 & 1 & | & 2 \\
      2 & -2 & -1 & | & -1 \\
      0 & 0& -3 & | & -8
    \end{matrix} \\
    \rightsquigarrow&
                      \begin{matrix}
                        1 & 1 & 1 & | & 2 \\
                        2 & -2 & -1 & | & -1 \\
                        0 & 0 & -3 & | & -8
                      \end{matrix} \\
    \rightsquigarrow&
                      \begin{matrix}
                        1 & 1 & 1 & | & 2 \\
                        0 & -4 & -3 & | & -5 \\
                        0 & 0 & -3 & | & -8
                      \end{matrix}
  \end{align*}
\end{itemize}
\begin{enumerate}
\item Die Funktion
  \[f\colon \mathbb{R}^{2} \to \mathbb{R}, \quad f(x, y) =
    \begin{cases}
      x \sin\left(\frac1y\right) + y\cos\left(\frac1x\right)
      & \text{falls \(x \ne 0\) und \(y \ne 0\)} \\
      0 & \text{falls \(x = 0\) und \(y = 0\)}
    \end{cases}
  \]
  ist stetig auf \(\mathbb{R}^{2}\) außer auf \(\{(x, y) \in \mathbb{R}^{2} \mid x = 0 \text{
    oder } y = 0 \} \setminus \{(0, 0)\}\).
\item Mit Hilfe des \texttt{\textbackslash{}cfrac}-Befehls setzen wir einen schönen
  Kettenbruch:
  \begin{align*}
    \frac4\pi = 1 + \cfrac{1^{2}}{3+
    \cfrac{2^{2}}{5+
    \cfrac{3^{2}}{7+
    \cfrac{4^{2}}{
    9 + \cfrac{5^{2}}{\ddots}
    }}}}
  \end{align*}
\item \texttt{split}-Umgebung.  Die folgende Rechnung bestätigt unsere
  Formel für den Laplace-Operator in Polarkoordinaten.
  \begin{align*}
          \p^{2}_{r}g + \frac1r \cdot \p_{r}g + \frac{1}{r^{2}} \cdot \p^{2}_{\varphi}g
      &= \cos(\varphi)^{2} \p_{x}^{2}f + 2 \cos(\varphi)\sin(\varphi)\p_{x}\p_{y} f \\
      &\quad + \sin(\varphi)^{2}\p_{y}^{2}f + \frac1r(\cos(\varphi)\p_{x}f +
        \sin(\varphi)\p_{y}f) \\
      &\quad + \frac{1}{r^{2}}(-r(\cos(\varphi)\partial_{x}f + \sin(\varphi)\p_{y}f)) \\
      &\quad + r^{2}(\sin(\varphi)^{2}\p_{x}^{2}f + \cos(\varphi)^{2}\p_{y}^{2}f) \\
      &\quad - 2r^{2}\sin(\varphi)\cos(\varphi)\p_{x}\p_{y}f) \\
      &= \p_{x}^{2}f(\cos(\varphi)^{2} + \sin(\varphi)^{2}) \tag{1}\\
      &\quad + \p_{y}^{2}f(\sin(\varphi)^{2} + \cos(\varphi)^{2}) \\
      &\quad + \p_{x}\p_{y}f(2 \cos(\varphi)\sin(\varphi) - 2 \sin(\varphi)\cos(\varphi)) \\
      &\quad + \p_{x}f(\frac1r \cos(\varphi) - \frac1r \cos(\varphi)) \\
      &\quad + \p_{y}f(\frac1r \sin(\varphi) - \frac1r \sin(\varphi)) \\
      &= \p_{x}^{2}f + \p_{y}^{2} f
  \end{align*}
\item Wir betrachten die Maxwellschen Gleichungen, wobei wir
  \(\bb{B}, \bb{E}, \bb{j}\) \textit{fett} setzen, statt sie mit
  Vektorpfeilen zu versehen, wie es in der Literatur auch vorkommt.
  De Gleichungen sollen zentriert gesetzt werden, also nicht nach dem
  Gleichheitszeichen ausgerichtet werden.
  \begin{gather*}
    \nabla \cdot \bb{E} = \frac{\rho}{\varepsilon_{0}} \\
    \nabla \cdot \bb{B} = 0 \\
    \nabla \times \bb{E} = -\frac{\partial B}{\partial t} \\
    \nabla \times \bb{B} = \mu_{0} \bb{j} + \mu_{0} \varepsilon_{0} \frac{\p \bb{E}}{\p t}    
  \end{gather*}
\end{enumerate}
\end{document}
