% 使用 chktex 检查 tex 文件中的语法错误
% settings for chktex
% chktex-file 3

\documentclass[fleqn,draft,a5paper]{article}

% 隐藏默认的章节序号:实际作业中与这个冲突
% https://tex.stackexchange.com/a/30225
\setcounter{secnumdepth}{0}

% 设置页边距,上下左右
\usepackage{geometry}
\geometry{left=1cm,
 right=1cm,
 top=1cm,
 bottom=2cm}

% (TeX-run-style-hooks "amsmath,mathtools,fontspec" "amsthm")

% 让 TeX 支持德语
\usepackage[ngerman]{babel}
\usepackage{amsmath,mathtools,fontspec,amssymb,amsthm,parskip,interval}

% 根据 AMS 建议,应为成对符号(比如绝对值)定义新命令
\providecommand{\skp}[1]{\langle#1\rangle}
\providecommand{\abs}[1]{\left\lvert#1\right\rvert}
\providecommand{\norm}[1]{\left\lVert#1\right\rVert}
\providecommand{\ceil}[1]{\left\lceil#1\right\rceil}
\providecommand{\floor}[1]{\left\lfloor#1\right\rfloor}

% 定理定义,依赖于 amsthm
\theoremstyle{remark}
\newtheorem*{Behauptung}{Behauptung}
\newtheorem*{Lemma}{Lemma}
\newtheorem*{Satz}{Satz}
\newtheorem*{Definition}{Definition}

% 定义新函数,依赖于AMSmath
\DeclareMathOperator{\card}{card}
\DeclareMathOperator{\Rang}{Rang}

\providecommand{\R}[1]{\mathrm{R#1}}

\newcommand{\wt}{\widetilde}
\newcommand{\dd}{\,\mathrm{d}}

% 标题与作者
\title{HA 2, LinA 2, Papp, Papp5}
\author{Shilong Yu 478912, Yuchen Guo 480788}

\begin{document}
\maketitle
\newpage
\subsection{1. Aufgabe}
\begin{enumerate}
\item
  \[
    A = \begin{bmatrix}
      1 & 2 & 3 & 4 \\ 2 & 4 & 6 & 8 \\ 1 & 2 & 3 & 4 \\ 1& 2 & 3 & 5
    \end{bmatrix}
  \]
  dann sind die erste Zeile und die dritte Zeile gleich. Daraus folgt,
  dass \(\Rang A < 4\) und wegen D10 gilt \(\det A = 0\).
\item \[
    B = \begin{bmatrix}
      1 & 2 & 3 \\ 0 & 3 & 4 \\ 1 & 2 & 5
    \end{bmatrix}
    \stackrel{\text{D7}}{\longrightarrow}
    \begin{bmatrix}
      1 & 2 & 3 \\ 0 & 3 & 4 \\ 0 & 0 & 2
    \end{bmatrix}
  \]
  wegen D8 folgt \(\det B = 1 \cdot 3 \cdot 2 = 6\).
\item
  \[C =
    \begin{bmatrix}
      1 & 0 & 0 \\ 0 & 3 & 2 \\ 0 & 2 & 3
    \end{bmatrix}
  \]
  wegen D9 folgt \(\det C = 1 \cdot (3 \cdot 3 - 2 \cdot 2) = 5\).
\end{enumerate}
\subsection{2. Aufgabe}
Wir definieren zwei Transpositionen
\[\tau_{1}(1) = 2, \; \tau_{1}(2) = 1; \quad \tau_{2}(1) = 4, \; \tau_{2}(4) = 1.\]
Dann ist die Permutation
\[\sigma =
  \begin{bmatrix}
    1 & 2 & 3 & 4 & 5 \\ 2 & 4 & 3 & 1 & 5
  \end{bmatrix}
  = \tau_{2} \circ \tau_{1}.
\]
\subsection{3. Aufgabe}
Wir berechnen die Determinante von
\[
  A =
  \begin{bmatrix}
    x & 1 & 1 \\ 1 & x & 1 \\ 1 & 1 & x
  \end{bmatrix}
\]
durch die Regel von \textsc{Sarrus},
\[\det A = x^{3} - 3x + 2.\]
Die Matrix \(A\) ist genau dann nicht invertierbar, falls \(\det A = 0\).
Daher berechnen die Nullstellen von \(\det A\), erhalten wir
\[\det A = (x+2)(x-1)^{2}.\]
Deshalb ist die Matrix \(A\) genau dann nicht invertierbar, falls \(x=-2\)
oder \(x=1\) gilt.

Wir berechnen die Determinante von
\[
  B =
  \begin{bmatrix}
    1 & x & x  \\ x & 1 & x \\ x & x & 1
  \end{bmatrix}
\]
durch die Regel von \textsc{Sarrus},
\[\det B = 2x^{3} - 3x^{2} + 1 = (2x+1)(x-1)^{2}.\]
Siehe oben. Deshalb ist die Matrix \(B\) genau dann nicht invertierbar, falls \(x=-1/2\)
oder \(x=1\) gilt.

\subsection{4. Aufgabe}

\begin{Behauptung}
  Ein Dreieck im \(\mathbb{R}^{2}\) sei gegeben durch die Eckpunkte \(u = (u_{1},
  u_{2}), v = (v_{1}, v_{2})\) und \(w = (w_{1}, w_{2})\).  Dann ist die
  Fläche des Dreiecks gleich
  \[\frac{1}{2}\abs{\det
      \begin{bmatrix}
        1 & u_{1} & u_{2} \\ 1 & v_{1} & v_{2} \\ 1 & w_{1} & w_{2}
      \end{bmatrix}
    }.\]
\end{Behauptung}
\begin{proof}
  Wir betrachten \(u,v,w\) als Punkte im komplexen Ebene.  Wir
  definieren die zwei Vektoren
  \[\overrightarrow{uv}=z_{1}=(v_{1}-u_{1})+ i (v_{2}-u_{2}), \quad
    \overrightarrow{uw} = z_{2} = (w_{1}-u_{1})+i (w_{2}-u_{2}).\]
  Die Fläche des Dreiecks ist genau Halbe von die Fläche des von
  \(z_{1},z_{2}\) aufgespannte Parallelograms.  Daraus folgt, dass
  \begin{align*}
    \frac{1}{2}z_{1}z_{2}
    &= \frac{1}{2}\abs{\mathfrak{Im}((v_{1}-u_{1})-i(v_{2}-u_{2}))((w_{1}-u_{1})-i(w_{2}-u_{2}))}  \\
    &=
      \frac{1}{2}\abs{(v_{1}-u_{1})(v_{2}-u_{2})-(w_{1}-u_{1})(w_{2}-u_{2})}
    \\
    &=\frac{1}{2}\abs{\det
      \begin{bmatrix}
        1 & u_{1} & u_{2} \\ 1 & v_{1} & v_{2} \\ 1 & w_{1} & w_{2}
      \end{bmatrix}
    }.
  \end{align*}
\end{proof}
\end{document}

