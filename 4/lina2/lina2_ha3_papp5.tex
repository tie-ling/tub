% 使用 chktex 检查 tex 文件中的语法错误
% settings for chktex
% chktex-file 3

\documentclass[fleqn,draft,a5paper]{article}

% 隐藏默认的章节序号:实际作业中与这个冲突
% https://tex.stackexchange.com/a/30225
\setcounter{secnumdepth}{0}

% 设置页边距,上下左右
\usepackage{geometry}
\geometry{left=1cm,
  right=1cm,
  top=1cm,
  bottom=2cm}

% (TeX-run-style-hooks "amsmath,mathtools,fontspec" "amsthm")

% 让 TeX 支持德语
\usepackage[ngerman]{babel}
\usepackage{amsmath,mathtools,fontspec,amssymb,amsthm,parskip,interval}

% 根据 AMS 建议,应为成对符号(比如绝对值)定义新命令
\providecommand{\skp}[1]{\langle#1\rangle}
\providecommand{\abs}[1]{\left\lvert#1\right\rvert}
\providecommand{\norm}[1]{\left\lVert#1\right\rVert}
\providecommand{\ceil}[1]{\left\lceil#1\right\rceil}
\providecommand{\floor}[1]{\left\lfloor#1\right\rfloor}

% 定理定义,依赖于 amsthm
\theoremstyle{remark}
\newtheorem*{Behauptung}{Behauptung}
\newtheorem*{Lemma}{Lemma}
\newtheorem*{Satz}{Satz}
\newtheorem*{Definition}{Definition}

% 定义新函数,依赖于AMSmath
\DeclareMathOperator{\card}{card}
\DeclareMathOperator{\Rang}{Rang}

\providecommand{\R}[1]{\mathrm{R#1}}

\newcommand{\wt}{\widetilde}
\newcommand{\dd}{\,\mathrm{d}}
\title{HA 3, LinA 2, Papp, Papp5}
\author{Shilong Yu 478912, Yuchen Guo 480788}

\begin{document}
\maketitle


\subsection{Aufgabe 1}

Wir berechnen die eindeutige Lösung des linearen Gleichungssystems $A
x = b$ mittels der Cramerschen Regel.
\begin{align*}
  A =
\begin{bmatrix}
   1 & 2 & 2 \\   3 & 1 & 0 \\  5 & 0 & 0 \\
 \end{bmatrix}, \quad
 b =
\begin{bmatrix}
  5 \\ 4 \\ 0 \\
\end{bmatrix}.
\end{align*}
\begin{proof}
  Zuerst berechnen wir die Determinante von $A$ durch Regel von
  Sarrus:
  \begin{align*}
    \det A = 1 \cdot 1 \cdot 0 + 2 \cdot 0 \cdot 5 + 2 \cdot 3 \cdot 0 - 2 \cdot 1 \cdot 5 - 0 - 0 = -10.
  \end{align*}
  Wir bezeichnen die Spaltenvektoren der Matrix $A$ als $a^{1}, a^{2},
  a^{3}$, dann folgt aus der Cramerschen Regel dass
  \begin{align*}
       x_{1}  & = \frac{\det(b,a^{2},a^{3})}{\det A} = \frac{0}{-10}
      = 0 \\
       x_{2} & = \frac{\det(a^{1},b,a^{3})}{\det A} = \frac{-40}{-10}
      = 4 \\
       x_{3} & = \frac{\det{(a^{1},a^{2},b)}}{\det A}= \frac{15}{-10}
      = -\frac{3}{2} \\
  \end{align*}
\end{proof}

\subsection{Aufgabe 2}
  Sei $A \in M(n \times n, \mathbb{Z})$.  Es existiert ein Inverse von $A$ mit ganzzählige
  Einträge $A^{-1} \in M(n\times n, \mathbb{Z})$, genau dann, wenn
  \begin{itemize}
   \item   $\det A \ne 0$
  (Existenz des Inverses wegen D10) und,
  \item      $\det A_{ij}'$ für alle $1 \le i,
  j \le n$ durch $\det A$ teilbar ist, wobei $A_{ij}'$ die Matrix ist,
  die man durch Weglassen der $i$-ten Zeile und der $j$-ten Spalte aus
  $A$ erhält.
  Denn wegen Satz 4.3.3 gilt für $A \in GL(n; K)$ und $C =
  (c_{ij})$ mit $c_{ij} := (-1)^{i+j} \det A_{ij}'$ dass
$ A^{-1} = \frac{1}{\det A} C^{\top}$.
\end{itemize}
\subsection{Aufgabe 3}
  Seien $P :=
  \begin{bmatrix}
     A & B \\ C & D \\
  \end{bmatrix} \in M(n \times n; \mathbb{R})
  $ und $ D \in M(m \times m; \mathbb{R})$, wobei $1 \le m < n$.
  \begin{Behauptung}
    Es gilt
    \begin{align*}
      \det P = \det (A - BD^{-1}C) \det D.
    \end{align*}
  \end{Behauptung}
  \begin{proof}
    Wir betrachten die untere Dreiecksmatrix
    \begin{align*}
      Q =
      \begin{bmatrix}
         E_{n-m} & 0 \\
         - D^{-1}C & E_{m} \\
      \end{bmatrix} .
    \end{align*}
    Dann gilt wegen D9 dass $\det Q = 1$.  Außerdem gilt wegen
    Determinanten Multiplikationssatz D11, dass
    \begin{align*}
       \det P & = \det P \cdot \det Q = \det (P \cdot Q) = \det
      \begin{bmatrix}
         A & B \\  C & D \\
      \end{bmatrix} \cdot
      \begin{bmatrix}
         E_{n-m} & 0 \\
         - D^{-1}C & E_{m} \\
      \end{bmatrix} \\
       &= \det
      \begin{bmatrix}
         A - BD^{-1}C & 0 \\  0 & D \\
      \end{bmatrix} = \det (A - BD^{-1}C) \det D
    \end{align*}
    wie behauptet.
  \end{proof}

\subsection{Aufgabe 4}
  \begin{Behauptung}
    Für beliebige $a, \ldots, z \in \mathbb{R}$ ist die Matrix
    \begin{align*}
      A =
      \begin{bmatrix}
        0 & 0 & 0 & a & b \\
        0 & 0 & 0 & c & d \\
        0 & 0 & 0 & e & f \\
        p & q & r & s & t \\
        v & w & x & y & z \\
      \end{bmatrix}
    \end{align*}
    nicht invertierbar.
  \end{Behauptung}
  \begin{proof}
    Die Matrix $A$ ist genau dann invertierbar, falls $\dim A \ne 0$
    gilt.  Diese Bedingung kann niemals erfüllt werden, denn
    betrachten wir die untere Dreiecksmatrix $A$ mit
    \begin{align*}
            A =
      \begin{bmatrix}
        0 & 0 & 0 & \mid & a & b \\
        0 & 0 & 0 & \mid & c & d \\
        - & - & - & \mid & - & - \\
        0 & 0 & 0 & \mid & e & f \\
        p & q & r & \mid & s & t \\
        v & w & x & \mid & y & z \\
      \end{bmatrix}
      =
      \begin{bmatrix}
        0 & M \\
        N & P \\
      \end{bmatrix}
    \end{align*}
    wobei aus Regel von Sarrus immer $\det N = 0$  folgt.   Durch
    endlich viele Zeilenvertauschungen erhalten wir
    \begin{align*}
      \det A' = \det
      \begin{bmatrix}
        N & P \\
        0 & M \\
      \end{bmatrix}
      = (-1)^{i} \det A
    \end{align*}
    wobei wegen $\det N = 0$ und D9 $\det A' = 0$ für alle $a, \ldots, z \in
    \mathbb{R}$ gilt.  Daraus folgt die Behauptung, dass $\det A = 0$ und $A$
    nicht invertierbar.
  \end{proof}
\end{document}
