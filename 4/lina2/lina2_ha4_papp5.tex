% 使用 chktex 检查 tex 文件中的语法错误
% settings for chktex
% chktex-file 3

\documentclass[fleqn,draft,a5paper,12pt]{article}

% 隐藏默认的章节序号:实际作业中与这个冲突
% https://tex.stackexchange.com/a/30225
\setcounter{secnumdepth}{0}

% 设置页边距,上下左右
\usepackage{geometry}
\geometry{left=1cm,
  right=1cm,
  top=1cm,
  bottom=2cm}

% (TeX-run-style-hooks "amsmath,mathtools,fontspec" "amsthm")

% 让 TeX 支持德语
\usepackage[ngerman]{babel}
\usepackage{amsmath,unicode-math,amsthm,parskip,interval}
\setmainfont{TeX Gyre Termes}
\setmathfont{TeX Gyre Termes Math}
\setmathfont[range=\mathcal]{latinmodern-math.otf}

% 根据 AMS 建议,应为成对符号(比如绝对值)定义新命令
\providecommand{\skp}[1]{\langle#1\rangle}
\providecommand{\abs}[1]{\left\lvert#1\right\rvert}
\providecommand{\norm}[1]{\left\lVert#1\right\rVert}
\providecommand{\ceil}[1]{\left\lceil#1\right\rceil}
\providecommand{\floor}[1]{\left\lfloor#1\right\rfloor}

% 定理定义,依赖于 amsthm
\theoremstyle{remark}
\newtheorem*{Behauptung}{Behauptung}
\newtheorem*{Lemma}{Lemma}
\newtheorem*{Satz}{Satz}
\newtheorem*{Definition}{Definition}

% 定义新函数,依赖于AMSmath
\DeclareMathOperator{\card}{card}
\DeclareMathOperator{\Rang}{Rang}

\providecommand{\R}[1]{\mathrm{R#1}}

\newcommand{\wt}{\widetilde}
\newcommand{\dd}{\,\mathrm{d}}
\title{HA 4, LinA 2, Papp, Papp5}
\author{Shilong Yu 478912, Yuchen Guo 480788}

\begin{document}
\maketitle


\subsection{Aufgabe 1}

\begin{Behauptung}
  Jede invertierbare Matrix \(A \in \mathrm{GL}(n; K)\) kann durch
  Spaltenumformungen vom Typ III (Addition einer \(\mu\)-fachen \(i\)-ten
  Spalte zu einer \(j\)-ten Spalte mit \(j \ne i\)) in eine Diagonalmatrix
  \[
    D =
    \begin{bmatrix}
      \lambda_{1} & & 0 \\
            &  \ddots & \\
      0 & & \lambda_{n}
    \end{bmatrix}
  \]
  überführt werden.
\end{Behauptung}
\begin{proof}
  Wir wenden das Eliminationsverfahren von Gauss an und bezeichnen die
  Spalten von \(A\) als
  \[ A =
    \begin{bmatrix}
      a_{1} & a_{2} & \cdots & a_{n}
    \end{bmatrix}.
  \]
  Weil \(A\) invertierbar ist, existiert mindestens eine Spalte \(a_{i}\),
  deren ersten Eintrag ungleich Null ist.  Wir ersetzen \(a_{1}\) mit
  \(a_{1}+a_{i}\):
  \[ A' =
    \begin{bmatrix}
      a_{1}+a_{i} & a_{2} & \cdots & a_{n}
    \end{bmatrix}
  \]
  um sicher zu stellen, dass der erste Eintrag der ersten Spalte
  \(A'_{1,1}\) ungleich Null ist.  Nun addieren zu alle andere Spalte
  \(j \ne 1\) die \(-\frac{A_{1,j}}{A_{1,1}}\)-fache \(A'_{\cdot,1}\) Spalte,
  erhalten wir
  \[ A'' =
    \begin{bmatrix}
      a_{1} + a_{i} & 0 & \cdots & 0 \\
      \vdots & a''_{2} & \cdots & a''_{n}
    \end{bmatrix}
  \]
  und so weiter.  Schließlich erhalten wir
  \[
    \begin{bmatrix}
      \lambda_{1} & 0 & 0 \\
      m      &  \ddots & 0 \\
      p & q & \lambda_{n}
    \end{bmatrix}.   
  \]
  Lemma.  Wir bemerken, dass alls \(\lambda_{i}\) ungleich Null ist, denn andersfalls
  ist
  \[
    \begin{bmatrix}
      \lambda_{1} & 0 & 0 & 0 \\
      m & \lambda_{2} & 0 & 0 \\
      n & p & [0] & 0 \\
      q & r & s & \lambda_{4}
    \end{bmatrix}
  \]
  und die Matrix ist im Widerspruch zur Voraussetzung nicht
  invertierbar.  Siehe HA3.

  Im letzten Schritt addieren wir, von rechts nach links, von unten
  nach oben, letzte Spalte zu alle andere nicht-diagonale Einträge, um
  solche Einträge zu beseitigen:
  \[
    \begin{bmatrix}
      \lambda_{1} & 0 & 0 & 0  & \\
      m & \lambda_{2} & 0 & 0 & \uparrow \\
      n & p & \lambda_{3} & 0 & \vdots \\
      q & 0 & 0 & \lambda_{4} & \\
      & \leftarrow & \cdots  & & 
    \end{bmatrix}
  \]
  Dieser Schritt ist möglich, denn \(\lambda_{i}, 1 \le i \le n\) sind ungleich
  Null.  Daraus folgt die Behauptung.
\end{proof}
\subsection{Aufgabe 2}
\begin{Behauptung}
  Je zwei Matrizen \(A, B \in M(m \times n; \mathbb{R}), m, n \in \mathbb{N}_{>0}\) sind durch
  einen stetige Kurve verbindbar.
\end{Behauptung}
\begin{proof}
  Wir definieren die Funktion
  \[ f\colon \interval{0}{1} \to M(m \times n; \mathbb{R}), \quad t \mapsto (1-t)A + tB. \]
  Aus Definition folgt, dass \(f(0) = A\) und \(f(1)=B\).

  Offensichtlich ist für alle \(t \in \interval{0}{1}\) der Funktionswert
  \(f(t)\) eine Matrix wegen Skalarmultiplikation und Matrixaddition.
 Die Funktion  \(f\)  ist stetig, denn \(f\) ist eine Komposition von
 elementare Funktionen.
\end{proof}
\subsection{Aufgabe 3}
Sei \(K\) Körper, \(V\) ein \(K\)-Vektorraum und \(f \in \mathrm{End}(V)\).
\begin{Behauptung}
  Falsch: Ist \(0\) im Kern von \(f\), so ist \(f\) injektiv.
\end{Behauptung}
\begin{proof}
  Angenommen, \(\dim V = 2\) und \(e_{1}, e_{2}\) sei die kanonische
  Basis.  Wir betrachten die Funktion \(f\) mit \(f(e_{1})=e_{2}\) und
  \(f(e_{2}) = 0\).  Dann ist \(f(0) = f(e_{2}) = 0\) und daher nicht injektiv.
\end{proof}
\begin{Behauptung}
  Falsch: Ist \(f \circ f\) die Nullabbildung, so ist auch \(f\) die Nullabbildung.
\end{Behauptung}
\begin{proof}
  Angenommen, \(\dim V = 2\) und \(e_{1}, e_{2}\) sei die kanonische
  Basis. Wir betrachten die Funktion \(f\) mit \(f(e_{1})=e_{2}\) und
  \(f(e_{2}) = 0\) mit der Darstellende Matrix
  \[
    M_{\mathcal{B}}(f) =
    \begin{bmatrix}
      0 & 0 \\ 1 & 0
    \end{bmatrix}.
  \]
  Es gilt \(f \circ f = 0\) aber \(f \ne 0\).
\end{proof}
\subsection{Aufgabe 4}
\begin{Behauptung}
  \(A, B \in \mathrm{GL}(2, \mathbb{R})\) mit \(A =
  \begin{bmatrix}
    1 & 0 \\ 0 & 1
  \end{bmatrix}
  \)
  und
  \(B =
  \begin{bmatrix}
    -1 & 0 \\ 0 & 1
  \end{bmatrix}
  \)
  können nicht durch eine stetige Kurve verbinden werden.
\end{Behauptung}
\begin{proof}
  Wegen Satz am Seite 234, sind \(A, B \in \mathrm{GL}(2, \mathbb{R})\) genau dann
  verbindbar, falls \(\det (A \cdot B) >0\).  Daher berechnen wir die
  Determinanten von beiden:
  \[\det A = 1 \cdot 1 - 0 \cdot 0 = 1, \quad \det B = -1 \cdot 1 - 0 \cdot 0 = -1.\]
  Die Bedingung \(\det (A \cdot B) > 0 \) ist nicht erfüllt.  Damit folgt
  die Behauptung.
\end{proof}
\end{document}
