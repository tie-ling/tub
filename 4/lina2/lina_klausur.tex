% 使用 chktex 检查 tex 文件中的语法错误
% settings for chktex
% chktex-file 3

\documentclass[fleqn,draft,a5paper]{article}

% 让每个章节 subsection 在新的一页上开始
% 而不是紧接着上一章节
\AddToHook{cmd/subsection/before}{\clearpage}

% 隐藏默认的章节序号:实际作业中与这个冲突
% https://tex.stackexchange.com/a/30225
\setcounter{secnumdepth}{0}

% 设置页边距,上下左右
\usepackage{geometry}
\geometry{left=1cm,
 right=1cm,
 top=1cm,
 bottom=2cm}

% 让 TeX 支持德语
\usepackage[ngerman]{babel}
\usepackage{unicode-math,amsthm,parskip,interval}
\setmainfont{TeX Gyre Termes}
\setmathfont{TeX Gyre Termes Math}
\setmathfont[range=\mathcal]{latinmodern-math.otf}

\DeclareMathOperator{\Span}{span}
\DeclareMathOperator{\M}{M}

\providecommand{\abs}[1]{\left\lvert#1\right\rvert}
% 标题与作者
\title{Lineare Algebra, Klausurvorbreitung}
\author{me}

\begin{document}
\subsection{Lineare Gleichungssysteme}
Die grundlegenden Operationen in der Linearen Algebra sind die
Addition von Vektoren und die Multiplikation mit Skalaren.

Das lineare Gleichungssystem kann man in der Form \(A \cdot x = b\)
schreiben.

Wir berechnen die Lösungsmenge
mittels das Eliminationsverfahren von Gauss, indem wir die erweiterte
Koeffizientenmatrix  durch elementaren
Zeilenumformungenin in der Zeilenstufenform bringt.

Die elementaren Zeilenumformungen: (1) Vertauschung von zwei Zeilen;
(2)  Addition der \(\lambda\)-fachen \(i\)-ten Zeilen zur \(k\)-ten Zeilen, wobei
\(0 \ne \lambda \in \mathbb{R}\) und \(i \ne k\).

Satz.  Elementaren Zeilenumformungen ändert die Lösungsmenge eines
linearen Gleichungssystems nicht.

Satz.  Jede Matrix \(A\) kann man durch elementare Zeilenumformungen in
eine Matrix \(\tilde{A}\) in Zeilenstufenform überführen.

Satz.  Die Parametrisierung von freien Variabeln nach Lösung des
linearen Gleichungssystems
ist bijektiv.
\subsection{Grundbegriffe}
\subsubsection{Gruppen, Ringe und Körper}
Definition.  Eine Menge \(G\) zusammen mit einer Verknüpfung \(\ast\) heißt
Gruppe, falls (G1) assoziativ; (G2a) existiert ein neutrales Element
\(e \in G\) sodass für alle \(a \in G\) gilt \(e \ast a = a\); (G2b) existiert ein
inverses Element \(a' \in G\) zu jedem \(a \in G\) sodass \(a' \ast a = e\); (G3)
falls kommutativ, dann abelsch.

Definition.  Eine Menge \(R\) mit zwei Verknüpfungen \(+\) und \(\cdot\) heißt
Ring, falls (R1) \(R\) mit \(+\) ist eine abelsche Gruppe; (R2) Die
Multiplikation ist assoziativ; (R3) distributiv, also für alle \(a,b,c
\in R\) gilt \(a(b+c)=ab+ac\) und \((a+b)c=ac+bc\); (R4) kommutativ, falls
multiplikation kommutativ ist. Einselement ist das neutrales Element
bzgl. Multiplikation.  Nullelement ist das neutrales Element
bzgl. Addition.

Definition.  Nullteilerfreier Ring ist ein Ring, sodass aus \(ab=0\)
stets \(a=0\) oder \(b=0\) folgt.

Definition.  Eine Menge \(K\) mit zwei Verknüpfungen \(+\) und \(\cdot\) heißt
Körper, wenn Folgendes gilt: (K1) \(K\) mit der Addition ist eine
abelsche Gruppe; (K2) \(K^{\ast} \coloneq K \setminus \{0\}\) mit der Multiplikation ist
eine abelsche Gruppe; (K3) Distributivgesetze, \(a(b+c)=ab+ac\) und
\((a+b)c=ac + bc\).

Definition.  Der Grad von Polynom \(f \in K[t]\) ist definiert durch
\[
  f = a_{0} + a_{1}t + \ldots + a_{n}t^{n}, \quad
  \deg f \coloneq
  \begin{cases}
    -\infty, & f = 0; \\
    \max\{\nu \in \mathbb{N}\colon a_{\nu} \ne 0\} & \text{sonst.}
  \end{cases}
\]

Satz.  Ist \(K\) ein Körper, so ist die Menge \(K[t]\) der Polynome über
\(K\) zusammen mit den oben definierten Verknüpfungen ein kommutativer
Ring ohne Nullteiler.  Weiter gilt für \(f,g \in K[t]\) dass \(\deg (f g) =
\deg f + \deg g.\)

Satz.  Teilung mit Rest.  Sind \(f,g \in K[t]\) und \(g \ne 0\), so gibt es
dazu eindeutig bestimmte Polynome \(q,r \in K[t]\) derart, dass
\[f=qg+r \text{ und } \deg r < \deg g.\]

Definition.  Integritätsring \(R\) ist ein Ring, der kommutativ mit
Einselement und nullteilerfrei ist.

Definition.  Die Vielfachheit der Nullstelle \(\lambda\) von \(0 \ne f \in K[t]\)
ist definiert durch \[\mu(f; \lambda) \coloneq \max \{r \in \mathbb{N}\colon f = (t - \lambda)^{r} \cdot g
  \text{ mit } g \in K[t]\}\]

Fundamentalsatz der Algebra.  Jedes Polynom \(f \in \mathbb{C}[t]\) mit \(\deg f >
0\) hat mindestens eine Nullstelle in \(\mathbb{C}\).

Definition.  Ein Körper \(K\) heißt algebraisch abgeschlossen, wenn
jedes Polynom \(f \in K[t]\) in Linearfaktoren zerfällt.

Fundamentalsatz der Algebra.  Der Körper \(\mathbb{C}\) ist algebraisch abgeschlossen.

\subsubsection{Vektorräume}

Definition.  Vektorraum.  Sei \(K\) ein Körper.  Eine Menge \(V\) zusammen
mit einer inneren Verknüpfung (Addition von Vektoren) \(+\) und einer
äußeren Verknüpfung (Multiplikation mit Skalaren) heißt
\(K\)-Vektorraum, falls (V1) \(V\) bzgl. Vektoraddition ist eine abelsche
Gruppe; (V2) für Skalaren \(\lambda, \mu \in K\) und Vektoren \(v, w \in V\) gilt
\begin{align*}
  &(\lambda + \mu)v=\lambda v + \mu v, && \lambda(v+w) = \lambda v+ \lambda w, \\
  &\lambda(\mu v) = (\lambda \mu) v, && 1v = v.
\end{align*}

Definition Untervektorraum.  Sei \(V\) ein \(K\)-Vektorraum, \(W \subseteq V\) heißt
Untervektorraum von \(V\) falls gilt (UV1) \(W \ne \emptyset\); (UV2) abgeschlossen
bzgl. Addition: für alle \(v, w
\in W\) gilt \(v+w \in W\); (UV3) abgeschlossen bzgl. Multiplikation mit
Skalaren: für alle \(\lambda \in K\) und \(v \in W\) gilt \(\lambda v \in W\).

Satz.  Durchschnitt von Untervektorräume ist wieder ein
Untervektorraum.

Definition.  Von der Familie \((v_{i})_{i \in I}\) aufgespannten
Vektorraum ist  die Menge, die alle Linearkombination von der Familie
enthält, oder Durchschnitt von alle Untervektorräume, die die Familie
enthält.

Definition.  Eine endliche Familie \((v_{1}, \ldots, v_{r})\) von Vektoren
aus \(V\) heißt linear unabhängig, falls sich der Nullvektor nur trivial
aus den \(v_{1}, \ldots, v_{r}\) linear kombinieren lässt.

\subsubsection{Basis und Dimension}

Definition.  Eine Familie \(\mathcal{B} = (v_{i})_{i \in I}\) in einem Vektorraum
\(V\) heißt Erzeugendensystem von \(V\), falls
\(V = \Span (v_{i})_{i\in I}\).  Eine Familie \(\mathcal{B}\) heißt Basis von
\(V\), falls \(\mathcal{B}\) linear unabhängig ist.  Falls ein endliches
Erzeugendensystem von \(V\) gibt, dann heißt \(V\) endlich erzeugt.  Ist
die Familie \(\mathcal{B}\) eine endliche Basis, so nennt man
\(\abs{\mathcal{B}}\) Länge der Basis.

Beispiel.  \((1, t, t^{2}, \ldots)\) ist eine Basis unendlicher Länge des
Polynomrings \(K[t]\).
 
Satz.  Eine Basis ist eine maximal linear unabhängiges
Erzeugendensystem.  Eine Basis ist eine minimal Erzeugendensystem, das
heißt, dass \(\mathcal{B}\) kein Erzeugendensystem mehr, wenn man um ein Vektor
davon verkürzt.

Basisauswahlsatz.  Aus jedem endlichen Erzeugendensystem eines
Vektorraums kann man eine Basis auswählen.

Existenz von Basen.  Jeder Vektorraum besitzt eine Basis.

\textsc{Zorn}sche Lemma.
\end{document}

