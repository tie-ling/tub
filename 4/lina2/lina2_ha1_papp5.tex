% 使用 chktex 检查 tex 文件中的语法错误
% settings for chktex
% chktex-file 3

\documentclass[fleqn,draft,a5paper]{article}

% 让每个章节 subsection 在新的一页上开始
% 而不是紧接着上一章节
\AddToHook{cmd/subsection/before}{\clearpage}

% 隐藏默认的章节序号:实际作业中与这个冲突
% https://tex.stackexchange.com/a/30225
\setcounter{secnumdepth}{0}

% 设置页边距,上下左右
\usepackage{geometry}
\geometry{left=1cm,
 right=1cm,
 top=1cm,
 bottom=2cm}

% 让 TeX 支持德语
\usepackage[ngerman]{babel}
\usepackage{amsmath,mathtools,fontspec,amssymb,amsthm,parskip,interval,mathtools,fontspec}
\setmainfont{TeX Gyre Termes}

% 根据 AMS 建议,应为成对符号(比如绝对值)定义新命令
\providecommand{\skp}[1]{\langle#1\rangle}
\providecommand{\abs}[1]{\left\lvert#1\right\rvert}
\providecommand{\norm}[1]{\left\lVert#1\right\rVert}
\providecommand{\ceil}[1]{\left\lceil#1\right\rceil}
\providecommand{\floor}[1]{\left\lfloor#1\right\rfloor}

% 定理定义,依赖于 amsthm
\theoremstyle{remark}
\newtheorem*{Behauptung}{Behauptung}
\newtheorem*{Lemma}{Lemma}
\newtheorem*{Satz}{Satz}
\newtheorem*{Definition}{Definition}

% 定义新函数,依赖于AMSmath
\DeclareMathOperator{\card}{card}
\DeclareMathOperator{\Rang}{Rang}

\providecommand{\R}[1]{\mathrm{R#1}}

\newcommand{\wt}{\widetilde}
\newcommand{\dd}{\,\mathrm{d}}

% 标题与作者
\title{HA 1, LinA 2, Papp, Papp5}
\author{Shilong Yu 478912, Yuchen Guo 480788}

\begin{document}
\maketitle
\newpage
\subsection{1. Aufgabe}
Es gibt keine lineare Abbildung \(F\colon \mathbb{R}^{2} \to \mathbb{R}^{2}\), sodass
\(F(3,0)=(0,1), F(2,2)=(3,1) \) und \(F(2,1)=(2,3)\).  Denn wegen Linearität
gilt
\[F(\lambda x) = \lambda F(x)\]
und
\[F(2,1)=F(1,0)+F(1,1) = \frac{1}{3}F(3,0) + \frac{1}{2}F(2,2) =
  (0,\frac{1}{3})+(\frac{3}{2}, \frac{1}{2}) \ne (2,3).\]
Damit ist der Existenz widerliegt.
\subsection{2. Aufgabe}
Berechne \(A \cdot B\).
\begin{enumerate}
\item \(A =
  \begin{bmatrix}
    1 & 2 \\ 3 & 4
  \end{bmatrix},
  B =
  \begin{bmatrix}
    1 & 1 \\ 0 & 1
  \end{bmatrix}.
  \)
  Wegen \(C_{ik} = \sum_{j=1}^{n}{A_{ij}B_{jk}}\) gilt
\[A \cdot B =
  \begin{bmatrix}
    1 \times 1 + 2 \times 0 & 1 \times 1 + 2 \times 1 \\
    3 \times 1 + 4 \times 0 & 3 \times 1 + 4 \times 1
  \end{bmatrix} =
  \begin{bmatrix}
    1 & 3 \\ 3 & 7
  \end{bmatrix}.
  \]
\item Berechne \(B \cdot A\), wobei \(A, B\) Matrizen wie in Teil Eins.
  \[B \cdot A =
    \begin{bmatrix}
      1 \times 1 + 1 \times 3 & 1 \times 2 + 1 \times 4 \\ 0 \times 1 + 1 \times 3 & 0 \times 2 + 1 \times 4
    \end{bmatrix}
    =
    \begin{bmatrix}
      4 & 6 \\ 3 & 4
    \end{bmatrix}.
  \]
\item \(A =
  \begin{bmatrix}
    0 & 1 & 0 \\ 1 & -1 & -1
  \end{bmatrix},
  B =
  \begin{bmatrix}
    1 & 0 & 2 \\ 0 & 0 & 1 \\  -1 & 0 & 0
  \end{bmatrix}.
  \)
  \begin{align*}
    A \cdot B
    &=
    \begin{bmatrix}
      0 \times 1 + 1 \times 0 + -1 \times 0
      & 0
      & 2 \times 0 + 1 \times 1 + 0 \times 0 \\
      1 \times 1 + -1 \times 0 + -1 \times -1
      & 0
      & 2 \times 1 + -1 \times 1 + -1 \times 0
    \end{bmatrix} \\
    &=
    \begin{bmatrix}
      0 & 0 & 1 \\ 2 & 0 & 1
    \end{bmatrix}.
  \end{align*}
\end{enumerate}
\subsection{3. Aufgabe}
Es sind
\[\Phi_{\mathcal{A}}=
  \begin{bmatrix}
    1 & 2 & 2 \\ -1 & 3 & 3 \\ 2 & 7 & 6
  \end{bmatrix},
  \quad
  \Phi_{\mathcal{B}} =
  \begin{bmatrix}
    1 & -1 & -2 \\ 2 & 3 & 7 \\ 2 & 3 & 6
  \end{bmatrix}.
\]
Dann erhalten wir
\[\Phi_{\mathcal{B}}^{-1}=
  \begin{bmatrix}
    3/5 & 0 & 1/5 \\ -2/5 & -2 & 11/5 \\ 0 & 1 & -1
  \end{bmatrix}, \quad
  T^{\mathcal{A}}_{\mathcal{B}} = \Phi_{\mathcal{B}}^{-1} \circ \Phi_{\mathcal{A}} =
  \begin{bmatrix}
    1 & 13/5 & 12/5 \\ 6 & 43/5 & 32/5 \\ -3 & -4 & -3
  \end{bmatrix}.
\]
indem wir die  elementare Zeilenumformungen gleichzeitig auf \(\Phi_{\mathcal{B}}\)
und \(E_{3}\) anwenden:
\begin{align*}
  \begin{bmatrix}
    1 & -1 & -2 \\ 2 & 3 & 7 \\ 2 & 3 & 6
  \end{bmatrix}
  &
    \begin{bmatrix}
      1 & 0 & 0 \\ 0 & 1  & 0 \\ 0 & 0 & 1
    \end{bmatrix}\\
  \begin{bmatrix}
    1 & -1 & -2 \\ 0 & 5 & 11 \\ 0 & 5 & 10
  \end{bmatrix}
  &
    \begin{bmatrix}
      1 & 0 & 0 \\ -2 & 1 & 0 \\ -2 & 0 & 1
    \end{bmatrix}\\
  \begin{bmatrix}
    1 & -1 & -2 \\ 0 & 0 & 1 \\ 0 & 5& 10 
  \end{bmatrix}
  &
    \begin{bmatrix}
      1 & 0 & 0 \\ 0 & 1 & -1 \\ -2 & 0 & 1
    \end{bmatrix}\\
  \begin{bmatrix}
    1 & -1 & -2 \\ 0 & 5 & 10 \\ 0 & 0& 1
  \end{bmatrix}
  &
    \begin{bmatrix}
      1 & 0 & 0 \\ -2 & 0 & 1 \\ 0 & 1 & -1
    \end{bmatrix}\\
  \begin{bmatrix}
    1 & -1 & -2 \\ 0 & 1 & 2 \\ 0 & 0 & 1
  \end{bmatrix}
  &
    \begin{bmatrix}
      1 & 0 & 0 \\ -2/5 & 0 & 1/5 \\ 0 & 1 & -1
    \end{bmatrix}\\
  \begin{bmatrix}
    1 & -1 & 0 \\ 0 & 1 & 0 \\ 0 & 0 & 1
  \end{bmatrix}
  &
    \begin{bmatrix}
      1 & 2 & -2 \\ -2/5 & -2 & 11/5 \\ 0& 1 & -1
    \end{bmatrix}\\
  \begin{bmatrix}
    1 & 0 & 0 \\ 0 & 1 & 0 \\ 0 & 0 & 1
  \end{bmatrix}
  &
    \begin{bmatrix}
      3/5 & 0 & 1/5 \\ -2/5 & -2 & 11/5 \\ 0 & 1 & -1
    \end{bmatrix}
\end{align*}


Für die zweiten Aufgabe erhalten wir
\[T^{\mathcal{A}}_{\mathcal{B}} \cdot
  \begin{bmatrix}
    2 \\ 9 \\ -8
  \end{bmatrix}
  =
  \begin{bmatrix}
    31/5 \\ 191/5 \\ -18
  \end{bmatrix}.
\]
\subsection{4. Aufgabe}
\begin{enumerate}
\item \(A =
  \left[
  \begin{smallmatrix}
    0 & 1 & 0 \\ 1 & -1 & -1 \\ -1 & 0 & 1
  \end{smallmatrix}\right]
  \) ist nicht invertierbar, denn bezeichen wir die drei Zeilenvektoren
  als \(a, b , c\) dann gilt \(b=-a-c\) und \(\Rang A = 2\).
\item \(B =
  \left[
  \begin{smallmatrix}
    1 & 2 & 0 \\ 0 & 1 & 2 \\ 2 & 0 & 1
  \end{smallmatrix}\right]
  \)
  ist invertierbar mit
  \[
    B^{-1} =
    \begin{bmatrix}
      1/9 & -2/9 & 4/9 \\ 4/9 & 1/9 & -2/9 \\ -2/9 & 4/9 & 1/9
    \end{bmatrix}.
  \]
\item \(C=
  \begin{bmatrix}
    1 & 2 \\ 0 & 3
  \end{bmatrix}
  \)
  ist invertierbar mit
  \[C^{-1}=
    \begin{bmatrix}
      1 & -2/3 \\ 0 & 1/3
    \end{bmatrix}.
  \]
\end{enumerate}
\end{document}

