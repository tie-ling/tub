% 使用 chktex 检查 tex 文件中的语法错误
% settings for chktex
% chktex-file 3

\documentclass[fleqn,draft,a5paper]{article}

% 让每个章节 subsection 在新的一页上开始
% 而不是紧接着上一章节
\AddToHook{cmd/subsection/before}{\clearpage}

% 隐藏默认的章节序号:实际作业中与这个冲突
% https://tex.stackexchange.com/a/30225
\setcounter{secnumdepth}{0}

% 设置页边距,上下左右
\usepackage{geometry}
\geometry{left=1cm,
 right=1cm,
 top=1cm,
 bottom=2cm}

% 让 TeX 支持德语
\usepackage[ngerman]{babel}
\usepackage{unicode-math,amsthm,parskip,interval}
\setmainfont{TeX Gyre Termes}
\setmathfont{TeX Gyre Termes Math}
\setmathfont[range=\mathcal]{latinmodern-math.otf}

% 根据 AMS 建议,应为成对符号(比如绝对值)定义新命令
\providecommand{\skp}[1]{\langle#1\rangle}
\providecommand{\abs}[1]{\left\lvert#1\right\rvert}
\providecommand{\ang}[1]{\left\langle#1\right\rangle}
\providecommand{\norm}[1]{\left\lVert#1\right\rVert}
\providecommand{\ceil}[1]{\left\lceil#1\right\rceil}
\providecommand{\floor}[1]{\left\lfloor#1\right\rfloor}

% 定理定义,依赖于 amsthm
\theoremstyle{remark}
\newtheorem*{Behauptung}{Behauptung}
\newtheorem*{Lemma}{Lemma}
\newtheorem*{Satz}{Satz}
\newtheorem*{Definition}{Definition}

% 定义新函数,依赖于AMSmath
\DeclareMathOperator{\card}{card}
\DeclareMathOperator{\rg}{rg}
\DeclareMathOperator{\grad}{grad}

\providecommand{\R}[1]{\mathrm{R#1}}
\newcommand{\wt}{\widetilde}
\newcommand{\dd}{\,\mathrm{d}}

% 标题与作者
\title{Analysis, Klausurvorbreitung}
\author{me}

\begin{document}
\maketitle
\newpage

Namenhafte Sätze zu beweisen.

[Normen]

Sei \(V\) ein \(K\)-Vektorraum.  \(\norm{\cdot}\colon V \to \mathbb{R}\) heißt Norm auf
\(V\), falls für alle \(v, w \in V\) und \(\lambda \in K\) gilt

\begin{itemize}
\item \(\norm{v} \ge 0\)
\item \(\norm{v} = 0\) genau dann \(v = 0\)
\item \(\norm{\lambda v} = \abs{\lambda} \norm{v}\)
\item \(\norm{v+w} \le \norm{v} + \norm{w}\)
\end{itemize}

[Stetigkeit]

Sei \((X, d)\) ein metrischer Raum und \((x_{n})\) eine Folge in X.
\((x_{n})\) heißt konvergent gegen \(a \in X\), falls
\(\forall \varepsilon > 0, \exists N \in \mathbb{N}, \forall n \ge N\colon d(x_{n}, a) < \varepsilon\).

Sei \((X, d)\) ein metrischer Raum.  Dann ist \(A\) abgeschlossen,
genau dann, wenn \(A\) folgenabgeschlossen ist.

Sei \((X, d)\) ein metrischer Raum.  Eine Folge \((x_{n})\) heißt
Cauchy-Folge, falls
\[\forall \varepsilon > 0, \exists N \in \mathbb{N}, \forall m, n \ge N\colon d(x_{n}, x_{m}) < \varepsilon.\]

Sei \((X, d)\) ein metrischer Raum.  \((X, d)\) heißt vollständig,
falls jede Cauchy-Folge konvergiert.

Ein vollständiger normierter Raum heißt Banachraum.

Seien \((X, d_{X})\) und \((Y, d_{Y})\) metrischer Vektorräume, \(f\colon X \to Y\)
eine Abbildung.  \(f\) heißt stetig in \(a \in X\), falls
\[\forall \varepsilon > 0, \exists \delta > 0, \forall a, b \in X, d_{X}(a, b) < \delta\colon
  d_{Y}(f(a),f(b))< \varepsilon.\]

Seien \(f\colon X \to Y\) stetig in \(a \in X\) und \(g\colon Y \to Z\) stetig in
\(f(a) \in Y\).  Dann ist \(g \circ f\) stetig in \(a\).

Stetige Abbildung: Urbilder offener Mengen sind offen.  Urbilder
abgeschlossener Mengen sind abgeschlossen.

[Kompaktheit]

Sei \((X, d)\) ein metrischer Raum und \(A \subseteq X\).  \(A\) heißt
kompakt, falls jede offene Überdeckung von \(A\) eine
\textbf{endliche} Teilüberdeckung besitzt.

Jede kompakte Teilmenge eines metrischen Raums ist beschränkt
und abgeschlossen.

Satz von Heine-Borel.  Eine Teilmenge von \(\mathbb{R}^{n}\) ist genau dann
kompakt, wenn sie beschränkt und abgeschlossen ist.

Sei \(f\colon X \to Y\) stetig und \(K \subseteq X\) kompakt.  Dann ist \(f(K)\)
kompakt.

Satz von Maximum und Minimum.  Sei \(X\) kompakt und \(f: X \to \mathbb{R}\)
stetig.  Dann gilt \(\max f = \sup f\) und \(\min f = \inf f\).

Satz von Heine.  Auf kompakte metrische Vektorräume stetige Abbildung
ist gleichmäßig stetig.

Satz von Bolzano-Weierstraß.  Sei \((X, d)\) ein metrischer Raum
und \(A \subseteq X\) kompakt.  Dann hat jede Folge \((a_{n}) \subseteq A\) eine
konvergente Teilfolge.

[Zusammenhang]

Definition.  Sei \((X, d)\) ein metrischer Raum.  \(X\) heißt
zusammenhängend, falls für alle \(U, V \subseteq X\) mit (1) \(U, V\) offen,
(2) \(U \cup V = X\), (3) \(U \cap V = \emptyset\) gilt: \(U = \emptyset\) oder \(V = \emptyset\).

\((X, d)\) ist nicht zusammenhängend, genau dann, wenn
\(C \subseteq X\) mit \(C \ne \emptyset\) und \(C \ne X\) existiert, sodass
\(C\) offen und abgeschlossen ist; genau dann, wenn
\(U, V \subseteq X\) existiert mit (1) \(U, V\) offen, (2)
\(U \cup V = X\), (3) \(U \cap V = \emptyset\) und (4) beide Mengen nicht leer.

Seien \((X, d_{X})\) und \((Y, d_{Y})\) metrische Vektorräumen und
\(f\colon X \to Y\) stetig.  Falls \(X\) zusammenhängend ist, ist auch
\(f(X)\) zusammenhängend.

Zwischenwertsatz.  Sei \((X, d)\) zusammenhängend und \(f: X \to \mathbb{R}\)
stetig.  Seien \(a, b \in X\) und \(c \in \mathbb{R}\) sodass \(f(a) < c < f(b)\).
Dann existiert ein \(x \in X\) mit \(f(x) = c\).

[Komponentenweise Konvergenz im \(\mathbb{R}^{n}\)]

Sei \(\mathbb{R}^{n}\) mit der durch die \(p\)-Norm induzierten Metrik.  Seien der
Punkt \(a \in \mathbb{R}^{n}\) und die Folge
\((x_{k}) \in \mathbb{R}^{n}\) .  Dann konvergiert \((x_{k})\) gegen
\(a\) genau dann, wenn die Folge komponentenweise gegen \(a\) konvergiert.

[Lineare Abbildungen]

Seien \(V, W\) normierte Vektorräume und \(f \in L(V, W)\).  Dann sind
die folgende Aussagen äquivalent:
\begin{itemize}
\item \(f\) ist stetig in \(0\).
\item \(f\) ist (Lipschitz-, gleichmäßig) stetig.
\item Existiert \(c \ge 0\) mit \(\norm{f(v)} \le c \norm{v}\) für alle
  \(v \in V\).
\end{itemize}

Seien \(V, W\) normierte Vektorräume und \(f \in L(V, W)\).  Falls
\(\dim V\) endlich ist, dann ist \(f\) stetig.

Seien \(V, W\) normierte Vektorräume und \(f \in L(V, W)\) stetig.
Dann wird ihre Norm, die Operatornorm definiert durch
\[\norm{f} \coloneq \sup \{\norm{f(v)}\colon v \in V, \norm{v} \le 1\}\]
und es gilt für alle \(f \in L(V, W)\) und \(v \in V\) dass \(\norm{f(v)}
\le \norm{f} \cdot \norm{v}\).

[Richtungsableitung]

Seien \(V, W\) end-dim Banachräume und \(U \subseteq V\) offen,
\(f\colon U \to W\), \(a \in U\), \(v \in V\) und
\(h \in \mathbb{R}\).  Die Richtungsableitung von \(f\) in \(a\) nach Richtung
\(v\) ist definiert durch
\[\partial_{v}f(a) = \lim_{h\to0}\frac{f(a+hv) - f(a)}{h} \in W\]

Seien \(V, W\) end-dim Banachräume und \(U \subseteq V\) offen,
\(f\colon U \to W\), \(a \in U\) und \(v \in V\).  \(f\) heißt total
differenzierbar in \(a\) falls es eine \(L \in L(V, W)\) und
\(R \colon U \to W\) mit \(\lim_{v\to0}\frac{R(a+v)}{\norm{v}}=0\) existiert
sodass für alle \(a+v \in U\) gilt
\[f(a+v) = f(a) + L(v) + R(a+v).\]

Zusammenhang von Richtungsableitung und totaler Ableitung:  ist
\(f\) diffbar in \(a\), so gilt für alle \(v \in V\) dass
\[Df(a)(v)=\partial_{v}f(a).\]

Seien \(V, W\) end-dim Banachräume und \(U \subseteq V\) offen, \(f\colon U \to W\)
diffbar in \(x \in U\).  Dann ist \(f\) stetig in \(x\).

Partielle Ableitung: Richtungsableitung in der Richtung der
Einheitsvektoren.  Dies hilft bei der Bestimmung von
Jacobi-Matrix.

Seien \(U \subseteq \mathbb{R}^{n}\) offen und \(f\colon U \to \mathbb{R}^{m}\) eine Funktion.  Ist
\(f\) stetig partiell diffbar, so ist \(f\) stetig total diffbar.

\(f\) ist genau dann in \(x \in U\) diffbar, falls alle Komponentenfunktion
\(f_{i}\) in \(x\) diffbar sind.

[Jacobi-Matrix]

Seien \(V, W\) end-dim Banachräume, \(U \subseteq V\) offen und \(f\colon U \to W\) mit
\(\dim V = n\) und \(\dim W = m\).  Dann ist
\begin{align*}
  \begin{bmatrix}
    Df(x)
  \end{bmatrix}
  =
  \begin{bmatrix}
    \frac{\partial f}{\partial x_{1}}(x) & \cdots & \frac{\partial f}{\partial x_{n}}(x)
  \end{bmatrix}
  =
  \begin{bmatrix}
    \frac{\partial f_{1}}{\partial x_{1}}(x) & \cdots & \frac{\partial f_{1}}{\partial x_{n}}(x) \\
    \vdots &  & \vdots \\
    \frac{\partial f_{m}}{\partial x_{1}}(x) & \cdots & \frac{\partial f_{m}}{\partial x_{n}}(x)
  \end{bmatrix}
\end{align*}

[Kettenregel]
\[D(g \circ f)(t) = Dg(f(t)) \circ Df(t).\]

Ableitung bilinearer Abbildungen.  Sei \(U, V, W\) end-dim
Banachräume. Sei \(f\colon U \times V \to W\) bilinear.  Dann ist \(f\) diffbar
mit
\[Df(v,w)(a,b) = f(v,b) + f(a,w).\]

Sei \(A \in M(m \times n)\).  Definiere \(f\colon \mathbb{R}^{m} \times \mathbb{R}^{n} \to \mathbb{R}\) mit \(f(v,w) =
v^{\top} A w\), dann gilt für die Ableitung von \(f\) dass
\[Df(v,w)(a,b) = v^{\top}Ab + a^{\top} A w.\]

[Der Schrankensatz]

Sei \(V\) ein Vektorraum und \(a, b \in V\).  Dann heißt
\[\overline{ab} = \{a+t(b-a)\mid t \in \interval{0}{1}\}\]
die Verbindungsstrecke zwischen \(a, b\).

Sei \(V, W\) end-dim Banachräume. \(U \subseteq V\) offen und \(f\colon U \to W\) diffbar.
Seien \(a, b \in U\) mit \(\overline{ab} \subseteq U\).  Dann gilt
\[\norm{f(b)-f(a)}\le \sup_{x \in \overline{ab}}\norm{Df(x)}\cdot\norm{b-a}.\]

[Höhere Ableitungen]

Definition \(k\)-mal stetig diffbar. \(f\) heißt \(k\)-mal stetig diffbar,
falls \(f\) in alle \(x \in U\) \(k\)-mal diffbar ist und \(D^{k}f\) stetig ist.

[Satz von Schwarz]

Sei \(U \subseteq V\) offen und \(f\colon U \to W\) zweimal stetig partiell diffbar.  Sei
\(\dim V = n\).  Dann gilt für alle \(a \in U\) und alle
\(i,j=1,2,\ldots,n\) dass \(D_{j}D_{i}f(a) = D_{i}D_{j}f(a)\).  In diesem Fall
ist Hesse-Matrix symmetrisch.

[Hessesche Matrix]
\[H_{f}(x) =
  \begin{bmatrix}
    D_{i}D_{j}f(x)
  \end{bmatrix}, \quad
  i , j = 1 \ldots, n.
\]

[Taylorformel]

Seien \(V, W\) end-dim Banachräume, \(U \subseteq V\) offen und
\(f\colon U \to W\) \(k\)-mal diffbar.  Sei \(x_{0} \in U\) und
\(D^{0}f(x_{0})=f(x_{0})\).  Dann existiert \(R_{k}\colon U \to W\) sodass für
alle \(x \in U\) gilt \(\lim_{x\to x_{0}}=\frac{R_{k}(x)}{\norm{x-x_{0}}^{k}}
= 0\) und
\[f(x) =
  \left[\sum_{l=0}^{k}\frac{1}{l!}D^{l}f(x_{0})(x-x_{0})^{l}\right] +
  R_{k}(x).\]

[lokales Extremum]

\(f\colon U \to \mathbb{R}\) hat ein lokales Maximum in \(x_{0}\), falls es eine Umgebung
\(U_{\varepsilon}(x_{0})\) von \(x_{0}\) existiert sodass
\(f(x_{0}) \ge f(x)\) für alle \(x \in U_{\varepsilon}(x_{0}) \cap U \).

[Gradient]

Sei \(U \subseteq V \) offen und \(f\colon U \to W\) partiell diffbar.  Dann heißt
\[\grad  f =
  \begin{bmatrix}
    \frac{\partial f}{\partial x_{1}}(x) & \cdots & \frac{\partial f}{\partial x_{n}}(x)
  \end{bmatrix}
\]
der Gradient von \(f\) im Punkt \(x \in U\).

Der Gradient zeigt in die Richtung des stärksten Anstiegs von \(f\).

Der Gradient steht senkrecht auf der jeweiligen Niveaumenge von \(f\).

[lokales Extremum, notwendiges Bedingung]

Sei \(U \subseteq V\) offen und \(f\colon U \to \mathbb{R}\) part diffbar Funktion.  Falls \(f\) in
\(x \in U\) ein lokales Extremum besitzt, so gilt
\[\grad  f(x) = 0.\]

[Definitheit einer quadratischer Form]

Sei \(A \in M(n \times n)\) eine symmetrische Matrix.  Dann heißt \(q(x) =
x^{\top}Ax\) eine quadratische Form.

\(A\) heißt positiv definit, falls \(\ang{x, Ax}>0\) für alle \(x \in V \setminus
\{0\}\).

\(A\) heißt positiv semidefinit, falls \(\ang{x, Ax} \ge 0\) für alle \(x \in
V\).

\(A\) heißt negativ (semi)definit, falls \(-A\) positiv (semi)definit ist.

[Hinreichendes Kriterium für lokales Extremum]

Sei \(V\) ein end-dim Banachraum, \(U \subseteq V\) offen und \(f\colon U \to \mathbb{R}\) \(k\)-mal
diffbar und \(x_{0} \in U\), sodass
\[Df(x_{0})= \ldots = D^{k-1}f(x_{0})=0 \quad \text{aber} \quad D^{k}f(x_{0}) \ne
  0.\]

Ist \(D^{k}f(x_{0})\) negativ definit, dann hat \(f\) in \(x_{0}\) ein
striktes lokales Minimum.

Ist positiv definit, dann hat \(f\) in \(x_{0}\) striktes lokales Maximum.

Ist indefinit, dann kein lokales Extremum.

[Umkehrsatz]

Sei \(U \subseteq \mathbb{R}^{n}\) offen und
\(f\colon U \to \mathbb{R}^{n}\) stetig diffbar.  Sei \(x_{0} \in U\), sodass
\(Df(x_{0})\colon \mathbb{R}^{n} \to \mathbb{R}^{n}\) invertierbar ist.  Dann ist
\(f\) um \(x_{0}\) lokal invertierbar mit stetig diffbarer Inverser.

D.h., es existiert \(W \subseteq U\) offen mit \(x_{0} \in W\), sodass
\(f\colon W \to f(W)\) bijektiv ist und die lokale Inverse
\(g\colon f(W) \to W\) von \(f\) stetig diffbar ist.

Weiter gilt für alle \(x \in V\) dass \(Df(x)\) invertierbar und es gilt
\(Dg(f(x)) = (Df(x))^{-1}\).

[Implizite Funktionen]

Seien \(W \subseteq \mathbb{R}^{n} \times \mathbb{R}^{m}\) offen, \(F\colon W \to \mathbb{R}^{m}\) stetig diffbar, sowie
\((x_{0}, y_{0}) \in W\) mit \(F(x_{0}, y_{0}) = 0\) und \(\frac{\partial F}{\partial
  y}(x_{0}, y_{0}) \in \mathbb{R}^{m \times m}\) invertierbar.

Wobei \(\frac{\partial F}{\partial y}(x_{0}, y_{0}) \) die rechte \(m\) Spalte von der
Jacobi-Matrix \(DF(x_{0},y_{0})\) ist, also gilt
\begin{align*}
  DF(x_{0}, y_{0}) &=
  \begin{bmatrix}
    \frac{\partial F}{\partial x}(x_{0}, y_{0}) & \frac{\partial F}{\partial y}(x_{0}, y_{0})
  \end{bmatrix} \\ &=
  \begin{bmatrix}
    \frac{\partial F_{1}}{\partial x_{1}}(x_{0}, y_{0})
    & \ldots
    & \frac{\partial F_{1}}{\partial x_{n}}(x_{0}, y_{0})
    & \frac{\partial F_{1}}{\partial y_{1}}(x_{0}, y_{0})
    & \ldots
    & \frac{\partial F_{1}}{\partial y_{m}}(x_{0}, y_{0}) \\
    \vdots & & & & & \vdots \\
    \frac{\partial F_{m}}{\partial x_{1}}(x_{0}, y_{0})
    & \ldots
    & \frac{\partial F_{m}}{\partial x_{n}}(x_{0}, y_{0})
    & \frac{\partial F_{m}}{\partial y_{1}}(x_{0}, y_{0})
    & \ldots
    & \frac{\partial F_{m}}{\partial y_{m}}(x_{0}, y_{0})
  \end{bmatrix}  
\end{align*}
Dann existiert \(U \subseteq \mathbb{R}^{n}\) offene Umgebung von \(x_{0}\) und \(V \subseteq \mathbb{R}^{m}\)
offene Umgebung von \(y_{0}\) mit \(U \times V \subseteq W\), sodass gilt:
\begin{itemize}
\item Es existiert \(f: U \to V, x \mapsto f(x)\) bijektiv sodass
  \[F(x, f(x)) = 0.\]
  In diesem Fall heißt \(x\) Parameter und \(y\) die von \(x\) eindeutig
  bestimmten Unbekannte.

\item Diese Funktion \(f\) ist stetig diffbar.
\item Für alle \(x \in U\) ist \(A \coloneq \frac{\partial F}{\partial y}(x, f(x))\) invertierbar und
  für alle \(x \in U\) ist
  \[Df(x) = - A^{-1} \cdot A.\]
\end{itemize}

[Untermannigfaltigkeiten des \(\mathbb{R}^{n}\) (übersprungen)]

Definition Homöomorphismus.  Seien \(X, Y\) metrische Räume.  \(X, Y\)
sind homöomorph und \(\varphi\colon X \to Y\) ein Homömorphismus, falls \(\varphi\) bijektiv
ist und \(\varphi\) und \(\varphi^{-1}\) beide stetig sind.

Definition Diffeomorphismus.  Seien \(U, V \subseteq \mathbb{R}^{n}\) offen.  \(U, V\)
sind diffeomorph und \(\varphi\colon U \to W\) ein Diffeomorphismus, falls \(\varphi\) bijektiv
ist und \(\varphi\) und \(\varphi^{-1}\) beide stetig diffbar sind.

Definition Immersion.  Sei \(T \subseteq \mathbb{R}^{k}\) offen.  Eine Abbildung \(\varphi\colon T
\to \mathbb{R}^{n}\) heißt Immersion, falls \(\varphi\) stetig diffbar ist und für alle \(t
\in T\) die Ableitung \(D\varphi(t)\) injektiv ist.  D.h., für alle \(t \in T\) dass
\(\dim \ker D\varphi(t) = 0\).

Satz.  Sei \(\varphi\colon T \to \mathbb{R}^{n}\) eine Immersion.  Dann gibt es zu jedem \(t \in
T\), eine offene Umgebung \(W \subseteq T\) von \(t\), so dass die Einschränkung
\(\varphi\vert_{W}\colon W \to \varphi(W)\) ein Homöomorphismus ist.

Definition Untermannigfaltigkeit.  Sei \(M \subseteq \mathbb{R}^{n}\).  \(M\) heißt
\(k\)-dimensionale Untermannigfaltigkeit des \(\mathbb{R}^{n}\), falls es zu jedem
\(a \in M\) eine offene Umgebung \(U \subseteq \mathbb{R}^{n}\) von \(a\), sowie eine offene
Menge \(T \subseteq \mathbb{R}^{k}\) und eine Immersion \(\varphi\colon T \to M\) gibt, sodass \(\varphi(T) =
M \cap U\) gilt und \(\varphi\colon T \to \varphi(T)\) ein Homöomorphismus ist.

\(\varphi\colon T \to M\) heißt Parameterdarstellung von \(M\) um \(a\).  Ist \((t_{1}, \ldots,
t_{k}) \in T\) mit \(\varphi(t_{1}, \ldots, t_{k}) = a\), so heißen \(t_{1}, \ldots, t_{k}\)
die lokalen Koordinaten von \(a\) bzgl. \(\varphi\).

[Extremwertaufgaben unter Nebenbedingungen]

Definition.  Sei \(U \subseteq \mathbb{R}^{n}\) offen und \(m \le n\), seien \(f\colon U \to \mathbb{R}\) und
\(g\colon U \to \mathbb{R}^{m}\) stetig diffbar.  Wir nennen \(\min f(x)\) mit \(g(x) = 0\)
eine Extremwertaufgabe mit Nebenbedingungen.

\(\mathcal{F} \coloneq \{x \in U \mid g(x) = 0\}\) heißt zulässiger Bereich von der
Extremwertaufgabe mit Nebenbedingungen.

Ein Punkt \(x \in \mathcal{F}\) heißt regulär, falls \(\rg Dg(x) = m\).

Ein Punkt \(x^{\ast} \in \mathcal{F}\) heißt lokale Lösung von der EmN oder Stelle
eines lokalen Minimums von EmN, falls ein \(\varepsilon > 0\) existiert, sodass
\(f(x^{\ast}) \le f(x)\) für alle \(x \in \mathcal{F}\) mit \(\norm{x - x^{\ast}} < \varepsilon\) gilt.

Satz.    Sei \(U \subseteq \mathbb{R}^{n}\) offen und seien \(f\colon U \to \mathbb{R}, g = (g_{1}, \ldots,
g_{m})\colon U \to \mathbb{R}^{m}\) stetig diffbar.  ferner sei \(x^{\ast} \in \{x \in U \mid g(x)
= 0\}\) eine Stelle eines lokalen Minimums der Extremwertaufgabe unter
Nebenbedingungen \(\min f(x)\) mit \(g(x) = 0\).  Falls \(x^{\ast}\) regulär
ist, so gibt es Koeffizienten \(\lambda_{1}, \ldots, \lambda_{m} \in \mathbb{R}\) sodass
\[\grad f(x^{\ast}) = \sum_{j=1}^{m}\lambda_{j} \grad g_{j}(x^{\ast}).\]

[Gewöhnliche Differentialgleichungen]

Definition.  Sei \(U \subseteq \mathbb{R} \times \mathbb{R}^{n}\) offen und \(f\colon U \to \mathbb{R}^{n}, (t, x)
\mapsto f(t,x)\) eine Abbildung.  Die Gleichung \(x' = f(t, x)\) heißt
gewöhnliche Differentialgleichung erster Ordnung.

Sei \(I \subseteq \mathbb{R}\) ein Intervall mit \(I^{\circ} \ne \emptyset\).  Eine Funktion \(x \colon I \to
\mathbb{R}^{n}\) heißt Lösung der Differentialgleichung \(x' = f(t, x)\), falls
gilt: (1) für alle \(t \in I\) ist \((t, x(t)) \in U\);  (2) \(x\colon I \to \mathbb{R}^{n}\)
ist diffbar; (3) für alle \(t \in I\) gilt \(x'(t) = f(t, x(t))\).

Sei \((t_{0}, x_{0}) \in U\).  Dann heißt das Gleichungssystem \(x' = f(t,
x)\) mit \(x(t_{0}) = x_{0}\) ein Anfangswertproblem.

Eine Lösung \(x\) der Differentialgleichung heißt Lösung des
Anfangswertproblems, falls \(x(t_{0}) = x_{0}\).

Definition.  Seien \(G \subseteq \mathbb{R} \times \mathbb{R}^{n}\) und \(f\colon G \to \mathbb{R}^{n}\).  Wir sagen:
\begin{itemize}
\item \(f\) genügt in \(G\) einer Lipschitzbedingung mit
  Lipschitz-Konstante \(L \ge 0\), falls für alle \((t, x), (t, \tilde{x})
  \in G\) gilt, dass \(\norm{f(t,x) - f (t, \tilde{x})} \le L \cdot \norm{x -
    \tilde{x}}\).
\item \(f\) genügt in \(G\) lokal einer Lipschitzbedingung, falls es zu
  jedem \((t, x) \in G\) eine offene Umgebung \(U \subseteq \mathbb{R} \times \mathbb{R}^{n}\) von \((t, x)\)
  gibt, sodass \(f\) in \(G \cap U\) einer Lipschitzbedingung mit von \(U\)
  abhängiger Lipschitz-Konstante \(L_{U} \ge 0\)genügt.
\end{itemize}

[Satz von Picard-Lindelöf]  Sei \(U \subseteq \mathbb{R} \times \mathbb{R}^{n}\) offen und sei \(f\colon U \to
\mathbb{R}^{n}\) stetig und genüge lokal einer Lipschitzbedingung.  Dann gilt:
\begin{itemize}
\item Lokale Existenz.  Zu jedem \((t_{0}, x_{0}) \in U\) existiert ein
  \(\varepsilon> 0\), sodass das Anfangswertproblem \(x' = f(t,x), x(t_{0}) =
  x_{0}\) eine Lösung \(x\colon \interval{t_{0} - \varepsilon}{t_{0} + \varepsilon} \to \mathbb{R}^{n}\)
  besitzt.
\item Globale Eindeutigkeit.  Sei \(I \subseteq \mathbb{R}\) ein Intervall mit \(t_{0} \in
  I\).  Sind \(x_{1}, x_{2} \colon I \to \mathbb{R}^{n}\) zwei Lösungen des
  Anfangswertproblems, so gilt \(x_{1} = x_{2}\).
\end{itemize}

[Maßtheorie]

[Ring]  Eine Menge \(R \subseteq P(\Omega)\) heißt Ring über \(\Omega\), falls gilt: (1) \(R
\ne \emptyset\); (2) für alle \(A, B \in R\) gilt \(A \cup B \in R\); (3) für alle \(A, B \in
R\) gilt \(A \setminus B \in R\).

[\(\sigma\)-Algebra]  \(\mathcal{A} \subseteq P(\Omega)\) heißt \(\sigma\)-Algebra über \(\Omega\), falls (1) \(\Omega \in
\mathcal{A}\); (2) aus \(A \in \mathcal{A}\) folgt \(A^{c} = \Omega \setminus A \in \mathcal{A}\); (3) für alle
\((A_{k}) \in \mathcal{A}, k \in \mathbb{N}\) ist \(\bigcup_{k}A_{k} \in \mathcal{A}\).

[von \(E\) erzeugte \(\sigma\)-Algebra] Sei \(E \subseteq P(\Omega)\).  Wir definieren
die Durchschnitt von alle \(\mathcal{A}\) mit \(E \subseteq \mathcal{A}\) die von \(E\) erzeugte
\(\sigma\)-Algebra, \(\sigma(E)\).

[Elementargeometrische Menge] \(A \subseteq \mathbb{R}^{n}\) heißt elementargeometrisch,
falls \(A\) endliche Vereinigung von Quadern \(Q\) ist, wobei
\(Q = I_{1} \times \cdots \times I_{n}\) mit \(I_{j}\) Intervalle (offen, abgeschlossen,
(links-, rechts-) halboffen) auf \(\mathbb{R}\).  Die Mengensystem
\(\mathcal{R}^{n}\) von aller elementargeometrischen Mengen in
\(\mathbb{R}^{n}\) ist ein Ring über \(\mathbb{R}^{n}\).

[Borelsche \(\sigma\)-Algebra] Die von \(\mathcal{R}^{n}\) erzeugte \(\sigma\)-Algebra heißt
Borelsche \(\sigma\)-Algebra \(\mathcal{B}^{n} = \sigma(\mathcal{R}^{n})\).  Ihre Elemente nennen wir
Borel-Mengen.

Sei \(A \subseteq \mathbb{R}^{n}\) offen oder abgeschlossen, dann gilt \(A \in \mathcal{B}^{n}\).

[Inhalt, Prämaß, Maß, Maßraum]

Sei \(R\) Ring über \(\Omega\) und \(\mu\colon R \to \interval{0}{\infty}\).  \(\mu\) ist additiv
(Inhalt) auf \(R\), falls \(\mu(\emptyset) = 0\) und für alle disjunkte \(A, B \in R\)
gilt \(\mu(A \cup B) = \mu(A) + \mu(B)\).

\(\mu\) ist \(\sigma\)-additiv (Prämaß) auf \(R\), falls \(\mu(\emptyset) = 0\) und alle
paarweise disjunkte \(A_{k} \in R, k \in \mathbb{N}\) gilt: falls \(\bigcup A_{k} \in R\) dann
folgt \(\mu\left(\bigcup A_{k}\right) = \sum \mu(A_{k})\).

Ein Prämaß \(\mu\) auf einer \(\sigma\)-Algebra \(\mathcal{A}\) heißt Maß auf \(\mathcal{A}\).

Ein Maßraum ist ein Tripel \((\Omega, \mathcal{A}, \mu)\) aus einer Menge \(\Omega\), einer
\(\sigma\)-Algebra \(\mathcal{A}\) über \(\Omega\) und einem Maß \(\mu\) auf \(\mathcal{A}\).

[elementargeometrische Inhalt]  \(\lambda\colon \mathcal{R}^{n} \to \rinterval{0}{\infty}\) ist
definiert durch \(\lambda(Q) \coloneq \prod_{j=1}^{n}(b_{j} - a_{j})\) wobei \(Q = I_{1} \times
\cdots \times I_{n}\), \(I_{j}\) Intervalle auf \(\mathbb{R}\).  Zerfällt \(A \in \mathcal{R}\) in disjunkte
Quader \(Q_{1}, \ldots, Q_{m}\), so ist \(\lambda(A) = \sum \lambda(Q_{k})\) und \(\lambda(\emptyset) = 0\).

Der elementargeometrische Inhalt \(\lambda\) auf \(\mathcal{R}^{n}\) ist ein Prämaß auf \(\mathcal{R}^{n}\).

[Fortsetzung von \(\mu\) auf \(R\) zum Maß auf \(\sigma(R)\)]

Sei \(R\) Ring über \(\Omega\) und \(\mu\) Prämaß auf \(R\).  \(E \subseteq P(\Omega)\) schöpft \(\Omega\)
aus, falls es existiert \((E_{k})_{k \in \mathbb{N}} \in E\) sodass \(\Omega = \bigcup E_{k}\).

Sei \(E \subseteq R\).  \(E\) schöpft \(\Omega\) bzgl. \(\mu\) aus, falls es abzählbar viele
Mengen \((E_{k})_{k \in \mathbb{N}}\) gibt, sodass \(\mu(E_{k}) < \infty\) für alle \(k \in \mathbb{N}\)
und \(\Omega = \bigcup E_{k}\).

Definition Äußeres Maß.  Sei \(R\) Ring über \(\Omega\) und \(\mu\) Prämaß auf \(R\),
sodass \(R\) die Menge \(\Omega\) bzgl. \(\mu\) ausschöpft.  Wir definieren das
äußere Maß auf \(\Omega\) als
\[\mu^{\ast}\colon P(\Omega) \to \interval{0}{\infty}, \quad \mu^{\ast}(A) \coloneq \inf \left\{ \sum \mu(E_{k});
    \text{ für alle } (E_{k})_{k \in \mathbb{N}} \in R \text{ mit } A \subseteq \bigcup E_{k}
  \right\}\]

Definition Abstandsmaß.  Seien \(A, B \subseteq \Omega\), dann heißt \(d(A, B) \coloneq
\mu^{\ast}(A \triangle B)\) das Abstandsmaß von \(A\) und \(B\).

Dann ist \(d(A,A) = 0\), \(d(A, B) = d(B, A)\) und \(d(A, B) \le d(A, C) +
d(C, B)\).

Definition messbare Menge.  Sei \(A \subseteq \Omega\).  \(A\) heißt endlich messbar
bzgl. \(\mu^{\ast}\), falls \(\mu^{\ast} (A) < \infty\) und falls es eine Folge \((E_{n})
\in R\) existiert, sodass \(\lim_{n \to \infty} d(E_{n}, A) = 0\) gilt.  Die Mengensystem
der endlich messbaren Teilmengen von \(\Omega\) nennt man \(\mathcal{M}_{\mu^{\ast}}\).

\(A\) heißt messbar bzgl. \(\mu^{\ast}\), falls \(A\) abzählbare Vereinigung von
endlich messbaren Mengen bzgl. \(\mu^{\ast}\) ist.  Die Mengensystem der
messbaren Teilmengen von \(\Omega\) nennt man \(\mathcal{A}_{\mu^{\ast}}\).

[Maßerweiterungssatz von Caratheodory]  Seien \(R \subseteq P(\Omega)\) Ring und \(\mu\)
Prämaß auf \(R\), sodass \(R\) die Menge \(\Omega\) bzgl. \(\mu\) ausschöpft und
\(\mu^{\ast}\) das durchh \(\mu\) gegebene äußere Maß auf \(\Omega\).  Dann ist die
Menge \(\mathcal{A}_{\mu^{\ast}}\) der bzgl. \(\mu^{\ast}\) messbaren Mengen eine \(\sigma\)-Algebra
und \(\mu^{\ast}\) ein Maß auf \(\mathcal{A}_{\mu^{\ast}}\).

Korollar.  Sei \(\lambda\) der elementargeometrische Inhalt auf \(\mathbb{R}^{n}\).  Dann
ist das dadurch auf dem \(\mathbb{R}^{n}\) definierte äußere Maß \(\lambda^{\ast}\)
\begin{itemize}
\item ein Maß auf der \(\sigma\)-Algebra \(\mathcal{A}_{\lambda^{\ast}}\) der bzgl. \(\lambda^{\ast}\)
  messbaren Mengen (genannt Lesbesgue-messbare Mengen), das sogenannte
  Lebesgue-Maß \(\lambda^{n} = \lambda = \lambda^{\ast} \vert_{\mathcal{A}_{\lambda^{\ast}}}\).
\item ein Maß auf der \(\sigma\)-Algebra \(\mathcal{B}\) der Borel-Mengen, das
  Lebesgue-Borelsche Maß \(\lambda^{n} = \lambda = \lambda^{\ast} \vert_{\mathcal{B}}\).
\end{itemize}

[Nullmenge]

Sei \((\Omega, \mathcal{A}, \mu)\) ein Maßraum.  Eine Menge \(N \in \mathcal{A}\) heißt Nullmenge,
falls \(\mu(N) = 0\).  Eine Menge heißt vernachlässigbar, falls es eine
Nullmenge \(N \in \mathcal{A}\) gibt, sodass \(T \subseteq N\).  \(\mu\) bzw. \((\Omega, \mathcal{A}, \mu)\) heißt
vollständig, falls jede vernachlässigbare Menge eine Nullmenge ist,
d.h. falls  für alle Nullmenge \(N \in \mathcal{A}\) gilt: für Teilmengen \(T \subseteq N\)
ist Element von \(\mathcal{A}\) und \(\mu(T) = 0\).
\end{document}

