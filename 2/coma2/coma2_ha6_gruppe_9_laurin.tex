% !TeX program = lualatex
%%% TeX-engine: luatex

\documentclass[draft,a5paper]{article}
\usepackage[margin=2cm]{geometry}
\linespread{1.2}

\setlength{\headheight}{15pt}
\usepackage{fancyhdr}
\pagestyle{fancy}
\lhead{Yiwen Yang 466096, Yuchen Guo 480788, Qing Wang 458040}
\rhead{HA 6, CoMa 2, \\ Laurin, Gruppe 9}

% set language to german
\usepackage[ngerman]{babel}

\usepackage[amsthm,vvarbb,varg]{newtx}

% used for integral with respect to a variable x
% like \int \dd{x}
% \usepackage{physics}

% theorem environment used in this document
\theoremstyle{remark}
\newtheorem*{beh}{Behauptung}
\newtheorem*{lem}{Lemma}

\newcommand{\envert}[1]{\left\lvert#1\right\rvert}
\let\abs=\envert

% pseudocode
\usepackage[commentColor=black]{algpseudocodex}

\begin{document}
\subsection*{3. Aufgabe (Iterative \texttt{Inorder-Tree-Walk} mit Stack)}

Wir schreiben einen iterativen Algorithmus \textsc{Inorder-Tree-Walk}
in Pseudocode.  Wir nehmen an, dass einen Stack \textit{S} mit den
Methoden \textsc{insert}, \textsc{pop}, \textsc{peek} und
\textsc{isEmpty} existiert.

\begin{algorithmic}
  \State \textbf{function} \textsc{Insert-Node-to-Stack}(\textit{current-node}, \textit{S})
  \While{\textit{current-node} \(\ne\) \textsc{None}}
  \State \textsc{insert}(\textit{S}, \textit{current-node})
  \State \textit{current-node} \(\gets\) \textit{current-node.left}
  \EndWhile

  \State \textbf{function} \textsc{Inorder-Tree-Walk}(\textit{T})

  \State \textit{current-node} \(\gets\) \textit{T.root}

  \State \textsc{Insert-Node-to-Stack}(\textit{current-node}, \textit{S})

  \If{\textit{current-node} \(=\) \textsc{None} \textbf{and}
    \textsc{isEmpty}(\textit{S}) \(=\) \textsc{False}}
  \State \textsc{print}(\textsc{pop}(\textit{S}))
  \State \textit{current-node} \(\gets\) \textit{current-node.right}
  \State \textsc{Insert-Node-to-Stack}(\textit{current-node}, \textit{S})
  \EndIf

\end{algorithmic}



\begin{beh}
  Der Algorithmus ist korrekt.
\end{beh}

\begin{proof}
  Am Anfang ist der Stack \textit{S} leer.  Während der ersten Aufruf
  von \textsc{Insert-Node-to-Stack} werden alle Nodes an der linken
  Seite zum Stack eingefügt.  Wegen der Suchbaum-Eigenschaft ist der
  letzte Node im Stack genau der kleinste.  Wir poppen dies, drücken
  der Wert aus, und im Fall des Existenz von
  \textit{current-node.right}, print der Baum der rechten Seite.
  Wegen der Suchbaum-Eigenschaft ist das Resultat sortiert.
\end{proof}
\end{document}