% !TeX program = lualatex
%%% TeX-engine: luatex

\documentclass[draft,a5paper]{article}
\usepackage[margin=2cm]{geometry}
\linespread{1.2}

% set language to german
\usepackage[ngerman]{babel}

% load math packages and unicode support
% must in this order
\usepackage{amsmath,mathtools,fontspec}
\usepackage{amsthm}
\usepackage{unicode-math,luatex85}

% used for integral with respect to a variable x
% like \int \dd{x}
%\usepackage{physics}

% set fonts
\setmainfont{STIX Two Text}

% theorem environment used in this document
\theoremstyle{remark}
\newtheorem*{beh}{Behauptung}
\newtheorem*{lem}{Lemma}

%    \interval is used to provide better spacing after a [ that
%    is used as a closing delimiter.
\newcommand{\interval}[1]{\mathinner{#1}}

%%
%% Julia definition (c) 2014 Jubobs
%%
\usepackage{listings}

\lstdefinelanguage{Julia}%
{morekeywords={function,for,in,end,while,true,false,
    while,else,elseif,return,mutable,struct,Vector,Union},
  sensitive=true,%
  morecomment=[l]\#,%
  morestring=[s]{"}{"},%
  morestring=[m]{'}{'},%
}[keywords,comments,strings]%

\lstset{%
  basicstyle       = \footnotesize,
  numbers          = left,
  language         = Julia,
  showstringspaces = false,
  frame            = lines,
}

% pseudocode
\usepackage[commentColor=black]{algpseudocodex}

%    Enclose the argument in vert-bar delimiters:
\newcommand{\envert}[1]{\left\lvert#1\right\rvert}
\let\abs=\envert

\author{Yiwen Yang 466096, Yuchen Guo 480788, Qing Wang 458040}
\date{\today}
\title{HA 3, CoMa 2 -- Gruppe 9, Laurin Geyer}

\begin{document}
\maketitle

\newpage

\subsection*{1. Aufgabe (Coma-Heaps)}


\subsection*{4. Aufgabe (Algorithmus von Ford)}

Es sei \(G = (V, E, \Psi)\) ein einfacher gerichteter Graph mit
Kantenlängen \(c_{e}\), \(e \in E\) und der Algorithmus von Ford wie
folgt definiert.

\begin{algorithmic}
  \For{\(v \in V\)}
    \State \(y_{v} \gets \infty\)
    \State \(p(V) \gets \text{None}\)
  \EndFor

  \State \(y_{a} \gets 0\)

  \While{\(\exists (v, w) \in E\) \textbf{with}
      \(y_{w} > y_{v} + c_{(v, w)}\)}
    \State \(y_{w} \gets y_{v} + c_{(v, w)}\)
    \State \(p(w) \gets v\)
  \EndWhile
\end{algorithmic}

\begin{beh}
  Es sei \(c\) ganzzahlig und es gibt keine negativen Kreise.  Der
  Algorithmus terminiert nach höchstens \(C \cdot \envert{V}^{2}\)
  Iterationen, wobei \(C \coloneq 2 \cdot \max_{e \in E}{\envert{c_{e}}} + 1\) sei.
\end{beh}

\begin{proof}
  Zuerst bemerken wir, dass die erste \textbf{for}-Schleife genau
  \(\envert{V}\) Iterationen benötigt.  In der \textbf{while}-Schleife
  müssen wir die Existenz von Paare \((v, w)\) mit \(v, w \in V\)
  prüfen und zwar in dieser Reihenfolge:
  \begin{algorithmic}
    \If{\((a, a) \in E\) \textbf{with} \(y_{a} > y_{a} + c_{(a, a)}\)}
      \State False
    \EndIf
    \If{\((a, b) \in E\) \textbf{with} \(y_{b} > y_{a} + c_{(a, b)}\)}
      \State False
    \EndIf
    \If{\((a, c) \in E\) \textbf{with} \(y_{c} > y_{a} + c_{(a, c)}\)}
      \State False
    \EndIf
    \If{\((a, d) \in E\) \textbf{with} \(y_{d} > y_{a} + c_{(a, d)}\)}
      \State True
      \State \(y_{d} \gets y_{a} + c_{(a, d)}\)
      \State \(p(d) \gets a\)
    \EndIf
    \LComment{Die \textbf{while}-Schleife startet erneut vom \((a, a)\) an.}
    \If{\((a, a) \in E\) \textbf{with} \(y_{a} > y_{a} + c_{(a, a)}\)}
      \State False
    \EndIf
    \If{\((a, b) \in E\) \textbf{with} \(y_{b} > y_{a} + c_{(a, b)}\)}
      \State False
    \EndIf
    \If{\((a, c) \in E\) \textbf{with} \(y_{c} > y_{a} + c_{(a, c)}\)}
      \State False
    \EndIf
    \State \(\vdots\)
    \LComment{In der letzten Aufruf von \textbf{while}-Schleife müssen alle Paare
      \((v, w)\) geprüft werden mit dem Resultat "`False"'. Dieser
      Schritt benötigt \(\envert{V}^{2}\)-mal Vergleich.}
    \If{\((a, a) \in E\) \textbf{with} \(y_{a} > y_{a} + c_{(a, a)}\)}
      \State False
    \EndIf
    \State \(\vdots\)
    \If{\((z, z) \in E\) \textbf{with} \(y_{z} > y_{z} + c_{(z, z)}\)}
      \State False
    \EndIf
    \LComment{Die \textbf{while}-Schleife terminiert.}
  \end{algorithmic}
  Wir können nicht feststellen, warum alle bis auf letzte Aufruf von
  \textbf{while}-Schleife insgesamt höchstens
  \(2 \cdot \max_{e \in E}{\envert{c_{e}}} \cdot \envert{V}^{2}\) Vergleich
  benötigt.
\end{proof}
\end{document}