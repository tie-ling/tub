% !TeX program = lualatex
%%% TeX-engine: luatex

\documentclass[draft,a5paper]{article}
\usepackage[margin=2cm]{geometry}
\linespread{1.2}

\usepackage{fancyhdr}
\pagestyle{fancy}
\lhead{Yiwen Yang 466096, Yuchen Guo 480788, Qing Wang 458040}
\rhead{HA 7, CoMa 2, Laurin, Gruppe 9}

% set language to german
\usepackage[ngerman]{babel}

\usepackage[amsthm,vvarbb,varg]{newtx}

% used for integral with respect to a variable x
% like \int \dd{x}
% \usepackage{physics}

% theorem environment used in this document
\theoremstyle{remark}
\newtheorem*{beh}{Behauptung}
\newtheorem*{lem}{Lemma}

\newcommand{\envert}[1]{\left\lvert#1\right\rvert}
\let\abs=\envert

\begin{document}
\subsection*{1. Aufgabe (Extremale AVL-Bäume)}
Zuerst ein paar Definitionen vorweg.

\begin{itemize}
\item Ein binärer Suchbaum \(T\) heißt
  \textbf{AVL-Baum}, falls \(\abs{\beta(v)} \le 1\) für alle Knoten
  \(v \in T\), d.h., falls alle Knoten balanciert ist.

\item Ein AVL-Baum \(T\) heißt
  \textbf{extremaler AVL-Baum} der Höhe \(h\), falls er die minimale
  Anzahl von Knoten enthält, sodass er noch ein AVL-Baum der Höhe
  \(h\) ist.

\item Ein Knoten vom Grad \(1\) in einem Baum heißt
  \textbf{Blatt.}
\end{itemize}

\subsubsection*{1. Aufgabe, Teil a}
\begin{beh}
  Die Blätteranzahl jedes extremale AVL-Baum der Höhe \(h\) ist
  minimal.
\end{beh}

\begin{proof}
  durch Widerspruch.  Angenommen, es existiert ein extremale AVL-Baum
  \(T\) der Höhe \(h\), sodass die Blätteranzahl von \(T\) nicht
  minimal ist, d.h., es existiert mindestens ein Blatt von \(T\), wenn
  man dieses Blatt löscht, ist \(T'\) wieder ein AVL-Baum der Höhe
  \(h\).  Damit haben wir einen AVL-Baum mit weniger Knoten gefunden,
  im Widerspruch zur Definition von extremaler AVL-Baum.
\end{proof}

\subsubsection*{1. Aufgabe, Teil b}

\begin{beh}
  Sei \(T\) ein AVL-Baum der Höhe \(h\) mit minimaler Blätteranzahl.
  Dann ist \(T\) extremal.
\end{beh}

\begin{proof}
  ???
\end{proof}

\subsubsection*{1. Aufgabe, Teil c}
\begin{lem}
  Die Knotenzahl \(n(h)\) eines AVL-Baumes der Höhe \(h\) erfüllt
  \[
    n(h) \ge \Phi^{h}
  \] mit \(\Phi \coloneq \frac{1+\sqrt{5}}{2}\).
\end{lem}

\begin{proof}
  In der Vorlesung haben wir festgestellt, dass
  \(n(h) \ge \sqrt{2}^{h}\) gilt. Wir suchen nach einer genauere Abschätzung
  in der Gestalt \(n(h) \ge p^{h}.\) Es gilt
  \[n(h) = 1 + n(h - 1) + n(h-2).\]  Da wir nach einer untere Schranke
  suchen, konnen wir das Konstant \(1\) weglassen und erhalten wir \[
    n(h) \ge n(h-1) + n(h-2) \] und daher \[p^{h} \ge p^{h-1} + p^{h-2}.\]
  Daraus erhalten wir \[p^{2} - p - 1 = 0,\] daraus folgt \(p =
  (1+\sqrt{5})/2\) und \[ n(h) \ge \Phi^{h}.\]
\end{proof}

\begin{beh}
  Ein AVL-Baum \(T\) mit \(n \ge 1\) Knoten hat Höhe \(h(T) \le (\log
  \Phi)^{-1} \log n\).
\end{beh}

\begin{proof}
  Diese folgt sofort aus Lemma.
\end{proof}



\end{document}