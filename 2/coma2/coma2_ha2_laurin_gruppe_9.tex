\documentclass[draft,a5paper]{article}
\usepackage[ngerman]{babel}
\usepackage[margin=2cm]{geometry}
\usepackage{amsmath,mathtools,fontspec,amssymb,mathtools}

% for theorems and lemma
\usepackage{amsthm}
\newtheorem*{beh}{Behauptung}
\newtheorem*{lem}{Lemma}

% times new roman font
\usepackage{newtx}

\linespread{1.2}

% multicol
\usepackage{multicol}

\author{Yiwen Yang 466096, Yuchen Guo 480788, Qing Wang 458040}
\date{\today}
\title{HA 1, CoMa 2 -- Gruppe 9, Laurin Geyer}

% syntax highlighting
\usepackage{minted}

% tikz pictures


\begin{document}
\maketitle

\newpage

\subsection*{1. Aufgabe. (Stack mit Queue)}
Wir implementieren ein \texttt{julia}-Konstruktor \texttt{Stack} mit
den Methoden \texttt{stack\textunderscore insert},
\texttt{stack\textunderscore pop} und \texttt{stack\textunderscore
  isEmpty}.  Wir nehmen an, dass ein Konstruktor \texttt{Queue} mit den
folgenden Methoden existiert, vgl. Folie 25.

\begin{minted}[frame=lines,fontsize=\footnotesize,linenos,]{julia}
## Konstruktor Queue von Folie 25
isEmpty(Q::Queue)
insert(Q::Queue, x)
pop(Q::Queue)
peek(Q::Queue)
## end of Konstruktor Queue von Folie 25

struct Stack
    q0::Queue
    q1::Queue
end

stack_isEmpty(S::Stack) =
    isEmpty(S.q0) && isEmpty(S.q1)

function stack_insert(S::Stack, x)
    insert(S.q1, x)
end

function move_to_other_queue(q0::Queue, q1::Queue)
    result = pop(q0)
    while ! isEmpty(q0)
        insert(q1, result)
        result = pop(q0)
    end
    return result
end

function stack_pop(S::Stack)
    if stack_isEmpty(S)
        return
    end
    if isEmpty(S.q1)
        return move_to_other_queue(S.q0, S.q1)
    else
        return move_to_other_queue(S.q1, S.q0)
    end
end
\end{minted}

\newpage

\begin{beh}
  Für die Methode \texttt{stack\textunderscore insert} werden im
  schlechtesten Fall bei \(n\) Elementen \(1\)-mal \texttt{insert}
  Operation benötigt.

  Für die Methode \texttt{stack\textunderscore pop} werden im
  schlechtesten Fall \(n\)-mal \texttt{insert} und \(n\)-mal
  \texttt{pop} Operationen
  benötigt.
\end{beh}

\begin{proof}
  Aus der Definition von der Methode \texttt{stack\textunderscore
    insert} folgt, dass nur \(1\)-mal \texttt{insert} Operation
  benötigt wird.

  Andereseits wird bei jedem Aufruf von \texttt{stack\textunderscore
    pop} im schlechtesten Fall \(n\)-mal \texttt{insert} und \(n\)-mal
  \texttt{pop} benötigt.  Denn, wir müssen jeder Element einer
  Queue \texttt{q0} "`poppen"' und das Element in die andere Queue \texttt{q1} einfügen.
\end{proof}

\subsection*{2. Aufgabe. (Catalan-Zahlen, Teil 1)}

Die Catalan-Zahlen sind durch \(C_{0} \coloneq 1\) und die rekursive Regel
\begin{align*}
  C_{n} \coloneq \sum_{k=0}^{n-1}{C_{k} \cdot C_{n-1-k}}
\end{align*}
für alle \(n \ge 1\) definiert.

\subsubsection*{2. Aufgabe, Teil i.}

\begin{beh}
  Es gilt für alle \(n \ge 1\) dass
  \begin{align*}
    2 \cdot \sum_{k=0}^{n-1}{3^{k}} = 3^{n} - 1.
  \end{align*}
\end{beh}

\begin{proof}
  Wir bemühen uns um einen Beweis mittels vollständigen Induktion.
  Induktionsanfang: \(n = 1\).  Es gilt \(2 \cdot 3^{0} = 3^{1} - 1 = 2
  \).  Induktionsvoraussetzung: sei \(n \ge 1\) festgewählt.  Es gelte die
  Behauptung für \(n\).  Induktionsschritt: es gilt
  \begin{align*}
    2 \cdot \sum_{k=0}^{(n+1)-1}{3^{k}}
    = \left[2 \cdot \sum_{k=0}^{n-1}{3^{k}}\right] + 2 \cdot 3^{n}
    = \left[3^{n} - 1\right] + 2 \cdot 3^{n}
    = 3^{n+1} -1.
  \end{align*}
  Damit gilt die Behauptung.
\end{proof}

\newpage

\subsubsection*{2. Aufgabe, Teil ii.}

\begin{minted}[frame=lines,fontsize=\footnotesize,linenos]{julia}
function Catalan(n::Int)
    if n == 0
        return 1
    end

    result = 0
    for x in 0:n-1
        # Die Anzahl von Additionen und Multiplikationen
        # sind gleich
        result += Catalan(x) * Catalan(n-1-x)
    end

    return result
end
\end{minted}

\begin{beh}
  Es gilt für die Laufzeit des Codes \(T(n)\) dass
  \(T(n) \in \Theta(3^{n})\).
\end{beh}

\begin{proof}
  Wir berechnen die gesamte Anzahl und Additionen und
  Multiplikationen.  Wir bemerken, dass die Anzahl von Additionen und
  Multiplikationen gleich sind.  Daher betrachten wir nur die Anzahl
  von Multiplikationen.

  Wir zeigen, dass in der Berechnung von \(C_{n}\) genau
  \(\sum_{i=1}^{n-1}{3^{i}}\)-mal Multiplikationen benötigt wird, mittels
  vollständige Induktion.  Induktionsanfang, \(n=1\).  Dann gilt
  \begin{align*}
    C_{1} &= C_{0} \cdot C_{0}, &&\sum_{i=1}^{n-1}{3^{i}} = 3^{0} = 1.
  \end{align*}
  Induktionsvoraussetzung.  Sei \(n \in \mathbb{N}\) fest gewählt.  Es gelte die
  Behauptung für alle \(i \le n - 1\).  Induktionsschritt, wir berechnen die
  Anzahl von Multiplikationen für \(C_{n}\).  Es gilt
  \begin{align*}
    C_{n} &= \sum_{k=0}^{n-1}{C_{k} \cdot C_{n-1-k}},
    &&n+2\sum_{i=1}^{n-1}{M_{C_{i}}} =
       n+2\sum_{i=1}^{n-1}{\sum_{j=0}^{i-1}{3^{j}}}
       = \sum_{i=1}^{n-1}{3^{i}}
  \end{align*}
  wo \(M_{C_{i}}\) für die Anzahl von Multiplikationen in der
  Berechnung von \(C_{i}\) steht.

  Damit haben wir gezeigt, dass die Anzahl von Multiplikationen bei
  der Berechnung von \(C_{n}\) genau \(\sum_{i=1}^{n-1}{3^{i}}\)-mal ist,
  und die gesamte Rechnungsoperationen genau zweimal dieser Anzahl
  ist.  Wegen \textbf{Teil i} gilt \(T(n) = 3^{n} - 1\). Damit gilt
  \(T(n) \in \Theta(3^{n})\).

  Eine übersichtlichere Darstellung ist wie folgt.  Die Spalten rechts
  sind die Anzahl von Multiplikationen.
  \begin{align*}
    C_{1} &= C_{0} \cdot C_{0}, &&3^{0}, &&1;\\
    C_{2} &= C_{0} \cdot C_{1} + C_{1} \cdot C_{0},
                            &&3^{0} + 3^{1},
                                     &&2 + 2 \cdot 3^{0}_{C_{1}}; \\
    C_{3} &= C_{0} \cdot C_{2} + C_{1} \cdot C_{1} + C_{2} \cdot C_{0},
                            &&3^{0} + 3^{1} + 3^{2},
                               &&3 + 2 \cdot 3^{0}_{C_{1}} + 2 \cdot
                                  (3^{0}+3^{1})_{C_{2}};\\
          & &&\vdots &&\\
    C_{n} &= \sum_{k=0}^{n-1}{C_{k} \cdot C_{n-1-k}}, &&\sum_{i=1}^{n-1}{3^{i}}, &&n+2\sum_{i=1}^{n-1}{\sum_{j=0}^{i-1}{3^{j}}}.
  \end{align*}
\end{proof}

\subsection*{3. Aufgabe (Standard-Young-Schema)}

\subsubsection*{3. Aufgabe, Teil i.}

\begin{align*}
  \begin{bmatrix}
    1 & 3 & 4 & \infty \\
    5 & 8 & 10 & \infty \\
    12 & 14 & 21 & \infty \\
    \infty & \infty & \infty & \infty
  \end{bmatrix}
\end{align*}

\subsubsection*{3. Aufgabe, Teil ii.}

Wir entwerfen einen Algorithmus in \texttt{julia} mit Laufzeit
\(O(n + m)\), der aus einem Standard-Young-Schema \(Y\) mit
\(m \times n\) Einträge ein minimales Element löscht und anschließend durch
Vertauschen von Einträgen die Monotonieeigenschaften wiederherstellt.

\begin{minted}[frame=lines,fontsize=\footnotesize,linenos]{julia}
function del_min_element(Y::Matrix)
    m, n = size(Y)
    x = 1
    y = 1

    while x < n && y < m
        if Y[x, y+1] < Y[x+1, y]
            Y[x, y] = Y[x, y+1]
            y += 1
            continue
        end
        if Y[x, y+1] > Y[x+1, y]
            Y[x, y] = Y[x+1, y]
            x += 1
            continue
        end
    end
    if x == n
        while y < m
            Y[x, y] = Y[x, y+1]
            y += 1
        end
    else
        while x < n
            Y[x, y] = Y[x+1, y]
            x += 1
        end
    end
    return Y
end
\end{minted}

\begin{beh}
  Der Algorithmus ist korrekt und es gilt \(T(m \times n) \in O(n + m)\).
\end{beh}

\begin{proof}
  Zuerst bemerken wir, dass das minimale Element sich wegen der
  Monotonieeigenschaft zum Beginn in der Position \((1, 1)\) befindet.
  Wir löschen dieser Eintrag, d.h., ersetzen durch \(\infty\).  Wir
  bezeichnen die Koordinaten von \(\infty\) mit \((x, y) \leftarrow (1, 1)\).

  Danach prüfen wir, welcher der zwei benachbarten Einträge in der
  Position \((x, y + 1)\) und \((x + 1, y)\) kleiner ist und tauschen
  den Wert dieses Eintrags mit den Wert des Eintrags \(\infty\) in der
  Position \((x, y)\).  Damit wird die Monotonieeigenschaft in der
  Position \((x, y)\) erfüllt.  Anschließend weisen wir die neue
  Position von \(\infty\) zu.

  Durch \((m - 1) + (n - 1)\)-mal Anwendung des obengenannten
  Algorithmus mit einer \texttt{while}-Schleife befindet sich der
  \(\infty\)-Eintrag nun entweder in der letzten Spalte oder in der letzten
  Zeile der Matrix.  Wir müssen nur den \(\infty\)-Eintrag in der Position
  \((m, n)\) bringen, um die Monotonieeigenschaft zu erfüllen.
  Dieser Schritt benötigt im schlechtesten Fall \(\max(m, n)\)-mal
  Zuweisungsoperationen.  Damit gilt
  \begin{align*}
    T(m \times n) = a[(m - 1) + (n - 1)] + \max(m, n) \in O(m + n).
  \end{align*}
  Damit gilt die Behauptung.
\end{proof}

\subsubsection*{3. Aufgabe, Teil iii.}

Wir entwerfen einen Algorithmus in \texttt{julia} mit Laufzeit
\(O(n + m)\), der in ein Standard-Young-Schema \(Y\) mit
\(m \times n\) Einträge \textemdash{} darunter genau ein leerer Eintrag in
der Position \((m, n)\) \textemdash{} ein Element (der leere Eintrag)
durch ein Integer ersetzt und anschließend durch Vertauschen von
Einträgen die Monotonieeigenschaften wiederherstellt.


\end{document}