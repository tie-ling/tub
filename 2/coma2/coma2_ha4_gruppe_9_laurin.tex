% !TeX program = lualatex
%%% TeX-engine: luatex

\documentclass[draft,a5paper]{article}
\usepackage[margin=2cm]{geometry}
\linespread{1.2}

% set language to german
\usepackage[ngerman]{babel}

% load math packages and unicode support
% must in this order
\usepackage{amsmath,mathtools,fontspec}
\usepackage{amsthm}
\usepackage{unicode-math,luatex85}

% used for integral with respect to a variable x
% like \int \dd{x}
%\usepackage{physics}

% set fonts
\setmainfont{STIX Two Text}

% theorem environment used in this document
\theoremstyle{remark}
\newtheorem*{beh}{Behauptung}
\newtheorem*{lem}{Lemma}

%    \interval is used to provide better spacing after a [ that
%    is used as a closing delimiter.
\newcommand{\interval}[1]{\mathinner{#1}}

% pseudocode
\usepackage[commentColor=black]{algpseudocodex}

%    Enclose the argument in vert-bar delimiters:
\newcommand{\envert}[1]{\left\lvert#1\right\rvert}
\let\abs=\envert

\author{Yiwen Yang 466096, Yuchen Guo 480788, Qing Wang 458040}
\date{\today}
\title{HA 4, CoMa 2 -- Gruppe 9, Laurin Geyer}

\begin{document}
\maketitle

\newpage

\subsection*{1. Aufgabe}

Sei \(G = (V, E)\) ein ungerichteter Graph und seien
\(c \in \mathbb{R}^{E}\) Kantenlängen.

\subsubsection*{1. Aufgabe, Teil i}

\begin{beh}
  Es gibt genau dann zwischen allen Paaren \(v, w \in V\) einen
  kürzesten Pfad, wenn \(G\) zusammenhängend und \(c_{e} \ge 0\) für
  alle \(e \in E\) ist.
\end{beh}

\begin{proof}
  Hinrichtung.  Angenommen, es gibt zwischen allen Paaren \(v, w \in V\)
  einen kürzesten Pfad.  Dann existiert mindestens einen Pfad zwischen
  alle Paaren \(v, w \in V\).  Damit ist \(G\) zusammenhängend.  Es gilt
  \(c_{e} \ge 0\) für alle \(e \in E\), denn angenommen, es existiert ein
  \(e \in E\) mit \(c_{e} < 0\).  Da \(G\) ungerichtet ist, können wir
  diese Pfad \(v_{0}, e, v_{1}\) unendlich Mal treten und erhalten wir
  die kleinste Länge eines Pfades \(P\) mit \(- \infty\).  Damit muss
  \(c_{e} \ge 0\) für alle \(e \in E\) sein.

  Rückrichtung.  Angenommen, der Graph \(G\) ist zusammenhängend und
  \(c_{e} \ge 0\) für alle \(e \in E\).  Weil \(G\) zusammenhängend ist,
  existiert mindestens einen Pfad zwischen allen Paaren
  \(v, w \in V\).  Weil \(c_{e} \ge 0\) für alle \(e \in E\), kann die Länge
  aller Pfade zwischen \(v, w \in V\) absteigend sortiert werden mit
  \(c(P) \ge 0\) für alle \(P\).  Damit existiert einen kürzesten Pfad
  zwischen allen Paaren \(v, w \in V\).
\end{proof}

\subsubsection*{1. Aufgabe, Teil ii}

\begin{beh}
  Sei \(G\) zusammenhängend und \(c_{e} > 0\) für alle \(e \in E\).  Die
  Abbildung
  \begin{align*}
    d\colon V \times V \to \mathbb{R}; \quad (v, w) \mapsto \min \{c(P) \mid P~\text{\(v\)-\(w\)-Pfad}\}
  \end{align*}
  ist eine Metrik.
\end{beh}

\begin{proof}
  Sei \(x, y, z \in V\) beliebig gewählt.

  Wegen \(c_{e} > 0\) für alle \(e \in E\) ist \(d(x, y) > 0\) für alle
  \(x \ne y\) und \(d(x, y) = 0\) für alle \(x = y\).  Damit ist die
  positive Definitheit erfüllt.  Weil \(G\) ungerichteter Graph ist,
  ist der kürzeste Pfad zwischen \(v, w\) auch der kürzeste Pfad
  zwischen \(w, v\).  Damit ist die Symmetrie erfüllt.  Die
  Dreiecksungleichung ist erfüllt, denn, angenommen, die
  Dreiecksungleichung gelte nicht.  Dann ist
  \(d(x, y) > d(x, z) + d(z, y)\), also der \(x,z,y\) Pfad ist kürzer
  als \(x,y\) Pfad im Widerspruch zur Voraussetzung dass \(d(x,y)\)
  das Minimum von Länge aller \(x, y\) Pfaden ist.
\end{proof}

\subsubsection*{1. Aufgabe, Teil iii}

\begin{beh}
  Sei \(G\) zusammenhängend und \(c \in \mathbb{R}^{E}\) beliebig.  Für \(v, w \in
  V\) sei \(d'(v, w)\) die Länge eines kürzesten Weges von \(v\) nach
  \(w\).  Positive Definitheit ist nicht erfüllt.
\end{beh}

\begin{proof}
  Positive Definitheit ist nicht erfüllt.  Denn, sei
  \(c \in \mathbb{R}^{E}\) die konstante Funktion \(0\).  Dann ist
  \(d'(x, y) = 0\) für alle \(x, y \in V\) und wir können \(x = y\)
  nicht folgern.  Weil \(G\) ungerichteter Graph ist, ist der kürzeste
  Weg zwischen \(v, w\) auch der kürzeste Weg zwischen \(w, v\).
  Damit ist die Symmetrie erfüllt.  Die
  Dreiecksungleichung ist erfüllt, denn, angenommen, die
  Dreiecksungleichung gelte nicht.  Dann ist
  \(d(x, y) > d(x, z) + d(z, y)\), also der \(x,z,y\) Pfad ist kürzer
  als \(x,y\) Pfad im Widerspruch zur Voraussetzung dass \(d(x,y)\)
  das Minimum von Länge aller \(x, y\) Pfaden ist.
\end{proof}



\end{document}