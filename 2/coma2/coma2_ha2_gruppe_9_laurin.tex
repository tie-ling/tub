\documentclass[draft,a5paper]{article}
\usepackage[ngerman]{babel}
\usepackage[margin=2cm]{geometry}
\usepackage{amsmath,mathtools,fontspec,amssymb,mathtools}

% for theorems and lemma
\usepackage{amsthm}
\newtheorem*{beh}{Behauptung}
\newtheorem*{lem}{Lemma}

% times new roman font
\usepackage{newtx}

\linespread{1.2}

% multicol
\usepackage{multicol}

\author{Yiwen Yang 466096, Yuchen Guo 480788, Qing Wang 458040}
\date{\today}
\title{HA 2, CoMa 2 -- Gruppe 9, Laurin Geyer}

% syntax highlighting
\usepackage{minted}

% tikz pictures


\begin{document}
\maketitle

\newpage

\subsection*{1. Aufgabe. (Queue mit Stack)}
Wir implementieren ein \texttt{julia}-Konstruktor \texttt{Queue} mit
den Methoden \texttt{queue\textunderscore insert},
\texttt{queue\textunderscore pop} und \texttt{queue\textunderscore
  isEmpty}.  Wir nehmen an, dass ein Konstruktor \texttt{Stack} mit den
folgenden Methoden existiert, vgl. Folie 25.

\begin{minted}[frame=lines,fontsize=\footnotesize,linenos,]{julia}
## Konstruktor Stack von Folie 25
mutable struct Stack{T}
function isEmpty(Q::Stack)
function insert(Q::Stack, x)
function pop(Q::Stack)
function peek(Q::Stack)
## end of Konstruktor Stack von Folie 25

mutable struct Queue
    reversed::Stack
    not_reversed::Stack
    function Queue(data::Vector)
        return new(Stack([]), Stack(data))
    end
end

function queue_isEmpty(Q::Queue)
    return isEmpty(Q.reversed) && isEmpty(Q.not_reversed)
end

function queue_insert(Q::Queue, x)
    insert(Q.not_reversed, x)
    return Q
end


function queue_pop(Q::Queue)
    #
    # Case 1:
    ## if the reversed stack is not empty,
    ## pop the reversed stack first.
    #
    if ! isEmpty(Q.reversed)
        return pop(Q.reversed)

    #
    # Case 2:
    ## if the reversed stack of queue is empty,
    ## and not_reversed stack is not empty,
    ##
    ## pop all elements in not_reversed stack, then
    ## insert all but last popped element into reversed stack.
    ## Return the last popped element as result.
    #
    elseif ! isEmpty(Q.not_reversed)
        result = pop(Q.not_reversed)
        while ! isEmpty(Q.not_reversed)
            insert(Q.reversed, result)
            result = pop(Q.not_reversed)
        end
        return result
    else
    #
    # Case 3:
    ## Both stacks is empty. Then the queue is empty.
    #
        return "Queue is empty!"
    end
end
\end{minted}
\begin{beh}
  Für die Methode \texttt{queue\textunderscore insert} werden im
  schlechtesten Fall bei \(n\) Elementen \(1\)-mal \texttt{insert}
  Operation benötigt.

  Für die Methode \texttt{queue\textunderscore pop} werden im
  schlechtesten Fall \(n\)-mal \texttt{insert} und \(n\)-mal
  \texttt{pop} Operationen benötigt.
\end{beh}

\begin{proof}
  Aus der Definition von der Methode \texttt{queue\textunderscore
    insert} folgt, dass es nur \(1\)-mal \texttt{insert} Operation
  benötigt wird.

  Andereseits wird bei Aufruf von \texttt{queue\textunderscore
    pop} im schlechtesten Fall \(n\)-mal \texttt{insert} und \(n\)-mal
  \texttt{pop} benötigt.  Denn, wir müssen jeder Element des Stacks
  \texttt{q.not\textunderscore reversed} "`poppen"' und das Element
  im Stack \texttt{q.reversed} einfügen.
\end{proof}

\newpage

\subsection*{2. Aufgabe. (Catalan-Zahlen, Teil 2)}

Die Catalan-Zahlen sind durch \(C_{0} \coloneq 1\) und die rekursive Regel
\begin{align*}
  C_{n} \coloneq \sum_{k=0}^{n-1}{C_{k} \cdot C_{n-1-k}}
\end{align*}
für alle \(n \ge 1\) definiert.

\subsubsection*{2. Aufgabe, Teil i.}

Wir entwerfen eine \texttt{julia}-Funktion \texttt{Catalan2(n)}, die
die \(n\)-te Catalan-Zahl \(C_{n}\) mittels der Rekursionsformel
berechnet. Dabei wird \(C_{i}, 1 \le i \le n - 1\) höchstens einmal berechnet.

\begin{minted}[frame=lines,fontsize=\footnotesize,linenos]{julia}
function Catalan(n::Int, list_of_results::Vector{Union{Nothing, Int}})
    if n == 0
        return 1
    end

    if list_of_results[n] != nothing
        return list_of_results[n]
    end

    result = 0
    for x in 0:n-1
        # Die Anzahl von Additionen und Multiplikationen
        # sind gleich
        result += Catalan(x, list_of_results) * Catalan(n-1-x, list_of_results)
    end

    return result
end

function Catalan2(n::Int)
    if n == 0
        return 1
    end

    # initialize the list of n-results with "nothing" entries
    list_of_results = Vector{Union{Nothing, Int}}(nothing, n)

    for x in 1:n
        # fill the list of n-results, smallest number first
        list_of_results[x] = Catalan(x, list_of_results)
    end

    return last(list_of_results)
end
\end{minted}

\begin{beh}
  Es gilt für die Laufzeit des Codes \(T(n)\) dass
  \(T(n) \in \Theta(n^{2})\).
\end{beh}

\begin{proof}
  Wir berechnen die gesamte Anzahl von Additionen und
  Multiplikationen.  Wir bemerken, dass die Anzahl von Additionen und
  Multiplikationen gleich sind und alle andere Operationen eine
  lineare Anzahl bzgl. \(n\) haben.

  Die gesamte Anzahl von Aufrufe der \texttt{Catalan} Funktion ist
  durch die \texttt{for}-Schleife der \texttt{Catalan2} Funktion
  bestimmt.  Diese \texttt{for}-Schleife läuft wegen Definition
  \((n-1)\) Mal.

  Wir betrachten nun die Funktion \texttt{Catalan}. Die gesamt Anzahl
  von Multiplikation innerhalb dieser Funktion wird durch die
  \texttt{for}-Schleife bestimmt.  Wir bemerken, dass die rekursive
  Aufrufe von \texttt{Catalan} innerhalb der \texttt{for}-Schleife
  ohne Multiplikation oder Addition beendet, wegen der Existenz des
  Variables \texttt{list\textunderscore of\textunderscore results}.
  Daraus folgt, dass für jede Aufrufe von \texttt{Catalan} mit
  \((n-1)\) Mal Multiplikation und \(n-1\) Mal Addition beendet.
  Insgesamt ergibt sich
  \begin{align*}
    T(n) = an + (n-1)^{2} \in \Theta(n^{2}).
  \end{align*}
  Damit gilt die Behauptung.
\end{proof}

\subsubsection*{2. Aufgabe, Teil ii.}

Wir entwerfen eine \texttt{julia}-Funktion
\texttt{CatalanTriangle(n)}, die die \(n\)-te Catalan-Zahl \(C_{n}\)
mittels des Catalan'schen Dreiecks berechnet.  Wir speichern nur die
letzten zwei Zeilen des Dreiecks während des gesamten Aufrufs.  Jeder
Eintrag \(C(n, k)\) wird höchstens einmal berechnet, denn, beim Beginn
der äußeren \texttt{for}-Schleife wird jedes Mal die letzte Zeile
\texttt{line1} als neue Zeile \texttt{line0} kopiert.

\begin{minted}[frame=lines,fontsize=\footnotesize,linenos]{julia}
function CatalanTriangle(n::Int)
    # only the last two lines of the triangle are needed
    line0 = Vector{Union{Nothing, Int}}(nothing, n+1)
    line1 = Vector{Union{Nothing, Int}}(nothing, n+1)

    # be careful, julia array indices begin with 1 (number one)
    for x in 0:n-1
        line0 = copy(line1)
        line0[1] = 1;  line1[1] = 1
        line0[2] = x;  line1[2] = x+1
        for k in 2:x+1
            line1[k] = line1[k-1] + line0[k]
        end
        if x > 0
            line1[x + 2] = line1[x + 1]
        end
    end

    return line1[n]
end
\end{minted}

\end{document}