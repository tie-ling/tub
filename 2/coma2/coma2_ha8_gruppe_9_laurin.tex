% !TeX program = lualatex
%%% TeX-engine: luatex

\documentclass[draft,a5paper]{article}
\usepackage[margin=2cm]{geometry}
\linespread{1.2}

\usepackage{fancyhdr}
\pagestyle{fancy}
\lhead{Yiwen Yang 466096, Yuchen Guo 480788, Qing Wang 458040}
\rhead{HA 8, CoMa 2, Laurin, Gruppe 9}

% set language to german
\usepackage[ngerman]{babel}

\usepackage[amsthm,vvarbb,varg]{newtx}

% used for integral with respect to a variable x
% like \int \dd{x}
% \usepackage{physics}

% theorem environment used in this document
\theoremstyle{remark}
\newtheorem*{beh}{Behauptung}
\newtheorem*{lem}{Lemma}

\newcommand{\envert}[1]{\left\lvert#1\right\rvert}
\let\abs=\envert


\usepackage{tikz}

\begin{document}
\subsection*{1. Aufgabe}
\subsubsection*{1. Aufgabe, Teil a}
\begin{center}
  \usetikzlibrary {graphs,graphdrawing} \usegdlibrary {trees}
  \tikz \graph [tree layout,
  nodes={draw,circle}]
  { a/"5" -> {b/"4" -> {d/"3" -> {h/"2" -> k/"B", i/"1" -> l/"B"},
        e/"2" -> j/"A"}, c/"3" -> {f/"2" -> m/"A", g/"1" -> n/"A"}}};
\end{center}
In der zweiten Zeil sind die Ergebnisse nach person A entweder ein
Streichholz oder zwei Streichhölzer genommen hat.  Die dritten und
vierten Zeilen sind analog in der Reihenfolge A, B, A, B.

\newpage
\subsection*{4. Aufgabe}
Seien \(h_{1}, h_{2}\colon U \to \{0, 1, \ldots, m - 1\}\) Hash-Funktionen und
eine Doppel-Hash definiert mit
\[h(k, i) \coloneq (h_{1}(k) + i\cdot h_{2}(k)) \bmod m.\]

\begin{beh}
  Sei \(k \in U\) beliebig gewählt.  Die Funktion
  \[h'(i)\colon \{0, 1, \ldots, m - 1\} \to \{0, 1, \ldots, m - 1\}, \quad i \mapsto h(k, i) \]
  ist genau dann bijektiv, falls \(m\) und \(h_{2}(k)\) teilerfremd
  sind.
\end{beh}

\begin{proof}
  Hinrichtung.  Beweis mit Kontraposition.  Angenommen, \(m\) und
  \(h_{2}(k)\) sind nicht teilerfremd.  Wegen \(h_{2}(k) \le m - 1\)
  existiert solche natürliche Zahlen \(n, p, q\) größer als Eins und
  kleiner als \(m\) sodass \(np = m\) und \(nq = h_{2}(k)\).  Dann
  gilt \[ \frac{mq}{p} = h_{2}(k) \] und insbesondere
  \[ 0 \cdot h_{2}(k) \bmod m = [p \cdot h_{2}(k)] \bmod m = 0. \] Damit ist
  \(h'(0) = h'(p)\) für \( p \ne 0\) und die Funktion \(h'(i)\) ist
  nicht bijektiv.

  Rückrichtung.  Angenommen, \(m\) und \(h_{2}(k)\) sind teilerfremd.
  Wir betrachten die Paare
  \[i, j \in \{0, 1, \ldots, m - 1\},~ i \ne j\] mit
  \(h'(i) = h'(j)\).  Daraus folgt, dass
  \begin{align*}
    h'(i)
    &= h'(j) \\
    (h_{1}(k) + i\cdot h_{2}(k)) \bmod m
    &= (h_{1}(k) + j \cdot h_{2}(k)) \bmod m \\
    (i\cdot h_{2}(k)) \bmod m
    &= (j \cdot h_{2}(k)) \bmod m \\
    ((i - j) \cdot h_{2}(k)) \bmod m
    &= 0.
  \end{align*}
  Weil \(m\) und \(h_{2}(k)\) teilerfremd sind, muss \(i - j = m\)
  sein im Widerspruch zur Voraussetzung dass
  \(i, j \in \{0, 1, \ldots, m - 1\}\).
\end{proof}
\end{document}