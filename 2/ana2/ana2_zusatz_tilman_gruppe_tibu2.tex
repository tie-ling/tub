% !TeX program = lualatex
%%% TeX-engine: luatex

\documentclass[draft,a5paper]{article}
\usepackage[margin=2cm]{geometry}
\linespread{1.2}

\setlength{\headheight}{15pt}
\usepackage{fancyhdr}
\pagestyle{fancy}
\lhead{Yiwen Yang 466096, Yuchen Guo 480788, Meng Zhang 484981}
\rhead{Zusatz, Ana 2, TiBu 2}

% set language to german
\usepackage[ngerman]{babel}

\usepackage[amsthm,vvarbb,varg]{newtx}

% used for integral with respect to a variable x
% like \int \dd{x}
% \usepackage{physics}

% theorem environment used in this document
\theoremstyle{remark}
\newtheorem*{beh}{Behauptung}
\newtheorem*{lem}{Lemma}

% \interval is used to provide better spacing after a [ that
% is used as a closing delimiter.
\newcommand{\interval}[1]{\mathinner{#1}}

% Enclose the argument in vert-bar delimiters:
\newcommand{\envert}[1]{\left\lvert#1\right\rvert}
\let\abs=\envert

% Enclose the argument in double-vert-bar delimiters:
\newcommand{\enVert}[1]{\left\lVert#1\right\rVert}
\let\norm=\enVert

\newcommand{\enAngle}[1]{\left\langle#1\right\rangle}
\let\scp=\enAngle

\begin{document}
\subsection*{Aufgabe Z.1}
Es sei \(f\colon \mathbb{R}^{3} \to \mathbb{R}\) mit \(f(x, y, z) \coloneq \sin (x) \cos(y) e^{z}\).
\subsubsection*{Aufgabe Z.1, Teil i}
Wir bestimmen zuerst
\[[Df(x, y, z)] = [\cos(x) \cos(y) e^{z}, - \sin(x) \sin(y) e^{z},
  \sin(x) \cos(y) e^{z}]\]
und
\[H_{f}(x, y, z) =
  \begin{bmatrix}
    - \sin(x) \cos(y) e^{z}
    & - \cos(x) \sin(y) e^{z}
    & \cos(x) \cos(y) e^{z} \\
    - \cos(x) \sin(y) e^{z}
    & - \sin(x) \cos(y) e^{z}
    & - \sin(x) \sin(y) e^{z} \\
    \cos(x) \cos(y) e^{z}
    & - \sin(x) \sin(y) e^{z}
    & \sin(x) \cos(y) e^{z}
  \end{bmatrix}
\]
Demnach ist
\begin{align*}
  T_{2}((0,0,0);(x,y,z))
  &= f(0,0,0) + \scp{\nabla f(0,0,0),
  [x, y, z]
  } + \frac{1}{2} \scp{
  [x, y, z],
  H_{f}(0,0,0)
    [x, y, z]} \\
  &= 0 + \scp{[1,0,0], [x, y, z]} + \frac{1}{2}
    \scp{[x, y, z],
    \begin{bmatrix}
      0 & 0 & 1 \\
      0 & 0 & 0 \\
      1 & 0 & 0
    \end{bmatrix}
    [x, y, z]
    } \\
  &= x + \frac{1}{2} \scp{[x, y, z], [z, 0, x]} \\
  &= x + xz.
\end{align*}
\subsubsection*{Aufgabe Z.1, Teil ii}

\newpage

\subsection*{Aufgabe Z.2}
Sei \(f\colon \mathbb{R}^{n} \to \mathbb{R}\), \(f \in \mathrm{C}^{2}(\mathbb{R})\).
\begin{beh}
  Falls \(x' \in \mathbb{R}^{n}\) ein Minimierer von \(f\) ist, und \(D^{2}f\)
  stetig in \(x'\) ist, so gilt für alle \(v \in \mathbb{R}^{n}\) dass
  \(D^{2}f(x')(v, v) \ge 0\).
\end{beh}
\begin{proof}

\end{proof}

\subsection*{Aufgabe Z.3}
\subsubsection*{Aufgabe Z.3, Teil i}
\begin{proof}
  Es gilt wegen Quotientenkriterium dass
  \begin{align*}
    \partial_{i}f(x) = \underbrace{(x_{1} \cdot \ldots \cdot x_{n})^{\frac{1}{n+1}}}_{(1)}
    \underbrace{\frac{\frac{1}{n+1}\frac{1}{x_{i}}(1+x_{1}+\ldots+x_{n}) - 1}{(1+x_{1}+\ldots+x_{n})^{2}}}_{(2)}.
  \end{align*}
  \(\partial_{i}f(x)\) ist für alle \(x_{i} > 0\) wohldefiniert und stetig.
  Daraus folgt, dass \(f\) auf \(\interval{]0, \infty[}^{n}\) diffbar ist.
  Wir bemerken, dass der Faktor (1) und der Nenner des Faktors (2)
  stets größer als Null sind.  Die einzige Möglichkeit, die Gleichung
  \(\partial_{i}f(x) = 0\) für alle \(i\) zu erfüllen, ist der Zähler von (2)
  gleich Null.

  Wir suchen nach einer kritische Punkt \(x\), sodass für alle \(i\)
  der Zähler gleich Null ist, nämlich
  \[ 1 = \frac{1}{n+1}\frac{1}{x_{i}}(1+x_{1}+\ldots+x_{n}). \]
  Diese ist nur dann erfüllt, falls \(x_{i} = x_{0}\) für ein
  \(x_{0}\) und für alle \(i\) gilt.  Also
  \[n + 1 = \frac{1}{x_{0}}(1+nx_{0}).\]
  Daraus folgt, dass \(x_{0} = 1 = x_{i}\) für alle \(i\) und \(x =
  (1, \ldots, 1)\) ist der einzige kritische Punkt.
\end{proof}
\end{document}