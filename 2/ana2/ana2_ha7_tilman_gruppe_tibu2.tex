% !TeX program = lualatex
%%% TeX-engine: luatex

\documentclass[draft,a5paper]{article}
\usepackage[margin=2cm]{geometry}
\linespread{1.2}

\setlength{\headheight}{15pt}
\usepackage{fancyhdr}
\pagestyle{fancy}
\lhead{Yiwen Yang 466096, Yuchen Guo 480788, Meng Zhang 484981}
\rhead{HA 7, Ana 2, TiBu 2}

% set language to german
\usepackage[ngerman]{babel}

\usepackage[amsthm,vvarbb,varg]{newtx}

% used for integral with respect to a variable x
% like \int \dd{x}
% \usepackage{physics}

% theorem environment used in this document
\theoremstyle{remark}
\newtheorem*{beh}{Behauptung}
\newtheorem*{lem}{Lemma}

% \interval is used to provide better spacing after a [ that
% is used as a closing delimiter.
\newcommand{\interval}[1]{\mathinner{#1}}

% Enclose the argument in vert-bar delimiters:
\newcommand{\envert}[1]{\left\lvert#1\right\rvert}
\let\abs=\envert

% Enclose the argument in double-vert-bar delimiters:
\newcommand{\enVert}[1]{\left\lVert#1\right\rVert}
\let\norm=\enVert

\begin{document}
\subsection*{H 7.1}
Seien die Funktionen \(f, g\colon \mathbb{R}^{2} \to \mathbb{R}\) mit
\begin{align*}
  f(x, y) &=
  \begin{cases}
    0 & \text{falls } (x, y) = (0, 0), \\
    xy \cdot (x^{2} + y^{2})^{-\frac{1}{2}} & \text{sonst.}
  \end{cases}\\
  g(x, y) &=
  \begin{cases}
    0 & \text{falls } (x, y) = (0, 0), \\
    (x^{2} + y^{2}) \sin{[(x^{2}+y^{2})^{-\frac{1}{2}}]} & \text{sonst.}
  \end{cases}
\end{align*}
\subsubsection*{H 7.1.i}
Es gilt für alle \((x, y) \ne (0, 0)\) dass
\[
  \partial_{x}f(x, y) = \lim_{h \to 0}{\frac{f((x, y) + h(1, 0)) - f(x, y)}{h}}
  = (x^{2} + y^{2})^{-\frac{1}{2}} y -(x^{2} +
  y^{2})^{-\frac{3}{2}}x^{2}y,
\]
\[
  \partial_{y}f(x, y) = \lim_{h \to 0}{\frac{f((x, y) + h(0, 1)) - f(x, y)}{h}}
  = (x^{2} + y^{2})^{-\frac{1}{2}} x -(x^{2} +
  y^{2})^{-\frac{3}{2}}xy^{2}.
\]
\[
  \partial_{x}g(x, y) = 2x \sin[(x^{2} + y^{2})^{-\frac{1}{2}}] - x (x^{2} +
  y^{2})^{-\frac{1}{2}} \cos[(x^{2} + y^{2})^{-\frac{1}{2}}],
\]
\[
  \partial_{y}g(x, y) = 2y \sin[(x^{2} + y^{2})^{-\frac{1}{2}}] - y (x^{2} +
  y^{2})^{-\frac{1}{2}} \cos[(x^{2} + y^{2})^{-\frac{1}{2}}].
\]
Es gilt für \((x, y) = (0, 0)\) dass
\[
  \partial_{x}f(0, 0) = \lim_{h \to 0}{\frac{f((0, 0) + h(1, 0)) - f(0, 0)}{h}}
  = 0,
\]
\[
  \partial_{y}f(0, 0) = \lim_{h \to 0}{\frac{f((0, 0) + h(0, 1)) - f(0, 0)}{h}}
  = 0.
\]
\[
  \partial_{x}g(0, 0) = \lim_{h \to 0}{\frac{g((0, 0) + h(1, 0)) - g(0, 0)}{h}}
  = \lim_{h \to 0}{\frac{h^{2} \sin (h^{-1})}{h}} = \lim_{h \to 0}{h \sin
    (h^{-1})} = 0,
\]
\[
  \partial_{y}g(0, 0) = \lim_{h \to 0}{\frac{g((0, 0) + h(0, 1)) - g(0, 0)}{h}}
  = \lim_{h \to 0}{\frac{h^{2} \sin (h^{-1})}{h}} = \lim_{h \to 0}{h \sin
    (h^{-1})} = 0.
\]
\subsubsection*{H 7.1.ii}
Wir zeigen, dass die partielle Ableitung \(\partial_{x}f(x, y)\) in \((0,
0)\) stetig ist, indem wir zeigen, dass für alle Folgen
\((x_{n})_{n \in \mathbb{N}}\) mit \(\lim_{n \to \infty}{x_{n}} = 0\) gilt
\[\lim_{n \to \infty}{\partial_{x}f(x_{n}, 0)} = \partial_{x}f(0, 0) = 0.\]
Diese gilt wegen \(y = 0\) und
\(\partial_{x}f(x, y) = (x^{2} + y^{2})^{-\frac{1}{2}} y -(x^{2} +
y^{2})^{-\frac{3}{2}}x^{2}y.\)  Analog ist \(\partial_{y}f(x, y)\) stetig in
\((0, 0)\).

Wir zeigen, dass die partielle Ableitung \(\partial_{x}g(x, y)\) in \((0,
0)\) stetig ist, indem wir zeigen, dass für alle Folgen
\((x_{n})_{n \in \mathbb{N}}\) mit \(\lim_{n \to \infty}{x_{n}} = 0\) gilt
\[\lim_{n \to \infty}{\partial_{x}g(x_{n}, 0)} = \partial_{x}g(0, 0) = 0.\]
Diese gilt wegen \(y = 0\) und
\[
  \partial_{x}g(x, 0) = 2x \sin(x^{-1}) - \cos(x^{-1}),
\]
da \(\sin(x^{-1})\) und \(\cos(x^{-1})\) beschränkt sind.  Analog ist
\(\partial_{y}g(x, y)\) stetig in \((0, 0)\).

\subsubsection*{H 7.1.iii}
Die Funktion \(g(x, y)\) ist diffbar im Nullpunkt.  Denn, für alle
\((x, y) \to (0, 0)\) gilt
\[ \lim_{(x, y) \to (0, 0)}{\frac{g(x, y) - g(0, 0)}{\norm{(x, y)}}} =
  \lim_{(x, y) \to (0, 0)}{\frac{(x^{2} + y^{2})
      \sin{[(x^{2}+y^{2})^{-\frac{1}{2}}]}}{\norm{(x, y)}}}
\] und die Funktion \(\sin{[(x^{2}+y^{2})^{-\frac{1}{2}}]}\) ist
beschränkt.  Weil alle Normen in \(\mathbb{R}^{2}\) äquivalent sind, nehmen wir
die euklidische Norm und erhalten wir
\[
  \lim_{(x, y) \to (0, 0)}{\frac{(x^{2} + y^{2})
      \sin{[(x^{2}+y^{2})^{-\frac{1}{2}}]}}{\norm{(x, y)}}}
  = \lim_{(x, y) \to (0, 0)}{\sqrt{x^{2} + y^{2}} \cdot
    \sin{[(x^{2}+y^{2})^{-\frac{1}{2}}]}} = 0.
\]

{\Huge Frage Frage Frage Frage Frage}

Sei \((x, y) = (\frac{1}{n}, \frac{1}{n})\).  Dann ist die Ableitung
von \(f(x, y)\) im Nullpunkt
\[ \lim_{n \to \infty}{\frac{f(x, y) - f(0, 0)}{\norm{(x, y)}}} = \lim_{n \to
    \infty}{\frac{f(x, y) - 0}{\sqrt{x^{2} + y^{2}}}} = \lim_{n \to
    \infty}{\frac{xy}{x^{2} + y^{2}}} = \lim_{n \to
    \infty}{\frac{1}{n^{2}}\cdot\frac{n^{2}}{2}} = \frac{1}{2}.
\]

Sei \((x, y) = (\frac{1}{n}, \frac{1}{n^{2}})\).  Dann ist die
Ableitung von \(f(x, y)\) im Nullpunkt
\[ \lim_{n \to \infty}{\frac{f(x, y) - f(0, 0)}{\norm{(x, y)}}} = \lim_{n \to
    \infty}{\frac{f(x, y) - 0}{\sqrt{x^{2} + y^{2}}}} = \lim_{n \to
    \infty}{\frac{xy}{x^{2} + y^{2}}} = \lim_{n \to
    \infty}{\frac{1}{n^{3}}\cdot\frac{n^{4}}{n^{2}+1}} = 0.
\]

\subsection*{H 7.2}
Zeigen, [1] Diffbarkeit, [2] Ableitungen, [3] Jacobimatrizen.

\subsubsection*{H 7.2.i}
\[f\colon \mathbb{R}^{3} \to \mathbb{R}^{2},~ f(x, y, z) =
  \begin{bmatrix}
    \sin(xy) + y \\ e^{x + y + z}
  \end{bmatrix}.
\]
Diffbarkeit.  Wir zeigen, dass die Funktion \(f\) diffbar ist, indem
wir zeigen, dass \(f\) stetig partiell diffbar ist, d.h., \(f\) ist
partiell diffbar und die partielle Ableitungen sind stetig.  Die
partielle Ableitungen sind
\begin{alignat*}{3}
\partial_{x}f_{1}(x, y, z) &= y \cos(xy) &&\quad
\partial_{y}f_{1}(x, y, z) = 1 + x \cos(xy) &&\quad
\partial_{z}f_{1}(x, y, z) = 0 \\
\partial_{x}f_{2}(x, y, z) &= e^{x + y + z} &&\quad
\partial_{y}f_{2}(x, y, z) = e^{x + y + z} &&\quad
\partial_{z}f_{2}(x, y, z) = e^{x + y + z}
\end{alignat*}
Alle partielle Ableitung sind wegen [Ana 1] auf \(\mathbb{R}\) stetig.  Daraus
folgt, dass \(f\) diffbar, wegen [Satz 2.18].
Daraus folgt, dass die Jacobimatrix ist
\[
  \begin{bmatrix}
    y \cos(xy) & 1 + x \cos(xy) & 0 \\
    e^{x + y + z} & e^{x + y + z} & e^{x + y + z}
  \end{bmatrix}.
\]

Wir bestimmen die Ableitung \(Df(x, y, z)\) von \(f\) im Punkt
\((x, y, z)\).  Wir bemerken, dass \(Df(x, y, z)\) eine lineare
Abbildung ist mit
\[
  Df(x, y, z)\colon \mathbb{R}^{3} \to \mathbb{R}^{2},~ (a, b, c) \mapsto Df(x, y, z)(a, b, c).
\]
Daraus folgt, dass \[
  Df(x, y, z)(a, b, c) =
  \begin{bmatrix}
    (ay+bx)\cos(xy) + b \\
    (a+b+c)e^{x+y+z}
  \end{bmatrix}.
\]

\subsubsection*{H 7.2.ii}
\[
  g\colon \mathbb{R}^{2} \to \mathbb{R},~ g(x, y) = x \cos(x + y^{2}).
\]

Diffbarkeit.  Wir zeigen, dass die Funktion \(g\) diffbar ist, indem
wir zeigen, dass \(g\) stetig partiell diffbar ist, d.h., \(g\) ist
partiell diffbar und die partielle Ableitungen sind stetig.  Die
partielle Ableitungen sind
\[
  \partial_{x}g(x, y) = \cos(x+y^{2}) - x \sin(x+y^{2}) \quad \text{und} \quad \partial_{y}g(x, y) = -2
  xy \sin(x+y^{2}).
\]
Alle partielle Ableitung sind wegen [Ana 1] auf \(\mathbb{R}\) stetig.  Daraus
folgt, dass \(g\) diffbar, wegen [Satz 2.18].
Daraus folgt, dass die Jacobimatrix ist
\[
  \begin{bmatrix}
    \cos(x+y^{2}) - x \sin(x+y^{2})
    & -2 xy \sin(x+y^{2})
  \end{bmatrix}.
\]
Wir bestimmen die Ableitung \(Dg(x, y)\) von \(g\) im Punkt
\((x, y)\).  Es gilt
\[ Dg(x, y)\colon \mathbb{R}^{2} \to \mathbb{R},~ (a, b) \mapsto
  \begin{bmatrix}
    Dg(x, y)
  \end{bmatrix}
  \cdot
  \begin{bmatrix}
    a \\ b
  \end{bmatrix}=
  (-ax-2bxy)\sin(x+y^{2}) + a\cos(x+y^{2}).
\]
Daraus folgt, dass
\(Dg(x, y)(a, b) = (-ax-2bxy)\sin(x+y^{2}) + a\cos(x+y^{2}).\)

\subsubsection*{H 7.2.iii}
\[
  h\colon \mathbb{R}^{2} \to \mathbb{R}^{3},~ h(x, y) =
  \begin{bmatrix}
    xy+y^{4} \\ \ln(1+y^{2}) \\ y \sin x
  \end{bmatrix}.
\]
Diffbarkeit.  Die partielle Ableitungen sind
\begin{alignat*}{2}
  \partial_{x}h_{1}(x, y) &= y &&\quad
  \partial_{y}h_{1}(x, y) = x + 4y^{3} \\
  \partial_{x}h_{2}(x, y) &= 0 &&\quad
  \partial_{y}h_{2}(x, y) = \frac{2y}{1+y^{2}} \\
  \partial_{x}h_{3}(x, y) &= y \cos x &&\quad
  \partial_{y}h_{3}(x, y) = \sin x
\end{alignat*}
Offenbar sind alle partielle Ableitung stetig.  Daraus folgt, dass
\(h\) diffbar ist.  Die Jacobimatrix ist
\[
  \begin{bmatrix}
    y & x + 4y^{3} \\
    0 & \frac{2y}{1+y^{2}} \\
    y \cos x & \sin x
  \end{bmatrix}.
\]
Die Ableitung \(Df(x,y)\) ist \[
  Df(x, y)\colon \mathbb{R}^{2} \to \mathbb{R}^{3},~ (a, b) \mapsto
  \begin{bmatrix}
    ay + b(x+4y^{3}) \\
    \frac{2by}{1+y^{2}} \\
    a y\cos x + b \sin x
  \end{bmatrix}.
\]

\subsection*{H 7.3}

Sei \(g\colon \mathbb{R} \to \mathbb{R}\) eine Funktion und \(f\colon \mathbb{R}^{2} \to \mathbb{R}\) definiert durch
\(f(x, y) \coloneq y \cdot g(x)\).

\begin{beh}
  Die Funktion \(f\) ist in \((0, 0)\) genau dann diffbar, falls \(g\)
  stetig in \(0\) ist.
\end{beh}

\begin{proof}
  Hinrichtung. Angenommen, \(f\) ist in \((0, 0)\) diffbar.  Dann
  existiert eine lineare Abbildung
  \(Df(0, 0)\colon \mathbb{R}^{2} \to \mathbb{R}\) und eine Restfunktion
  \(R\colon \mathbb{R}^{2} \to \mathbb{R}\) sodass für alle Folgen
  \((x_{n}, y_{n})_{n \in \mathbb{N}} \in \mathbb{R}^{2}\) mit
  \(\lim_{n \to \mathbb{N}}{(x_{n}, y_{n})} = (0, 0)\) gilt
  \[
    f(x_{n}, y_{n}) = f(0, 0) + Df(0, 0)(x_{n}, y_{n}) + R(x_{n},
    y_{n}) \quad \text{mit} \quad \lim_{n \to \mathbb{N}}{\frac{R(x_{n},
        y_{n})}{\norm{(x_{n}, y_{n})}}} = 0.
  \]
  Wir setzen \(f(x, y) \coloneq y \cdot g(x)\) ein.  Dann erhalten wir,
  \[
    y_{n} \cdot g(x_{n}) = 0 + Df(0, 0)(x_{n}, y_{n}) + R(x_{n}, y_{n}).
  \]
  Weil \(f\) diffbar im Punkt \((0, 0)\) ist, ist \(f\) insbesondere
  partiell diffbar in \((0, 0)\) nach aller Richtung.  Es gilt
  \[\partial_{x}f(0, 0) = \lim_{h \to 0}{\frac{f(h, 0) - f(0, 0)}{h}} = 0, \quad
    \partial_{y}f(0, 0) = \lim_{h \to 0}{\frac{f(0, h) - f(0, 0)}{h}} = g(0). \]
  Damit ist die Jacobimatrix \(
  \begin{bmatrix}
    0 & g(0)
  \end{bmatrix}
  \) und \(Df(0, 0)(x_{n}, y_{n}) = y_{n} \cdot g(0).\)
  Damit gilt
  \[
    y_{n} \cdot g(x_{n}) = y_{n} \cdot g(0) + R(x_{n}, y_{n}) \quad \text{mit} \quad
    \lim_{n \to \infty}{\frac{R(x_{n}, y_{n})}{\norm{(x_{n}, y_{n})}}} = 0.
  \]
  Es gilt insbesondere
  \[
    \lim_{n \to \infty}{x_{n}} = 0 \quad \text{und} \quad \lim_{n \to \infty}{g(x_{n})} =
    g(0) + \lim_{n \to \infty}{\frac{R(x_{n}, y_{n})}{y_{n}}} \quad \text{mit} \quad
    \lim_{n \to \infty}{\frac{R(x_{n}, y_{n})}{\norm{(x_{n}, y_{n})}}} = 0.
  \]
  {\Huge Warum \(\lim_{n \to \infty}{\frac{R(x_{n}, y_{n})}{y_{n}}} = 0\)?}
  Damit ist
  \(\lim_{n \to \infty}{\frac{R(x_{n}, y_{n})}{y_{n}}} = 0\) und die Funktion
  \(g\) ist stetig in \(0\).

  Rückrichtung.
\end{proof}

\subsection*{H 7.4}

Eine Matrix \(A \in K^{n \times n}\) heißt schiefsymmetrisch, falls
\(A^{\mathrm{T}} = -A\) gilt.

\end{document}