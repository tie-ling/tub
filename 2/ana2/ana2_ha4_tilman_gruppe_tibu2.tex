% !TeX program = lualatex
%%% TeX-engine: luatex

\documentclass[draft,a5paper]{article}
\usepackage[margin=2cm]{geometry}
\linespread{1.2}

% set language to german
\usepackage[ngerman]{babel}

% load math packages and unicode support
% must in this order
\usepackage{amsmath,mathtools,fontspec}
\usepackage{amsthm}
\usepackage{unicode-math,luatex85}

% used for integral with respect to a variable x
% like \int \dd{x}
% \usepackage{physics}

% set fonts
\setmainfont{STIX Two Text}

% theorem environment used in this document
\theoremstyle{remark}
\newtheorem*{beh}{Behauptung}
\newtheorem*{lem}{Lemma}

% \interval is used to provide better spacing after a [ that
% is used as a closing delimiter.
\newcommand{\interval}[1]{\mathinner{#1}}

% Enclose the argument in vert-bar delimiters:
\newcommand{\envert}[1]{\left\lvert#1\right\rvert}
\let\abs=\envert

% Enclose the argument in double-vert-bar delimiters:
\newcommand{\enVert}[1]{\left\lVert#1\right\rVert}
\let\norm=\enVert

\author{Yiwen Yang 466096, Yuchen Guo 480788, Meng Zhang 484981}
\date{\today}
\title{HA 4, Ana 2 -- Gruppe TiBu 2, Tilman Burghoff}

\begin{document}
\maketitle

\newpage

\subsection*{H 4.1}

Wir bestimmen alle kompakten Teilmengen der Menge
\begin{align*}
  M \coloneq \{0\} \cup \{1/n \mid n \in \mathbb{N} \setminus \{0\}\}
\end{align*}
bezüglich der von \(\envert{\cdot}\) induzierten Metrik.  Sei \(A \subseteq M\)
eine Teilmenge von \(M\).  Wir betrachten vier Fällen.

\begin{beh}
  Endliche Teilmengen von \(M\) und unendliche Teilmengen, die \(0\)
  enthält, sind kompakt.  Unendliche Teilmengen ohne \(0\) ist nicht
  kompakt.
\end{beh}

\subsubsection*{H 4.1, Teil 1}

\begin{beh}
  Falls \(\envert{A} = 0\).  Dann ist \(A\) kompakt.
\end{beh}

\begin{proof}
  Es ist \(A = \emptyset\).  Sei die Indexmenge
  \(I\) und die Familie offener Mengen \((U_{i})_{i \in I}\) eine
  beliebige offene Überdeckung von \(A\).  Wir zeigen, dass \(A\)
  kompakt ist, indem wir zeigen, dass eine endliche Menge
  \(J \subseteq I\) existiert, sodass \((U_{i})_{i \in J}\) eine offene
  Überdeckung von \(A\) ist.

  Wir wählen \(p \in I\) beliebig und definieren \(J \coloneq \{p\}\).  Dann
  gilt \(A = \emptyset \in U_{p}\).  Damit haben wir eine endliche
  Teilüberdeckung von \(A\) aus eine beliebige offene Überdeckung von
  \(A\) konstruiert. Daraus folgt, dass \(A\) kompakt ist.
\end{proof}

\subsubsection*{H 4.1, Teil 2}

\begin{beh}
  Falls \(\envert{A} \in \mathbb{N}_{\ge 1}\), dann ist \(A\) kompakt.
\end{beh}

\begin{proof}
  Die Menge \(A\) ist endlich. Sei die Indexmenge \(I\) und die
  Familie offener Mengen \((U_{i})_{i \in I}\) eine beliebige offene
  Überdeckung von \(A\).

  Wir zeigen, dass \(A\) kompakt ist, indem
  wir zeigen, dass eine endliche Menge \(J \subseteq I\) existiert, sodass
  \((U_{i})_{i \in J}\) eine offene Überdeckung von \(A\) ist.  Weil
  \((U_{i})_{i \in I}\) eine offene Überdeckung von \(A\) ist, existiert
  zu jedem Element \(x_{k} \in A\) eine offene Menge \(U_{i_{k}}\),
  \(i_{k} \in I\) sodass \(x_{k} \in U_{i_{k}}\).

  In dieser Art und Weise wählen wir zu jedem Element
  \(x_{1}, \ldots, x_{n} \in A\) die Indizes
  \(i_{1}, \ldots, i_{n} \in I\) und definieren
  \(J \coloneq \{i_{1}, \ldots, i_{n}\}\).  Dann ist die Indexmenge
  \(J \subseteq I\) endlich und gilt
  \begin{align*}
    A \subseteq \bigcup_{k \in J}{U_{k}}.
  \end{align*}
  Damit haben wir eine endliche Teilüberdeckung von
  \(A\) aus eine beliebige offene Überdeckung von \(A\)
  konstruiert. Daraus folgt, dass \(A\) kompakt ist.
\end{proof}

\subsubsection*{H 4.1, Teil 3}

\begin{lem}
  Sei \(\varepsilon > 0\).  Dann ist die Menge \(M \setminus U_{\varepsilon}(0)\) endlich.
\end{lem}

\begin{proof}
  Es gilt \(\lim_{n \to \infty}{1/n} = 0\).  Dann existiert zu jedem
  \(\varepsilon > 0\) ein \(N \in \mathbb{N}\) sodass für alle
  \(n \ge N\) dass \(1/n \in U_{\varepsilon}(0)\).  Daraus folgt, dass \(\envert{M \setminus
    U_{\varepsilon}(0)} = N - 1\).  Die Menge \(M \setminus U_{\varepsilon}(0)\) ist endlich.
\end{proof}

\begin{beh}
  Falls \(\envert{A} = \infty\) und \(0 \in A\).  Dann ist \(A\) kompakt.
\end{beh}

\begin{proof}
  Sei \(\varepsilon > 0\). Zuerst bemerken wir, dass die Menge \(A\) unendlich
  viele Elemente von die Menge \(U_{\varepsilon}(0)\) enthalten muss, denn sonst
  wäre \(A\) endlich, wegen [Lemma].

  Sei die Indexmenge \(I\) und die Familie offener Mengen
  \((U_{i})_{i \in I}\) eine beliebige offene Überdeckung von \(A\).
  Wir zeigen, dass \(A\) kompakt ist, indem wir zeigen, dass eine
  endliche Menge \(J \subseteq I\) existiert, sodass
  \((U_{i})_{i \in J}\) eine offene Überdeckung von \(A\) ist.  Es
  existiert ein \(j \in I\) mit \(0 \in U_{j}\).  Es gilt auch, dass die
  Menge \(U_{j}\) offen ist.  Dann ist unendlich viele Elemente von
  \(A\) in \(U_{j}\) enthalten.  Die Menge \(A \setminus U_{j}\) ist analog zu
  [Lemma] endlich. Zu jedem Element \(x_{k} \in (A \setminus U_{j})\) existiert
  eine offene Menge \(U_{i_{k}}\), \(i_{k} \in I\) sodass
  \(x_{k} \in U_{i_{k}}\).

  In dieser Art und Weise wählen wir zu jedem Element
  \(x_{1}, \ldots, x_{n} \in A \setminus U_{j}\) die Indizes
  \(i_{1}, \ldots, i_{n} \in I\) und definieren
  \(J \coloneq \{i_{1}, \ldots, i_{n}\} \cup \{j\}\).  Dann ist die Indexmenge
  \(J \subseteq I\) endlich und gilt
  \begin{align*}
    A \subseteq \bigcup_{k \in J}{U_{k}}.
  \end{align*}
  Damit haben wir eine endliche Teilüberdeckung von
  \(A\) aus eine beliebige offene Überdeckung von \(A\)
  konstruiert. Daraus folgt, dass \(A\) kompakt ist.
\end{proof}

\subsubsection*{H 4.1, Teil 4}

\begin{beh}
  Falls \(\envert{A} = \infty\) und \(0 \notin A\).  Dann ist \(A\) nicht kompakt.
\end{beh}

\begin{proof}
  Sei \(\varepsilon > 0\). Zuerst bemerken wir, dass die Menge \(A\) unendlich
  viele Elemente von die Menge \(U_{\varepsilon}(0)\) enthalten muss, denn sonst
  wäre \(A\) endlich, wegen [Lemma].

  Wir nummerieren die Elemente in \(A\) monoton fallend durch.  Das
  heißt,
  \begin{align*}
    a_{1} > a_{2} >  \ldots > a_{n} > a_{n+1} > \ldots  \in A
  \end{align*}
  und setzen \(I = \mathbb{N}_{\ge 1}\),
  \begin{align*}
    U_{i} =
    \begin{cases}
      ~ \interval{]a_{2}, a_{1} + 1[}, & \text{falls } i = 1, \\
      ~ \interval{]a_{i + 1}, a_{i - 1}[}, & \text{sonst.}
    \end{cases}
  \end{align*}
  \(U_{i}\) ist offen, also \((U_{i})_{i \ge 1}\) eine offene
  Überdeckung von \(A\).  Jedes \(U_{i}\) enthält genau einen Punkt
  von \(A\), nämlich \(a_{i}\).  Deshalb wird \(A\) von keinem
  endlichen Teilsystem \((U_{1}, U_{2}, \ldots, U_{k})\) überdeckt.  Die
  Menge \(A\) ist nicht kompakt.
\end{proof}

\subsection*{H 4.2}

Sei \((X, d)\) ein metrischer Raum.  Eine Menge \(A \subseteq X\) heißt
\textit{total beschränkt}, falls zu jedem \(\varepsilon > 0\) ein \(n \in \mathbb{N} \setminus
\{0\}\) sowie gewisse \(x_{1}, \ldots, x_{n} \in X\) existieren, mit
\begin{align*}
  A \subseteq \bigcup_{i = 1}^{n}{U_{\varepsilon}(x_{i})}.
\end{align*}

\begin{beh}
  Sei nun der metrische Raum \((X, d)\) vollständig.  Eine Menge
  \(A \subseteq X\) ist genau dann kompakt, wenn sie abgeschlossen und total
  beschränkt ist. Wir beweisen diese Behauptung in drei Schritte.
\end{beh}

\begin{lem}
  Aus Kompaktheit von \(A\) folgt die totale Beschränktheit von \(A\).
\end{lem}

\begin{proof}
  Falls \(A\) leer ist, ist die Behauptung trivial.  Sei
  \(\varepsilon > 0\) beliebig und sei die Familie von Mengen definiert durch
  \begin{align*}
    M_{1} \coloneq \bigcup_{x \in \partial A} U_{\varepsilon/2}(x), \quad
    M_{n} \coloneq \bigcup_{x \in \partial M_{n - 1}}U_{\varepsilon/2}(x).
  \end{align*}
  Dann ist die Vereinigung von \((M_{n})_{n \in \mathbb{N}}\) eine offene
  Überdeckung von \(A\).  Weil \(A\) kompakt ist, existiert eine
  endliche Teilmenge \(I \subseteq \mathbb{N}\) sodass
  \begin{align*}
    A \subseteq \bigcup_{i \in I}{M_{i}}.
  \end{align*}
  Wir wählen \(x_{i} \in M_{i}\) beliebig, dann ist
  \begin{align*}
    A \subseteq \bigcup_{i \in I}{U_{\varepsilon}(x_{i})}.
  \end{align*}
  Damit gilt die Behauptung.
\end{proof}

\begin{lem}
  Aus Kompaktheit von \(A\) folgt die Abgeschlossenheit von \(A\).
\end{lem}

\begin{proof}
  Wir zeigen, dass \(A\) abgeschlossen ist, indem wir zeigen, dass \(X
  \setminus A\) offen ist.  Sei \(b \in X \setminus A\) ein vorgegebener Punkt.  Da
  \(X\) ein metrischer Raum ist, gilt das Hausdorffsche Trennungsaxiom
  und gibt es zu jedem Punkt \(a \in A\) offene Umgebungen \(U_{a} \coloneq
  U_{\varepsilon}(a)\) und \(V_{a} \coloneq U_{\varepsilon}(b)\) mit \(\varepsilon \coloneq \frac{1}{3}d(a, b)\)
  sodass \(U_{a} \cap V_{a} = \emptyset\).  Da
  \begin{align*}
    A \subseteq \bigcup_{a \in A}{U_{a}},
  \end{align*}
  und \(A\) kompakt ist, gibt es endlich viele Punkte \(a_{1},\ldots, a_{s}
  \in A\) mit
  \begin{align*}
    A \subseteq U_{a_{1}} \cup U_{a_{2}} \cup \ldots \cup U_{a_{s}}.
  \end{align*}
  Dann ist
  \begin{align*}
    V \coloneq V_{a_{1}} \cup V_{a_{2}} \cup \ldots \cup V_{a_{s}}
  \end{align*}
  eine offene Umgebung von \(b\) mit \(V \cap \bigcup_{i=1}^{s}{U_{a_{i}}} =
  \emptyset\), also \(V \subseteq X \setminus A\).  Da \(b \in X \setminus A\) beliebig war, ist \(X \setminus
  A\) Umgebung jedes seiner Punkte, also offen.
\end{proof}

\begin{beh}
  Aus die Abgeschlossenheit und totale Beschränktheit von \(A\) folgt
  Kompaktheit von \(A\).
\end{beh}
\begin{proof}
  Angenommen, \(A\) nicht kompakt, d.h., sei
  \((U_{i})_{i \in \mathbb{N}}\) eine offene Überdeckung von \(A\), die keine
  endliche Teilüberdeckung besitzt.
\end{proof}

\subsection*{H 4.3}

Sei \((X, d)\) ein metrischer Raum und \(A, B \subseteq X\) nicht leer.  Wir
definieren \(d(A, B)\) mit

\begin{align*}
  d(A, B) \coloneq \inf \{d(x, y) \mid x \in A, y \in B \}.
\end{align*}

\subsubsection*{H 4.3, Teil i}

\begin{beh}
  Falls \(A\) kompakt und \(B\) abgeschlossen ist mit \(A \cap B \ne \emptyset\),
  dann gilt \(d(A, B) > 0\).
\end{beh}

\begin{proof}
  Wir bemerken, dass aus \(A \cap B = \emptyset\) folgt,
  \begin{align*}
    d(A, B) =
    \inf~\{d(x, y) \mid
    x \in A, y \in B\} = \min~\{d(x, y) \mid x \in \partial A, y \in \partial B\}.
  \end{align*}
  Wir bemühen uns um einen Widerspruchsbeweis.  Angenommen, es gilt
  \(d(A, B) = 0\).  Dann existiert mindestens ein paar Punkte
  \(x \in \partial A, y \in \partial B\) mit \(d(x, y) = 0\).  Aus der Definition von
  Metrik folgt, dass \(x = y\).  Aus der Kompaktheit von \(A\) folgt,
  dass \(A\) abgeschlossen ist.  Aus der Abgeschlossenheit von \(A\)
  und \(B\) folgt, dass \(\partial A \subseteq A\) und
  \(\partial B \subseteq B\).  Daraus folgt, dass \(x \in A\) und
  \(y \in B\).  Wegen \(x = y\) gilt
  \(\{x, y\} \subseteq A \cap B \ne \emptyset\).  Diese ist ein Widerspruch.
\end{proof}


\subsubsection*{H 4.3, Teil ii}

\begin{beh}
  Falls \(A\) und \(B\) abgeschlossen ist mit
  \(A \cap B \ne \emptyset\), dann gilt \(d(A, B) > 0\) nicht.
\end{beh}

\begin{proof}
  ???
\end{proof}

\subsection*{H 4.4}

Sei \((X, d)\) ein metrischer Raum, der kompakt ist. Wegen [Korollar
1.73] ist \(X\) vollständig. Sei \(f\colon X \to X\) eine schwache
Kontraktion, d.h., für alle \(x, y \in X\) mit \(x \ne y\) gilt
\begin{align*}
  d(f(x), f(y)) < d(x, y).
\end{align*}
Diese Voraussetzung ist anders als die Voraussetzung des Banach'schen
Fixpunktsatzes, denn die Ungleichung gilt nur für \(x \ne y\), wobei in
Banach'schen Fixpunktsatz gilt die Ungleichung für alle \(x \in X\).

\begin{beh}
  Die Funktion \(f\) hat genau einen Fixpunkt.
\end{beh}

\begin{proof}
  Beweis der Eindeutigkeit. Wir zeigen zunächst die Eindeutigkeit des
  Fixpunkts.  Angenommen, es existiert zwei Fixpunkte \(a, b\) von
  \(f\) mit \(a \ne b\), d.h., es gilt \(f(a) = a\), \(f(b) = b\) und
  \(f(a) \ne f(b)\).  Dann gilt
  \begin{align*}
    d(a, b) = d(f(a), f(b)) < d(a, b).
  \end{align*}
  Diese ist im Widerspruch zur \(d(a, b) = d(a, b)\).  Daraus folgt
  die Eindeutigkeit des Fixpunkts.

  Beweis der Existenz.  Wir definieren die Folge \((x_{n})\) mit
  \(x_{0} \in X\) beliebig gewählt und \(x_{n+1} = f(x_{n})\), \(n \in
  \mathbb{N}\).  Dann gilt für alle \(n \in \mathbb{N} \setminus \{0\}\) dass
  \begin{align*}
    d(x_{n}, x_{n+1}) = d(f(x_{n-1}), f(x_{n})) < d(x_{n-1}, x_{n}).
  \end{align*}
\end{proof}

\end{document}