%%% TeX-engine: luatex
% !TeX program = lualatex
% !TeX encoding = UTF-8 Unicode

\documentclass[draft,a5paper]{article}
\usepackage[ngerman]{babel}
\usepackage[margin=2cm]{geometry}
\linespread{1.2}
% let auctex detect environments
\usepackage{amsmath,mathtools,fontspec,amsthm}
\newtheorem*{beh}{Behauptung}
\newtheorem*{lem}{Lemma}

% use straight "d" in integrals
% like \int\dd{x}
\usepackage{physics}


\usepackage{unicode-math,luatex85}
\setmainfont{STIX Two Text}

\author{Yiwen Yang 466096, Yuchen Guo 480788, Meng Zhang 484981}
\date{\today}
\title{HA 2, Ana 2 -- Gruppe TiBu 2, Tilman Burghoff}

\begin{document}
\maketitle

\newpage

\subsection*{H 2.1}

Sei \((l_{\infty}, \| \cdot \|_{l_{\infty}})\) der metrischer Raum aller beschränkten
Zahlenfolgen.

\subsubsection*{H 2.1, Teil i}

\begin{beh}
  Die Menge \(A \coloneq \{x \in l_{\infty} \mid x \text{ konvergent in } \mathbb{R}\}\) ist
  abgeschlossen und unbeschränkt.
\end{beh}

\begin{proof}
  Wir zeigen, dass die Menge \(A\) abgeschlossen ist, indem wir
  zeigen, dass \(A\) folgenabgeschlossen ist. D.h., für jede Folge
  \((x^{(n)})\) in \(A\) mit \(\lim_{n \to \infty}{x^{(n)}} = a\) gilt, dass
  \(a \in A\).

  Sei \(\varepsilon > 0\) beliebig gewählt.  Weil \((x^{(n)})\) gegen
  \(a\) konvergiert, existiert ein \(N \in \mathbb{N}\) sodass für alle \(n \ge N\)
  gilt \(\sup_{k \in \mathbb{N}}{|x_{k}^{(n)} - a_{k}|} < \varepsilon\).  Dann gilt
  \begin{align*}
    |x_{k}^{(n)} - a_{k}| \le \sup_{k \in \mathbb{N}}{|x_{k}^{(n)} - a_{k}|} =
    \|x^{(n)} - a\|_{l_{\infty}} \stackrel{n \to \infty}{\to} 0.
  \end{align*}
  Insbesondere existiert zu jedem \(\varepsilon > 0\) ein
  \(N \in \mathbb{N}\) sodass \(|a_{k} - x_{k}^{(n)}| = \varepsilon\).  Daraus folgt, dass
  \(a\) konvergiert gegen \(x^{(n)}\).  Da \(x^{(n)}\) konvergiert,
  gilt \(a \in A\).

  Die Menge \(A\) ist unbeschränkt.  Sei \(a\) die Nullfolge.  Wir
  zeigen, dass \(A\) unbeschränkt ist, indem wir zeigen, dass es zu
  jedem \(M \ge 0\) ein \(x \in A\) existiert mit \(\|x - a\| > M\).  Sei
  dann die Folge \(x_{n} \coloneq 1/n\) mit
  \(n \in \mathbb{N}_{> 0}\).  Die Folge \(x_{n}\) konvergiert gegen
  \(0\) und es gilt
  \(\|x - a\| = \sup_{n \in \mathbb{N}}{(x_{n} - a_{n})} = 1\).  Dann ist die Folge
  \(x'\) mit
  \begin{align*}
    x'_{n} =
    \begin{cases}
      x_{n} + 1, & n = 1; \\
      x_{n}, & \text{sonst.}
    \end{cases}
  \end{align*}
  eine konvergente Folge mit Grenzwert \(0\) und
  \(\|x' - a\| = \sup_{n \in \mathbb{N}}{(x'_{n} - a_{n})} = 2\).  Analog können wir
  für ein beliebige \(M \in \mathbb{R}\) eine Folge konstruieren mit
  \(\|x - a\| > M, x \in A, a \in l_{\infty}\).
\end{proof}

\subsubsection*{H 2.1, Teil ii}

\begin{beh}
  Die Menge \(B \coloneq \{x \in l_{\infty} \mid \exists n_{0} \in \mathbb{N}, \forall n \ge n_{0} \colon x_{n} =
  0\}\) ist abgeschlossen und unbeschränkt.
\end{beh}

\begin{proof}
  Wir bemerken, dass die Menge \(B\) die Menge alle Folgen, die gegen
  \(0\) konvergiert ist.  Denn sei \(\varepsilon > 0\) beliebig gewählt, es
  existiert wegen Voraussetzung ein \(n_{0} \in \mathbb{N}\), sodass für alle \(n
  \ge n_{0}\) gilt \(|x_{n} - 0| < \varepsilon\).  Damit konvergiert alle Folgen in
  \(B\) gegen \(0\) und die Behauptung von [Teil i] gilt.
\end{proof}

\newpage
\subsection*{H 2.2}

\subsubsection*{H 2.2, Teil i}

\begin{beh}
  Für \(\alpha \ge 2\) ist die Funktion \(f\colon \mathbb{R}^{2} \to \mathbb{R}, (0, 0) \mapsto 0\), sonst \((x, y) \mapsto
  |xy|^{\alpha}/(x^{2}+y^{2})\)  stetig.
\end{beh}

\begin{proof}
  Auf \(\mathbb{R}^{2} \setminus \{(0, 0)\}\) ist \(f\) als Komposition stetiger
  Funktionen stetig bzgl. der Standardmetrik.  Es bleibt die
  Stetigkeit in \((0, 0)\) zu untersuchen.  Es gilt für \(\alpha \ge 2\) und
  alle Paare \((x, y) \in \mathbb{R}^{2} \setminus \{(0, 0)\}\) stets
  \begin{align*}
    0 \le |f(x, y)| = |x|^{\alpha} \cdot |y|^{\alpha-2} \cdot \underbrace{\frac{y^{2}}{x^{2} +
    y^{2}}}_{< 1} \le |x|^{\alpha} \cdot |y|^{\alpha-2} \stackrel{(x, y) \to (0, 0)}{\to} 0.
  \end{align*}
  Damit haben wir die Stetigkeit gezeigt.
\end{proof}

\subsubsection*{H 2.2, Teil ii}

\begin{beh}
  Für \(\beta = 1\) ist die Funktion \(g\colon \mathbb{R}^{2} \to \mathbb{R}, (0, 0) \mapsto  \beta\), sonst
  \((x, y) \mapsto  (x^{4} - y^{4})/(x^{4} + y^{4})\) stetig.
\end{beh}

\begin{proof}
  Auf \(\mathbb{R}^{2} \setminus \{(0, 0)\}\) ist \(g\) als Komposition stetiger
  Funktionen stetig bzgl. der Standardmetrik.  Es bleibt die
  Stetigkeit in \((0, 0)\) zu untersuchen.  Es gilt für alle Paare
  \((x, y) \in \mathbb{R}^{2} \setminus \{(0, 0)\}\) stets
  \begin{align*}
    \frac{x^{4} + y^{4} - 2y^{4}}{x^{4} + y^{4}} = 1 - 2 \cdot
    \underbrace{\frac{y^{4}}{x^{4}+y^{4}}}_{< 1} \stackrel{(x, y) \to
    (0, 0)}{\to} 1 = \beta.
  \end{align*}
  Damit haben wir die Stetigkeit gezeigt.
\end{proof}

\newpage
\subsection*{H 2.4}
Sei \((\mathcal{C} ([a, b], \mathbb{R}), \|\cdot\|)\) ein metrischer Raum und
\(f \in \mathcal{C} (\mathbb{R}, \mathbb{R}) \).  Weiter sei die Funktion \(A\) definiert durch
  \begin{align*}
    A\colon \mathcal{C} ([a, b], \mathbb{R}) \to \mathcal{C} ([a, b], \mathbb{R}), \quad u \mapsto f \circ u.
  \end{align*}

\begin{beh}
  Die Funktion \(A\) ist stetig.
\end{beh}

\begin{proof}
  Wir zeigen, dass die Funktion
  \(f \circ u\) stetig ist, indem wir zeigen, dass die folgende
  Implikation gilt
  \begin{align*}
    \lim_{n \to \infty}{u_{n}} = u \implies \lim_{n \to \infty}{(f \circ u_{n})} =
    (f \circ u).
  \end{align*}
  Sei \((u_{n})\) eine beliebige Folge in
  \(\mathcal{C} ([a, b], \mathbb{R})\) mit dem Grenzwert
  \(u \in \mathcal{C} ([a, b], \mathbb{R})\). Daraus folgt, dass
  \begin{align*}
    \forall \varepsilon > 0, \exists N \in \mathbb{N}, \forall n \ge N\colon \sup_{t \in [a, b]}{|u_{n}(t) - u(t)|} < \varepsilon.
  \end{align*}
  Wegen \(f\) stetig in \(\mathbb{R}\), gilt
  \begin{align*}
    \forall a \in \mathbb{R}, \forall \varepsilon > 0, \exists \delta > 0, \forall x \in \mathbb{R} \text{ mit } |x - a| < \delta
    \implies |f(x) - f(a)| < \varepsilon.
  \end{align*}
  Wegen \(u_{n}(t), u(t) \in \mathbb{R}\) für alle \(n \in \mathbb{N}, t \in [a, b]\), gilt
  dann
  \begin{align*}
    &\forall \varepsilon > 0, \exists \delta > 0, \forall u_{n}(t) \text{ mit } |u_{n}(t) - u(t)| < \delta
    \text{ für alle } t \in [a, b] \\
    &\implies |(f \circ u_{n})(t) - (f \circ
    u)(t)| < \varepsilon \text{ für alle } t \in [a, b].
  \end{align*}
  Nun sei \(\varepsilon > 0\) beliebig.  Dann existiert wegen der Stetigkeit von
  \(f\) ein \(\delta > 0\), die das obige Kriterium erfüllt.  Wir setzen
  \(\delta\) in der Definition der Grenzwert von \(\lim_{n \to \infty}{u_{n}} =
  u\).  Wir erhalten
  \begin{align*}
    \exists N \in \mathbb{N}, \forall n \ge N\colon \sup_{t \in [a, b]}{|u_{n}(t) - u(t)|} < \delta,
  \end{align*}
  woraus wir \(|u_{n}(t) - u(t)| < \delta\) für alle \(t \in [a, b]\)
  erhalten kann.  Damit ist \(\varepsilon\) eine obere Schranke von
  \(|(f \circ u_{n})(t) - (f \circ u)(t)|\) und es gilt
  \begin{align*}
    \sup_{t \in [a, b]}{|(f \circ u_{n})(t) - (f \circ
    u)(t)|} < \varepsilon.
  \end{align*}
  Da \(\varepsilon > 0\) beliebig gewählt war, folgt
  \begin{align*}
    \forall \varepsilon > 0, \exists N \in \mathbb{N}, \forall n \ge N\colon \sup_{t \in [a, b]}{|(f \circ u_{n})(t) - (f \circ
    u)(t)|} < \varepsilon.
  \end{align*}
  Damit gilt \(\lim_{n \to \infty}{(f \circ u_{n}) = f \circ u}\) und die
  Stetigkeit von \(A\) ist bewiesen.
\end{proof}
\end{document}