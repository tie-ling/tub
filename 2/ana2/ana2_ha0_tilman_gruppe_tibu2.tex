\documentclass[draft,a5paper]{article}
\usepackage[ngerman]{babel}
\usepackage[margin=2cm]{geometry}
\usepackage{amsmath,mathtools,fontspec,amssymb,mathtools}

% for theorems and lemma
\usepackage{amsthm}
\newtheorem*{beh}{Behauptung}
\newtheorem*{lem}{Lemma}

% times new roman font
\usepackage{newtx}

\linespread{1.2}

\author{Yiwen Yang 466096, Yuchen Guo 480788, Meng Zhang 484981}
\date{\today}
\title{HA 0, Ana 2 -- Gruppe TiBu 2, Tilman Friedrich Burghoff}

\begin{document}
\maketitle

\newpage

\subsection*{1. Aufgabe.}
Für \(f \colon [0, 1] \to \mathbb{R}\) und
\(n \in \mathbb{N} \setminus \left\{ 0 \right\}\) definieren wir das
\(n\)-te Bernstein-Polynom von \(f\) durch
\begin{align*}
  B_n f \colon [0, 1] \to \mathbb{R}, \quad (B_n f)(x) \coloneq \sum_{k=0}^n
  {f \left( \frac{k}{n} \right)
  \begin{pmatrix} n \\ k \end{pmatrix} x^k (1-x)^{n - k}}.
\end{align*}

\subsubsection*{1. Aufgabe, Teil i.}

\begin{lem}
  Für alle \(x \in \mathbb{R}\) gilt
  \begin{align}
    \label{eq:1}
    \sum_{k = 0}^{n}
    {\begin{pmatrix} n \\ k \end{pmatrix} x^k (1 - x)^{n - k}} &= 1; \\
    \label{eq:2}
    \sum_{k=0}^n{k \begin{pmatrix} n \\ k \end{pmatrix} x^k (1 -
    x)^{n - k}} &= n x; \\
    \label{eq:3}
    \sum_{k=0}^n{k^2 \begin{pmatrix} n \\ k \end{pmatrix} x^k (1 -
    x)^{n - k}} &= n (n - 1) x^2 + n x.
  \end{align}
\end{lem}

\begin{proof}
  Die Gleichung \eqref{eq:1} folgt sofort aus binomischen
  Lehrsatz. Die Gleichung \eqref{eq:2} gilt wegen
  \begin{align*}
    \sum_{k=0}^n{k \begin{pmatrix} n \\ k \end{pmatrix} x^k (1 - x)^{n -
    k}}
    &= \sum_{k=0}^n{k \cdot  \frac{n!}{k! (n-k)!} \cdot x^k (1 -
      x)^{n - k}} \\
    &= \sum_{k=0}^n{k \cdot \frac{n (n-1)!}{k (k-1)! [(n-1) - (k-1)]!} \cdot x^k (1 -
      x)^{n - k}} \\
    &= n \cdot \sum_{k=0}^n {\frac{(n-1)!}{(k-1)! [(n-1) - (k-1)]!} \cdot x^k (1 -
      x)^{n - k}} \\
    &= n \cdot x \cdot \sum_{k=0}^n {\frac{(n-1)!}{(k-1)! [(n-1) - (k-1)]!} \cdot x^{k-1} (1 -
      x)^{(n - 1) - (k - 1)}} \\
    &= n \cdot x \cdot (x + 1 - x)^{n-1} \\
    &= nx.
  \end{align*}
  Die Gleichung \eqref{eq:3} folgt wegen \(k^2 = k ( k - 1 ) + k\) dass
  \begin{align*}
    & \sum_{k=0}^n{k^2 \begin{pmatrix} n \\ k \end{pmatrix} x^k (1 -
      x)^{n - k}} \\
    &= \sum_{k=0}^n{k (k - 1) \frac{n (n - 1) (n - 2)!}{k (k - 1) (k -
      2)! [(n - 2) - (k - 2)]!} \cdot x^k (1 -  x)^{n - k}}
      + \text{Gleichung \eqref{eq:2}} \\
    &= n \cdot (n - 1) x^2 \sum_{k=0}^n{\frac{(n - 2)!}{(k - 2)! [(n -
      2) - (k - 2)]!} \cdot x^{k - 2} (1 - x)^{(n - 2) - (k - 2)}}
      + \text{Gleichung \eqref{eq:2}} \\
    &= n(n-1)x^2 (x + 1 - x)^{n-2} + \text{Gleichung \eqref{eq:2}} \\
    &= n(n-1)x^2 + nx.
  \end{align*}
\end{proof}

\begin{beh}
  Falls \(f\) stetig ist, konvergiert \((B_n f)_{n \in \mathbb{N}}\)
  gleichmäßig gegen \(f\).
\end{beh}

\begin{proof}
  Wir zeigen, dass die Funktionenfolge \(B_n f\) gegen \(f\)
  gleichmäßig konvergiert, indem wir zeigen, dass zu jedem
  \(\varepsilon > 0\) ein \(N_{\varepsilon} \in \mathbb{N}\) existiert, sodass
  für alle \(n \in \mathbb{N}\), \(n \ge N_{\varepsilon}\) und alle
  \(x \in [0, 1]\) gilt \(\left| (B_n f)(x) - f(x) \right| < \varepsilon\).
  Es gilt
  \begin{align*}
    (B_n f)(x) - f(x)
    &= \sum_{k = 0}^n {\left[ f \left( \frac{k}{n}
      \right) - f(x) \right]
      \begin{pmatrix} n \\ k \end{pmatrix} x^k (1-x)^{n-k}}.
  \end{align*}
  Wegen der Stetigkeit von der Funktion \(f\) können wir ein \(\delta > 0\)
  so wählen, dass für alle \(x, x' \in [0, 1]\) mit \(|x - x'| < \delta\) gilt
  \(|f(x) - f(x')|< \varepsilon / 2\).  Dann gilt
  \begin{align}
    (B_n f)(x) - f(x) =
    \label{eq:4}
    &\sum_{|x - k/n| < \delta} {\left[ f \left( \frac{k}{n}
      \right) - f(x) \right]
      \begin{pmatrix} n \\ k \end{pmatrix} x^k (1-x)^{n-k}} \\
    \label{eq:5}
    &+ \sum_{|x - k/n| \ge \delta} {\left[ f \left( \frac{k}{n}
      \right) - f(x) \right]
      \begin{pmatrix} n \\ k \end{pmatrix} x^k (1-x)^{n-k}}.
  \end{align}
  Wegen der Voraussetzung dass für alle \(x, x' \in [0, 1]\) mit
  \(|x - x'| < \delta\) die Ungleichung \(|f(x) - f(x')|< \varepsilon / 2\)
  gilt, folgt daraus, dass
  \begin{align*}
    \text{Summe \eqref{eq:4}} < \frac{\varepsilon}{2} \cdot \sum_{k=0}^n
    {\begin{pmatrix} n \\ k \end{pmatrix} x^k (1-x)^{n-k}}
    = \frac{\varepsilon}{2}.
  \end{align*}
  Wir betrachten nun die Summe \eqref{eq:5}.  Aus der Definition von
  Supremumsnorm folgt \(|f(x)| \le \|f\|\) für alle \(x \in [0, 1]\).  Damit
  gilt
  \begin{align*}
    \text{Summe \eqref{eq:5}} \le 2 \|f\|
    \sum_{|x - k/n| \ge \delta}
    {\begin{pmatrix} n \\ k \end{pmatrix} x^k (1-x)^{n-k}}.
  \end{align*}
  Jetzt gilt wegen \(\delta^{-2}(x - k/n)^2 \ge 0\) für alle \(k\) und
  \(\delta^{-2}(x - k/n)^2 \ge 1\) für alle \(|x - k/n| \ge \delta\) dass
  \begin{align*}
    \sum_{|x - k/n| \ge \delta}
    {\begin{pmatrix} n \\ k \end{pmatrix} x^k (1-x)^{n-k}}
    &\le \sum_{k = 0}^{n} \delta^{-2}(x - k/n)^2
      {\begin{pmatrix} n \\ k \end{pmatrix} x^k (1-x)^{n-k}} \\
    &= \delta^{-2} \sum_{k = 0}^{n}
      \left( x^2 - \frac{2kx}{n} + \frac{k}{n}^2 \right)
      {\begin{pmatrix} n \\ k \end{pmatrix} x^k (1-x)^{n-k}}
  \end{align*}
  Nun nutzen wir Lemma von \eqref{eq:1} bis \eqref{eq:3} aus, erhalten wir
  \begin{align*}
    \delta^{-2} \sum_{k = 0}^{n}
    \left( x^2 - \frac{2kx}{n} + \frac{k}{n}^2 \right)
    {\begin{pmatrix} n \\ k \end{pmatrix} x^k (1-x)^{n-k}}
    &= \delta^{-2} \left[ x^2 - \frac{2x}{n} \cdot (nx) + \frac{1}{n^2}
      \cdot (n(n-1)x^2 + nx) \right]\\
    &= \frac{x(1-x)}{n\delta^2} \le \frac{1}{4n\delta^2}.
  \end{align*}
  Damit haben wir gezeigt, dass für alle
  \(n \ge \frac{\varepsilon}{2 \|f\| \delta^2}\) gilt Summe \eqref{eq:4} +
  Summe \eqref{eq:5} \(\le \varepsilon\).  Damit konvergiert die
  Funktionenfolge \(B_n f\) gleichmäßig gegen \(f\).
\end{proof}

\subsubsection*{1. Aufgabe, Teil ii.}

\begin{beh}
  Zu jeder stetigen Funktion \(f \colon [a, b] \to \mathbb{R}\) existiert
  eine Funktionenfolge \((p_n)_{n \in \mathbb{N}}\), $p_n \colon [a, b] \to
  \mathbb{R}$, die gleichmäßig gegen \(f\) konvergiert.
\end{beh}

\begin{proof}
  Sei die Zahlen \(a, b \in \mathbb{R}, a < b\) und die Funktion
  \(g(x) \colon [a, b] \to \mathbb{R}, x \mapsto \frac{(x-a)}{(b-a)}\).  Dann
  ist der Wertbereich von \(g\) gleich \([0, 1]\) und die Verknüpfung
  \(f \circ g = g(f(x))\) ist stetig.  Aus Teil i dieser Aufgabe folgt
  dann die Existenz einer Funktionenfolge, die gleichmäßig gegen \(f\)
  konvergiert.
\end{proof}

\subsection*{2. Aufgabe.}

Gegeben ist die Fibonacci-Folge mit
\begin{align*}
  a_0 \coloneq 0, \quad a_1 \coloneq 1, \quad a_n \coloneq a_{n-1} + a_{n-2}, \quad n \ge 2.
\end{align*}

\begin{beh}
  Es gilt für alle \(n \in \mathbb{N}\) dass
  \begin{align*}
    a_n = \frac{1}{\sqrt{5}}
    \left[ \left( \frac{1 + \sqrt{5}}{2} \right)^{n}
    - \left( \frac{1 - \sqrt{5}}{2} \right)^n \right].
  \end{align*}
\end{beh}

\begin{proof}
  Wir bemühen uns um einen Beweis mit vollständigen Induktion.

  \begin{itemize}
  \item Induktionsanfang.  Es gilt \(a_0 = 0 = 1/\sqrt{5} \cdot 0\) und
    \(a_1 = 1 = 1/\sqrt{5} \cdot \sqrt{5}\).
  \item Induktionsvoraussetzung.  Sei \(n\) fest gewählt mit
    \(n \in \mathbb{N}_{\ge 2}\).  Die Behauptung gelte für alle natürliche
    Zahlen kleiner gleich \(n\).
  \item Induktionsschritt. Wir zeigen, dass
    \begin{align*}
      a_{n+1} = \frac{1}{\sqrt{5}}
      \left[ \left( \frac{1 + \sqrt{5}}{2} \right)^{n+1}
      - \left( \frac{1 - \sqrt{5}}{2} \right)^{n+1} \right].
    \end{align*}
    gilt.  Wegen der Definition von Fibonacci-Folgen und
    Induktionsvoraussetzung gilt
    \begin{align*}
      a_{n+1}
      &= a_n + a_{n - 1} \\
      &= \frac{1}{\sqrt{5}}
        \left[ \left( \frac{1 + \sqrt{5}}{2} \right)^{n+1}
        - \left( \frac{1 - \sqrt{5}}{2} \right)^{n+1} \right]
      + \frac{1}{\sqrt{5}}
        \left[ \left( \frac{1 + \sqrt{5}}{2} \right)^{n-1}
        - \left( \frac{1 - \sqrt{5}}{2} \right)^{n-1} \right] \\
      &= \frac{1}{\sqrt{5}} \cdot \frac{1}{2^{n+1}} \cdot [(1 +
        \sqrt{5})^{n-1} \cdot (2 (1 + \sqrt{5}) + 4) - (1 -
        \sqrt{5})^{n-1} \cdot (2 (1 - \sqrt{5}) + 4)] \\
      &= \frac{1}{\sqrt{5}} \cdot \frac{1}{2^{n+1}} \cdot [(1 +
        \sqrt{5})^{n+1} - (1 - \sqrt{5})^{n+1}] \\
      &= \frac{1}{\sqrt{5}}
        \left[ \left( \frac{1 + \sqrt{5}}{2} \right)^{n+1}
        - \left( \frac{1 - \sqrt{5}}{2} \right)^{n+1} \right].
    \end{align*}
  \end{itemize}
  Damit gilt die Behauptung.
\end{proof}
\end{document}