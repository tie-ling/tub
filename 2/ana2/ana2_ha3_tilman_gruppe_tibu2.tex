% !TeX program = lualatex
%%% TeX-engine: luatex

\documentclass[draft,a5paper]{article}
\usepackage[margin=2cm]{geometry}
\linespread{1.2}

% set language to german
\usepackage[ngerman]{babel}

% load math packages and unicode support
% must in this order
\usepackage{amsmath,mathtools,fontspec}
\usepackage{amsthm}
\usepackage{unicode-math,luatex85}

% used for integral with respect to a variable x
% like \int \dd{x}
%\usepackage{physics}

% set fonts
\setmainfont{STIX Two Text}

% theorem environment used in this document
\theoremstyle{remark}
\newtheorem*{beh}{Behauptung}
\newtheorem*{lem}{Lemma}

%    \interval is used to provide better spacing after a [ that
%    is used as a closing delimiter.
\newcommand{\interval}[1]{\mathinner{#1}}

%    Enclose the argument in vert-bar delimiters:
\newcommand{\envert}[1]{\left\lvert#1\right\rvert}
\let\abs=\envert

%    Enclose the argument in double-vert-bar delimiters:
\newcommand{\enVert}[1]{\left\lVert#1\right\rVert}
\let\norm=\enVert

\author{Yiwen Yang 466096, Yuchen Guo 480788, Meng Zhang 484981}
\date{\today}
\title{HA 3, Ana 2 -- Gruppe TiBu 2, Tilman Burghoff}

\begin{document}
\maketitle

\newpage

\subsection*{H 3.1}

\subsubsection*{H 3.1, Teil i}

\begin{beh}
  Die Abbildung \(f\colon \interval{]0, 1]} \to \mathbb{R}, f(x) = x + 1/x\) ist keine
  Kontraktion bzgl. der Standardmetrik.
\end{beh}

\begin{proof}
  Denn \(f\) ist keine Selbstabbildung wegen \(f(1/2) = 5/2 \notin \interval{]0,
  1]}\).
\end{proof}

\subsubsection*{H 3.1, Teil ii}

\begin{beh}
  Die Funktion \(g\colon \mathbb{R}^{2} \to \mathbb{R}^{2},~ (x, y) \mapsto \frac{1}{2}(x+y, y)\) ist
  eine Kontraktion bzgl. der Standardmetrik.
\end{beh}

\begin{proof}
  Die Funktion \(g\) ist eine Selbstabbildung wegen der Definition von
  \(g\).  Wir zeigen, dass die Funktion eine Kontraktion ist, indem
  wir zeigen, dass die Funktion Lipschitz-stetig mit
  Lipschitz-Konstante \(L < 1\) ist.

  Angenommen, die Funktion ist nicht Lipschitz-stetig mit \(L < 1\).
  Dann existiert zu jedem \(0 \le L < 1\) ein Paar \((a, b), (c, d) \in
  \mathbb{R}^{2}\) mit \(d(g(a, b), g(c, d)) > L \cdot d((a, b), (c, d))\).  Zur
  Abkürzung setzen wir \(m \coloneq a - c\) und \(n \coloneq b - d\).  Dann gilt
  \begin{align*}
    d(g(a, b), g(c, d))
    &> L \cdot d((a, b), (c, d)) \\
    d\left(\frac{1}{2}(a+b, b), \frac{1}{2}(c+d, d)\right)
    &> L \cdot d((a,b), (c, d)) \\
    \frac{1}{2} \sqrt{(m+n)^{2} + n^{2}} &> L \cdot \sqrt{(m^{2} + n^{2})} \\
    \frac{1}{4} [(m + n)^{2} + n^{2}]
    &> L \cdot (m^{2} + n^{2}) \\
    (1 - 4L)m^{2} + (2 - 4L)n^{2} + 2mn
    &> 0 \tag{1}
  \end{align*}
  Aber die letzte Ungleichung (1) ist nicht gültig, wenn \(L \in
  \interval{[1/2, 1[}\). Denn wegen \(L \in \interval{[1/2, 1[}\) gilt
  \(-4L \in \interval{]-4, -2]}\), \(1-4L \in \interval{]-3, -1]}\) und
  \(2-4L \in \interval{]-2, 0]}\).  Insbesondere, wenn \(L \to 1\) ist,
  gilt
  \begin{align*}
    -3m^{2} - 2n^{2} + 2mn &> 0 \\
    2m^{2} + n^{2} + (m-n)^{2} &< 0.
  \end{align*}
  Diese letzte Ungleichung ist nicht gültig.  Daraus folgt, dass nicht
  zu jedem \(0 \le L < 1\) die Paare \((a,b), (c,d)\) existiert. Die
  Annahme ist falsch und die Funktion ist eine Kontraktion.
\end{proof}

\subsubsection*{H 3.1, Teil iii}

\begin{beh}
  Die Funktion \(g\) wie in (ii) bzgl. \(d_{\infty}\) ist keine Kontraktion.
\end{beh}

\begin{proof}
  Wir zeigen, dass \(g\) bzgl. \(d_{\infty}\) keine Kontraktion ist, indem
  wir zeigen, dass \(g\) nicht Lipschitz-stetig mit \(0 \le L < 1\) ist,
  also zu jedem \(L \in \interval{[0, 1[}\) existiert ein Paar \((a, b),
  (c, d) \in \mathbb{R}^{2}\) mit
  \[
    d(g(a, b), g(c, d)) > L \cdot d((a, b), (c, d)).
  \]
  Zuerst bemerken wir, dass \(d_{\infty}\) für
  \(x = (x_{1}, \ldots, x_{n}) \in \mathbb{R}^{n}\) durch [Beispiel 1.2], also
  \( \enVert{x}_{\infty} \coloneq \max\{\envert{x_{i}} \mid i = 1, \ldots, n \} \)
  definiert ist. Es ist
  \[
    d(g(a, b), g(c, d))
    = d\left(\frac{1}{2}(a+b, b), \frac{1}{2}(c+d, d)\right)
    = \max\{\envert{a-c + b-d}/2, \envert{b-d}/2 \}
  \]
  und
  \[
    L \cdot d((a, b), (c, d))
    = L \cdot \max\{\envert{a-c}, \envert{b-d} \}.
  \]
  Wir betrachten den Fall
  \begin{align*}
    d(g(a, b), g(c, d)) &= \envert{a-c + b-d}/2, \\
    L \cdot d((a, b), (c, d)) &= L \cdot \envert{a-c}.
  \end{align*}
  Es folgt wegen Dreiecksungleichung dass???
  \begin{align*}
    \envert{a-c}/2 + \envert{a-c}/2 \ge \envert{a-c}/2 + \envert{b-d}/2
    \ge \envert{a-c+b-d}/2.
  \end{align*}
\end{proof}

\subsection*{H 3.2}

\begin{beh}
  Seien \(X, d_{X}\) und \(Y, d_{Y}\) metrische Räume.  Die Funktion
  \(f\colon X \to Y\) ist genau dann stetig, wenn für alle Mengen \(M \subseteq X\)
  die Relation \(f(\overline{M}) \subseteq \overline{f(M)}\) gilt.
\end{beh}

\begin{proof}
  Wir zeigen, dass die Behauptung gilt, indem wir zeigen, dass die
  Hinrichtung und die Rückrichtung gilt.

  Hinrichtung.  Sei \(M \subseteq X\) beliebig.  Wir zeigen, dass aus der
  Stetigkeit von \(f\) folgt
  \(f(\overline{M}) \subseteq \overline{f(M)}\).  Zuerst bemerken wir, dass
  wegen [Korollar 1.24] gilt \(\overline{M} = M \cup \partial M\) und
  \(\overline{f(M)} = f(M) \cup \partial f(M)\).  Daraus folgt, dass
  \(f(M) \subseteq \overline{f(M)}\).  Danach zeigen wir, dass
  \(f(\partial M) \subseteq \overline{f(M)}\) gilt.

  Sei \(a \in \partial M\) beliebig.  Wegen [Satz 1.23] gilt für alle
  \(\varepsilon > 0\) dass
  \(U_{\varepsilon}(a) \cap M \ne \emptyset\).  Sei
  \((x_{n})_{n \in \mathbb{N}}\) eine Folge in
  \(U_{\varepsilon}(a) \cap M\) mit
  \(\lim_{n \to \infty}{x_{n}} = a\).  Wegen der Stetigkeit von \(f\) folgt
  \(\lim_{n \to \infty}{f(x_{n})} = f(a)\).  Es ist
  \(f(x_{n}) \in \overline{f(M)}\) für alle \(n \in \mathbb{N}\) wegen
  \(x_{n} \in M\) für alle \(n \in \mathbb{N}\). Nun ist die Menge
  \(\overline{f(M)}\) abgeschlossen und insbesondere
  folgenabgeschlossen.  Daraus folgt, dass
  \(f(a) \in \overline{f(M)}\).  Weil \(a \in \partial M\) beliebig gewählt war,
  folgt die Behauptung dass
  \(f(\partial M) \subseteq \overline{f(M)}\). Es gilt
  \(M \cup \partial M = \overline{M}\).  Es gilt auch, dass
  \(f(\overline{M}) = f(M \cup \partial M) = f(M) \cup f(\partial M)\).  Wir haben
  gezeigt, dass \(f(M) \subseteq \overline{f(M)}\) und
  \(f(\partial M) \subseteq \overline{f(M)}\).  Daraus folgt, dass
  \begin{align*}
    f(\overline{M}) = f(M) \cup f(\partial M) \subseteq \overline{f(M)}.
  \end{align*}

  Rückrichtung.
\end{proof}

\end{document}