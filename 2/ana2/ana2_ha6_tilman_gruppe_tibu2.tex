% !TeX program = lualatex
%%% TeX-engine: luatex

\documentclass[draft,a5paper]{article}
\usepackage[margin=2cm]{geometry}
\linespread{1.2}

\setlength{\headheight}{15pt}
\usepackage{fancyhdr}
\pagestyle{fancy}
\lhead{Yiwen Yang 466096, Yuchen Guo 480788, Meng Zhang 484981}
\rhead{HA 6, Ana 2, TiBu 2}

% set language to german
\usepackage[ngerman]{babel}

\usepackage[amsthm,vvarbb,varg]{newtx}

% used for integral with respect to a variable x
% like \int \dd{x}
% \usepackage{physics}

% theorem environment used in this document
\theoremstyle{remark}
\newtheorem*{beh}{Behauptung}
\newtheorem*{lem}{Lemma}

% \interval is used to provide better spacing after a [ that
% is used as a closing delimiter.
\newcommand{\interval}[1]{\mathinner{#1}}

% Enclose the argument in vert-bar delimiters:
\newcommand{\envert}[1]{\left\lvert#1\right\rvert}
\let\abs=\envert

% Enclose the argument in double-vert-bar delimiters:
\newcommand{\enVert}[1]{\left\lVert#1\right\rVert}
\let\norm=\enVert

\begin{document}
\subsection*{H 6.1}

\subsubsection*{H 6.1, Teil 1}

\begin{beh}
  Die Abbildung
  \[f\colon \mathbb{R}^{n \times n} \to \mathbb{R}^{n \times n},~ X \mapsto (X - B)^{2} + X + C\] mit
  \(B, C \in \mathbb{R}^{n \times n}\) ist differenzierbar für alle
  \(x_{0} \in \mathbb{R}^{n \times n}\).
\end{beh}

\begin{proof}
  Wir zeigen, dass \(f\) in alle \(x_{0} \in \mathbb{R}^{n \times n}\) diffbar ist,
  indem wir zeigen, dass es zu jedem
  \(x \coloneq x_{0} + v, v \in \mathbb{R}^{n \times n}\) eine lineare Abbildung
  \(F \colon \mathbb{R}^{n \times n} \to \mathbb{R}^{n \times n}\) und eine Funktion
  \(R \colon \mathbb{R}^{n \times n} \to \mathbb{R}^{n \times n}\) existiert, sodass gilt
\[ f(x_{0} + v) = f(x_{0}) + F(v) + R(x_{0} + v)
  \quad\text{mit}\quad
  \lim_{v \to 0}{\frac{R(x)}{\enVert{v}}} = 0.
\]
Es gilt
\begin{align*}
  f(x_{0} + v)
  &= (x_{0} + v - B)^{2} + (x_{0} + v) + C \\
  &= (x_{0} - B + v)^{2} + x_{0} + v + C \\
  &= \underbrace{(x_{0} - B)^{2} + x_{0}
    + C}_{f(x_{0})}
    + \underbrace{(x_{0} - B)v + v(x_{0} - B) + v}_{\text{linear in }v}
    + v^{2}.
\end{align*}
Wir setzen \(R(x_{0} + v) \coloneq v^{2}\) und prüfen die
Restgliedeigenschaften: Es gilt
\begin{align*}
  \lim_{v \to 0}{\frac{R(x_{0} + v)}{\norm{v}}} = \lim_{v \to
  0}{\frac{\norm{v^{2}}}{\norm{v}}} = \lim_{v \to 0}{\norm{v}} = 0.
\end{align*}
Es gilt daher \(Df(x_{0})v = (x_{0} - B)v + v(x_{0} - B) + v\).
\end{proof}

\subsubsection*{H 6.1, Teil 2}

\begin{beh}
  Die Abbildung
  \[g\colon \mathbb{R}_{\le 1}[x] \to \mathbb{R}^{2 \times 2},~ ax+b \mapsto
    \begin{bmatrix}
      a + b & a - b \\
      b - a & -(a + b)
    \end{bmatrix}
  \] ist differenzierbar für alle
  \((ax+b) \in \mathbb{R}_{\le 1}[x]\).
\end{beh}

\begin{proof}
  Zuerst bemerken wir, dass die Funktion \(g\) linear ist, denn es
  gilt für alle \((ax+b), (cx+d) \in \mathbb{R}_{\le 1}[x]\) und \(\lambda \in \mathbb{R}\) dass
  \begin{align*}
    g(\lambda(ax+b)+(cx+d)) = \lambda g(ax+b) + g(cx + d) =
    \begin{bmatrix}
      \lambda(a+b) + (c+d) & \lambda(a-b) + (c-d) \\
      \lambda(b-a) + (d-c) & -\lambda(a+b) - (c+d)
    \end{bmatrix}.
  \end{align*}
  Wegen [Beispiel 2.7] ist eine lineare Abbildung diffbar und die
  Ableitung ist sich selbst.
\end{proof}

\subsubsection*{H 6.1, Teil 3}

\begin{beh}
  Die Abbildung
  \[h\colon \mathbb{R}_{\le 1}[x] \to \mathbb{R}^{2 \times 1},~ ax+b \mapsto
    \begin{bmatrix}
      a^{2} + b^{2} \\
      a
    \end{bmatrix}
  \] ist differenzierbar für alle
  \((ax+b) \in \mathbb{R}_{\le 1}[x]\).
\end{beh}

\begin{proof}
  Wir zeigen, dass \(h\) in alle \((ax+b) \in \mathbb{R}_{\le 1}[x]\) diffbar ist,
  indem wir zeigen, dass es zu jedem
  \((ax+b) \in \mathbb{R}_{\le 1}[x]\) eine lineare Abbildung
  \(F \colon \mathbb{R}^{n \times n} \to \mathbb{R}^{n \times n}\) und eine Funktion
  \(R \colon \mathbb{R}^{n \times n} \to \mathbb{R}^{n \times n}\) existiert, sodass gilt
  \[ f((ax+b) + (cx+d)) = f(ax+b) + F(cx+d) + R((ax+b) + (cx+d)) \]
  mit
  \[
    \lim_{(cx+d) \to 0}{\frac{R((ax+b)(cx+d))}{\enVert{cx+d}}} = 0.
  \]
Es gilt
\begin{align*}
  f((ax+b) + (cx+d))
  &=
    \begin{bmatrix}
      (a+c)^{2} + (b+d)^{2} \\
      a + c
    \end{bmatrix}
    =
    \underbrace{
    \begin{bmatrix}
      a^{2} + b^{2} \\
      a
    \end{bmatrix}}_{h(ax+b)}
    +
    \underbrace{
    \begin{bmatrix}
      2ac+2bd \\
      c
    \end{bmatrix}}_{\text{linear in } cx+b}
    +
    \underbrace{
    \begin{bmatrix}
      c^{2} + d^{2} \\
      0
    \end{bmatrix}}_{\eqcolon R((ax+b) + (cx+d))}
\end{align*}
Wir prüfen die Restgliedeigenschaften von \(R\).  Zuerst bemerken wir,
dass die Norm auf dem Raum der Polynome nicht definiert ist.  Aus der
linearen Algebra ist uns bekannt, dass der Raum der Polynome mit
gegebenem Grad endlich dimensional ist.  Aus der VL wissen wir, dass
Normen auf endlich dimensionalen Vektorräumen äquivalent sind.  Daher
definieren wir die Norm auf \(\mathbb{R}_{\le 1}[x]\) mit
\[ax+b \mapsto \sqrt{a^{2} + b^{2}}.\] Es gilt
\begin{align*}
  \lim_{(cx+d) \to 0}{\frac{R((ax+b)+(cx+d))}{\norm{cx+d}}} =
  \lim_{(cx+d) \to 0}{\frac{
  \begin{bmatrix}
    c^{2} + d^{2} \\
    0
  \end{bmatrix}
  }{\sqrt{c^{2} + d^{2}}}} =
  \lim_{(cx+d) \to 0}{
  \begin{bmatrix}
    \sqrt{c^{2} + d^{2}} \\
    0
  \end{bmatrix}}
  =  0.
\end{align*}
Es gilt daher \(Df(ax+b)(cx+d) =
\begin{bmatrix}
  2ac+2bd \\
  c
\end{bmatrix}\).
\end{proof}

\subsection*{H 6.2}

Sei \(f\colon \mathbb{R}^{2} \to \mathbb{R}\) eine Funktion.  Weiter gilt für alle
\(v = (v_{1}, v_{2}) \in \mathbb{R}^{2}\setminus\{(0, 0)\}\) dass
\[\partial_{v}f(0, 0) = \frac{\abs{v_{1}}v_{2}}{\abs{v_{1}} + \abs{v_{2}}} +
  3v_{1} - 2v_{2}.\]

Wir berechnen die partiellen Ableitungen von \(f\) in \((0, 0)\).  Es
gilt
\[\frac{\partial f}{\partial v_{i}}(x) \coloneq \partial_{e_{i}}f(x).\] Daraus folgt, dass die
partiellen Ableitung von \(f\) in \((0, 0)\) nach \(v_{1}\) ist \(3\)
und nach \(v_{2}\) ist \(-2\).

Die Funktion \(f\) ist in \((0, 0)\) nicht differenzierbar.
Angenommen, die Funktion \(f\) ist in \((0, 0)\) differenzierbar.  Aus
der VL ist bekannt, dass dann für alle \((v_{1}, v_{2}) \in \mathbb{R}^{2}\) die
Gleichung
\[Df(0, 0)(v_{1}, v_{2}) = \partial_{(v_{1}, v_{2})}f(0, 0)\]
gelten muss.  Aufgrund der Linearität von \(Df(0, 0)\) muss dann also
insbesondere
\[Df(0, 0)(v_{1}, v_{2}) = Df(0, 0)(v_{1}, 0) + Df(0, 0)(0, v_{2})\]
gelten.  Dies ist aber für \(v_{1} = v_{2} = 1\) falsch, wegen
\[Df(0, 0)(1, 1) = \frac{1}{2} + 3 - 2 \ne 3 - 2 = Df(0, 0)(1, 0) +
  Df(0, 0)(0, 1).\]

\subsection*{H 6.3}

Seien \(V, W\) endlichdimensionale Banachräume und sei die Abbildung
\(f\colon V \to W \) sodass für ein festes
\(k \in \mathbb{N}\) und für alle \(t > 0\) und alle \(v \in V\) gilt
\[f(tv) = t^{k}f(v).\] Weiter sei \(f\) in \(v \in V \) differenzierbar.

\begin{beh}
  Es gilt \[Df(v)v = kf(v).\]
\end{beh}

\begin{proof}
  Wir betrachten die Funktion \(g\) von
  \(t \in \interval{]0, \infty[}, \) mit
  \[ g\colon \interval{]0, \infty[} \to W, \quad t \mapsto t^{-k}f(tv) \] Nach der
  Kettenregel und \(f\) differenzierbar in \(v\) ist \(g\)
  differenzierbar in \(v\) mit
  \[Dg(t) = -kt^{-k-1}f(tv) + t^{-k}Df(tv)v\] für alle \(t>0\).  Wegen
  Voraussetzung dass \(f\) homogen ist und \(f(tv) = t^{k}f(v)\) gilt,
  folgt \(g(t) = f(v)\) konstant. Also, der Wert von \(g(t)\) ist das
  Gleiche egal welcher Wert \(t\) ist.  Weil \(g(t)\) konstant ist,
  ist \(Dg(t) = 0\).  Wir setzen \(t=1\) in \(Dg(t) = 0\) ein.  Daraus
  folgt, dass
  \[0 = -kf(v) + Df(v)v. \] Damit gilt die Behauptung.
\end{proof}

\end{document}