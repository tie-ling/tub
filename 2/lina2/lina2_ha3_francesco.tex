\documentclass[draft,a5paper]{article}
\usepackage[ngerman]{babel}
\usepackage[margin=2cm]{geometry}
\usepackage{amsmath,mathtools,fontspec,amssymb,mathtools}

% for theorems and lemma
\usepackage{amsthm}
\newtheorem*{beh}{Behauptung}
\newtheorem*{lem}{Lemma}

% times new roman font
\usepackage{newtx}

\linespread{1.2}

\author{Yuchen Guo 480788, Meng Zhang 484981}

\date{\today}

\title{HA 3, LinA 2 -- Francesco Patrick Nowell}

% fuctions exclusive to this HA
\DeclareMathOperator{\sign}{sign}

\begin{document}
\maketitle

\newpage

\subsection*{Aufgabe 3.1}

Es seien \(n \in \mathbb{N}\), \(n \ge 2\) und
\(a_{1}, \ldots, a_{n} \in \mathbb{R}\).  Weiter sei
\(A = I_{n} + B\), indem \(B \in \mathbb{R}^{m \times n}\) mit
\(B_{i, j} = a_{j}\) gilt.

\begin{beh}
  Falls \(m > n\), dann ist \(A\) linear abhängig und es gilt \(\det (A)
= 0\).
\end{beh}

\begin{proof}
  Die Funktion \(\det\) ist alternierend.
\end{proof}

\begin{beh}
Falls \(m \le n\) und \(a_{i} = 0\) für alle \(1 \le i \le n\), dann gilt
\(\det(A) = 1\).
\end{beh}

\begin{proof}
  In diesem Fall ist \(A = I_{n}\).
\end{proof}

\begin{beh}
Falls \(m \le n\) und \(a_{j} \ne 0\) und \(a_{i} = 0\) für alle \(1 \le i \le
n, i \ne j\), dann gilt
\(\det(A) = 2\).
\end{beh}

\begin{proof}
Wir bezeichnen die Zeilen
von \(A\) als \(\alpha_{i}\), die Zeilen von \(I_{n}\) als
\(\beta_{i}\) und die Zeilen von \(B\) als \(\gamma_{i}\).  Dann gilt,
\begin{align*}
  \det(A)
  &= \det(I_{n} + B) \\
  \det(\alpha_{1}, \ldots, \alpha_{n})
  &= \det(\beta_{1} + \gamma_{1}, \beta_{2} + \gamma_{2}, \ldots, \beta_{m}
    + \gamma_{m}, \ldots \beta_{n}) \\
  &= \det(\beta_{1}, \ldots) + \det(\gamma_{1}, \ldots)
\end{align*}

Es gibt nämlich nur zwei Kombinationen, die linear unabhägig sind und
äquivalent zur \(I_{n}\) sind.  Die sind
\begin{align*}
  \det(\beta_{1}, \ldots, \beta_{j}, \ldots, \beta_{m}, \beta_{m+1}, \ldots, \beta_{n}) &= 1, \\
  \det(\beta_{1}, \ldots, \gamma_{j}, \ldots, \beta_{m}, \beta_{m+1}, \ldots, \beta_{n}) &= 1.
\end{align*}

Daraus folgt \(\det(A) = 2\).
\end{proof}

\begin{beh}
Falls \(m \le n\) und \(a_{i} \ne 0\) für alle \(1 \le i \le n\), dann gilt
\(\det(A) = m + 1\).
\end{beh}

\begin{proof}
  Analog zum oben, bezeichnen wir die Zeilen von \(A\) als
  \(\alpha_{i}\), die Zeilen von \(I_{n}\) als \(\beta_{i}\) und die Zeilen von
  \(B\) als \(\gamma_{i}\).  Dann gilt,
\begin{align*}
  \det(A)
  &= \det(I_{n} + B) \\
  \det(\alpha_{1}, \ldots, \alpha_{n})
  &= \det(\beta_{1} + \gamma_{1}, \beta_{2} + \gamma_{2}, \ldots, \beta_{m}
    + \gamma_{m}, \ldots \beta_{n}) \\
  &= \det(\beta_{1}, \ldots) + \det(\gamma_{1}, \ldots)
\end{align*}

Wir betrachten die Determinant
\(\det(\beta_{1}, \ldots) + \det(\gamma_{1}, \ldots)\).  Um die Matrix linear unabhängig
zu sein, dürfen wir höchstens ein Mal ein \(\gamma_{i}\) Vektor in den
\(m\) Zeilen setzen.  Das heißt
\begin{align*}
  \det(\beta_{1}, \beta_{2}, \beta_{3}, \ldots, \beta_{m}, \beta_{m + 1}, \ldots, \beta_{n}) &= 1 \\
  \det(\gamma_{1}, \beta_{2}, \beta_{3}, \ldots, \beta_{m}, \beta_{m + 1}, \ldots, \beta_{n}) &= 1 \\
  \det(\beta_{1}, \gamma_{2}, \beta_{3}, \ldots, \beta_{m}, \beta_{m + 1}, \ldots, \beta_{n}) &= 1 \\
  \det(\beta_{1}, \beta_{2}, \gamma_{3}, \ldots, \beta_{m}, \beta_{m + 1}, \ldots, \beta_{n}) &= 1 \\
                               &\vdots \\
  \det(\beta_{1}, \beta_{2}, \beta_{3}, \ldots, \gamma_{m}, \beta_{m + 1}, \ldots, \beta_{n}) &= 1
\end{align*}
Daraus folgt, dass es insgesamt \(m+1\) Kombinationen von Vektoren
gibt, die linear unabhängig und äquivalent zu \(I_{n}\) sind.  Daraus
folgt, dass \(\det(A) = m + 1\).
\end{proof}

\subsection*{Aufgabe 3.2}

\begin{beh}
Sei \(\mathbb{K}\) ein Körper mit \(1 + 1 \ne 0\) und sei \(S_{n} \in \mathbb{K}^{n \times n}\)
mit \(S_{n} = - S_{n}^{T}\).  Dann existiert ein \(\lambda \in \mathbb{K}\) mit
\(\det(S_{n}) = \lambda^{2}\).
\end{beh}

\begin{proof}
  Wegen Definition von Körper gilt insbesondere \(\lambda^{2} \ge 0\) für alle
  \(\lambda \in \mathbb{K}\).  Wir zeigen, dass \(\det(S_{n}) \ge 0\) gilt.

  Es gilt wegen Satz 18.1 dass \(\det(S_{n}) = \det(S_{n}^{T})\).  Es
  gilt wegen Voraussetzung dass \(\det(S_{n}) = \det(-S_{n}^{T})\).
  Daraus folgt, dass \(\det(S_{n}^{T})  = \det(-S_{n}^{T})\).

  Daraus folgt, dass
  \begin{align*}
    \det(S_{n}^{T})  &= \det(-S_{n}^{T}) \\
    \sum_{\sigma \in \mathcal{S}_{n} }{\sign(\sigma) \prod_{i = 1}^{n}{a_{\sigma(i), i}}}
                     &=  \sum_{\sigma \in \mathcal{S}_{n} }{\sign(\sigma) \prod_{i =
                       1}^{n}{-a_{\sigma(i), i}}} \\
    2 \det(S_{n}) &= 0.
  \end{align*}
  Damit existiert ein \(\lambda = 0\) mit \(\lambda^{2} = \det(S_{n})\).
\end{proof}

\subsection*{Aufgabe 3.3}

Sei der Matrix \(B_{n} = (b_{k, l}) \in \mathbb{R}^{n \times n}\).  Für
\(k = 1\) oder \(l = 1\) gilt \(b_{k, l} = 1\) und sonst
\(b_{k, l} = b_{k-1, l} + b_{k, l - 1}\).

\begin{beh}
  Es gilt \(\det(B_{n}) = 1\).
\end{beh}

\begin{proof}
  Wir zeigen, dass die Matrix \(B_{n}\) symmetrisch ist, mittels
  vollständigen Induktion über \(k, l\).
  \begin{itemize}
  \item Induktionsanfang, \(k = 1\).  Dann ist wegen Definition dass \((B_{n})_{k, l} = (B_{n})_{l,
      k} = 1\) für alle \(l \in \mathbb{N}\).
  \item   Induktionsvoraussetzung, es gilt für ein fest gewähltes \(k\) dass
    \((B_{n})_{k, l} = (B_{n})_{l, k}\) für alle \(l \in \mathbb{N}\).
  \item   Induktionsschritt. Es gilt     \((B_{n})_{k + 1, l} = (B_{n})_{l, k + 1}\).
  \item Induktionsanfang, \(l = 1\).  Dann ist wegen Definition dass \((B_{n})_{k, l} = (B_{n})_{l,
      k} = 1\) für alle \(k \in \mathbb{N}\).
  \item   Induktionsvoraussetzung, es gilt für ein fest gewähltes \(k\) dass
    \((B_{n})_{k, l} = (B_{n})_{l, k}\).
  \item   Induktionsschritt. Es gilt     \((B_{n})_{k + 1, l} = (B_{n})_{l, k + 1}\).
  \end{itemize}



  Die Matrix \(B_{n}\) äquivalent zur Identitätsmatrix
  \(I_{n}\) ist, indem wir die Matrix \(B_{n}\) im Zeilenstufenform von
  Definition 14.6 bringen.
  \begin{align*}
    \begin{bmatrix}
      1 & 1 & 1 & 1 \\
      1 & 2 & 3 & 4 \\
      1 & 3 & 6 & 10 \\
      1 & 4 & 10 & 20
    \end{bmatrix}
    \to
    \begin{bmatrix}
      1 & 1 & 1 & 1 \\
      0 & 1 & 2 & 3 \\
      0 & 2 & 5 & 9 \\
      0 & 3 & 9 & 19
    \end{bmatrix}
    \to
    \begin{bmatrix}
      1 & 1 & 1 & 1 \\
      0 & 1 & 2 & 3 \\
      0 & 0 & 1 & 3 \\
      0 & 0 & 3 & 10
    \end{bmatrix}
    \to
    \begin{bmatrix}
      1 & 1 & 1 & 1 \\
      0 & 1 & 2 & 3 \\
      0 & 0 & 1 & 3 \\
      0 & 0 & 0 & 1
    \end{bmatrix}
    \to
    \begin{bmatrix}
      1 & 0 & 0 & 0 \\
      0 & 1 & 0 & 0 \\
      0 & 0 & 1 & 0 \\
      0 & 0 & 0 & 1
    \end{bmatrix}
  \end{align*}
\end{proof}

\end{document}