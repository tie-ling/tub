\documentclass[draft,a5paper]{article}
\usepackage[ngerman]{babel}
\usepackage[margin=2cm]{geometry}
\usepackage{amsmath,mathtools,fontspec,amssymb,mathtools}

% for theorems and lemma
\usepackage{amsthm}
\newtheorem*{beh}{Behauptung}
\newtheorem*{lem}{Lemma}

% times new roman font
\usepackage{newtx}

% for pics
\usepackage{graphicx}
\graphicspath{{./}}

\linespread{1.2}

% \author{Yuchen Guo 480788, Meng Zhang 484981}

\author{Ruihao Luan 479823, Yuchen Guo 480788}

\date{\today}

\title{HA 1, LinA 2 -- Francesco Patrick Nowell}

% fuctions exclusive to this HA
\DeclareMathOperator{\sign}{sign}
\DeclareMathOperator{\id}{id}

\begin{document}
\maketitle

\newpage

\subsection*{Aufgabe 1.1}

\includegraphics[width=\textwidth]{aufgabe1}

\newpage

\subsection*{Aufgabe 1.2}

Sei \(\sigma \in \mathcal{S}_{n}\) eine Permutation.  Die zugehörige Permutationsmatrix
wird spaltenweise durch \(P_{\sigma} = (\mathbf{e}_{\sigma(i)})_{i = 1, \ldots, n} \in
\mathbb{K}^{n \times n}\) definiert.  \(\mathbf{e}_{i} \in \mathbb{K}^{n}\)
ist der \(i\)-ten Standardeinheitsvektor.

\subsubsection*{Aufgabe 1.2, Teil i}

\begin{lem}
  Seien \(A, B, C\) Matrizen in \( \mathbb{K}^{n \times n}\).  Dann gilt \(
  A(BC) = (AB)C \).
\end{lem}

\begin{proof}
  Es gilt \((BC)_{i,j} = \sum_{h = 1}^{n}{B_{i,h} C_{h,j}}\) und
  \((AM)_{k,l} = \sum_{q = 1}^{n}{A_{k,q} M_{q,l}}\).  Daraus folgt,
  \begin{align*}
    [A(BC)]_{k,l}
    &= \sum_{q=1}^{n}{A_{k,q}(BC)_{q,l}} \\
    &= \sum_{q=1}^{n}{A_{k,q}\sum_{p=1}^{n}{B_{q,p}C_{p,l}}} \\
    &= \sum_{p=1}^{n}{\left(\sum_{q=1}^{n}{A_{k,q}B_{q,p}}\right)C_{p,l}} \\
    &= \sum_{p=1}^{n}{(AB)_{k,p}C_{p,l}} \\
    &= [(AB)C]_{k,l}.
  \end{align*}
  Damit gilt die Behauptung.
\end{proof}

\begin{beh}
  Die Menge aller Permutationsmatrizen \(\mathcal{P} = \{ P_{\sigma} \mid \sigma \in
  \mathcal{S}_{n}\}\) bildet eine Gruppe zusammen mit der Matrixmultiplikation.
\end{beh}

\begin{proof}
  Wir zeigen, dass die Menge \(\mathcal{P} \) mit diese Verknüpfung eine Gruppe
  bildet, indem wir zeigen, dass die Gruppenaxiome erfüllt sind.

  \begin{itemize}
  \item Für alle \(A, B, C \in \mathcal{P} \) gilt \( A(BC) = (AB)C \).  Diese
    Aussage folgt direkt aus dem Lemma.
  \item Es gibt ein \( E \in \mathcal{P} \), sodass für alle \( A \in \mathcal{P} \) gilt \(
    EA = A\).

    Sei \( E \coloneq (\mathbf{e}_{\id (i)}) =
    (\mathbf{e}_{i}) \). Dann ist \(E\) die \(n \times n\)
    Einheitsmatrix und es gilt \(EA=A\) für alle \(A \in \mathbb{K}^{n
    \times n}\).
  \item Zu jedem \( A \in \mathcal{P} \) gibt es ein \( B \in \mathcal{P} \) mit \( B A = E \).

    Seien \( A \coloneq (\mathbf{e}_{\sigma(i)}) \) und \( B \coloneq
    (\mathbf{e}_{\sigma^{-1}(i)})\), weil eine Permutation \(\sigma\colon \mathbb{N} \to
    \mathbb{N} \) bijektiv ist. Dann ist \( B A = E\), denn
    \begin{align*}
      (BA)_{i,j}
      = \sum_{k=1}^{n}{B_{i,k}A_{k,j}}.
    \end{align*}
    Wegen Voraussetzung ist \(B_{i,k} = \delta_{i, \sigma^{-1}(i)}\) und \(A_{k,j} =
    \delta_{\sigma(j), j}\).  Dann ist \(BA_{i,j}\) genau \(1\), falls \(i =
    \sigma^{-1}(i)\) und \(\sigma(j) = j\).  D.h.,
    \begin{align*}
      \sigma(i) = \sigma(\sigma^{-1}(i)) &= i, \\
      \sigma(j) &= j.
    \end{align*}
  \end{itemize}
  Das heißt, \((BA)_{i,j}\) ist genau dann \(1\), wenn \(i = j\) gilt.
  Damit ist \(BA = E\).
\end{proof}

\subsubsection*{Aufgabe 1.2, Teil ii}

\begin{beh}
  Die Gruppen \((\mathcal{S}_{n}, \circ)\) und \((\mathcal{P}, \cdot)\) sind isomorph.
\end{beh}

\begin{proof}
  Wegen Satz von Cayley (Jede Gruppe \(G\) mit \(n\) Elementen ist
  isomorph zu einer Untergruppe von \(\mathcal{S}_{n}\)), zeigen wir, dass es
  genau \(n!\) Elementen in \(\mathcal{P} \) gibt.
  Diese folgt aber sofort aus der bijektivität von Permutationen,
  d.h., eine Permutationsmatrix ist durch eine Permutation eindeutig
  bestimmt und umgekehrt.
\end{proof}

\subsubsection*{Aufgabe 1.3}

\begin{beh}
  Die Relation \(R\) auf der Menge \(\mathbb{K}^{n \times n}\) mit
  \begin{align*}
    (A, B) \in R \quad \iff  \quad \text{es existiert eine Permutationsmatrix }
    P_{\sigma} \in \mathcal{P} \text{ mit } A = P_{\sigma}^{T}BP_{\sigma}
  \end{align*}
  ist eine Äquivalenzrelation.
\end{beh}

\begin{proof}
  Wir zeigen, dass diese Relation reflexiv, symmetrisch und transitiv
  ist.  Es gilt \(P^{-1} = P^{T}\).
  \begin{itemize}
  \item   Diese Relation ist reflexiv, denn sei \(\sigma \coloneq \id\).

    Dann ist \(P_{\sigma}^{T} = P_{\sigma} \) die Einheitsmatrix und es gilt
    \(A = P_{\sigma}^{T}BP_{\sigma}\) für alle \(A \in \mathbb{K}^{n \times n}\).

  \item   Diese Relation ist symmetrisch, denn sei \(A, B \in
    \mathbb{K}^{n \times n}\) mit \((A, B) \in R\).  Wir zeigen, dass \((B,
    A) \in R\) gilt, d.h., es existiert eine Permutationsmatrix \(P_{\sigma^{-1}}
    \in \mathcal{P} \) mit \(B = P_{\sigma^{-1}}^{T}AP_{\sigma^{-1}}\).  Diese gilt wegen
    \begin{align*}
      B &= P_{\sigma^{-1}}^{T}AP_{\sigma^{-1}} \\
        &= P_{\sigma^{-1}}^{T}(P_{\sigma}^{T}BP_{\sigma})P_{\sigma^{-1}} \\
        &= EBE \\
        &= B.
    \end{align*}
  \item Diese Relation ist transitiv, denn sei \((A, B) \in R\) und
    \((B, C) \in R\) mit
    \begin{align*}
      A &= P_{\sigma_{1}}^{T} B P_{\sigma_{1}}, \\
      B &= P_{\sigma_{2}}^{T} C P_{\sigma_{2}}.
    \end{align*}
    Wir zeigen \((A, C) \in R\).  Das heißt, es existiert ein \(\sigma_{3} \in
    \mathcal{S}_{n}\) mit \(A = P_{\sigma_{3}}^{T} C P_{\sigma_{3}}\).  Es gilt
    \begin{align*}
      A &= P_{\sigma_{1}}^{T} B P_{\sigma_{1}} \\
        &= P_{\sigma_{1}}^{T} \left(P_{\sigma_{2}}^{T} C P_{\sigma_{2}}\right) P_{\sigma_{1}}.
    \end{align*}
  \end{itemize}
  Daraus folgt, dass \(\sigma_{3}\) existiert und es gilt \(\sigma_{3} \coloneq \sigma_{2} \circ \sigma_{1}\).
\end{proof}

\subsection*{Aufgabe 1.4}

Sei \(n \ge 2\) und \(\sigma \in \mathcal{S}_{n}\).

\subsubsection*{Aufgabe 1.4, Teil i}

\begin{beh}
  Falls \(\sigma \ne \id\), dann existieren \(i, j \in \{1, \ldots, n\}\) mit \(\sigma(i)
  > i\) und \(\sigma(j) < j\).
\end{beh}

\begin{proof}
  Wir bemühen uns um einen Widerspruchsbeweis.  Angenommen, es gilt
  für alle \(i, j \in \{1, \ldots, n\}\) dass \(\sigma(i) \le i\) und
  \(\sigma(j) \ge j\).  Dann ist insbesondere \(\sigma(i) = i\) und
  \(\sigma(j) = j\) für alle \(i, j \in \{1, \ldots, n\}\).  Dann gilt im
  Widerspruch zur Voraussetzung dass \(\sigma = \id\).
\end{proof}

\subsubsection*{Aufgabe 1.4, Teil ii}

\begin{beh}
  Es existieren \(i_{1}, \ldots, i_{k} \in \{1, \ldots, n - 1\}\), sodass \(\sigma =
  \tau_{i_{k}} \circ \cdots \circ \tau_{i_{1}}\) gilt, wobei \(\tau_{i_{j}} = (i_{j}, i_{j}
  + 1)\) ist.  Diese Aussage ist falsch.
\end{beh}

\begin{proof}
  ???
\end{proof}

\end{document}