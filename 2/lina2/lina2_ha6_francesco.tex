% !TeX program = lualatex

\documentclass[draft,a5paper]{article}
\usepackage[margin=2cm]{geometry}
\linespread{1.2}

% set language to german
\usepackage[ngerman]{babel}

% load math packages and unicode support
% must in this order
\usepackage{amsmath,mathtools,fontspec}
\usepackage{amsthm}
\usepackage{unicode-math,luatex85}

% set fonts
\setmainfont{STIX Two Text}

% theorem environment used in this document
\theoremstyle{remark}
\newtheorem*{beh}{Behauptung}
\newtheorem*{lem}{Lemma}

\author{Meng Zhang 484981, Yuchen Guo 480788}
\date{\today}
\title{HA 6, LinA 2 -- Francesco}

% math operators
\DeclareMathOperator{\Spek}{Spek}
\DeclareMathOperator{\Ker}{Ker}
\DeclareMathOperator{\GL}{GL}
\DeclareMathOperator{\rg}{rg}

%    Enclose the argument in vert-bar delimiters:
\newcommand{\envert}[1]{\left\lvert#1\right\rvert}
\let\abs=\envert

%    is used as a closing delimiter.
\newcommand{\interval}[1]{\mathinner{#1}}

\begin{document}
\maketitle

\subsection*{Aufgabe 6.1}

Sei \(A \in \mathbb{C}^{n \times n}\). Wir nennen die Matrix \(A\) nilpotent, falls es
ein \(m \in \mathbb{N}_{\ge 1}\) gibt mit \(A^{m} = 0\).

\subsubsection*{Aufgabe 6.1, Teil i}

\begin{lem}
  Die Matrix \(A\) ist nilpotent genau dann, wenn \(A\) nicht
  invertierbar ist.
\end{lem}

\begin{proof}
  Weil \(A\) nilpotent ist, existiert ein \(m \in \mathbb{N}_{\ge 1}\) mit
  \(A^{m} = 0\).  Weil die Determinant-Funktion alternierend ist, gilt
  \(\det(A^{m}) = 0\).  Es gilt
  \begin{align*}
    \det(A^{m}) = \det(A)\det(A^{m-1}) = \det(A)\det(A)\det(A^{m-2}) =
    \ldots = \det(A)^{m} = 0.
  \end{align*}
  Damit gilt \(\det(A) = 0\) und \(A\) ist nicht invertierbar.  Damit
  gilt die Behauptung.
\end{proof}

\begin{beh}
  Es gilt \(\Spek(A) = \{0\}\) genau dann, wenn \(A\) nilpotent ist.
\end{beh}

\begin{proof}
  Hinrichtung.  Angenommen, \(\Spek(A) = \{0\}\).  Dann ist die Matrix
  \(A - \lambda I\) nicht invertierbar für \(\lambda = 0\) und invertierbar für
  alle \(\lambda \in \mathbb{C} \setminus \{0\}\).  Wegen [Lemma], ist
  \(A\) nilpotent genau dann, wenn \(A\) nicht invertierbar ist.
  Daraus folgt, dass \(A - 0I = A\) nilpotent.  Damit gilt die Behauptung.

  Rückrichtung.  Angenommen, \(A\) ist nilpotent.  Wir betrachten die
  Matrix \(A - \lambda I\) für alle \(\lambda \in \mathbb{C}\).  Zuerst bemerken wir, dass
  wegen \(\det(A - 0I) = \det(A) = 0\) gilt \(0 \in \Spek(A)\).  Sei nun
  \(\lambda \in \mathbb{C} \setminus \{0\}\) beliebig.  Dann gilt
  \begin{align*}
    A^{m} - \lambda^{m}I = 0 - \lambda^{m}I = -\lambda^{m}I = (A - \lambda I)\sum_{k=0}^{n-1}{\lambda^{k}A^{n-1-k}}.
  \end{align*}
  Insbesondere gilt
  \begin{align*}
    \det(A - \lambda I) \det(\sum_{k=0}^{n-1}{\lambda^{k}A^{n-1-k}}) = \det(-\lambda^{m} I)
    \ne 0.
  \end{align*}
  Daraus folgt, dass \(A - \lambda I \ne 0\) für alle
  \(\lambda \in \mathbb{C} \setminus \{0\}\).  Damit gilt
  \(\lambda \notin \Spek(A)\) für alle \(\lambda \in \mathbb{C} \setminus \{0\}\).  Damit gilt die Behauptung.
\end{proof}

\subsubsection*{Aufgabe 6.1, Teil ii}

\begin{beh}
  Sei \(A\) nilpotent.  Dann ist \(\chi_{A}(t) = t^{n}\). Das
  Minimalpolynom von \(A\) ist \(t\).
\end{beh}

\begin{proof}
  Wegen [Teil i] ist \( \Spek(A) = \{0\} \).  Es gilt
  \(\chi_{A}(t) = 0\) genau dann, wenn \(t=0\) ist.  Die Gleichung
  \(\chi_{A}(t) = 0\) hat keine Nullstellen in \(t \in \mathbb{C}_{\ne 0}\).  Daraus
  folgt, dass \(\chi_{A}(t) = t^{n}\) und \(\mu_{A} = t\) ist
  Minimalpolynom von \(A\).
\end{proof}

\subsubsection*{Aufgabe 6.1, Teil iii}

\begin{beh}
  Sei \(A\) nilpotent.  Dann ist \(\det(A + I_{n}) = 1\).
\end{beh}

\begin{proof}
  Wir bemühen uns um einen Widerspruchsbeweis.  Angenommen,
  \(\det(A+I_{n}) \ne 1\).  Dann existiert mindestens einen Eigenwert
  \(\lambda \ne 1\) von der Matrix \(A+I_{n}\) und einen Eigenvektor
  \(v \ne 0\) zu \(\lambda\) dass
  \[ \lambda v = (A + I_{n}) v = Av + I_{n} v. \] Daraus folgt, dass
  \((\lambda-1)v = Av\).  Wegen \(\lambda \ne 1\) und \(v \ne 0\) gilt
  \(Av \ne 0\).  Durch Induktion erhalten wir \(A^{m}v \ne 0\) im
  Widerspruch zur Voraussetzung dass \(A\) nilpotent ist.
\end{proof}

\subsection*{Aufgabe 6.3}

Sei \(V\) Vektorraum mit \(\dim V < \infty\) und \(f \in L(V, V)\).

\subsubsection*{Aufgabe 6.3, Teil i}

\begin{beh}
  Die Funktion \(f\) ist eine Projektion genau dann, wenn das
  Minimalpolynom \(\mu_{f}\) das Polynom \(t(1-t)\) teilt.
\end{beh}

\begin{proof}
  Hinrichtung.  Angenommen, \(f\) ist eine Projektion.  Wegen [Aufgabe
  5.4] ist \(\{1\} \subseteq \Spek(f) \subseteq \{0, 1\}\).  Wir betrachten zwei
  Fällen.  Falls \(\Spek(f) = \{1\}\).  Dann ist
  \(\chi_{f}(t) = (t-1)^{n}\) und das Minimalpolynom teilt
  \(\chi_{f}(t)\). Damit folgt die Behauptung.  Falls
  \(\Spek(f) = \{0, 1\}\).  Dann ist \(\chi_{f}(t) = t^{m}(t-1)^{n}\) und
  das Minimalpolynom teilt \(\chi_{f}(t)\).  Damit folgt die Behauptung.

  Rückrichtung.  Angenommen, das Minimalpolynom \(\mu_{f}\) teilt das
  Polynom \(t(1-t)\).
\end{proof}

\end{document}
