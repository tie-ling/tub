%% page style
\documentclass[12pt]{extarticle}
\usepackage[margin=2cm]{geometry}
\usepackage{fancyhdr,parskip}
\pagestyle{fancy}
\usepackage[onehalfspacing]{setspace}
\setlength{\parindent}{0pt}
\lhead{\myAuthor}
\rhead{\mySubject \ \myHausaufgaben. Übungsblatt \\ \myTutor}
\renewcommand*\familydefault{\sfdefault} %% Only if the base font of the document is to be sans serif

%% language
\usepackage[utf8]{inputenc}
\usepackage{xcharter-otf}
\usepackage[ngerman]{babel}

%% default packages
\usepackage{amsmath,mathtools,fontspec,amsthm,amssymb,amsfonts,
  stmaryrd, % for the lightning symbol used in proof by contraction
  tikz,     % used to draw diagrams
}

%% metadata
\newcommand{\myAuthor}{Yuchen Guo 480788 | Meng Zhang 484981}
\newcommand{\myHausaufgaben}{10}
\newcommand{\mySubject}{LinA}
\newcommand{\myTutor}{Saskia}


%% custom commands
\newcommand{\mg}[1]{\mathbb{#1}}
\newcommand{\lin}{\operatorname{lin}}
\newcommand{\aufgn}[1]{\textbf{Aufgabe #1.}}
\newcommand{\beh}{\textit{Behauptung.}\ }
\newcommand{\Bild}{\operatorname{Bild}}

\begin{document}
\aufgn{10.1.i}

\beh Wenn \(\dim(U) = \dim(W) = n - 1\) gilt, dann gilt
\(\dim(U \cap W) \ge n - 2\).

\begin{proof}
Es gilt wegen (Satz 9.4) dass
\begin{align*}
  \dim(U + W)
  &= \dim U + \dim W - \dim (U \cap W) \\
  &= 2n - 2 - \dim (U \cap W)
\end{align*}
Wegen \(U, W \subseteq V\) gilt \(\dim(U + W) \le n\).
Daraus folgt, dass
\begin{align*}
\dim(U \cap W) \ge n - 2.
\end{align*}
\end{proof}

\aufgn{10.1.ii}

\beh Wenn \(\dim(U) + \dim(W) > n\) gilt, dann gilt $U
\cap W \ne \left\{ 0 \right\}$.

\begin{proof}
  Wir beweisen diese Aussage mittels Kontraposition.  Zu
  zeigen: $U \cap W \ne \left\{ 0 \right\} \implies \dim
  U + \dim W \le n$.

  Angenommen, \(U \cap W = \left\{ 0 \right\}\).  Dann
  gilt \(\dim(U \cap W) = 0\).

Daraus folgt, dass
\begin{align*}
  \dim(U + W)
  &= \dim U + \dim W - \dim (U \cap W) \\
  &= \dim U + \dim W \\
  &\le n
\end{align*}
\end{proof}

\aufgn{10.1.iii}

\beh Sei \(V = \mg{R}^3\) und \(\dim U = \dim W = 2\).  Dann
gilt \(U = W\) genau dann, wenn \(\dim (U \cap W) = 2\)
gilt.

\begin{proof}
\textit{Hinrichtung.} Es gilt \(U = W\).  Daraus folgt, $U
+ U = U + W = U$.  Wegen (Satz 9.3) gilt
\begin{align*}
  \dim(U+W)
  &= \dim U = 2 \\
  &= \dim U + \dim W - \dim (U \cap W) \\
  &= 2 + 2 - \dim (U \cap W).
\end{align*}
Daraus folgt \(\dim (U \cap W) = 2\).

\beh Rückrichtung Lemma.  Sei \(V = \mg{R}^3\) und
\(\dim U = \dim W = 2\) und \(U\), \(W\) Untervektorräume von
\(V\) mit \(U \ne W\).  Dann gilt \(V = U + W\).

\begin{proof}
Angenommen, \(V \ne U + W\).  Weil \(U\), \(W\)
Untervektorräume von \(V\) sind, gilt \((U + W) \subset V\)
und \(\dim (U+W) < 3\).

Falls \(\dim (U+W) = 2\).  Dann gilt im Widerspruch zur
Voraussetzung dass \(U = W\).  (Hinrichtung.)

Falls \(\dim (U+W) \le 1\).  Dann gilt
\begin{align*}
\dim(U \cap W) = \dim U + \dim W - \dim(U+W) \ge 3.
\end{align*}
Aber es gilt \(U \cap W \subseteq U\) und \(\dim U = 2\).
Diese ist daher ein Widerspruch zur \(\dim U = 2\).
\end{proof}

\textit{Rückrichtung.}  Wir beweisen diese Aussage
mittels Kontraposition.  Angenommen, \(U \ne W\).  Zu
zeigen: \(\dim(U \cap W) \ne 2\).  Wegen $\dim U = \dim W
= 2$ und \(\dim V = 3\) und \(U\), \(W\) Untervektorräume von
\(V\), gilt \(U + W = V\).  Daraus folgt,
\begin{align*}
  \dim V
  &= \dim (U + W) = 3\\
  &= \dim U + \dim W - \dim (U \cap W) \\
  &= 2 + 2 - \dim (U \cap W)
\end{align*}
Daraus folgt \(\dim (U \cap W) = 1 \ne 2\).
\end{proof}

\newpage
\aufgn{10.2.i}

\beh \(V^V = U \oplus W\).

\begin{proof}
Sei \(f \in V^V\) beliebig gewählt.

Es gilt \(U \cap W = \left\{ 0 \right\}\).  Denn, sei $f
\in U \cap W$ beliebig gewählt.  Die Funktionen
erfüllen die folgenden Eigenschaften.  Für alle $v \in
V$ gilt
\begin{align*}
  f(v) &= - f(-v) \\
  f(v) &= f(-v).
\end{align*}
Addieren wir die beide Gleichungen erhalten wir $f(v) =
0$.  Also \(U \cap W = \left\{ 0 \right\}\).
\end{proof}

\newpage
\aufgn{10.4}

\beh Die Aussagen (i) und (ii) sind äquivalent.

\begin{proof}
\textit{Hinrichtung.}  Für alle \(v \in V\) mit
\(\phi(\phi(v)) = 0\) folgt \(\phi(v) \in \Bild(\phi)\) und
\(\phi(v) \in \ker(\phi)\).  Wegen Voraussetzung dass
\(\ker(\phi) \cap \Bild(\phi) = \left\{ 0 \right\}\) folgt
dann \(\phi(v) = 0\).

\textit{Rückrichtung.}  Sei $u \in \ker(\phi) \cap
\Bild(\phi)\( beliebig gewählt.  Dann existiert ein \)w
\in V$ mit \(\phi(w) = u\) und es gilt \(\phi(u) = 0\).  Es
gilt dann \(\phi(\phi(w)) = 0\).  Wegen Voraussetzung gilt
dann \(\phi(w) = 0 = u\).  Also, es gilt \(u = 0\).
\end{proof}

\end{document}