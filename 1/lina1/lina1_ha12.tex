%% page style
\documentclass[12pt]{extarticle}
\usepackage[margin=2cm]{geometry}
\usepackage{fancyhdr,parskip}
\pagestyle{fancy}
\usepackage[onehalfspacing]{setspace}
\setlength{\parindent}{0pt}
\lhead{\myAuthor}
\rhead{\mySubject \ \myHausaufgaben. Übungsblatt \\ \myTutor}
\renewcommand*\familydefault{\sfdefault} %% Only if the base font of the document is to be sans serif

%% language
\usepackage[utf8]{inputenc}
\usepackage{xcharter-otf}
\usepackage[ngerman]{babel}

%% default packages
\usepackage{amsmath,mathtools,fontspec,amsthm,amssymb,amsfonts,
  stmaryrd, % for the lightning symbol used in proof by contraction
  tikz,     % used to draw diagrams
}

%% metadata
\newcommand{\myAuthor}{Yuchen Guo 480788 | Meng Zhang 484981}
\newcommand{\myHausaufgaben}{12}
\newcommand{\mySubject}{LinA}
\newcommand{\myTutor}{Saskia}


%% custom commands
\newcommand{\mg}[1]{\mathbb{#1}}
\newcommand{\mc}[1]{\mathcal{#1}}
\newcommand{\lin}{\operatorname{lin}}
\newcommand{\aufgn}[1]{\textbf{Aufgabe #1.}}
\newcommand{\beh}{\textit{Behauptung.}\ }
\newcommand{\Bild}{\operatorname{Bild}}
\newcommand{\rg}{\operatorname{rg}}

\begin{document}
\aufgn{12.1}

Zur Abkürzung setzen wir
$A =
\begin{pmatrix}
  2 + i & 1 + i \\
  1 - i & 2
\end{pmatrix}$. Bestimmen \(f(A)\), \(\ker(f)\) und \(\rg(f)\)
sowie \(M_{\mc{B}_2,\mc{C}_2}(f)\).

\begin{proof}
  Wir nehmen drei Schritte.
\begin{itemize}
\item Bestimmen \(f(A)\).

Wegen (Übung 12) gilt
\begin{align*}
f(A) = \Phi_{\mc{C}_1}^{-1}(M_{\mc{B}_1,\mc{C}_1}(f) \cdot \Phi_{\mc{B}_1}(A)).
\end{align*}
Zunächst bestimmen wir \(\Phi_{\mc{B}_1}(A)\).
Angenommen, $\Phi_{\mc{B}_1}(A) =
\begin{pmatrix}
  a \\
  b \\
  c \\
  d
\end{pmatrix}
$.  Dann gilt
\begin{align*}
a
\begin{pmatrix}
  1 & 0 \\
  0 & 0
\end{pmatrix}
  + b
\begin{pmatrix}
  0 & 1 \\
  1 & 0
\end{pmatrix}
  + c
\begin{pmatrix}
  0 & 1 \\
  -1 & 0
\end{pmatrix}
  + d
\begin{pmatrix}
  1 & 1 \\
  1 & 1
\end{pmatrix}
  =
\begin{pmatrix}
  2 + i & 1 + i \\
  1 - i & 2
\end{pmatrix}.
\end{align*}
Daraus folgt,
$
  \Phi_{\mc{B}_1}(A) =
\begin{pmatrix}
  i \\
  -1 \\
  i \\
  2
\end{pmatrix}.
\( und \)M_{\mc{B}_1,\mc{C}_1}(f) \cdot \Phi_{\mc{B}_1}(A)
=
\begin{pmatrix}
  2 \\
  9 \\
  3i + 2 \alpha
\end{pmatrix}
$.

Dann bestimmen wir \(\Phi_{\mc{C}_1}\) und
\(\Phi_{\mc{C}_1}^{-1}\).  Angenommen, es gilt
$\Phi_{\mc{C}_1} \colon \mg{C}_{\le 2}[t] \to \mg{C}^3,
(at^2 + bt + c) \mapsto
\begin{pmatrix}
  m \\
  n \\
  p
\end{pmatrix}$.  Dann gilt
\begin{align*}
at^2 + bt + c = m(t^2 - t) + n(t - 1) + p = mt^2 +
  (n-m)t + p - n.
\end{align*}
Daraus folgt
\begin{align*}
  \Phi_{\mc{C}_1} \colon
   (at^2 + bt + c) \mapsto
\begin{pmatrix}
  a \\
  a + b \\
  a + b + c
\end{pmatrix}\\
\Phi_{\mc{C}_1}^{-1} \colon
\begin{pmatrix}
  a \\
  a + b \\
  a + b + c
\end{pmatrix} \mapsto (at^2 + bt + c).
\end{align*}  Daraus folgt,
\begin{align*}
f(A) = 2t^2 + 7t + 3i + 2\alpha - 9.
\end{align*}
\item Bestimmen \(\ker(f)\).

Wegen (Übung 12) gilt
\begin{align*}
\ker(f) = \Phi_{\mc{B}_1}^{-1}(\ker(M_{\mc{B}_1,\mc{C}_1}(f))).
\end{align*}

Zuerst bestimmen wir \(\ker(M_{\mc{B}_1,\mc{C}_1}(f))\).
\begin{align*}
\ker(M_{\mc{B}_1,\mc{C}_1}(f)) &= \ker
\begin{pmatrix}
  1 & -2 & -1 & 0\\
  0 & 1 & 0 & 5 \\
  0 & 0 & 3 & \alpha
\end{pmatrix}\\
  &=
\begin{pmatrix}
  -10 - \frac{\alpha}{3} \\
  -5 \\
  - \frac{\alpha}{3} \\
  1
\end{pmatrix} \cdot \mg{C}.
\end{align*}
Daraus folgt, dass
\begin{align*}
  \ker(f) =   \Phi_{\mc{B}_1}^{-1}(\ker(M_{\mc{B}_1,\mc{C}_1}(f)))
  =
\begin{pmatrix}
  -9 - \frac{\alpha}{3} & -4 -\frac{\alpha}{3} \\
  \frac{\alpha}{3} - 4 & 1
\end{pmatrix}.
\end{align*}
\item Bestimmen \(\rg(f)\).

  Wegen (Übung 12) gilt
\begin{align*}
  \rg(f) = \dim(\Bild(f)) = \dim V - \dim \ker f = 4 - 4
  = 0.
\end{align*}
\item Bestimmen \(M_{\mc{B}_2,\mc{C}_2}(f)\).

  Es gilt
\begin{align*}
  M_{\mc{B}_2,\mc{C}_2}(f)
  &= M(T_{C_1,C_2}) \cdot  M_{B_1,C_1}(f) \cdot
    M(T_{B_2,B_1}).
\end{align*}
Weiter gilt
\begin{align*}
  M(T_{C_1,C_2})_{\cdot,j} = \Phi_{C_2}(p_j) \text{ mit }
  p_j \in \mc{C}_1, \\
  M(T_{B_2,B_1})_{\cdot,j} = \Phi_{B_1}(p_j) \text{ mit }  p_j \in \mc{B}_2.
\end{align*}
Wir bestimmen jetzt \(\Phi_{C_2}\) und \(\Phi_{B_1}\).

Es gilt
\begin{align*}
\Phi_{C_2} \colon \mg{C}_{\le 2}[t] \to \mg{C}^3, at^2 +
  bt + c \mapsto
\begin{pmatrix}
  a \\
  b - a - c \\
  a + c
\end{pmatrix}.
\end{align*}

Es gilt
\begin{align*}
  \Phi_{B_1} \colon \mg{C}^{2 \times 2} \to \mg{C}^4,
\begin{pmatrix}
  a & b \\
  c & d
\end{pmatrix}
  \mapsto
\begin{pmatrix}
  a - d \\
  b + c - d \\
  b \\
  d
\end{pmatrix}.
\end{align*}

Dann gilt
\begin{align*}
  M(T_{C_1,C_2}) =
\begin{pmatrix}
  1 & 0 & 0 \\
  -2 & 2 & -1 \\
  1 & -1 & 1
\end{pmatrix}
 \\
  M(T_{B_2,B_1}) =
\begin{pmatrix}
  1 & 0 & 0 & -1 \\
  0 & 1 & 1 & -1 \\
  0 & 1 & 0 & 0 \\
  0 & 0 & 0 & 1
\end{pmatrix}
\end{align*}
Daraus folgt
\begin{align*}
  M_{\mc{B}_2,\mc{C}_2}(f) =
  \begin{pmatrix}
  1 & 0 & 0 \\
  -2 & 2 & -1 \\
  1 & -1 & 1
  \end{pmatrix}
  \cdot
\begin{pmatrix}
  1 & -2 & -1 & 0\\
  0 & 1 & 0 & 5 \\
  0 & 0 & 3 & \alpha
\end{pmatrix}
  \cdot
\begin{pmatrix}
  1 & 0 & 0 & -1 \\
  0 & 1 & 1 & -1 \\
  0 & 1 & 0 & 0 \\
  0 & 0 & 0 & 1
\end{pmatrix}
\end{align*}
\end{itemize}
\end{proof}

\newpage

\aufgn{12.2.i}

Es gilt zu jedem \(v \in V\) dass $v = \sum_{i =
  1}^n{\lambda_i v_i}$ für \(\lambda_i \in \mg{R}\).
Weil die Abbildung \(f\) eine lineare Abbildung ist, gilt
dann
\begin{align*}
  f(v)
  &= \sum_{i=1}^{n-1}{\lambda_i
    (v_i + v_{i+1}) + \lambda_n(v_n+v_1)} \\
  &= (\lambda_1 + \lambda_n) v_1 + (\lambda_2 +
    \lambda_1) v_2 + \cdots + (\lambda_{n-1} +
    \lambda_{n-2}) v_{n-1} +
    (\lambda_n + \lambda_{n-1})v_n.
\end{align*}

Bestimmen \(M_{\mc{B},\mc{B}}(f)\).

Es gilt
$M_{\mc{B},\mc{B}}(f)=M(\Phi_B \circ f \circ
\Phi_B^{-1})$.  Weil \(\Phi_B\) bijektiv ist, gilt für
alle \(e_i\), \(1 \le i \le n\) dass $\Phi_B^{-1}(e_i) =
v_i$.  Es gilt außerdem
\begin{align*}
M_{\mc{B},\mc{B}}(f) = \Phi_B \circ f(v_i) =
\begin{cases}
  e_i + e_{i+1}, & 1 \le i \le n - 1, \\
  e_1 + e_n, & i = n.
\end{cases}
\end{align*}
Daraus folgt,
\begin{align*}
M_{\mc{B},\mc{B}}(f) =
\begin{pmatrix}
  1 & 0 & 0 & \cdots & 0 & 1 \\
  1 & 1 & 0 & \cdots & 0 & 0 \\
  0 & 1 & 1 & \cdots & 0 & 0 \\
    &   &   & \vdots &   &   \\
  0 & 0 & 0 & 0      & 1 & 0 \\
  0 & 0 & 0 & 0      & 1 & 1
\end{pmatrix} \in \left\{ 0, 1 \right\}^{n \times n}.
\end{align*}
\aufgn{12.2.ii}

Zu zeigen: \(\lin(\mc{C}) = V\) und
\(\mc{C}= \left\{ w_1, \ldots, w_n \right\}\) mit
\(w_j = j v_{n+1-j}\) und \(1 \le j \le n\) linear
unabhängig.

Linear Unabhängigkeit:  sei \(\lambda_i \in \mg{R}\), $1
\le i \le n$ mit
\begin{align*}
\sum_{k = 1}^n{\lambda_k w_k} = \sum_{k=1}^n{\lambda_k
  \cdot k \cdot v_{n+1-k}} = 0.
\end{align*}
Weil \(\mc{B} = \left( v_1, \ldots, v_n \right)\) eine
Basis von \(V\) ist, sind die Vektoren \(v_1, \ldots, v_n\)
linear unabhängig und \(\lambda_k \cdot k = 0\) für alle
\(1 \le k \le n\).  Wegen \(k \ne 0\) gilt dann
\(\lambda_k = 0\) für alle \(1 \le k \le n\) und \(\mc{C}\)
ist daher linear unabhängig.

Zu zeigen \(\lin(\mc{C}) \subseteq V\).  Sei $u \in
\lin(\mc{C})$ beliebig gewählt.  Dann existiert
\(\lambda_i \in \mg{R}\) mit \(1 \le i \le n\) sodass
\begin{align*}
u = \sum_{i = 1}^n{\lambda_i w_i} = \sum_{i =
  1}^n{\lambda_i \cdot i \cdot v_{n+1-i}}.
\end{align*}
Es gilt insbesondere dass \(v_{n+1-i} \in \mc{B}\) und
\(\lambda_i \cdot i \in \mg{R}\) für alle $1 \le i \le
n$.  Daraus folgt, dass \(u \in \lin \mc{B}\).  Daraus
folgt, dass \(\lin(\mc{C}) \subseteq V\).

Zu zeigen \(V \subseteq \lin(\mc{C})\).  Sei \(u \in V\)
beliebig gewählt.  Weil \(\mc{C}\) eine Basis von \(V\) ist, existiert
\(\lambda_i \in \mg{R}\) mit \(1 \le i \le n\) sodass
\begin{align*}
u = \sum_{i=1}^n{\lambda_i v_i} = \sum_{i =
  1}^n{\left( \frac{\lambda_i}{n+1-i} \cdot w_{n+1-i}  \right)}.
\end{align*}
Es gilt wegen \(1 \le i \le n\) dass \(n+1-i \ne 0\) und
\(\frac{\lambda_i}{n+1-i} \in \mg{R}\).  Daraus folgt,
dass \(u \in \lin \mc{C}\) und \(V \subseteq \lin(\mc{C})\).

Bestimmen \(M_{\mc{B}, \mc{C}}(f)\),
\(M_{\mc{C}, \mc{B}}(f)\) und \(M(T_{\mc{B}, \mc{C}})\).
Zuerst bestimmen wir \(\Phi_{\mc{B}}\) und \(\Phi_{\mc{C}}\).
Sei \(u \in \lin \mc{B}\) und
\(w \in \lin \mc{C}\)
beliebig gewählt.  Dann ist
\(u = \sum_{i=1}^n{\lambda_i v_i}\) und $t = \sum_{i =
  1}^n{\mu_i w_i}$ und
\begin{align*}
\Phi_\mc{B}(u) =
\begin{pmatrix}
  \lambda_1 \\
  \vdots \\
  \lambda_n
\end{pmatrix},
\Phi_\mc{C}(t) =
\begin{pmatrix}
  \mu_1 \\
  \vdots \\
  \mu_n
\end{pmatrix}.
\end{align*}
Dann ist
\begin{align*}
  M_{\mc{B}, \mc{C}}(f)_{\cdot , j}
  &= M(\Phi_{\mc{C}} \circ f \circ
    \Phi_{\mc{B}}^{-1})_{\cdot, j} \\
  &= (\Phi_{\mc{C}} \circ f \circ
    \Phi_{\mc{B}}^{-1})(e_i) \\
  &= (\Phi_{\mc{C}} \circ f)(v_i) \\
  &= \Phi_{\mc{C}}(v_i + v_{i+1}) \\
  &= \Phi_{\mc{C}} \left( \frac{1}{n+1-i}w_{n+1-i} + \frac{1}{n-i}w_{n-i} \right) \\
  &=
\begin{pmatrix}
  0 \\
  0 \\
  \vdots \\
  \frac{1}{n-i} \\
  \frac{1}{n+1-i} \\
  \vdots \\
  0
\end{pmatrix} \in \mg{R}^n.
\end{align*}
Analog gilt
\begin{align*}
  M_{\mc{C}, \mc{B}}(f)_{\cdot , j}
  &= M(\Phi_{\mc{B}} \circ f \circ
    \Phi_{\mc{C}}^{-1})_{\cdot, j} \\
  &= (\Phi_{\mc{B}} \circ f \circ
    \Phi_{\mc{C}}^{-1})(e_i) \\
  &= (\Phi_{\mc{B}} \circ f)(w_i) \\
  &= (\Phi_{\mc{B}} \circ f)(i \cdot v_{n+1 - i}) \\
  &= \Phi_{\mc{B}}(i \cdot (v_{n+1-i} + v_{n+2-i})) \\
  &=
\begin{pmatrix}
  0 \\
  0 \\
  \vdots \\
  i \\
  i \\
  \vdots \\
  0
\end{pmatrix} \in \mg{R}^n.
\end{align*}
Analog gilt
\begin{align*}
  M(T_{\mc{B}, \mc{C}})_{\cdot, j}
  &= M(\Phi_{\mc{C}} \circ
    \Phi_{\mc{B}}^{-1})_{\cdot, j} \\
  &= (\Phi_{\mc{C}} \circ
    \Phi_{\mc{B}}^{-1})(e_i) \\
  &= \Phi_{\mc{C}}(v_i) \\
  &= \frac{1}{n+1-i}w_{n+1-i} \\
  &=
\begin{pmatrix}
  0 \\
  \vdots \\
  \frac{1}{n+1-i} \\
  \vdots \\
  0
\end{pmatrix}.
\end{align*}

\newpage

\aufgn{12.3}

Es sei \(A \in \mg{K}^{n \times n}\).  Zur Abkürzung
setzen wir die Menge aller Spalten der Matrix \(A\) als
$S := \left\{ A_{\cdot, 1}, \ldots, A_{\cdot, n}
\right\}$. Es gilt wegen (Übung 12) dass
\begin{align*}
  \rg(A) &= \dim (\Bild (A)) \\
         &= \dim (\lin S).
\end{align*}
\beh Es existiert ein \(1 \le p \le n\) sodass
\(A_{\cdot, p} \in \lin (S \setminus \left\{ A_{\cdot, p} \right\})\)
\(\iff\) Es existiert eine Matrix
\(B \in \mg{K}^{n \times n} \setminus \left\{ 0 \right\}\)
mit \(AB = 0\).

\begin{proof}
Hinrichtung.  Angenommen, es existiert ein $1 \le p \le
n\( sodass \)A_{\cdot, p} \in \lin (S \setminus \left\{
  A_{\cdot, p} \right\})\(.  Sei außerdem die Matrix \)B
\in \mg{R}^{n \times n}$ so definiert:
\begin{align*}
B_{\cdot, j} =
\begin{cases}
0, & \text{ für alle } 1 \le j \le n \text{ und } j \ne p;
  \\
  D \in \mg{R}^n \text{ mit } \sum_{i=1}^n{A_{i, \cdot}
  \cdot D} = 0, & \text{ falls } j = p.
\end{cases}
\end{align*}
Es gilt
\begin{align*}
A_{\cdot, p} = \sum_{i=1}^{p-1}{\lambda_i A_{\cdot, i}}
  + \sum_{i=p-1}^{n}{\lambda_i A_{\cdot, i}}
\end{align*}
\end{proof}
\end{document}