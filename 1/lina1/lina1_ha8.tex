%% page style
\documentclass[12pt]{extarticle}
\usepackage[margin=2cm]{geometry}
\usepackage{fancyhdr,parskip}
\pagestyle{fancy}
\usepackage[onehalfspacing]{setspace}
\setlength{\parindent}{0pt}
\lhead{\myAuthor}
\rhead{\mySubject \ \myHausaufgaben. Übungsblatt \\ \myTutor}
\renewcommand*\familydefault{\sfdefault} %% Only if the base font of the document is to be sans serif

%% language
\usepackage[utf8]{inputenc}
\usepackage{xcharter-otf}
\usepackage[ngerman]{babel}

%% default packages
\usepackage{amsmath,mathtools,fontspec,amsthm,amssymb,amsfonts,
  stmaryrd, % for the lightning symbol used in proof by contraction
  tikz,     % used to draw diagrams
}


%% metadata
\newcommand{\myAuthor}{Yuchen Guo 480788 | Meng Zhang 484981}
\newcommand{\myHausaufgaben}{8}
\newcommand{\mySubject}{LinA}
\newcommand{\myTutor}{Saskia}


%% custom commands
\newcommand{\mg}[1]{\mathbb{#1}}
\newcommand{\lin}{\operatorname{lin}}

\begin{document}
\textbf{Aufgabe 8.1}

\textbf{Aufgabe 8.2}

\textit{Behauptung.}  In dem \(\mg{Q}\)-Vektorraum
\(\mg{R}\) sind die drei Vektoren \(1, \sqrt{2}, \sqrt{3}\)
linear unabhängig.

\begin{proof}
  Wir bemühen uns um einen Widerspruchsbeweis.
  Angenommen, die drei Vektoren sind linear abhängig.
  Dann existiert $\alpha, \beta, \gamma \in \mg{Q}
  \setminus \left\{ 0 \right\}$ dass
\begin{align*}
\alpha \cdot 1 + \beta \cdot \sqrt{2} + \gamma \cdot
  \sqrt{3} = 0.
\end{align*}
Durch Umformungen erhalten wir dann
\begin{align*}
  \sqrt{6} =
  -2 \cdot \frac{\beta}{\alpha} - 3 \cdot \frac{\gamma}{\alpha}
\end{align*} mit \(\sqrt{6} \notin \mg{Q}\) und $-2 \cdot
\frac{\beta}{\alpha} - 3 \cdot \frac{\gamma}{\alpha}
\in \mg{Q}$.  Dies ist ein Widerspruch. \(\lightning\)

Daraus folgt, dass solche  $\alpha, \beta, \gamma \in \mg{Q}
\setminus \left\{ 0 \right\}$  existiert nicht und es
gilt \(\alpha = \beta = \gamma = 0\).
\end{proof}

\textit{Behauptung.}  \(\sqrt{6}\) ist keine
Linearkombination dieser drei Vektoren.

\begin{proof}
  Wir bemühen uns um einen Widerspruchsbeweis.
  Angenommen, \(\sqrt{6}\) ist eine Linearkombination der
  drei Vektoren \(1, \sqrt{2}, \sqrt{3}\). Dann existiert
  \(\alpha, \beta, \gamma \in \mg{Q}\) sodass
\begin{align*}
\alpha \cdot 1 + \beta \cdot \sqrt{2} +
  \gamma \cdot \sqrt{3} = \sqrt{6}.
\end{align*}
Falls \(\alpha=0\), dann gilt
\(\beta=\frac{\sqrt{3}}{2}\) und
\(\gamma=\frac{\sqrt{2}}{2}\) und damit $\beta,
\gamma \notin \mg{Q}$.

Falls \(\alpha \neq 0\), dann gilt
\begin{align*}
2 \sqrt{3} \frac{\beta}{\alpha} + 3 \sqrt{2}
  \frac{\gamma}{\alpha} = \frac{6}{\alpha}
\end{align*}
mit \(\frac{6}{\alpha} \in \mg{Q}\) und
$2 \sqrt{3} \frac{\beta}{\alpha} + 3 \sqrt{2}
\frac{\gamma}{\alpha} \notin \mg{Q}$.
Dies ist ein Widerspruch. \(\lightning\)

Daraus folgt, dass solche  \(\alpha, \beta, \gamma \in \mg{Q}\)  existiert nicht.
\end{proof}

\textbf{Aufgabe 8.3.i}

\textit{Behauptung.}  Wenn $U_i \neq \left\{ 0
\right\}$ für \(i = 1,2,3\) paarweise verschieden sind,
dann existieren drei linear unabhängige Vektoren $u_i
\in U_i$ für \(i = 1,2,3\).

\begin{proof}
  Wir bemühen uns um einen Widerspruchsbeweis.
  Angenommen, für alle \(u_1 \in U_1\), \(u_2 \in U_2\),
  \(u_3 \in U_3\) gilt, dass \(u_1, u_2, u_3\) linear
  unabhängig sind.  Dann existiert
  $\alpha, \beta, \gamma \in \mg{K} \setminus \left\{ 0
  \right\}$ sodass
\begin{align*}
\alpha u_1 + \beta u_2 + \gamma u_3 = 0
\end{align*}
für alle \(u_1 \in U_1\), \(u_2 \in U_2\), \(u_3 \in U_3\).
Insbesondere, es existiert
\(\alpha, \beta, \gamma \in \mg{K}\) für alle $u_1 \in
U_1$, \(u_2 \in U_2\) sodass
\begin{align*}
\alpha u_1 + \beta u_2 + \gamma 0 = 0.
\end{align*}

Damit gilt \(u_1 = -\frac{\beta}{\alpha}u_2\) für alle
\(u_1 \in U_1\), \(u_2 \in U_2\).  Das heißt, \(U_1 = U_2\)
im Widerspruch zur Voraussetzung \(U_1 \neq U_2\). \(\lightning\)
\end{proof}

\textbf{Aufgabe 8.3.ii}

\textit{Behauptung.} Wenn für alle
\(u_1 \in U_1 \setminus \left\{ 0 \right\}\) und
\(u_2 \in U_2 \setminus \left\{ 0 \right\}\) die Vektoren
\(u_1, u_2\) linear unabhängig sind, dann gilt
\(U_1 \cap U_2 = \left\{ 0 \right\}\).

\begin{proof}
  Wir bemühen uns um einen Beweis mittels
  Kontraposition.

  Angenommen, es gilt \(v \in U_1 \cap U_2\) mit
  \(v \neq 0\).  Daraus folgt, \(v \in U_1\) und
  \(v \in U_2\).  Es gilt dann insbesondere,
\begin{align*}
v + (-1) v = 0.
\end{align*}
Daraus folgt, dass nicht alle \(u_1 \in U_1\) und $u_2
\in U_2$ linear unabhängig sind.
\end{proof}

\textbf{Aufgabe 8.3.iii}

\textit{Behauptung.}  Seien $v_1, \ldots, v_n \in V
\setminus \left\{ 0 \right\}$ linear abhängig, dann
existieren zwei Vektornen \(v_i, v_j\) mit $i, j \in
\left\{ 1, \ldots, n \right\}, i \neq j$, die sich
jeweils als Linearkombination der restlichen \(n-1\)
Vektoren darstellen lassen.

\begin{proof}
  Diese Aussage ist wahr.  Aus der lineare Abhängigkeit
  folgt, dass es existiert
  $\lambda_1, \ldots, \lambda_n \in \mg{K} \setminus
  \left\{ 0 \right\}$ sodass
\begin{align*}
\sum_{i=1}^n{\lambda_iv_i}=0.
\end{align*}

Daraus folgt, dass zu jedem Vektor \(v_i\) mit $i \in
\left\{ 1, \ldots, n \right\}$ gibt es eine
Linearkombination der restlichen \(n-1\) Vektoren mit
\begin{align*}
v_i = \frac{1}{\lambda_i} \left(
  \sum_{j=1}^{i-1}{\lambda_jv_j} + \sum_{k=i+1}^n{\lambda_kv_k} \right).
\end{align*}
\end{proof}

\textbf{Aufgabe 8.4.i}

\textit{Behauptung.}  Eine Basis von \(\mg{R}^4/U_0\) mit
$U_0= \left\{ x \in \mg{R}^4 \colon x_1+2x_2 = 0 \wedge
  x_3 + x_4 = 0\right\}$ ist
\begin{align*}
B := \left\{ \left( 1, 0, 0, 0
  \right) + U_0, \left( 0, 1, 0, 0 \right) + U_0,
  \left( 0, 0, 1, 0 \right) + U_0, \left( 0, 0, 0, 1
  \right) + U_0 \right\}.
\end{align*}

\begin{proof}
Wir beweisen, dass \(B\) eine Basis von \(\mg{R}^4/U_0\)
ist, indem wir zeigen, dass \(B\) linear unabhängig ist,
und \(\lin(B) = \mg{R}^4/U_0\) gilt.

???
\end{proof}

\end{document}