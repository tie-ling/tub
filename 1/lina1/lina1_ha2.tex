\documentclass[12pt]{extarticle}
\usepackage{amsmath,mathtools,fontspec,amsthm,amssymb,amsfonts,fancyhdr,color,graphicx,}
\usepackage[margin=2.5cm]{geometry}
\usepackage[utf8]{inputenc}
\usepackage{xcharter-otf}
\usepackage[ngerman]{babel}
\usepackage[onehalfspacing]{setspace}
\usepackage{tikz}
\renewcommand{\familydefault}{\sfdefault}
\pagestyle{fancy}
\setlength{\parindent}{0pt}
\lhead{Yuchen Guo 480788, Meng Zhang 484981}
\rhead{2. Hausaufgabenblatt, LinA I\\Tutorin: Saskia, Mi 2-4 p.m.}
\begin{document}
\textbf{Lemma Aufgabe 2.1.1.1} Seien \(A\) und \(B\) Mengen.  Es gilt \(A \setminus B = A \cap \overline{B}\).
\begin{proof}
  \begin{align*}
    A \setminus B
    &\iff \left\{ x|x \in A \wedge x \notin B \right\}
      \tag*{Definition von Differenz}\\
    &\iff \left\{ x|x \in A \wedge x \in \overline{B} \right\}
      \tag*{Definition von \(\overline{B}\), 1.19.iv}\\
    &\iff A \cap \overline{B}
  \end{align*}
\end{proof}

\textbf{Aufgabe 2.1.1}
\begin{proof}
  \begin{align*}
    M \setminus (N \cap L)
    &\iff M \cap \overline{\left( N \cap L \right)}
      \tag*{Lemma} \\
    &\iff M \cap \left( \overline{N} \cup \overline{L} \right)
      \tag*{De Morgan'sche Regeln}\\
    &\iff (M \cap \overline{N}) \cup (M \cap \overline{L}) \tag*{Distributivität}\\
    &\iff (M \setminus N) \cup (M \setminus L) \tag*{Lemma}
  \end{align*}
\end{proof}

\textbf{Aufgabe 2.1.2}
\begin{proof}
  \begin{align*}
    M \setminus (N \cup L)
    &\iff M \cap \overline{\left( N \cup L \right)}
      \tag*{Lemma} \\
    &\iff M \cap \left( \overline{N} \cap \overline{L} \right)
      \tag*{De Morgan'sche Regeln}\\
    &\iff (M \cap \overline{N}) \cap (M \cap \overline{L}) \tag*{Distributivität}\\
    &\iff (M \setminus N) \cap (M \setminus L) \tag*{Lemma}
  \end{align*}
\end{proof}


\textbf{Aufgabe 2.2}

Um zu zeigen, dass drei Aussagen äquivalent sind, ist es ausreichend,
die Aussage \(A \implies B \implies C \implies A\) als wahr zu beweisen.

\begin{itemize}
\item \(N \subseteq M \implies M \cap N = N\)

  \begin{proof}

    Wir zeigen, aus \(N \subseteq M \) folgt \(M \cap N = N\), indem wir
    zeigen, dass die Aussagen  \(M \cap N \subseteq N\) und $N \subseteq M
    \cap N$ wahr sind.
    \begin{itemize}
    \item \(N \subseteq M \cap N\)

      Aus \(N \subseteq M\) erhalten wir, dass für alle \(x \in N\) gilt $x
      \in M$.  Es gilt aber auch \(N \subseteq N\), denn für alle \(x \in N\) gilt
      \(x \in N\).

      Es gilt für alle \(x \in N\) die Aussage $(x \in
      M) \wedge (x \in N)$, d.h., \(N \subseteq M \cap N\).
    \item \(M \cap N \subseteq N\)

      Aus der Definition von \(M \cap N\) erhalten wir, dass es für alle $x \in
      M \cap N$, die Aussage \(x \in N\) gilt.  D.h., \(M \cap N \subseteq N\).
    \end{itemize}


    Weil die beide Aussagen  \(M \cap N \subseteq N\) und $N \subseteq M
    \cap N$ wahr sind, ist auch die Aussage \(M \cap N = N\) wahr.
  \end{proof}
\item \(M \cap N = N \implies M \cup N = M\)

  \begin{proof}

    Wir zeigen, aus \(M \cap N = N \) folgt \(M \cup N = M\), indem wir
    zeigen, dass die Aussagen  \(M \cup N \subseteq M\) und $M \subseteq M
    \cup N$ wahr sind.
    \begin{itemize}
    \item \(M \subseteq M \cup N\)


      Aus der Definition von \(M \cup N\) erhalten wir, dass es für alle $x \in
      M$, die Aussage \(x \in M \cup N\) gilt.  D.h., \(M \subseteq M \cup N\).

    \item \(M \cup N \subseteq M\)

      Aus \(M \cap N = N\) erhalten wir, \(N \subseteq M \cap N\).  D.h., dass
      für alle \(x \in N\) gilt \(x \in N \wedge x \in M\).  Insbosondere, es
      gilt für alle \(x \in N\) die Aussage \(x \in M\), also
      \(N \subseteq M\).

      Wegen \(N \subseteq M\), gilt \((M \cup N) \subseteq (M \cup M) = M\),
      also, \(M \cup N \subseteq M\).
    \end{itemize}


    Weil die beide Aussagen \(M \subseteq M \cup N\) und
    \(M \cup N \subseteq M\) wahr sind, ist auch die Aussage
    \(M \cup N = M\) wahr.
  \end{proof}

\item \(M \cup N = M \implies N \subseteq M\)

  \begin{proof}
    Aus \(M \cup N = M\) folgt, dass \(M \cup N \subseteq M\) gilt.
    Es gilt für alle \(x \in M \cup N\) die Aussage \(x \in M\).
    Insbesondere, für alle \(x \in N\) gilt \(x \in M\).  D.h., $N \subseteq
    M$.
  \end{proof}
\end{itemize}

\textbf{Aufgabe 2.3.1}

Wir beweisen diese Aufgabe, indem wir zeigen, dass
$(8+24\mathbb{Z}) \cup (16+24\mathbb{Z}) \subseteq 8\mathbb{Z}
\setminus 3\mathbb{Z}$ und
$ 8\mathbb{Z} \setminus 3\mathbb{Z} \subseteq (8+24\mathbb{Z}) \cup
(16+24\mathbb{Z})$.
\begin{itemize}
\item $(8+24\mathbb{Z}) \cup (16+24\mathbb{Z}) \subseteq 8\mathbb{Z}
  \setminus 3\mathbb{Z}$

  \begin{proof}

    \begin{align*}
      (8+24\mathbb{Z}) \cup (16+24\mathbb{Z})
      &\iff (3(8\mathbb{Z}+2)+2) \cup (3(8\mathbb{Z}+5)+1)
        \tag{1}\\
      &\iff 8(3\mathbb{Z}+1) \cup 8(3\mathbb{Z}+2)
        \tag{2}\\
    \end{align*}

    Der Formel (1) ist durch \(3\) nicht teilbar und der Formel (2) ist
    durch \(8\) teilbar.  Deshalb gilt, $(8+24\mathbb{Z}) \cup (16+24\mathbb{Z}) \subseteq 8\mathbb{Z}
    \setminus 3\mathbb{Z}$.
  \end{proof}
\item
  $ 8\mathbb{Z} \setminus 3\mathbb{Z} \subseteq (8+24\mathbb{Z})
  \cup (16+24\mathbb{Z})$
  \begin{proof}
    \begin{align*}
      8\mathbb{Z} \setminus 3\mathbb{Z}
      &\iff 8\mathbb{Z} \cap \overline{3\mathbb{Z}}
        \tag*{Lemma Aufgabe 2.1.1.1}\\
      &\iff 8\mathbb{Z} \cap \left( \left( 3\mathbb{Z} + 1 \right) \cup
        \left( 3\mathbb{Z} + 2 \right) \right) \\
      &\iff \left( 8\mathbb{Z} \cap \left( 3\mathbb{Z} + 1 \right) \right)
        \cup \left( 8\mathbb{Z} \cap \left( 3\mathbb{Z} + 2 \right)
        \right)
        \tag*{Distributivität}
    \end{align*}
    \begin{itemize}
    \item Falls $x \in 8\mathbb{Z} \cap \left( 3\mathbb{Z} + 1
      \right)$

      Seien \(x = 8p = 3q+1\) mit \(p, q \in \mathbb{Z}\).

      Die Gleichung \(q = \frac{8p-1}{3}\) ist in (1) lösbar, denn, aus
      \(\frac{8p-1}{3}=8\mathbb{Z}+5\) folgt \(p = 3\mathbb{Z}+2\).

      Deshalb gilt \(x \in (8+24\mathbb{Z}) \cup (16+24\mathbb{Z})\).
    \item Falls $x \in 8\mathbb{Z} \cap \left( 3\mathbb{Z} + 2
      \right)$


      Seien \(x = 8p = 3q+2\) mit \(p, q \in \mathbb{Z}\).

      Die Gleichung \(q = \frac{8p-2}{3}\) ist in (1) lösbar, denn, aus
      \(\frac{8p-2}{3}=8\mathbb{Z}+2\) folgt \(p = 3\mathbb{Z}+1\).

      Deshalb gilt \(x \in (8+24\mathbb{Z}) \cup (16+24\mathbb{Z})\).
    \end{itemize}

  \end{proof}
\end{itemize}

\textbf{Aufgabe 2.3.2}
\begin{itemize}
\item \((8+11\mathbb{Z}) \cap 4\mathbb{Z} \subseteq 8+44\mathbb{Z}\)

  \begin{proof}
    Sei \(x \in (8+11\mathbb{Z}) \cap 4\mathbb{Z}\).  Dann gilt
    \(x=8+11p=4q\) mit \(p, q \in \mathbb{Z}\).  Es gilt also
    \(q=2+\frac{11}{4}p\) mit \(p,q \in \mathbb{Z}\).

    Wir zeigen, dass diese Gleichung in \(8+44\mathbb{Z}\) eine Lösung
    hat.

    Wegen \(8+44\mathbb{Z} \iff 4(2+11\mathbb{Z})\), ist es ausreichend,
    zu zeigen, dass \(q=2+\frac{11}{4}p=2+11\mathbb{Z}\) lösbar ist.

    Diese ist tatsächlich lösbar, denn,

    \begin{align*}
      2+11\mathbb{Z} &= 2+\frac{11}{4}p\\
      11\mathbb{Z} &= \frac{11}{4}p\\
      p &= \frac{44\mathbb{Z}}{11} = 4\mathbb{Z}
    \end{align*}

    Deshalb gilt die Behauptung \((8+11\mathbb{Z}) \cap 4\mathbb{Z} \subseteq 8+44\mathbb{Z}\).
  \end{proof}
\item \(8+44\mathbb{Z} \subseteq (8+11\mathbb{Z}) \cap 4\mathbb{Z}\).

  \begin{proof}
    Wegen \(8+44\mathbb{Z} \iff 4(2+11\mathbb{Z})\), gilt für alle $x
    \in 8+44\mathbb{Z}\( dass \(x\) durch \(4\) teilbar ist, also \)x \in
    4\mathbb{Z}$.

    Wegen \(8+44\mathbb{Z} \iff 8+4\cdot 11 \mathbb{Z}\), gilt für alle
    \(x \in 8+44\mathbb{Z}\) dass \(x\) durch \(11\) geteilt einen Rest von
    \(8\) hat, also \(x \in (8+11\mathbb{Z})\).

    Deshalb gilt die Behauptung $8+44\mathbb{Z} \subseteq
    (8+11\mathbb{Z}) \cap 4\mathbb{Z}$.
  \end{proof}
\end{itemize}

\textbf{Aufgaben 2.3.3}

\textit{Behauptung}: Seien \(k, m, n \in \mathbb{Z}\).  Dann gilt
entweder \(m + k\mathbb{Z}=n+k\mathbb{Z}\) oder $(m+k\mathbb{Z}) \cap
(n+k\mathbb{Z})=\emptyset$.

\begin{proof}
  Wir zeigen, dass diese Aussage wahr ist, indem wir zeigen, dass die
  beide Behauptung nicht gleichzeitig wahr oder gleichzeitig falsch
  sein darf.

  \begin{itemize}
  \item Gleichzeitig wahr: $(m+k\mathbb{Z} = n+k\mathbb{Z}) \wedge
    ((m+k\mathbb{Z}) \cap (n+k\mathbb{Z})=\emptyset)$

    Seien \(A =m+k\mathbb{Z}\) und  \(B=n+k\mathbb{Z}\).  Wegen
    Voraussetzungen gilt \(A=B\) und \(A \cap B = \emptyset\).

    Wegen \(A=B\), gilt \(A \cap B = A = B\).  Es gilt aber auch
    \(A \cap B = \emptyset\).  Daraus folgt, dass \(A = B = \emptyset\).
    Dies ist aber im Widerspruch zur Voraussetzung \(A =m+k\mathbb{Z}\)
    und \(B=n+k\mathbb{Z}\), denn \(A\) und \(B\) sind eindeutig nicht
    leere Mengen.

    Deshalb können die beide Aussagen nicht gleichzeitig wahr sein.

  \item Gleichzeitig falsch: $(m+k\mathbb{Z} \neq n+k\mathbb{Z}) \wedge
    ((m+k\mathbb{Z}) \cap (n+k\mathbb{Z})\neq \emptyset)$

    Wegen \((m+k\mathbb{Z}) \cap (n+k\mathbb{Z})\neq \emptyset\),
    existiert ein Element \(x\) mit der Eigenschaften
    \(x \in m + k\mathbb{Z}\) und \(x \in n+k\mathbb{Z}\).

    Seien \(x = m+kp = n+k(p+q)\) mit \(p,q \in\mathbb{Z}\).  Daraus
    folgt, \(m=n+kq\).

    Weil \(m\) gleich \(n+kq\) ist, gilt \(m+k\mathbb{Z}=n+kq+k\mathbb{Z}\)
    mit \(q \in \mathbb{Z}\).  D.h., es gilt
    \(m + k\mathbb{Z} = n+k\mathbb{Z}\).  Dies ist aber im Widerspruch
    zur Voraussetzung \(m+k\mathbb{Z} \neq n+k\mathbb{Z}\).

    Deshalb können die beide Aussagen nicht gleichzeitig falsch sein.

  \end{itemize}
\end{proof}

\textbf{Aufgabe 2.4.1}

\textit{Behauptung.}  Seien \(M, N\) Mengen.  Dann gilt $\mathcal{P}(M)
\cap \mathcal{P}(N) = \mathcal{P}(M \cap N)$.

\begin{proof}
\begin{align*}
  X \in \mathcal{P}(M) \cap \mathcal{P}(N)
  &\iff (X \in \mathcal{P}(M)) \wedge (X \in \mathcal{P}(N))
  \tag*{Definition von Durchschnitt}\\
  &\iff (X \subseteq M) \wedge (X \subseteq N)
    \tag*{Definition von Potenzmenge}\\
  &\iff \forall x \in X: x \in M \wedge x \in N
    \tag*{Definition von Teilmenge}\\
  &\iff \forall x \in X: x \in M \cap N
  \tag*{Definition von Durchschnitt}\\
  &\iff X \subseteq M \cap N
    \tag*{Definition von Teilmenge}\\
  &\iff X \in \mathcal{P}(M \cap N)
    \tag*{Definition von Potenzmenge}
\end{align*}

D.h., für alle \(X \in \mathcal{P}(M) \cap \mathcal{P}(N)\) gilt auch
\(X \in \mathcal{P}(M \cap N)\) und umgekehrt.

Deshalb gilt
\(\mathcal{P}(M) \cap \mathcal{P}(N) = \mathcal{P}(M\cap N)\).
  \end{proof}

\textbf{Aufgabe 2.4.2}

\textit{Behauptung.}  Seien \(M, N\) Mengen.  Dann gilt $\mathcal{P}(M)
\cup \mathcal{P}(N) \subseteq \mathcal{P}(M \cup N)$.

\begin{proof}

\begin{align*}
  X \in \mathcal{P}(M) \cup \mathcal{P}(N)
  &\Leftrightarrow X \in \mathcal{P}(M) \lor X \in \mathcal{P}(N)
    \tag*{Definition von Vereinigung}\\
  &\Leftrightarrow X \subseteq M \lor X \subseteq N
    \tag*{Definition von Potenzmengen}\\
  &\Rightarrow \forall x \in X: x \in M \lor x \in N
    \tag*{Definition von Teilmengen}\\
  &\Leftrightarrow X \subseteq M \cup N
    \tag*{Definition von Vereinigung}\\
  &\Leftrightarrow X \in \mathcal{P}(M \cup N)
    \tag*{Definition von Potenzmengen}
\end{align*}
  \end{proof}

  Seien \(M= \left\{ 0 \right\}\) und \(N= \left\{ 1 \right\}\).  Dann
  gilt \(\left\{ 0,1 \right\} \subseteq \mathcal{P}(M \cup N)\).

  Aber es gilt auch \(\left\{ 0,1 \right\} \nsubseteq M\) und
  \(\left\{ 0,1 \right\} \nsubseteq N\).  Deshalb gilt die Gleichheit
  \(\mathcal{P}(M) \cup \mathcal{P}(N) = \mathcal{P}(M \cup N)\) nicht.
\\\\
\textbf{Lemma 2.4.3.1}  Seien \(A, B, C\) Mengen mit \(A \subseteq B\) und
\(B \subseteq C\).  Dann gilt \(A \subseteq C\).

\begin{proof}
  Aus der Definition von Teilmengen erhalten wir, dass für alle $a \in
  A$, es gilt \(a \in B\).  Für alle \(b \in B\) gilt auch \(b \in C\).

  Daraus folgt, dass für alle \(a \in A\), es gilt \(a \in B\) und
  \(a \in C\).  D.h., \(A \subseteq C\).
  \end{proof}

\textbf{Aufgabe 2.4.3}

\textit{Behauptung.}  Seien \(M, N\) Mengen.  Dann gilt
\(M \subseteq N \iff \mathcal{P}(M) \subseteq \mathcal{P}(N)\).
\begin{proof}
\begin{align*}
  M \subseteq N
  &\iff \forall X \in \mathcal{P}(M): X \subseteq M \subseteq N
    \tag*{Lemma 2.4.3.1}\\
  &\iff \forall X \in \mathcal{P}(M): X \in \mathcal{P}(N)
    \tag*{Definition von Potenzmengen}\\
  &\iff \mathcal{P}(M) \subseteq \mathcal{P}(N)
    \tag*{Definition von Teilmengen}
\end{align*}
\end{proof}

\end{document}