%% page style
\documentclass[12pt]{extarticle}
\usepackage[margin=2cm]{geometry}
\usepackage{fancyhdr,parskip}
\pagestyle{fancy}
\usepackage[onehalfspacing]{setspace}
\setlength{\parindent}{0pt}
\lhead{\myAuthor}
\rhead{\mySubject \ \myHausaufgaben. Übungsblatt \\ \myTutor}
\renewcommand*\familydefault{\sfdefault} %% Only if the base font of the document is to be sans serif

%% language
\usepackage[utf8]{inputenc}
\usepackage{xcharter-otf}
\usepackage[ngerman]{babel}

%% default packages
\usepackage{amsmath,mathtools,fontspec,amsthm,amssymb,amsfonts,
  stmaryrd, % for the lightning symbol used in proof by contraction
  tikz,     % used to draw diagrams
}

%% metadata
\newcommand{\myAuthor}{Yuchen Guo 480788 | Meng Zhang 484981}
\newcommand{\myHausaufgaben}{10}
\newcommand{\mySubject}{LinA}
\newcommand{\myTutor}{Saskia}


%% custom commands
\newcommand{\mg}[1]{\mathbb{#1}}
\newcommand{\lin}{\operatorname{lin}}
\newcommand{\aufgn}[1]{\textbf{Aufgabe #1.}}
\newcommand{\beh}{\textit{Behauptung.}\ }
\newcommand{\Bild}{\operatorname{Bild}}

\begin{document}

\aufgn{11.1.i}

Es sei
\begin{align*}
  A =
\begin{pmatrix}
  [0]_3 & [1]_3 & [1]_3 \\
  [0]_3 & [0]_3 & [1]_3 \\
  [0]_3 & [0]_3 & [0]_3
\end{pmatrix}.
\end{align*}
Daraus folgt,
\begin{align*}
  A^0 = I_3, \quad
A^2 =
\begin{pmatrix}
  [0]_3 & [0]_3 & [1]_3 \\
  [0]_3 & [0]_3 & [0]_3 \\
  [0]_3 & [0]_3 & [0]_3
\end{pmatrix}, \quad
A^m =
\begin{pmatrix}
  [0]_3 & [0]_3 & [0]_3 \\
  [0]_3 & [0]_3 & [0]_3 \\
  [0]_3 & [0]_3 & [0]_3
\end{pmatrix} \quad \text{für alle } m \ge 3.
\end{align*}

\aufgn{11.1.ii}

Es sei
\begin{align*}
  A =
\begin{pmatrix}
  [1]_5 & [1]_5 & [1]_5 \\
  [0]_5 & [1]_5 & [1]_5 \\
  [0]_5 & [0]_5 & [1]_5
\end{pmatrix}.
\end{align*}
Es gibt kein \(m \in \mg{N}\) mit \(A^m = 0\).
\begin{proof}
Es gilt insbesondere, dass \(a_{i,j}^m \ge 1\) für alle $m
\in \mg{N}_{\ge 1}$ und \(i = j\) wegen
\begin{align*}
a_{i,j}^m = \underbrace{a_{i,j}^{m-1}}_{=1} +
  \underbrace{a_p a_q + a_g a_h}_{\ge 0}.
\end{align*}
Daraus folgt, dass kein \(m \in \mg{N}\) mit \(A^m = 0\) existiert.
\end{proof}

\aufgn{11.2.i}

\beh \(U_1, U_2\) sind Unterräume von
\(\mg{K}^{2 \times 2}\).

\begin{proof}
  Zuerst möchten wir feststellen, welche Eigenschaften
  die Matrix \(B \in U_1\) hat.  Sei
\begin{align*}
B :=
  \begin{pmatrix}
    a & b\\
    c & d
  \end{pmatrix}
\end{align*}
in \(U_1\) beliebig gewählt. Wegen \(A_1B=BA_1\) gilt
\begin{align*}
A_1B=
\begin{pmatrix}
  a+c & b+d \\
  c & d\\
\end{pmatrix} =
  BA_1 =
\begin{pmatrix}
  a & a+b \\
  c & c+d
\end{pmatrix}
\end{align*}
daraus folgt \(c=0\) und \(a = d\).

Nun seien  \(X, Y \in U_1\), \(\alpha \in \mg{K}\) beliebig
gewählt.  Zu zeigen: \(X + Y \in U_1\) und $\alpha X \in
U_1$.  Diese ist wahr, denn es gilt
\begin{align*}
X :=
  \begin{pmatrix}
    a & b \\
    0 & a
  \end{pmatrix}, \quad
Y:=
\begin{pmatrix}
  e & f \\
  0 & e
\end{pmatrix}
\end{align*}
Daraus folgt
\begin{align*}
X + Y =
\begin{pmatrix}
  a + e & b + f \\
  0 & a + e
\end{pmatrix}, \quad
  \alpha X =
  \begin{pmatrix}
    \alpha a & \alpha b \\
    0 & \alpha a
  \end{pmatrix}.
\end{align*}
Die Matrizen \(X + Y\) und \(\alpha X\)  erfüllt die
Eigenschaft dass \(c = 0\) und \(a = d\) und damit gehören
zur Menge \(U_1\).  Die Menge erfüllt das Unterräumenkriterium.

Zuerst möchten wir feststellen, welche Eigenschaften die
Matrix \(B \in U_2\) hat.  Sei
\begin{align*}
B :=
  \begin{pmatrix}
    a & b\\
    c & d
  \end{pmatrix}
\end{align*}
in \(U_2\) beliebig gewählt. Wegen \(A_2B=BA_2\) gilt
\begin{align*}
A_2B=
\begin{pmatrix}
  a & b \\
  a+c & b+d\\
\end{pmatrix} =
  BA_2 =
\begin{pmatrix}
  a+b & b \\
  c+d & d
\end{pmatrix}
\end{align*}
daraus folgt \(b=0\) und \(a = d\).

Nun seien  \(X, Y \in U_2\), \(\alpha \in \mg{K}\) beliebig
gewählt.  Zu zeigen: \(X + Y \in U_2\) und $\alpha X \in
U_2$.  Diese ist wahr, denn es gilt
\begin{align*}
X :=
  \begin{pmatrix}
    a & 0 \\
    c & a
  \end{pmatrix}, \quad
Y:=
\begin{pmatrix}
  e & 0 \\
  g & e
\end{pmatrix}
\end{align*}
Daraus folgt
\begin{align*}
X + Y =
\begin{pmatrix}
  a + e & 0 \\
  c + g & a + e
\end{pmatrix}, \quad
  \alpha X =
  \begin{pmatrix}
    \alpha a & 0 \\
    \alpha c & \alpha a
  \end{pmatrix}.
\end{align*}
Die Matrizen \(X + Y\) und \(\alpha X\)  erfüllt die
Eigenschaft dass \(d = 0\) und \(a = d\) und damit gehören
zur Menge \(U_2\).  Die Menge erfüllt das Unterräumenkriterium.
\end{proof}

\aufgn{11.2.ii}

\beh Basen von \(U_1\) und \(U_2\) sind jeweils
\begin{align*}
  B_1 := \left\{
  \begin{pmatrix}
    1 & 0 \\
    0 & 1
  \end{pmatrix},
  \begin{pmatrix}
    0 & 1 \\
    0 & 0
  \end{pmatrix}
  \right\}, \quad
  B_2 := \left\{
  \begin{pmatrix}
    1 & 0 \\
    0 & 1
  \end{pmatrix},
  \begin{pmatrix}
    0 & 0 \\
    1 & 0
  \end{pmatrix}
  \right\}.
\end{align*}

\begin{proof}
Zu zeigen: die Matrizen in \(B_1\) und \(B_2\) sind linear
unabhängig und \(U_1 = \lin B_1\) und \(U_2 = \lin B_2\).

Linear Unabhängigkeit.  Sei \(\alpha, \beta \in \mg{K}\).
Dann wegen
\begin{align*}
  \alpha
  \begin{pmatrix}
    1 & 0 \\
    0 & 1
  \end{pmatrix} +
  \beta \begin{pmatrix}
    0 & 1 \\
    0 & 0
  \end{pmatrix} =
\begin{pmatrix}
    0 & 0 \\
    0 & 0
  \end{pmatrix}
\end{align*}
gilt \(\alpha = \beta = 0\).  Analog ist auch \(B_2\) linear
unabhängig.

Zu zeigen: \(\lin B_1 \subseteq U_1\) und
\(U_1 \subseteq \lin B_1\).  Sei \(X \in \lin B_1\)
beliebig.  Dann erfüllt \(X\) die Bedingungen dass \(b=0\)
und \(a = d\) und damit \(X \in U_1\).  Sei \(Y \in U_1\)
beliebig.  Dann erfüllt \(Y\) die Bedingungen dass \(b=0\)
und \(a = d\) und damit \(Y \in \lin B_1\).  Analog folgt
\(U_2 = \lin B_2\).
\end{proof}

\aufgn{11.2.iii}

\begin{proof}
  Es gilt
  $\begin{pmatrix}
     1 & 0 \\
     0 & 1
   \end{pmatrix}
   \in U_1 \cap U_2$ und damit
   \(U_1 \cap U_2 \ne \left\{ 0 \right\}\).  Wegen
   Proposition 9.2 dann ist \(U \ne U_1 \oplus U_2\).
\end{proof}

\aufgn{11.2.iv}

\aufgn{11.3.i}

\begin{proof}
  Induktionsanfang: Es gilt
\begin{align*}
A^0 = I_3 =
  \begin{pmatrix}
    1 & 0 & 0 \\
    0 & 1 & 0 \\
    0 & 0 & 1
  \end{pmatrix}.
\end{align*}

  Induktionsannahme: Es gilt für ein \(n \in \mg{N}\) dass
\begin{align*}
  A^n =
\begin{pmatrix}
  1 & n & \frac{n(n-1)}{2} \\
  0 & 1 & n \\
  0 & 0 & 1
\end{pmatrix}.
\end{align*}
Induktionsschritt: \(n \to n+1\).
\begin{align*}
  A^{n+1} = A^n \times A =
  \begin{pmatrix}
  1 & n & \frac{n(n-1)}{2} \\
  0 & 1 & n \\
  0 & 0 & 1
  \end{pmatrix} \times
\begin{pmatrix}
  1 & 1 & 0 \\
  0 & 1 & 1 \\
  0 & 0 & 1
\end{pmatrix} =
\begin{pmatrix}
  1 & n + 1 & n + \frac{n(n-1)}{2} \\
  0 & 1 & n + 1 \\
  0 & 0 & 1
\end{pmatrix} = A^{n+1}.
\end{align*}
\end{proof}

\aufgn{11.3.ii}

\aufgn{11.4.i}

\begin{align*}
M(f) =
\begin{pmatrix}
  1 & -1 & 2 \\
  2 & 0 & -2 \\
  -1 & -1 & 4 \\
  3 & -1 & 0
\end{pmatrix}
\end{align*}

\end{document}