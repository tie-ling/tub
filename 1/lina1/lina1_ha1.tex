\documentclass{article}

% formatting and layout
\usepackage[left=2.5cm, right=2.5cm, bottom=2.5cm]{geometry}
\usepackage[onehalfspacing]{setspace}
\setlength{\parindent}{0pt}

% input/output language
\usepackage[utf8]{inputenc}
\usepackage{xcharter-otf}
\usepackage[ngerman]{babel}

% math packages
\usepackage{amsmath,mathtools,fontspec, amsfonts, amsthm}


\begin{document}
\begin{center}
    \Huge \textbf{Hausaufgabenblatt 1} % hier die Aufgabennummer
    % anpassen
    \\
    \tiny Lineare Algebra 1 WS 22/23
\end{center}
\begin{center}
Eingereicht von Yuchen Guo (Matr.Nr. 480788) und Meng Zhang (Matr.Nr. 484981).\\
Tutor*in: Name
\end{center}

\hrule
\section*{Aufgabe 1.1}
\subsection*{(i)}
\begin{proof}
  Wir setzen zur Abkürzung \(S(n)=\sum_{k=0}^n{k^3}\) und zeigen die
  Gleichung \(S(n) = \frac{n^2(n+1)^2}{4}\) durch vollständige
  Induktion.

\begin{itemize}
\item[(i)] Induktions-Anfang \(n=0\).

  Es ist \(S(0)=0\) und \(\frac{0^2(0+1)^2}{4}=0\), also gilt die Formel für \(n=0\).
\item[(ii)] Induktions-Schritt \(n \rightarrow n+1\).  Wir nehmen an,
  dass \(S(n)=\frac{n^2(n+1)^2}{4}\) gilt für alle \(n \in \mathbb{N}\)
  (Induktions-Voraussetzung) und müssen zeigen, dass daraus die Formel
  \(S(n+1)=\frac{(n+1)^2(n+2)^2}{4}\) folgt.

  Dies sieht man so:
\begin{align*}
  S(n+1) &= S(n) + (n+1)^3 \\
         &= \frac{n^2(n+1)^2}{4} + (n+1)^3 \tag*{Induktions-Voraussetzung} \\
         &= \frac{n^2(n+1)^2}{4} + \frac{4(n+1)^3}{4} \\
         &= \frac{(n+1)^2(n^2+4(n+1))}{4} \\
         &= \frac{(n+1)^2(n+2)^2}{4}.
\end{align*}
\end{itemize}
\end{proof}

\subsection*{(ii)}
\begin{proof}
  Wir setzen zur Abkürzung \(S(n)=\prod_{k=1}^n{k^k}\) und zeigen die
  Ungleichheit \(S(n) \leq n^{\frac{n(n+1)}{2}}\) durch vollständige
  Induktion.

\begin{itemize}
\item[(i)] Induktions-Anfang \(n=1\).

  Es ist \(S(1)=1\) und \(n^{\frac{1(1+1)}{2}} = 1\), also gilt die Formel für \(n=1\).
\item[(ii)] Induktions-Schritt \(n \rightarrow n+1\).

  Wir nehmen an, dass \(S(n)\leq n^{\frac{n(n+1)}{2}}\) gilt für alle
  \(n \in \mathbb{N}, n \geq 1\) (Induktions-Voraussetzung) und müssen
  zeigen, dass daraus die Formel \(S(n+1) \leq (n+1)^{\frac{(n+1)(n+2)}{2}}\)
  folgt.

  Dies sieht man so:
\begin{align*}
  S(n+1) &= S(n) \cdot (n+1)^{(n+1)} \\
         &\leq n^{\frac{n(n+1)}{2}} \cdot (n+1)^{(n+1)}
           \tag*{Induktions-Voraussetzung} \\
         &= n^{\frac{n^2+n}{2}}(n+1)^{(n+1)} \\
  &= n^{\frac{n^2}{2}} \cdot n^{\frac{n}{2}} \cdot (n+1)^n \cdot (n+1)
  \\
  &< (n+1)^{\frac{n^2}{2}} \cdot (n+1)^{\frac{n}{2}} \cdot (n+1)^n
    \cdot (n+1) \tag*{\text{Lemma Aufgabe 1.1.ii.1}}\\
  &= (n+1)^{\frac{(n+1)(n+2)}{2}}
\end{align*}
\end{itemize}
\end{proof}

\textbf{\textit{Bemerkung}} \(n(n+1)\) und \((n+1)(n+2)\) sind durch \(2\)
teilbar.

\textbf{Lemma Aufgabe 1.1.ii.1} Für alle
\(n, k \in \mathbb{N} \setminus \{0\}\) gilt
\begin{align*}
n^k < (n+1)^k.
\end{align*}
\begin{proof}
  Diese Behauptung folgt aus der Anordnungs-Axiome, nämlich:
\begin{align*}
0 \leq x < y \quad \text{und} \quad 0 \leq a < b \implies ax < by.
\end{align*}
\begin{itemize}

\item[(i)] Induktions-Anfang \(k=1\).

  Es ist \(n^1 < (n+1)^1\), also gilt die Formel für \(k=1\).

\item[(ii)] Induktions-Schritt \(k \rightarrow k+1\).

  Wir nehmen an, dass \(n^k < (n+1)^k\) gilt für alle
  \(k \in \mathbb{N}, k \geq 1\) (Induktions-Voraussetzung) und müssen
  zeigen, dass daraus die Formel \(n^{k+1} < (n+1)^{k+1}\) folgt.
\begin{align*}
  n^{k+1} &= n^k \cdot n \\
          &< n \cdot (n+1)^k \\
          &< (n+1) \cdot (n+1)^k \\
  &= (n+1)^{k+1}
\end{align*}
\end{itemize}
\end{proof}

\section*{Aufgabe 1.2}
\subsection*{(i)}
Zuerst formen wir die Aussage explizit in Form \(A \implies B\) um.

\(A\): \(n\) ist eine natürliche Zahl.

\(B\): \(n\) und \(n+1\) sind teilerfremd.

\begin{proof}
  Die Beweisführung erfolgt nach der Methode des Widerspruchsbeweises,
  das heißt, es wird gezeigt, dass die Annahme, die natürliche Zahlen
  \(n\) und \(n+1\) nicht teilerfremd sind, zu einem Widerspruch führt.

  Angenommen, dass \(n\) und \(n+1\) nicht teilerfremd sind und es somit
  einer Teiler \(k \in \mathbb{N}\) gibt.  Seien \(p, q \in \mathbb{N}\),
  dann gilt \(n = p \cdot k\) und \(n+1 = q \cdot k\). Insbesondere,
\begin{align*}
  (n+1) - n = (p - q) \cdot k = 1
\end{align*}
Es existiert aber nur eine Lösung in der Menge der natürlichen Zahlen,
die diese Gleichung erfüllt, nämlich
\begin{align*}
p - q = 1 \quad \wedge \quad k = 1.
\end{align*}

Daraus folgt, dass \(k = 1\) gilt und im Widerspruch zur Annahme \(n\) und
\(n+1\) teilerfremd sind.
\end{proof}

\subsection*{(ii)}

\(A\): Es gibt keine zwei natürliche Zahlen \(a, b \in \mathbb{N}\).

\(B\): \(a^2 - b^2 = 10\).

\begin{proof}
  Es gilt für alle \(a, k \in \mathbb{N}, k \geq 2\), die Aussagen
\begin{align}
  (a+k)^2 - a^2 &>  (a+1)^2 - a^2\\
  (a+1)^2-a^2 &> a^2 - (a-1)^2
\end{align}

Aus (1) folgt, dass \(a^2 - b^2\) ist genau dann minimum, wenn
\(b = a - 1\) ist.  Aus (2) folgt, dass je größer \(a\) ist, desto größer
\(a^2 - b^2\) ist.  Es gilt \(6^2 - 5^2 = 11\) und \(5^2 - 4^2 = 9\).  Wir
wissen auch von (1) und (2), dass für alle andere Kombinationen von
zwei natürlichen Zahlen, die Differenz ihrer Quadrate immer größer als
\(11\) oder kleiner als \(9\) ist. Deshalb gibt es keine zwei natürlichen
Zahlen, sodass die Differenz ihrer Quadrate gleich \(10\) ist.
\end{proof}

\subsection*{(iii)}
\(A\): Für alle \(a, b, c \in \mathbb{R} \setminus \mathbb{Q}\)

\(B\): $a + b  \in \mathbb{R} \setminus \mathbb{Q}
\quad \lor \quad a + c \in \mathbb{R} \setminus \mathbb{Q}
\quad \lor \quad b + c \in \mathbb{R} \setminus \mathbb{Q}$.

  Die Beweisführung erfolgt nach der Methode des indirekten Beweises,
  das heißt, es wird gezeigt, dass die Aussage
  \(\neg B \implies \neg A\) gilt.

\begin{proof}

  \(\neg B\):  Es gibt \(a, b, c \in \mathbb{R}\).  Die Summen \(a+b\),
  \(a+c\), \(b+c\) sind rational.

  \(\neg A\):  \(a, b, c \in \mathbb{Q}\).

  Angenommen, \(p_{1,2,3},q_{1,2,3}\) sind ganze Zahlen, \(p_{n}\) und
  \(q_{n}\) sind teilerfremd.


\begin{align*}
  a+b &= \frac{p_1}{q_1}\\
  a+c &= \frac{p_2}{q_2}\\
  b+c &= \frac{p_3}{q_3}\\
  b&= \frac{1}{2}(a+b)+(b+c)-(a+c)\\
  &=\frac{p_1q_2q_3+p_3q_1q_2-p_2q_1q_3}{q_1q_2q_3}\\
\end{align*}

Damit sind \(a,b,c\) rational, also \(\neg A\) ist wahr.  $\neg B \implies
\neg A$ ist bewiesen.
\end{proof}
\section*{Aufgabe 1.3}
\subsection*{(i)}
\begin{center}
\begin{tabular}{ c  c c c ||c| c c | c|| }
  \(A\) & \(B\) & \(C\) & \(A \lor B\) & \((A \lor B) \wedge C\) & \(A \wedge C\)
  & \(B \wedge C\) & \((A \wedge C) \lor (B \wedge C)\)  \\
  \hline
  w & w & w & w & w & w & w & w \\
  w & w & f & w & f & f & f & f \\
  w & f & w & w & w & w & f & w \\
  w & f & f & w & f & f & f & f \\
  f & w & w & w & w & f & w & w \\
  f & w & f & w & f & f & f & f \\
  f & f & w & f & f & f & f & f \\
  f & f & f & f & f & f & f & f \\
\end{tabular}
\end{center}
\subsection*{(ii)}

Aus Wahrheitstafel \(\ldots\)
\begin{center}
\begin{tabular}{ c  c c c c c c c }
  \(A\) & \(B\) & \(A \lor B\) & \(\neg A \wedge B\) & \(A \lor \neg B\)
  & \(A \wedge (A \lor B)\) & \(A \lor (\neg A \wedge B) \)
  & \((A \lor B) \wedge (A \lor \neg B) \) \\
  \hline
  w & w & w & f & w & w & w & w \\
  w & f & w & f & w & w & w & w \\
  f & w & w & w & f & f & w & f \\
  f & f & f & f & w & f & f & f \\
\end{tabular}
\end{center}

\(\ldots\) folgt:
\begin{align*}
  A \wedge (A \lor B) &= A\\
  A \lor (\neg A \wedge B) &= A \lor B\\
  (A \lor B) \wedge (A \lor \neg B) &= A
\end{align*}

\section*{Aufgabe 1.4}
\begin{center}
  \begin{tabular}{ c c c c c }
    A(n) & Daisy wohnt hier & Gustav schüchtern & Farbe des Autos &
                                                                    Geschwindigkeit \\
    Do & w & f & r & r\\
    Ti & f & f & g & r\\
    Tri & f & w & b & r\\
    Tra & f & f & b & l\\
    Nr. & m & n & p & q\\
\end{tabular}
\end{center}

Angenommen, dass zwei Aussagen von Donald sind wahr, dann

\begin{itemize}
\item m, n wahr;

  dann Aussagen p, q von Tri müssen wahr sein. also w,f,b,r ist wahr.
  In diesem Fall hätte Don 3 wahre Aussagen gemacht.

\item m, p wahr;

  dann Aussagen n, q von Tri und Tra müssen gleichzeitig wahr sein,
  nicht möglich.

\item m, q wahr;

  dann Aussagen n, p von Tra müssen wahr sein, Don hätte 3 wahre Aussagen.
\item n, p wahr;

  dann Aussagen m, q von Tri müssen wahr sein, Don hätte 3 wahre Aussagen.
\item n, q wahr;

  dann wäre f f g/b r wahr.  Dann hätte Ti  3 wahre Aussagen.
\item p, q wahr;

  dann hätte Tri 3 wahre Aussagen gemacht.
\end{itemize}

Alle Möglichkeiten werden ausgeschlossen.  Dagobert hat sich vertan.
\end{document}

