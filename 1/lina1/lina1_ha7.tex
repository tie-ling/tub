\documentclass[12pt]{extarticle}
\usepackage{amsmath,mathtools,fontspec,amsthm,amssymb,amsfonts,fancyhdr,color,graphicx,}
\usepackage[margin=2.5cm]{geometry}
\usepackage[utf8]{inputenc}
\usepackage{xcharter-otf}
\usepackage[ngerman]{babel}
\usepackage[onehalfspacing]{setspace}
\newcommand{\mg}[1]{\mathbb{#1}}
\newcommand{\lin}{\operatorname{lin}}
\renewcommand{\familydefault}{\sfdefault}
\usepackage{parskip}
\pagestyle{fancy}
\setlength{\parindent}{0pt}
\lhead{Yuchen Guo 480788, Meng Zhang 484981}
\rhead{7. Hausaufgabenblatt, LinA I\\Tutorin: Saskia, Mi 2-4 p.m.}
\begin{document}
\textbf{Aufgabe 7.1.i}

\textit{Behauptung.}  $v \in \lin (\left\{ v_1, \ldots, v_n \right\})
\iff \lin (\left\{ v, v_1, \ldots, v_n \right\}) = \lin(\left\{ v_1,
  \ldots, v_n \right\})$.

\begin{proof}
  Wir beweisen diese Aussage als wahr, indem wir zeigen, dass die
  Hinrichtung und Rückrichtung wahr sind.

  \textbf{Hinrichtung.}

  Sei \(x \in \lin (\left\{ v, v_1, \ldots, v_n \right\})\) beliebig
  gewählt.  Es gilt,
\begin{align*}
x = \lambda_0v+\sum_{i=1}^n{\lambda_iv_i} \text{ mit } n \in
  \mg{N}_{\geq 1}, 1 \leq i \leq n, \lambda_i \in \mg{K}, \lambda_0
  \in \mg{K}, v_i \in \left\{ v_1, \ldots, v_n \right\}. \tag{1}
\end{align*}

Wegen \(v \in \lin (\left\{ v_1, \ldots, v_n \right\})\), es gilt
\begin{align*}
  v = \sum_{i=1}^n{\lambda_iv_i} \text{ mit } n \in
  \mg{N}_{\geq 1}, 1 \leq i \leq n, \lambda_i \in \mg{K},
  v_i \in \left\{ v_1, \ldots, v_n \right\}. \tag{2}
\end{align*}

Wir setzen (2) in (1) ein, erhalten wir
\begin{align*}
  x &= \lambda_0\sum_{i=1}^n{\lambda_iv_i}+\sum_{i=1}^n{\lambda_iv_i}\\
  &= \sum_{i=1}^n{\underbrace{(\lambda_i+\lambda_0\lambda_i)}_{\text{(3)}}v_i}.
\end{align*}

Weil \(\mg{K}\) ein Körper ist, gilt (3) \(\in \mg{K}\).  Es gilt dann
\(x \in \lin(\left\{ v_1, \ldots, v_n \right\})\).  Daraus folgt,
$ \lin (\left\{ v, v_1, \ldots, v_n \right\}) \subseteq \lin(\left\{
  v_1, \ldots, v_n \right\})$.

Es gilt auch
$ \lin(\left\{ v_1, \ldots, v_n \right\}) \subseteq \lin (\left\{ v,
  v_1, \ldots, v_n \right\})$, denn sei
\(x \in \lin(\left\{ v_1, \ldots, v_n \right\}) \) beliebig gewählt, es
gilt
$x = 0v_0 + \sum_{i=1}^n{\lambda_iv_i} \in \lin (\left\{ v, v_1,
  \ldots, v_n \right\})$.  Daraus folgt
$\lin (\left\{ v, v_1, \ldots, v_n \right\}) = \lin(\left\{ v_1,
  \ldots, v_n \right\})$.

\textbf{Rückrichtung.}

Es gilt $ \lin (\left\{ v, v_1, \ldots, v_n \right\}) = \lin(\left\{ v_1,
  \ldots, v_n \right\})\( und \)v \in \lin (\left\{ v, v_1, \ldots, v_n
\right\}) \(, daraus folgt, \)v \in \lin(\left\{ v_1,
  \ldots, v_n \right\})$.
\end{proof}

\textbf{Aufgabe 7.1.ii}

\textit{Behauptung.}
$v \notin \lin (\left\{ v_1, \ldots, v_n \right\}) \iff \lin (\left\{
  v_1, \ldots, v_n \right\}) \subsetneq \lin(\left\{ v, v_1, \ldots,
  v_n \right\})$.
\begin{proof}
    Wir beweisen diese Aussage als wahr, indem wir zeigen, dass die
  Hinrichtung und Rückrichtung wahr sind.

  Zur Abkürzung setzen wir $A=\lin (\left\{
  v_1, \ldots, v_n \right\})\( und \)B= \lin(\left\{ v, v_1, \ldots,
  v_n \right\})$.

  \textbf{Hinrichtung.}

  Wir zeigen, dass \( A \subseteq B\) und
  \( B \setminus A \neq \emptyset\) gilt.

  Sei \(x \in A \) beliebig gewählt.  Dann gilt
  \(x = 0v+\sum_{i=1}^n{\lambda_iv_i}\) und damit \(x \in B\) und
  \( A \subseteq B\).

Wir zeigen, dass
\(B \setminus A \neq \emptyset\) gilt, mittels
Widerspruchsbeweis.  Angenommen,
\(B \setminus A = \emptyset\).  Dann gilt wegen
\( A \subseteq B\) dass \(A = B\).  Aus (i) folgt aber im Widerspruch zur
Voraussetzung, dass \(v \in A\) gilt.  Daraus folgt, dass $A \subsetneq
B$ muss.

\textbf{Rückrichtung.}

Die Negation von (i) ist, \(v \notin A \iff B \neq A\).  Wir haben
bewiesen, dass \(A \subseteq B\).  Daraus folgt, dass
\(v \notin A \iff A \subsetneq B\).
\end{proof}

\textbf{Aufgabe 7.1.iii}

\textit{Behauptung.} \(\lin(M \cap N) = \lin(M) \cap \lin(N)\).
\begin{proof}
  Wir zeigen, dass diese Aussage wahr ist, indem wir zeigen, dass
  \(\lin(M \cap N) \subseteq \lin(M) \cap \lin(N)\) und
  \(\lin(M) \cap \lin(N) \subseteq \lin(M \cap N)\) gilt.

  \begin{itemize}

  \item Falls \(M = \emptyset\) oder \(N = \emptyset\), dann gilt
    $\lin(\emptyset \cap N) = \lin(\emptyset) = \left\{ 0 \right\} =
    \left\{ 0 \right\} \cap \lin(N)$.
  \item \(\lin(M \cap N) \subseteq \lin(M) \cap \lin(N)\).

    Sei \(M \cap N = \left\{ v_1, \ldots, v_n \right\}\) und $x \in
    \lin(M \cap N)\( beliebig gewählt.  Dann gilt \)x =
    \sum_{i=1}^n{\lambda_iv_i}$.  Es gilt, \(\lambda_i \in \mg{K}\),
    \(v_i \in M\) und \(v_i \in N\).  Daraus folgt, dass \(x \in \lin(M)\)
    und \(x \in \lin(N)\).  Damit gilt \(x \in \lin(M) \cap \lin(N)\).

  \item   \(\lin(M) \cap \lin(N) \subseteq \lin(M \cap N)\)

???
  \end{itemize}
\end{proof}

\textbf{Aufgabe 7.1.iv}

\textit{Behauptung.} \(\lin(M + N) \subseteq \lin(M) + \lin(N)\).
\begin{proof}
  Sei \(x \in \lin(M+N)\) beliebig gewählt.  Dann gilt
\begin{align*}
  x &= \sum_{i=1}^n{\lambda_i(m_i+n_i)} \text{ mit } n \in
  \mg{N}_{\geq 1}, 1 \leq i \leq n, \lambda_i \in \mg{K}, m_i \in M,
      n_i \in N.\\
  &= \underbrace{\sum_{i=1}^n{\lambda_im_i}}_{\text{(1)}}+\underbrace{\sum_{i=1}^n{\lambda_in_i}}_{\text{(2)}}
\end{align*}

Es gilt (1) \(\in \lin{M}\) und (2) \(\in \lin{N}\).  Daraus folgt, $x \in
\lin(M)+\lin(N)$.
\end{proof}

\textbf{Aufgabe 7.2.i}

\textit{Behauptung.}  \(U/T\) ist ein Unterraum von \(V/T\).

\begin{proof}
  Wegen \(U \subseteq V\) gilt \(U/T \subseteq V/T\).  Sei $\overline{a},
  \overline{b} \in U/T$ und \(\lambda \in \mg{K}\) beliebig gewählt.  Wir
  zeigen, dass \(\overline{a} + \overline{b} \in U/T\) und $\lambda
  \overline{a} \in U/T$ gilt.

  \begin{itemize}
  \item \(\overline{a} + \overline{b} \in U/T\).

    Es gilt $\overline{a} = \left\{ v \in U \colon a - v \in T
    \right\}\( und \)\overline{b} = \left\{ w \in U \colon b - w \in T
    \right\}\(.  Daraus folgt, \)\overline{a} + \overline{b} = \left\{ v
      + w \colon v \in \overline{a}, w \in \overline{b}\right\}$.

    Weil \(T\) ein Unterraum ist, gilt wegen \(a-v\in T\) und \(b-w \in T\)
    dass \((a-v)+(b-w) \in T\).  Daraus folgt, \((a+b)-(v+w) \in T\) und
    $\overline{a} + \overline{b} = \left\{ v + w \colon (a+b)-(v+w)
      \in T \right\}$.  Wegen \(a+b \in U\) und \(v+w \in U\) gilt dann
    \(\overline{a} + \overline{b} \in U/T\).
  \item \(\lambda \overline{a} \in U/T\).

    Es gilt wegen (Definition 5.15) dass $\lambda \overline{a} =
    \overline{\lambda a}\(.  Weiter gilt \)\overline{\lambda a}= \left\{
      v \in U \colon \lambda a - v \in T  \right\}\(.  Es gilt \)\lambda
    a  \in U$, weil \(U\) ein Unterraum ist.  Daraus folgt, dass
    \(\lambda \overline{a} \in U/T\).
  \end{itemize}
\end{proof}

\textbf{Aufgabe 7.2.ii}

\textbf{Aufgabe 7.3.i}

Sei \(U:=V\) und \(W:=V\).  Dann gilt \(V=U+W\), weil \(V\) abelsch ist.

\textbf{Aufgabe 7.3.ii}

Sei \(U:= \left\{ 0 \right\}\) und \(W:= \left\{ 0 \right\}\).  Dann gilt
\(U \cap W = \left\{ 0 \right\}\).

\textbf{Aufgabe 7.3.iii}

\textbf{Aufgabe 7.3.iv}

\textbf{Aufgabe 7.4.i}

\textbf{Aufgabe 7.4.ii}
\end{document}