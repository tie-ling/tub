\documentclass[12pt]{extarticle}
\usepackage{amsmath,mathtools,fontspec,amsthm,amssymb,amsfonts,fancyhdr,color,graphicx,}
\usepackage[margin=2.5cm]{geometry}
\usepackage[utf8]{inputenc}
\usepackage{xcharter-otf}
\usepackage[ngerman]{babel}
\usepackage[onehalfspacing]{setspace}
\usepackage{tikz}
\renewcommand{\familydefault}{\sfdefault}
\pagestyle{fancy}
\setlength{\parindent}{0pt}
\lhead{Yuchen Guo 480788, Meng Zhang 484981}
\rhead{3. Hausaufgabenblatt, LinA I\\Tutorin: Saskia, Mi 2-4 p.m.}
\begin{document}
\textbf{Aufgabe 3.1}
\begin{itemize}
\item \(f_1\): \(\mathbb{R} \rightarrow \mathbb{R}^2\), \(x \mapsto (x,x^2)\).

  \begin{itemize}
  \item Injektivität

    Seien \(x_1, x_2 \in \mathbb{R}\) mit \(x_1 \neq x_2\).  Zu zeigen:
    \(x_1 \neq x_2 \implies (x_1,x_1^2) \neq (x_2,x_2^2)\).

    Diese Aussage folgt unmittelbar aus \(x_1 \neq x_2\).  Deshalb ist
    diese Abbildung injektiv.
  \item Surjektivität

    Seien \((y, y^2) \in \mathbb{R}^2\).  Zu zeigen: es existiert
    mindestens ein \(x \in \mathbb{R}\) mit \(x \mapsto (y,y^2)\).

    Setzen wir \(x=y\), dann erhalten wir \(x=y \mapsto (y,y^2)\).  Deshalb
    ist diese Abbildung surjektiv.
  \end{itemize}

  Diese Abbildung ist daher injektiv, surjektiv und bijektiv.
\item \(f_2\): \(\mathbb{N} \rightarrow \mathbb{N}\), $n \mapsto
  \begin{cases}
     \frac{n}{2}, & n \text{ gerade} \\
     \frac{n+1}{2}, & n \text{ ungerade}
  \end{cases}$

  \begin{itemize}
  \item Nicht Injektiv

  Denn, mit \(x_1=1\) und \(x_2=2\) erhalten wir \(y_1=y_2=1\).

  \item Surjektivität

    Sei \(y \in \mathbb{N}\) beliebig gewählt.  Zu zeigen: es existiert
    mindestens eine Lösung \(x \in \mathbb{N}\) mit \(f(x)=y\).

    \begin{itemize}
    \item Fall 1:  Sei \(y=\frac{n+1}{2}\) mit \(n=2k+1\), $k \in
      \mathbb{N}$.

      Mit Umformungen erhalten wir \(k=y-1\).  Wegen \(n=2k+1\), $k \in
      \mathbb{N}\(, es gilt \(n \geq 1\) und \(y \geq 1\).  D.h., \)k=y-1
      \geq 0$, also \(k \in \mathbb{N}\).

      Deshalb existiert mindestens eine Lösung \(k \in \mathbb{N}\),
      sodass \(f(x)=f(2k+1)=y\) gilt.
    \item Fall 2:  Sei \(y=\frac{n}{2}\) mit \(n=2k\), $k \in
      \mathbb{N}$.

      Mit Umformungen erhalten wir \(k=y\).

      Deshalb existiert mindestens eine Lösung \(k \in \mathbb{N}\),
      sodass \(f(x)=f(2k)=y\) gilt.
    \end{itemize}
    Daher ist die Abbildung surjektiv.

    Die Abbildung ist injektiv, surjektiv und bijektiv.
  \end{itemize}

\item \(f\): \(\mathbb{R}^2 \rightarrow \mathbb{R}\), $(x,y) \mapsto
  x^2-y^2+1$.

  \begin{itemize}
  \item Nicht injektiv.

    Denn, für alle \((x,y)\) in \(\mathbb{R}^2\) gilt $f
    \left( (x,y) \right) = f \left( (-x,-y) \right)$.
  \item Surjektiv.

    Sei \(p \in \mathbb{R}\) beliebig gewählt.
\begin{align*}
  x^2-y^2+1&=p\\
  y^2&=x^2+1-p\\
  y&=\pm \sqrt{x^2+1-p}
\end{align*}
Diese Abbildung ist surjektiv, denn, für alle \(p \in \mathbb{R}\)
existiert ein \(x \in \mathbb{R}\) mit der Eigenschaft \(x^2+1-p \geq 0\).
Damit existiert ein Tupel \((x,y) \in \mathbb{R}^2\) für alle \(p \in \mathbb{R}\).
  \end{itemize}
\end{itemize}

\textbf{Aufgabe 3.2}

\begin{itemize}
\item[i \(\implies\) ii]
  \begin{proof}
    Wegen der Injektivität der Abbildung, gilt für alle $x_1 \in X
    \setminus N$ und \(x_2 \in N\), \(f(x_1) \neq f(x_2)\).

    Daraus folgt,
\begin{align*}
f(X \setminus N) \cap f(N) = \emptyset \tag{1}
\end{align*}

Nach Voraussetzung gilt \(N \subseteq X\). Dann für alle
\(x \in (X \setminus N) \cup N\) gilt, \(x \in X\).

Daraus folgt,
\begin{align*}
f(X \setminus N) \cup f(N) = f(X) \tag{2}
\end{align*}

Wegen (1) und (2), gilt
\begin{align*}
f(X) \setminus f(N)   &= (f(X \setminus N) \cup f(N)) \setminus f(N)
                                                \tag*{Wegen (2)}\\
&=  f(X \setminus N)  \tag*{Wegen (1)}
\end{align*}
    \end{proof}
\item[ii \(\implies\) iii]
Wir zeigen, dass die beide Aussage $f(M\cap N) \subseteq f(M) \cap
f(N)$ und \(f(M) \cap f(N) \subseteq f(M \cap N)\) gelten.

\begin{itemize}
\item \(f(M\cap N) \subseteq f(M) \cap f(N)\)
  \begin{proof}
    Wir setzen zur Abkürzung \(A=M \cap N\) ein.

    Aus der Voraussetzung folgt, dass $f(M \setminus A)=f(M) \setminus
    f(A)$ und \(f(N \setminus A)=f(N) \setminus f(A)\).

    Daraus folgt:
\begin{align*}
       f(M \setminus A) \cup f(A) &= f(M)\\
       f(N \setminus A) \cup f(A) &= f(N)\\
       f(M \setminus A) \cap f(A) &= \emptyset\\
       f(N \setminus A) \cap f(A) &= \emptyset
\end{align*}

Wir erhalten $f(M) \cap f(N) = f(A) \cup (f(M \setminus A) \cap f(N
\setminus A))$.  Damit gilt \(f(A) \subseteq f(M) \cap f(N)\).
    \end{proof}
\item \(f(M) \cap f(N) \subseteq f(M \cap N)\)

  ??? Wir können diese Aussage leider nicht beweisen.

\end{itemize}
\item[iii \(\implies\) i]
  \begin{proof}
Wegen der Voraussetzung gilt für alle \(M, N \in X\) die Aussage $f(M
\cap N)=f(M) \cap f(N)$.

Insbesondere, gilt für alle \(M, (X \setminus M) \in X\) die Aussage
\(f(M \cap (X \setminus M))=f(\emptyset)=f(M) \cap f(X \setminus M)\).

Das heißt, für alle \(x_1, x_2 \in X\) mit \(x_1 \neq x_2\) gilt $f(x_1)
\neq f(x_2)$.  Damit ist diese Abbildung injektiv.
    \end{proof}
\end{itemize}

\textbf{Aufgabe 3.3}
\begin{itemize}
\item[i \(\implies\) ii]

  \begin{proof}
    Weil die Abbildung surjektiv ist, gibt es für jedes \(y \in Y\)
    mindestens ein \(x \in X\) mit \(f(x)=y\).

    Daraus folgt, dass es für jedes \(a \in A\) mit \(A \subseteq Y\)
    mindestens ein \(b \in X\) mit \(f(b)=a\) gibt.

    Sei die Menge \(B = \left\{ b \in X|f(b)=a, a \in A \right\}\).  Es
    gilt daher \(f^{-1}(A)=B\).  Aus der Definition von \(B\) folgt, dass \(f(B)=A\).
    \end{proof}
  \item[ii \(\implies\) iii]

    ???
  \item[iii \(\implies\) i]

    ???
  \end{itemize}


  \textbf{Aufgabe 3.4.1}

  \begin{proof}
    Wir bemühen einen Widerspruchsbeweis.

    Seien \(X,Y\) nichtleere endliche Mengen mit $\left| X \right| >
    \left| Y \right|\(.  Es existiert eine injektive Abbildung \)f: X
    \rightarrow Y$.

    Aus der Definition von Abbildung folgt, dass für jedes \(x \in X\)
    muss ein \(y \in Y\) zugeordnet sein.  Weil die Abbildung injektiv
    ist, gilt \(x_1 \neq x_2 \implies f(x_1) \neq f(x_2)\).

    Es gilt daher, dass \(\left| Y \right| \geq \left| X \right|\).
    Dies ist im Widerspruch zur Voraussetzung  $\left| X \right| >
    \left| Y \right|$.  Deshalb existiert keine injektive Abbildung
    \(f: X \rightarrow Y\).
    \end{proof}


\textbf{Aufgabe 3.4.2}

Es gilt \(\left| \mathbb{Z} \right| > \left| \mathbb{N} \right|\).
Denn, für alle \(n \in \mathbb{N}_{>0}\) gibt es eine Inverse von \(n\) in
\(\mathbb{Z}\), nämlich \(-n\).  Daher gibt es wegen des vorherigen Beweis
keine injektive und daher keine bijektive Abbildung
\(f: \mathbb{Z} \rightarrow \mathbb{N}\).
\end{document}