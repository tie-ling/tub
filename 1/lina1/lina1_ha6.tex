\documentclass[12pt]{extarticle}
\usepackage{amsmath,mathtools,fontspec,amsthm,amssymb,amsfonts,fancyhdr,color,graphicx,}
\usepackage[margin=2.5cm]{geometry}
\usepackage[utf8]{inputenc}
\usepackage{xcharter-otf}
\usepackage[ngerman]{babel}
\usepackage[onehalfspacing]{setspace}
\newcommand{\ib}[1]{\boldsymbol{\mathit{#1}}}
\newcommand{\im}{\operatorname{\textbf{i}}}
\newcommand{\mg}[1]{\mathbb{#1}}
\renewcommand{\familydefault}{\sfdefault}
\usepackage{parskip}
\pagestyle{fancy}
\setlength{\parindent}{0pt}
\lhead{Yuchen Guo 480788, Meng Zhang 484981}
\rhead{6. Hausaufgabenblatt, LinA I\\Tutorin: Saskia, Mi 2-4 p.m.}
\begin{document}
\textbf{Aufgabe 6.1.i}

\textit{Behauptung.}  \((\mg{Z}, \oplus, \otimes)\) ist kein Körper.

\begin{proof}
  Wir zeigen, dass \((\mg{Z}, \oplus, \otimes)\) kein Körper ist,
  indem wir zeigen, dass die Relation
  \((\mg{Z}\setminus \left\{ 1 \right\}, \otimes)\) keine abelsche
  Gruppe ist.

  Wir zeigen zuerst, dass die Relation \((\mg{Z}, \oplus)\) eine
  abelsche Gruppe ist mit neutralem Element \(1\) und inversem Element
  \(a'=2-a\) für alle \(a \in \mg{Z}\).

  \begin{itemize}
  \item Die Menge \(\mg{Z}\) ist nicht leer.

  \item Die Relation \((\mg{Z}, \oplus)\) ist assoziativ.
    Denn, für alle \(x, y, z \in \mg{Z}\) gilt,
\begin{align*}
  (x \oplus y) \oplus z &= (x+y-1)+z-1\\
                        &= x+(y+z-1)-1\\
                        &= x \oplus(y+z-1)\\
  &= x \oplus(y\oplus z).
\end{align*}
\item  Die Relation \((\mg{Z}, \oplus)\) hat ein neutrales Element
  \(e:=1\), sodass für alle \(x \in \mg{Z}\) gilt $e \oplus x = x + 1 -
  1 =x$.
\item Es existiert zu jedem \(x \in \mg{Z}\) ein inverses Element
  \(x':=2-x\), sodass \(x' \oplus x=(2-x) + x - 1 = 1 = e\).
\item Für alle \(x, y \in \mg{Z}\) gilt, $a \oplus b = a + b - 1 = b
  + a - 1 = b \oplus a$.  Damit ist diese Gruppe abelsch.
\item Die Relation \((\mg{Z}\setminus \left\{ 1 \right\}, \otimes)\)
  ist assoziativ.
  Denn, für alle \(x, y, z \in \mg{Z}\setminus \left\{ 1 \right\}\)
  gilt,
\begin{align*}
  a \otimes (b \otimes c) &= a \otimes (b+c - bc)\\
                          &= (a+b+c - bc) - a(b+c-bc)\\
                          &= a+b+c-bc-ab-ac+abc \\
                          &= (a+b-ab)+c - (a+b-ab)c\\
                          &= (a+b - ab) \otimes c\\
  &= (a \otimes b) \otimes c.
\end{align*}
\item Die Relation
  \((\mg{Z} \setminus \left\{ 1 \right\}, \otimes)\) hat ein
  neutrales Element \(e:=0\), sodass für alle
  \(x \in \mg{Z} \setminus \left\{ 1 \right\}\) gilt
  \(e \otimes x = 0+ x - 0x =x\).
\item Es existiert nicht zu jedem
  \(x \in \mg{Z} \setminus \left\{ 1 \right\}\) ein inverses Element
  \(x':=\frac{x}{x-1} \in \mg{Z} \setminus \left\{ 1 \right\}\),
  sodass \(x' \otimes x= \frac{x}{x-1} + x - \frac{x^2}{x-1} = 0 =e\).
  Die Relation ist keine Gruppe.
\item Die Relation
  \((\mg{Z} \setminus \left\{ 1 \right\}, \otimes)\) ist kommutativ,
  denn für alle \(x, y \in \mg{Z} \setminus \left\{ 1 \right\}\)
  gilt \(x \otimes y = x + y - xy = y+x - yx = y \otimes x\).
\end{itemize}

Weil \((\mg{Z} \setminus \left\{ 1 \right\}, \otimes)\) keine
abelsche Gruppe ist, ist \((\mg{Z}, \oplus, \otimes)\) kein Körper.
\end{proof}

\textbf{Aufgabe 6.1.ii}

\textit{Behauptung.}  \((\mg{Q}, \oplus, \otimes)\) ist ein Körper.

\begin{proof}
  Wir zeigen zuerst, dass die Relation \((\mg{Q}, \oplus)\) eine
  abelsche Gruppe ist mit neutralem Element \(1\) und inversem Element
  \(a'=2-a\) für alle \(a \in \mg{Q}\).

  \begin{itemize}
  \item Die Menge \(\mg{Q}\) ist nicht leer.
  \item Die Relation \((\mg{Q}, \oplus)\) ist assoziativ.
    Denn, für alle \(x, y, z \in \mg{Q}\) gilt,
\begin{align*}
  (x \oplus y) \oplus z &= (x+y-1)+z-1\\
                        &= x+(y+z-1)-1\\
                        &= x \oplus(y+z-1)\\
  &= x \oplus(y\oplus z).
\end{align*}
\item  Die Relation \((\mg{Q}, \oplus)\) hat ein neutrales Element
  \(e:=1\), sodass für alle \(x \in \mg{Q}\) gilt $e \oplus x = x + 1 -
  1 =x$.
\item Es existiert zu jedem \(x \in \mg{Q}\) ein inverses Element
  \(x':=2-x\), sodass \(x' \oplus x=(2-x) + x - 1 = 1 = e\).
\item Für alle \(x, y \in \mg{Q}\) gilt, $a \oplus b = a + b - 1 = b
  + a - 1 = b \oplus a$.  Damit ist diese Gruppe abelsch.
\item Die Relation \((\mg{Q}\setminus \left\{ 1 \right\}, \otimes)\)
  ist assoziativ.

  Denn, für alle \(x, y, z \in \mg{Q}\setminus \left\{ 1 \right\}\)
  gilt,
\begin{align*}
  a \otimes (b \otimes c) &= a \otimes (b+c - bc)\\
                          &= (a+b+c - bc) - a(b+c-bc)\\
                          &= a+b+c-bc-ab-ac+abc \\
                          &= (a+b-ab)+c - (a+b-ab)c\\
                          &= (a+b - ab) \otimes c\\
  &= (a \otimes b) \otimes c.
\end{align*}
\item Die Relation
  \((\mg{Q} \setminus \left\{ 1 \right\}, \otimes)\) hat ein
  neutrales Element \(e:=0\), sodass für alle
  \(x \in \mg{Q} \setminus \left\{ 1 \right\}\) gilt
  \(e \otimes x = 0+ x - 0x =x\).
\item Es existiert zu jedem
  \(x \in \mg{Q} \setminus \left\{ 1 \right\}\) ein inverses Element
  \(x':=\frac{x}{x-1} \in \mg{Q} \setminus \left\{ 1 \right\}\),
  sodass \(x' \otimes x= \frac{x}{x-1} + x - \frac{x^2}{x-1} = 0 =e\).
\item Die Relation
  \((\mg{Q} \setminus \left\{ 1 \right\}, \otimes)\) ist kommutativ,
  denn für alle \(x, y \in \mg{Q} \setminus \left\{ 1 \right\}\)
  gilt \(x \otimes y = x + y - xy = y+x - yx = y \otimes x\).
\item Für alle \(x, y, z \in \mg{Q}\) gilt
\begin{align*}
  x \otimes (y \oplus z) &= x \otimes (y + z - 1)\\
                         &= x + (y + z - 1) - x(y+z-1)\\
                         &= x+y+z-1-xy-xz-x\\
                         &= (x + y - xy) \oplus (x + z - xz)\\
  &= x \otimes y \oplus x \otimes z.
\end{align*}
\end{itemize}

Daher ist \((\mg{Q}, \oplus, \otimes)\) ein Körper.
\end{proof}

\textbf{Aufgabe 6.2}

\begin{proof}
  Wir zeigen, dass \((W, \oplus, \otimes)\) ein \(\mg{K}\)-Vektorraum
  ist, indem wir zeigen, dass die Relation die Definition von
  Vektorraum erfüllt.

  Zuerst zeigen wir, dass \((W, \oplus)\) eine abelsche Gruppe ist.
  Weil die Abbildung \(f \colon V \to W\) bijektiv ist, existiert eine
  bijektive Umkehrabbildung \(f^{-1}\colon W \rightarrow V\).
  \begin{itemize}
  \item \((W, \oplus)\) ist assoziativ.

    Denn, es gilt für alle \(\ib{v}, \ib{w}, \ib{x} \in \mg{W}\),
\begin{align*}
  (\ib{v} \oplus \ib{w}) \oplus \ib{x}
  &= f(f^{-1}(\ib{v}) + f^{-1}(\ib{w})) \oplus \ib{x}\\
  &= f(f^{-1}(f(f^{-1}(\ib{v}) + f^{-1}(\ib{w}))) +
    f^{-1}(\ib{x}))\\
  &= f(f^{-1}(\ib{v}) + f^{-1}(\ib{w}) + f^{-1}(\ib{x}))
    \tag*{\(f\) bijektiv}\\
  &= f(f^{-1}(\ib{v}) + (f^{-1}(\ib{w}) + f^{-1}(\ib{x})))
    \tag*{\((V, \oplus)\) abelsche Gruppe}\\
  &= f(f^{-1}(\ib{v}) + f^{-1}(f(f^{-1}(\ib{w}) + f^{-1}(\ib{x}))))\\
  &= \ib{v} \oplus (\ib{w} \oplus \ib{x}).
\end{align*}
\item \((W, \oplus)\) hat ein neutrales Element.

    Weil \(V\) ein \(\mg{K}\)-Vektorraum mit Addition und skalarer
    Multiplikation ist, es existiert ein neutrales Element \(\ib{e_V} \in V\)
    sodass für alle \(\ib{v} \in V\) gilt (1): \(\ib{e_V} + \ib{v} = \ib{v}\).
    Weil \(f^{-1}\) bijektiv ist, existiert ein \(\ib{e_W} \in W\) mit
    (2): \(f^{-1}(\ib{e_W})=\ib{e_V}\).

    Sei \(\ib{x} \in W\) beliebig gewählt.  Dann gilt
\begin{align*}
  \ib{e_W} \oplus \ib{x} &= f(f^{-1}{(\ib{e_w})}+f^{-1}{(\ib{x})})\\
                         &= f(f^{-1}{(\ib{x})}) \tag*{Wegen (1)}\\
                         &= \ib{x}.
\end{align*}
\item Jedes Element in \((W, \oplus)\) hat ein inverses Element.

  Weil \(V\) ein \(\mg{K}\)-Vektorraum mit Addition und skalarer
  Multiplikation ist, es existiert zu jedes Element \(\ib{v} \in V\) ein
  \(\ib{-v} \in V\) sodass (3): \((\ib{-v}) + \ib{v} = \ib{e_V}\) gilt.
  Weil \(f^{-1}\) bijektiv ist, existiert ein \((\ib{-v})_W \in W\) mit
  \(f^{-1}((\ib{-v})_W)=\ib{-v}\).

    Sei \(\ib{x} \in W\) beliebig gewählt.  Dann gilt
\begin{align*}
  (\ib{-x})_{W} \oplus \ib{x}
  &= f(f^{-1}((\ib{-x})_W) + f^{-1}(\ib{x}))\\
  &= f(\ib{e}_V) \tag*{Wegen (3)}\\
  &= \ib{e}_W. \tag*{Wegen (2)}
\end{align*}
  \item \((W, \oplus)\) ist kommutativ.

    Weil \(V\) ein \(\mg{K}\)-Vektorraum mit Addition und skalarer
    Multiplikation ist, gilt für alle \(\ib{a}, \ib{b} \in V\)
    dass \(\ib{a}+\ib{b}=\ib{b}+\ib{a}\).

    Daraus folgt, dass
    \(f^{-1}(\ib{v})+f^{-1}(\ib{w})=f^{-1}(\ib{w})+f^{-1}(\ib{v})\)
    und damit
    \(\ib{v} \oplus \ib{w} = \ib{w} \oplus \ib{v}\)
    gilt.  Die Relation \((W, \oplus)\) ist kommutativ.
  \end{itemize}

  Danach zeigen wir, dass die folgende Behauptung gilt.

  Für alle \(\ib{v}, \ib{w} \in W\) und alle $\alpha, \beta \in
  \mg{K}$ gilt
  \begin{itemize}
  \item \((\alpha + \beta) \ib{v}=\alpha \ib{v} \oplus \beta \ib{v}\).
    Denn, es gilt
\begin{align*}
  \alpha \ib{v} \oplus \beta \ib{v}
  &= f(\alpha f^{-1}(\ib{v})) \oplus f(\beta f^{-1}(\ib{v}))\\
  &= f(f^{-1}(f(\alpha f^{-1}{(\ib{v})}))+f^{-1}(f(\beta
    f^{-1}{(\ib{v})})))\\
  &= f(\alpha f^{-1}{(\ib{v})} + \beta f^{-1}{(\ib{v})})
    \tag*{\(f\) bijektiv}\\
  &= f((\alpha+\beta)f^{-1}{(\ib{v})})\\
  &= (\alpha+\beta)\ib{v}.
\end{align*}
  \item
    $\alpha (\ib{v} \oplus \ib{w}) = \alpha \ib{v} \oplus \alpha
    \ib{w} $.
    Denn, es gilt
\begin{align*}
  \alpha ( \ib{v} \oplus \ib{w} )
  &= \alpha ( f ( f^{-1}(\ib{v}) + f^{-1}(\ib{w})))\\
  &= f(\alpha f^{-1}(f ( f^{-1}(\ib{v}) + f^{-1}(\ib{w}))))\\
  &= f(\alpha( f^{-1}(\ib{v}) + f^{-1}(\ib{w})))\\
  &= f(\alpha f^{-1}(\ib{v}) + \alpha f^{-1}(\ib{w}))
    \tag*{Weil \(A\) ist \(\mg{K}\)-Vektorraum}\\
  &= f(\alpha f^{-1}(\ib{v})) \oplus f(\alpha f^{-1}(\ib{w}))\\
  &= \alpha \ib{v} \oplus \alpha \ib{w}.
\end{align*}
  \item \((\alpha\beta)\ib{v}=\alpha(\beta\ib{v})\).  Denn, es gilt
\begin{align*}
  \alpha(\beta\ib{v}) &= \alpha(f(\beta f^{-1}(\ib{v}))) \\
                      &= f(\alpha f^{-1}(f(\beta f^{-1}(\ib{v}))))\\
                      &= f(\alpha (\beta f^{-1}(\ib{v})))\\
                      &= f(\alpha \beta f^{-1}(\ib{v}))
                        \tag*{Weil \(A\) ist \(\mg{K}\)-Vektorraum}\\
                      &= f((\alpha \beta) f^{-1}(\ib{v}))\\
                      &= (\alpha \beta)\ib{v}.
\end{align*}
  \item \(1_{\mg{K}}\ib{v}=\ib{v}\).  Denn, es gilt $1_{\mg{K}}
    \ib{x}$ für alle \(\ib{x} \in V\) und \(f^{-1}(\ib{v}) \in V\).
    Daraus folgt, dass $1_{\mg{K}}\ib{v} =
    f(1_{\mg{K}}f^{-1}(\ib{v})) = f(f^{-1}(\ib{v})) = \ib{v}$.
  \end{itemize}
\end{proof}

\textbf{Aufgabe 6.3.i}

\textit{Behauptung.}  Sei \((\mg{K}, +, \cdot)\) ein Körper,
\((R, +, \cdot)\) ein kommutativer Ring mit Eins bezüglich der
Multiplikation \(\cdot\) in \(\mg{K}\) und \(\mg{K} \subseteq R\).  Dann
ist \(R\) ein \(\mg{K}\)-Vektorraum.

Es bezeichne \(1\) das Einselement und \(0\) das Nullelement in \(\mg{K}\).

\begin{proof}
  Diese Aussage ist wahr.  Zuerst zeigen wir, dass \(1_R = 1\).

  Weil \(\mg{K}\) ein Körper ist, ist $(\mg{K} \setminus \left\{
    0 \right\}, \cdot)$ eine abelsche Gruppe mit neutralem Element
  \(1\).  Also, es gilt für alle \(x \in \mg{K}\) dass \(1 \cdot x = x\).
  Sei also \(1'\) ein anderes neutrales Element bzgl. Multiplikation.
  Dann gilt wegen der Kommutativität dass $1' \cdot 1 = 1 = 1 \cdot 1'
  = 1'$, also \(1 = 1'\).

  Der Ring \((R, +, \cdot)\) erfüllt dann die Definition von
  \(\mg{K}\)-Vektorraum.  Sei \(\alpha, \beta \in \mg{K}\) und
  \(\ib{v}, \ib{w} \in R\).  Es gilt insbesondere
  \(\alpha, \beta, \ib{v}, \ib{w} \in R\).
  \begin{itemize}
  \item Weil \((R, +, \cdot)\) ein Ring ist, ist die Gruppe \((R, +)\)
    abelsch.
  \item Weil \((R, +, \cdot)\) ein Ring ist, gilt
    \((\alpha+\beta)\ib{v}=\alpha\ib{v}+\beta\ib{v}\).
  \item Weil \((R, +, \cdot)\) ein Ring ist, gilt
    \(\alpha(\ib{v}+\ib{w})=\alpha\ib{v}+\alpha\ib{w}\).
  \item Weil \((R, +, \cdot)\) ein Ring ist, gilt
    \((\alpha\beta)\ib{v}=\alpha(\beta\ib{v})\).
  \item Wir haben oben bewiesen, dass \(1\) auch das Einelement von $(R
    \setminus \left\{ 0 \right\}, \cdot)$ ist.  Dann gilt
    \(1\ib{v}=\ib{v}\) für alle \(\ib{v} \in R\).
  \end{itemize}
\end{proof}

\textbf{Aufgabe 6.3.ii}

Diese Aussage ist falsch.  Wir widerlegen diese Aussage mit einem
Beispiel.

Sei \(\mg{K}:=\mg{R}\) und \(V:=\mg{R}\). Wegen (Bemerkung Seite 24: Jede
Körper \(\mg{K}\) ist ein \(\mg{K}\)-Vektorraum) ist \(V\) ein \(\mg{K}\)-Vektorraum.

Sei \(\lambda_1 := 0\), \(\lambda_2:=1\) und \(\ib{v}:=0\).  Dann gilt
\(\lambda_1 \neq \lambda_2\) und
\(\lambda_1 \ib{v} = \lambda_2 \ib{v} = 0\).

\textbf{Aufgabe 6.3.iii}

Diese Aussage ist falsch. Wir widerlegen diese Aussage mit einem Beispiel.

Sei \(\mg{K}:=\mg{R}\) und \(V:=\mg{R}\). Wegen (Bemerkung Seite 24: Jede
Körper \(\mg{K}\)) ist ein \(\mg{K}\)-Vektorraum ist \(V\) ein \(\mg{K}\)-Vektorraum.

Sei \(\lambda:=0\), \(\ib{v}:=0\) und \(\ib{w}:=1\).  Dann gilt $\ib{v} \neq
\ib{w}$ und \(\lambda\ib{v} = \lambda\ib{w} = 0\).

\textbf{Aufgabe 6.3.iii}

\textit{Behauptung.}  Die Aussage \(1+1=0\) und \(1+1+1+1=0\) ist
äquivalent.

\begin{proof}
  Wir beweisen die Hinrichtung und Rückrichtung als wahr.
  \begin{itemize}
  \item \(1+1=0 \implies 1+1+1+1=0\).

    Weil \((\mg{K},+)\) eine abelsche Gruppe ist, gilt
\begin{align*}
  1+1+1+1 &= (1+1) + (1+1) \tag*{Assoziativität}\\
          &= 0 + 0 \tag*{Voraussetzung}\\
          &= 0 \tag*{Existenz des neutralen Elements}
\end{align*}
\item \(1+1+1+1 = 0 \implies 1+1=0\)

    Weil \((\mg{K},+)\) eine abelsche Gruppe ist, gilt
\begin{align*}
1+1+1+1 &= (1+1) + (1+1)
\end{align*}
Es gilt auch, dass \(1+1+1+1=0\).  Daraus folgt, dass \((1+1)=-(1+1)\).
Daraus folgt, dass \(1+1=0\) muss.
  \end{itemize}
\end{proof}
\end{document}