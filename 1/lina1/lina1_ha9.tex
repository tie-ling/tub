%% page style
\documentclass[12pt]{extarticle}
\usepackage[margin=2cm]{geometry}
\usepackage{fancyhdr,parskip}
\pagestyle{fancy}
\usepackage[onehalfspacing]{setspace}
\setlength{\parindent}{0pt}
\lhead{\myAuthor}
\rhead{\mySubject \ \myHausaufgaben. Übungsblatt \\ \myTutor}
\renewcommand*\familydefault{\sfdefault} %% Only if the base font of the document is to be sans serif

%% language
\usepackage[utf8]{inputenc}
\usepackage{xcharter-otf}
\usepackage[ngerman]{babel}

%% default packages
\usepackage{amsmath,mathtools,fontspec,amsthm,amssymb,amsfonts,
  stmaryrd, % for the lightning symbol used in proof by contraction
  tikz,     % used to draw diagrams
}

%% metadata
\newcommand{\myAuthor}{Yuchen Guo 480788 | Meng Zhang 484981}
\newcommand{\myHausaufgaben}{9}
\newcommand{\mySubject}{LinA}
\newcommand{\myTutor}{Saskia}


%% custom commands
\newcommand{\mg}[1]{\mathbb{#1}}
\newcommand{\lin}{\operatorname{lin}}
\newcommand{\aufgn}[1]{\textbf{Aufgabe #1.}}
\newcommand{\beh}{\textit{Behauptung.}\ }

\begin{document}
\aufgn{9.1}

Bestimmen die Basen von \(U, W, U+W\) sowie eine Basis von \(\mg{R}^4\),
die eine Basis von \(U \cap W\) enthält, mit
\begin{align*}
  U=
  \lin \left( \left\{
    u_1, u_2, u_3
  \right\} \right), \quad
    W=
  \lin \left( \left\{
    w_1, w_2, w_3
   \right\} \right).
\end{align*}

\beh Eine Basis von \(U\) ist \(\left\{ u_1, u_2 \right\}\) mit Dimension
\(2\) und eine Basis von \(W\) ist \(\left\{ w_1, w_2, w_3 \right\}\) mit
Dimension \(3\).

\begin{proof}
  Zuerst bemerken wir, dass \(U\) endlich erzeugt ist.  Wegen
  Basis-Auswahlsatz müssen wir nur Vektoren von dem Erzeugendensystem
  weglassen, bis das Erzeugendensystem linear unabhängig ist.  Es gilt
  \(-1 \cdot u_1 + -1 \cdot u_2 + 1 \cdot u_3 = 0\).  Also \(u_3\) ist
  eine Linearkombination von \(u_1\) und \(u_2\).  Andereseits lösen wir
  die Gleichung \(\lambda_1 u_1 + \lambda_2 u_2 = 0\) erhalten wir
  \(\lambda_1 = \lambda_2 = 0\).  Das Erzeugendensystem
  \(\left\{ u_1, u_2 \right\}\) ist linear unabhängig und ist deshalb
  eine Basis von \(U\).

Das Erzeugendensystem \(\left\{ w_1, w_2, w_3 \right\}\) ist eine Basis
der Unterraum \(W\).  Denn, aus $\lambda_1 w_1 + \lambda_2w_2 +
\lambda_3w_3=0$ folgt
\begin{align*}
  3 \lambda_1 = 0 \\
  1 \lambda_1 + 3 \lambda_2 = 0 \\
  3 \lambda_1 + 3 \lambda_2 + 1 \lambda_3 = 0
\end{align*}
also \(\lambda_1 = \lambda_2 = \lambda_3 = 0\).  Das Erzeugendensystem
ist daher linear unabhängig und ist eine Basis von dem Unterraum \(W\).
\end{proof}

\beh Eine Basis von \(U + W\) ist $\left\{ u_1, u_2, w_1, w_2, w_3
\right\}$ mit Dimension \(5\).

\begin{proof}
Es gilt
\begin{align*}
U + W = \left\{ v \in V \mid \text{es gibt } u \in U \text{ und } w
  \in W \text{ mit } v = u + w \right\}.
\end{align*}

Wegen \(u = \alpha_1 u_1 + \alpha_2 u_2\) und
\(w = \beta_1 w_1 + \beta_2 w_2 + \beta_3 w_3\) für
\(\alpha_1, \alpha_2, \beta_1, \beta_2, \beta_3 \in \mg{K}\) gilt für
alle \(v \in U + W\) dass
\begin{align*}
v = (\alpha_1 u_1 + \alpha_2 u_2) + (\beta_1 w_1 + \beta_2 w_2 +
\beta_3 w_3).
\end{align*}
Also $v \in \lin \left(  \left\{ u_1, u_2, w_1, w_2, w_3 \right\}
\right)\(.  Es ist ausreichend, zu zeigen, dass \)\left\{ u_1, u_2, w_1,
  w_2, w_3 \right\}$ linear unabhängig ist.  Diese ist tatsächlich
linear unabhängig, denn aus der Koeffizientenmatrix erhalten wir diese
Matrix in Zeilenstufenform:
\begin{align*}
\begin{pmatrix}
  1 & 0 & 3 & 0 & 0\\
  1 & -2 & 1 & 3 & 0 \\
  2 & 1 & 7 & 2 & 0\\
  1 & 0 & 3 & 2 & 1
\end{pmatrix} \to
\begin{pmatrix}
  1 & 0 & 3 & 0 & 0\\
  0 & 1 & 1 & 0 & -1\\
  0 & 0 & 0 & 2 & 0\\
  0 & 0 & 0 & 0 & 0,5
\end{pmatrix}.
\end{align*}

Weil die Zeilenstufenform wie die original Form eine
\(4 \times 5\)-Matrix ist, ist
\(\left\{ u_1, u_2, w_1, w_2, w_3 \right\}\) linear unabhängig.
\end{proof}

\beh (Unklar???) Die Dimension von \(U \cap W\) ist \(0\).

\begin{proof}
  Diese folgt aus der Dimensionsformel.

  Für endlich dimensionale
  Untervektorräume \(U, W \subset V\) gilt
\begin{align*}
\dim(U + W) = \dim U + \dim W - \dim \left( U \cap W \right).
\end{align*}
Daraus folgt \(\dim \left( U \cap W \right)=0\).  Eine Basis von
\(\mg{R}^4\), die eine Basis von \(U \cap W\) enthält, wäre die
Standardbasis
\begin{align*}
\left\{ \left( 1, 0, 0, 0 \right), \left( 0, 1, 0, 0 \right), \left(
  0, 0, 1, 0 \right), \left( 0, 0, 0, 1 \right) \right\}.
\end{align*}
\end{proof}

\aufgn{9.2.i}

Sei \(f \colon V \to V\) eine lineare Abbildung, sowie \(v \in V\), sodass
für eine natürliche Zahl \(n\) gilt
\begin{align*}
f^n(v) \ne 0 \quad \text{und} \quad f^{n+1}(v) = 0
\end{align*}

\beh Sei das Unterraum
$U := \lin \left( \left\{ f^k(v) \colon k \in
    \mg{N}\right\} \right)$.  Dann gilt
\( 1 \leq \dim U \leq n\).

\begin{proof}
Es gilt \(f^k(v) = 0\) für alle \(k \geq n + 1\).  Denn, es
gilt
\begin{align*}
  f^{n+2}(v) &= f \left( f^{n+1}(v) \right) \\
             &= f (0) \\
             &= f(0 \cdot 0)\\
             &= 0 \cdot f(0)\\
             &= 0.
\end{align*}

Es gilt \(\dim U \geq 1\), denn es gibt mindestens einen
von Null verschieden Vektor in dem Erzeugendensystem
\(\left\{ f^k(v) \colon k \in \mg{N}\right\}\), nämlich
\(f^n(v) \ne 0\).

Es gibt wegen  \(f^k(v) = 0\) für alle \(k \geq n + 1\)
höchstens \(n\) verschiedene, lineare unabhängige
Vektoren \(f^k(v)\) mit \(k \in \mg{N}_{\leq n}\), die das
Erzeugendensystem bilden.
\end{proof}

\aufgn{9.2.ii}

Was bedeutet \(n\) in \(\dim(\ker(f))=n-1\)?

\aufgn{9.3}

Gegeben sei der Unterraum \(U = \lin M\) von \(\mg{R}[t]\).

\aufgn{9.3.1}

\beh Eine Basis \(B \subseteq M\) von \(U\) ist $\left\{
  t^3 - 1, 2t^3 - t, t^3 - t^2 \right\}$
und es gilt \(\dim_{\mg{R}}(U) = 3\).

\begin{proof}
  Aus der Koeffizientenmatrix erhalten wir diese Matrix
  in Zeilenstufenform:
\begin{align*}
\begin{pmatrix}
  -1 & 0 & 0 & 1 \\
  0 & 0 & -1 & 1 \\
  1 & -1 & 0 & 1 \\
  0 & -1 & -1 & 3 \\
  0 & 1 & -1 & -1
\end{pmatrix} \to
\begin{pmatrix}
  -1 & 0 & 0 & 1\\
  0 & -1 & 0 & 2\\
  0 & 0 & -1 & 1\\
  0 & 0 & 0 & 0 \\
  0 & 0 & 0 & 0
\end{pmatrix}.
\end{align*}
Daraus folgt, dass eine Basis $B := \left\{
  t^3 - 1, 2t^3 - t, t^3 - t^2 \right\}$ ist
und es gilt \(\dim_{\mg{R}}(U) = 3\).
\end{proof}

\aufgn{9.3.2}

Es gilt
\begin{align*}
p := 2t^3 - t^2 - t + 1 = (2t^3 - t) + -(t^3 -
  1) + (t^3 - t^2)
\end{align*}
Wegen Austauschlemma können wir die Polynome
\(\mathbf{p}\) gegen genau einen beliebigen Vektor in \(B\)
tauschen.

Andereseit  gilt \(q \notin \lin B\),  daraus folgt, dass
\begin{align*}
\lin \left( (B \setminus \left\{ v \right\}) \cup
  \left\{ q \right\}  \right) \ne \lin B
\end{align*}
für alle \(v \in B\).  Daraus folgt, dass keinen Vektor
in \(B\) gegen \(q\) getauscht werden kann.

\aufgn{9.4}

Zu zeigen: \(U\) ist ein Unterraum von \(\mg{R}[t]\) mit
\begin{align*}
U = \left\{ p \in \mg{R}[t] \colon p(-1) = p(0) = p(1)
  = 0 \right\}.
\end{align*}

\begin{proof}
Wir zeigen, dass \(U\) ein Unterraum von \(\mg{R}[t]\) ist,
indem wir zeigen, dass die drei
Untervektorraumkriterien erfüllt sind.

\begin{itemize}
\item \(U \ne \emptyset\),

  Sei \(p := 0\).  Dann gilt \(p \in U\).  Daraus folgt, $U
  \ne \emptyset$.
\item \(v, w \in U \implies v + w \in U\),

  Sei \(v, w \in U\) beliebig gewählt.  Dann gilt
\begin{align*}
  v(-1) = v(0) = v(1) &= 0\\
  w(-1) = w(0) = w(1) &= 0\\
  (v+w)(-1) = (v+w)(0) = (v+w)(1) &= 0.
\end{align*}
Also \(v+w \in U\).
\item $v \in U, \lambda \in \mg{K} \implies \lambda v
  \in U$.

  Sei \(v \in U\) und \(\lambda \in \mg{K}\) beliebig
  gewählt.  Dann gilt wegen der Definition von \(U\) dass
\begin{align*}
  v(-1) = v(0) = v(1) &= 0\\
  (\lambda v)(-1) = (\lambda v)(0) = (\lambda v)(1) &= 0.
\end{align*}
Also \(\lambda v \in U\).
\end{itemize}
\end{proof}

\beh Die Dimension von dem Quotientenvektorraum
\(\mg{R}[t]/U\) ist sieben.

\begin{proof}
  (Unklar???)
Es gilt
\begin{align*}
  \mg{R}[t] / U &= \left\{ v + U \mid v \in (\mg{R}[t]
                  \setminus U) \right\} \\
                &= \left\{ v + U \mid v \in \mg{R}[t]
                  \wedge (v(-1) \ne 0 \lor v(0) \ne 0
                  \lor v(1) \ne 0) \right\}
\end{align*}
Seien die Aussagen
\begin{align*}
  a &:= v(-1) \ne 0 \\
  b &:= v(0) \ne 0 \\
  c &:= v(1) \ne 0
\end{align*}
Dann gibt es insgesamt sieben Möglichkeiten, nämlich,
\begin{itemize}
\item a wahr, bc nicht wahr
\item b wahr, ac nicht wahr
\item c wahr, ab nicht wahr
\item ab wahr, c nicht wahr
\item ac wahr, b nicht wahr
\item bc wahr, a nicht wahr
\item abc wahr
\end{itemize}

Eine Basis von \(\mg{R}[t]/U\) ist dann
\begin{align*}
\left\{ v + U \mid v \in \mg{R}[t] \wedge v \text{
  erfüllt einer der sieben Möglichkeiten} \right\}.
\end{align*}
Die Dimension ist deshalb sieben.
\end{proof}

\aufgn{9.5.i}

\beh Sei \(v \in \mg{R}^3\) beliebig gewählt.  Dann ist
\(U := \lin (v)\) eine \(1\)-dimensionalen Unterraum des
\(\mg{R}^3\).

\begin{proof}
Wir zeigen, dass \(U\) ein Unterraum von \(\mg{R}^3\) ist,
indem wir zeigen, dass die drei
Untervektorraumkriterien erfüllt sind.

\begin{itemize}
\item \(U \ne \emptyset\),

  Es gilt \(v \in U\).  Daraus folgt, $U
  \ne \emptyset$.
\item \(v, w \in U \implies v + w \in U\),

  Diese folgt aus der Definition von linearen Hülle.
\item $v \in U, \lambda \in \mg{K} \implies \lambda v
  \in U$.

  Diese folgt aus der Definition von linearen Hülle.
\end{itemize}

Außerdem gilt \(\dim U = 1\), denn \(U\) ist von \(v\) erzeugt.
\end{proof}


\aufgn{9.5.ii}

Was bedeutet \(H(a,0)\)?

\aufgn{9.6.i}

\beh \(U_f\) ist ein Untervektorraum von \(V\).
\begin{proof}
  Wir zeigen, dass \(U_f\) ein Unterraum von \(V\) ist,
  indem wir zeigen, dass die drei
  Untervektorraumkriterien erfüllt sind.

\begin{itemize}
\item \(U_f \ne \emptyset\),

Weil \(f\) eine lineare Abbildung ist, gilt insbesondere
\begin{align*}
f(0) = f(0v) = 0f(v) = 0
\end{align*}
also \(f(0)=0\) und damit \(0 \in U_f\) und
\(U_f \ne \emptyset\).
\item \(v, w \in U_f \implies v + w \in U_f\),

  Sei \(v, w \in U_f\) beliebig gewählt.  Dann gilt
  \(f(v) = v\) und \(f(w) = w\).  Weil \(f\) eine lineare
  Abbildung ist, gilt \(v+w = f(v) + f(w) = f(v + w)\).
  Daraus folgt, \(v + w \in U_f\).
\item $v \in U_f, \lambda \in \mg{K} \implies \lambda v
  \in U_f$.

  Sei \(v \in U_f\) beliebig gewählt.  Dann gilt
  \(f(v) = v\).  Weil \(f\) eine lineare Abbildung ist,
  gilt \(\lambda v = \lambda f(v) = f(\lambda v)\).
  Daraus folgt, dass \(\lambda v \in U_f\).
\end{itemize}
\end{proof}

\aufgn{9.6.ii}

\beh Eine Basis von \(U_f\) ist
\(\left\{ \left( 0, 1, -1 \right) \right\}\).

\begin{proof}
  Sei \(v \in V\) beliebig gewählt.  Wegen
  \(V := \mg{R}^3\) und \(f\colon \mg{R}^3 \to \mg{R}^3\)
  linear gilt
\begin{align*}
  v &= \alpha e_1 + \beta e_2 + \gamma  e_3 \\
  f(v) &= \alpha f(e_1) + \beta f(e_2) + \gamma f(e_3)
\end{align*} für \(\alpha, \beta, \gamma \in \mg{K}\).

Sei \(v = f(v)\), dann gilt
\begin{align*}
  \alpha &= \alpha + 2 \beta + 2 \gamma, \\
  \beta &= \beta, \\
  \gamma &= 3 \alpha + \gamma.
\end{align*}

Daraus folgt \(\alpha = 0\), \(\beta = \beta\) und
\(\gamma = - \beta\).  Das heißt,
\(U_f = \lin (\left\{ (0, 1, -1) \right\})\) und eine
Basis von \(U_f\) ist \(\left\{ (0, 1, -1) \right\}\).
\end{proof}

\aufgn{9.7}

\textbf{Definition linearer Abbildung.}  Sind \(V, W\)
Vektorräume über einem Körper \(K\), so heißt eine
Abbildung \(F \colon V \to W\) \textit{linear}, oder
genauer \textit{\(K\)-linear}, wenn sie folgende
Eigenschaften hat:

\begin{itemize}
\item Für \(v, w \in V\) gilt \(F(v + w) = F(v) + F(w)\).
\item Für \(v \in V\) und \(\lambda \in K\) gilt $F(\lambda
  v) = \lambda F(v)$.
\end{itemize}

\aufgn{9.7.i}

\beh \(\Phi_1\) ist eine lineare Abbildung.

\begin{proof}
Wir zeigen, dass \(\Phi_1\) eine lineare Abbildung ist,
indem wir zeigen, dass die Definition einer linearen
Abbildung erfüllt ist.

  Seien \(v:=(a, b, c), w:=(x, y, z) \in \mg{Z}^3_p\) beliebig
  gewählt.  Dann gilt $a, b, c, x, y, z \in \mg{Z}_p =
  \left\{ [0]_p, \cdots, [p-1]_p \right\}$.

  Es gilt dann \(\Phi_1(v+w)=\Phi_1(v)+\Phi_1(w)\) und
  \(\lambda \Phi_1(v) = \Phi_1(\lambda v)\).   ???
\end{proof}

\aufgn{9.7.ii}

\beh \(\Phi_2\) ist keine lineare Abbildung.

\begin{proof}
  Seien \(v:=(a, b, c), w:=(x, y, z) \in \mg{Z}^3_p\) beliebig
  gewählt.  Dann gilt $a, b, c, x, y, z \in \mg{Z}_p =
  \left\{ [0]_p, \cdots, [p-1]_p \right\}$.

  Es gilt
\begin{align*}
  \Phi_2(v+w) &= (3(a+x)(b+y), a+x, c + z) \\
              &\ne (3xy+3ab, a+x, c + z)\\
              &= \Phi_2(v) + \Phi_2(w).
\end{align*}  Damit ist \(\Phi_2\) keine lineare Abbildung.
\end{proof}
\aufgn{9.7.iii}

\beh \(\Phi_{3}\) ist keine lineare Abbildung.

\begin{proof}
  Seien \(v:=(a, b, c), w:=(x, y, z) \in \mg{Z}^3_p\) beliebig
  gewählt.  Dann gilt $a, b, c, x, y, z \in \mg{Z}_p =
  \left\{ [0]_p, \cdots, [p-1]_p \right\}$.

  Es gilt
\begin{align*}
  \Phi_3(v+w) &= ((a+x)^2, a+x+b+y, -2c-2z)\\
              &\ne (a^2+x^2, a+x+b+y, -2c-2z)\\
              &= \Phi_3(v) + \Phi_3(w).
\end{align*}  Damit ist \(\Phi_3\) keine lineare Abbildung.
\end{proof}

\aufgn{9.8}

\beh Es gilt \(\ker (f) = \mg{R} \cdot (2, -1, 1) \).

\begin{proof}
  Sei \(v \in V\) beliebig gewählt.  Wegen
  \(V := \mg{R}^3\) und \(f\colon \mg{R}^3 \to \mg{R}[t]\)
  linear gilt
\begin{align*}
  v &= \alpha e_1 + \beta e_2 + \gamma  e_3 \\
  f(v) &= \alpha f(e_1) + \beta f(e_2) + \gamma f(e_3)
\end{align*} für \(\alpha, \beta, \gamma \in \mg{K}\).

Sei \(f(v) = 0\), dann gilt
\begin{align*}
  \beta + \gamma &= 0\\
  2\alpha - 4\gamma &= 0\\
  -\alpha - 2 \beta &= 0\\
  \alpha - 2\gamma &= 0
\end{align*}

Daraus folgt \(\alpha = -2 \beta = 2\gamma\).  D.h., alle
Vektoren mit dieser Eigenschaft ist Teil der Menge
\(\ker(f)\).
\end{proof}

\beh Eine Basis von \(\operatorname{Bild}(f)\) ist
\(\left\{ 1-t+2t^2 , -2t + t^3 \right\}\) und die
Dimension ist \(2\).

\begin{proof}
  Sei \(v \in \mg{R}^3\) beliebig gewählt.  Wegen
\begin{align*}
v := \alpha e_1 + \beta e_2 + \gamma e_3
\end{align*} und \(f\) linear gilt für alle \(f(v)\) dass
\begin{align*}
  f(v) &= \alpha f(e_1) + \beta f(e_2) + \gamma f(e_3)
\end{align*} mit \(\alpha, \beta, \gamma \in \mg{K}\).

Also \(\operatorname{Bild}(f)\) ist endlich erzeugt und
es gilt
\begin{align*}
  \lin (\left\{ f(e_1), f(e_2), f(e_3) \right\} )
  = \operatorname{Bild}(f).
\end{align*}
Wegen Basis-Auswahlsatz müssen wir nur Vektoren von dem
Erzeugendensystem weglassen, bis das Erzeugendensystem
linear unabhängig ist.

  Aus der Koeffizientenmatrix erhalten wir diese Matrix
  in Zeilenstufenform:
\begin{align*}
\begin{pmatrix}
  1 & -1 & 2 & 0 \\
  0 & -2 & 0 & 1 \\
  -2 & 0 & -4 & 1
\end{pmatrix} \to
\begin{pmatrix}
  1 & -1 & 2 & 0 \\
  0 & -2 & 0 & 1 \\
  0 & 0 & 0 & 0
\end{pmatrix}.
\end{align*}
Also eine Basis ist
\begin{align*}
\left\{ 1-t+2t^2 , -2t + t^3 \right\}
\end{align*}
und die Dimension ist \(2\).
\end{proof}
\end{document}