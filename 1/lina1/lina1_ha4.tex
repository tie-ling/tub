\documentclass[12pt]{extarticle}
\usepackage{amsmath,mathtools,fontspec,amsthm,amssymb,amsfonts,fancyhdr,color,graphicx,}
\usepackage[margin=2.5cm]{geometry}
\usepackage[utf8]{inputenc}
\usepackage{xcharter-otf}
\usepackage[ngerman]{babel}
\usepackage[onehalfspacing]{setspace}
\usepackage{tikz}
\renewcommand{\familydefault}{\sfdefault}
\pagestyle{fancy}
\setlength{\parindent}{0pt}
\lhead{Yuchen Guo 480788, Meng Zhang 484981}
\rhead{4. Hausaufgabenblatt, LinA I\\Tutorin: Saskia, Mi 2-4 p.m.}
\newcommand{\idf}{\operatorname{id}}
\begin{document}
\textbf{H 4.1.i}

Wir zeigen, dass \(R\) eine Äquivalenzrelation ist, indem wir zeigen,
dass diese Relation reflexiv, symmetrisch und transitiv ist.

\begin{proof}

Zur Abkürzung setzen wir die Mengen \(M := \left\{ 1,\ldots,m \right\}\)
und \(N := \left\{ 1,\ldots,n \right\}\) ein.
\begin{itemize}
\item Diese Relation ist reflexiv.

  Es existiert eine Permutation der Zahlen \(1, \ldots, m\), nämlich
  \(\idf_M\).  Es gilt für alle \(f \in X\) dass \(f = \idf_M \circ f\).
  Daraus folgt, dass \((f,f) \in R\) für alle \(f \in X\).  Diese Relation
  ist daher reflexiv.
\item Diese Relation ist symmetrisch.

  Sei \((f,g) \in R\).  Daraus folgt, dass es eine Permutation
  \(\sigma \in S_m\) gibt mit \(f=\sigma \circ g\).  Aus der Definition
  der Permutation folgt, dass \(\sigma\) bijektiv ist mit
  \(\sigma: M \rightarrow M\).  Wegen der Bijektivität von \(\sigma\)
  existiert eine Umkehrabbildung von \(\sigma\).  Wir bezeichnen diese
  Abbildung als \(\sigma^{-1}\). Die Abbildung \(\sigma^{-1}\) ist eine
  bijektive Abbildung mit \(\sigma^{-1}: M \rightarrow M\).  D.h.,
  \(\sigma^{-1} \in S_m\).

  Es gilt für alle \(n \in N\) dass \(f(n)=\sigma(g(n))\).  Daraus folgt,
  dass \(\sigma^{-1}(f(n))=g(n)\) und \((g,f) \in R\).  Damit ist diese
  Abbildung symmetrisch.
\item Diese Relation ist transitiv.

  Seien \((f,g),(g,h) \in R\).  Aus der Definition der Relation \(R\)
  folgt, dass es \(\sigma_1,\sigma_2 \in S_m\) existiert mit $f=\sigma_1
  \circ g\( und \(g = \sigma_2 \circ h\).  Daraus folgt, dass für alle \)n
  \in N$ gilt \(f(n)=\sigma_1(g(n))\) und \(g(n)=\sigma_2(h(n))\) und
  daher \(f(n)=\sigma_1(\sigma_2(h(n)))\).

  Es existiert daher eine Permutation der Menge \(M\), nämlich
  \(\sigma = \sigma_1 \circ \sigma_2\) mit \(f = \sigma \circ h\).  Damit
  ist diese Relation transitiv.
\end{itemize}
\end{proof}

\textbf{H 4.1.ii}

\textit{Behauptung.}  Es gilt \(m=n\) und \(f \in X\) bijektiv.  Die
Aussage \((f,g) \in R\) ist äquivalent zu der Aussage „\(g\) ist
bijektiv“.
\begin{proof}
  Wir setzen zur Abkürzung die Menge \(M:= \left\{ 1,\ldots,m \right\}\) ein.

\vspace{4mm}
\textit{Hinrichtung.}
  \begin{itemize}
  \item \(g\) ist injektiv

    Seien \(x_1, x_2 \in M\) mit \(x_1 \neq x_2\).  Aus der Bijektivität
    von \(f\) folgt, \(f(x_1) \neq f(x_2)\).  Die Abbildung \(\sigma\) ist
    eine Permutation und besitzt daher eine bijektive Umkehrabbildung
    \(\sigma^{-1}\).

    Weil \(\sigma^{-1}\) bijektiv ist, gilt
    \(\sigma^{-1}(f(x_1)) \neq \sigma^{-1}(f(x_2))\).  Es gilt auch,
    \(g=\sigma^{-1} \circ f\).  Es gilt daher \(g(x_1) \neq g(x_2)\) für
    alle \(x_1 \neq x_2\).  Damit ist \(g\) injektiv.

  \item \(g\) ist surjektiv

    Wir bemühen uns einen Widerspruchsbeweis.

    Sei \(g\) nicht surjektiv.  Dann existiert ein \(y \in M\) sodass
    \(g(x) \neq y\) für alle \(x \in M\).  Weil \(\sigma\) bijektiv ist,
    dann gilt \(f(x)=(\sigma \circ g)(x) \neq \sigma(y)\) für alle
    \(x \in M\).

    Es gilt \(\sigma: M \rightarrow M\) und \(y \in M\).  Daraus folgt,
    dass \(\sigma(y) \in M\).  Daraus folgt, dass \(f(x) \neq \sigma(y)\)
    für alle \(x \in M\) mit \(\sigma(y) \in M\).  Die Abbildung \(f\) ist
    damit im Widerspruch zur Voraussetzung nicht surjektiv.

    Daraus folgt, dass \(g\) surjektiv sein muss.
  \end{itemize}
\textit{Rückrichtung.}

  Seien \(f, g \in X\) bijektive Abbildungen.  Weil \(f\) und \(g\) bijektiv
  sind, existiert die Umkehrabbildungen \(f^{-1}\) und \(g^{-1}\).

  Es gilt \(f^{-1} \circ f = g^{-1} \circ g = \idf_M\).  Dann existiert
  eine Permutation der Menge \(M\), nämlich \(\sigma = f \circ g^{-1}\)
  mit \(f = \sigma \circ g\).
\end{proof}

\textbf{H 4.2.i}

Sei \(m, n \in \mathbb{N}\) mit \(m, n \geq 1\).
\begin{align*}
F: \mathbb{Z}_m\rightarrow\mathbb{Z}_n, \quad [x]_m\mapsto[x]_n
\end{align*}

\textit{Behauptung.}  Eine Abbildung ist durch \(F\) definiert, wenn \(m\)
durch \(n\) teilbar ist.

\begin{proof}
  Aus der Definition der Quotientenmenge folgt, dass $\mathbb{Z}_m= \left\{
    [0]_m, \ldots, [m-1]_m \right\}\( und \)\mathbb{Z}_n= \left\{
    [0]_n, \ldots, [n-1]_n \right\}$.

  Sei \([x]_m \in \mathbb{Z}_m\) beliebig gewählt und \(m\) durch \(n\)
  teilbar.  Wir zeigen, dass es existiert genau eine Restklasse,
  nämlich \([x]_n\), als Abbildung von \([x]_m\).

  Es gilt, \(n|m\).  Daraus folgt, dass \(m = kn\) mit \(k \in \mathbb{N}\).
  Es gilt dann,
\begin{align*}
  \mathbb{Z}_m&= \left\{ [0]_m, \ldots, [m-1]_m \right\}\\
              &= \left\{[0]_m, \ldots, [n-1]_m, \right\} \cup \ldots
                \cup \left\{ [(k-1)n]_m, \ldots, [kn-1]_m \right\}\\
  &= \bigcup_{i=0}^{(k-1)}{\bigcup_{j=in}^{(i+1)n-1}{[j]_m}}.
\end{align*}

Sei \(p,q \in \mathbb{N}\) mit \(0 \leq p \leq k-1\) und
\(0 \leq q \leq n-1\). Es gilt \([pn+q]_m = [q]_n\), denn für alle
\(x \in [pn+q]_m\) gilt
\(x = pn+q+m\mathbb{Z}=pn+q+kn\mathbb{Z}=q+n\mathbb{Z}'\).

D.h., für alle \([pn+q]_m \in \mathbb{Z}_m\) existiert genau eine
\([q]_n \in \mathbb{Z}_n\) als Abbildung, wenn \(n|m\) gilt.
\end{proof}

Die Relation \(F\) definiert keine Abbildung, wenn \(n|m\) nicht gilt.
Ein Gegenbeispiel wäre \(m=3\) und \(n=2\).  Zur Restklasse
\([2]_3= \left\{ 2, 5, \ldots, 2+3\mathbb{Z}\right\}\) existiert keine
Abbildung in \(\mathbb{Z}_2= \left\{ [0]_2, [1]_2 \right\}\).

\vspace{4mm}

\textbf{H 4.2.ii}

\textit{Behauptung.}  Wenn \(n|m\) gilt, ist die Relation \(F\) eine
surjektive Abbildung.  Wenn zusätzlich \(m=n\) gilt, ist \(F\) ist eine
bijektive Abbildung.


\begin{proof}
    Die Relation \(F\) definiert zuerst eine Abbildung, siehe (H 4.2.i).
  \begin{itemize}
\item Falls \(m \neq n\).
  Sei \(p,q,k \in \mathbb{N}\) mit \(m=kn\), \(0 \leq p \leq k-1\) und
  \(0 \leq q \leq n-1\).

  Es gilt \([pn+q]_m = [q]_n\), denn für alle \(x \in [pn+q]_m\) gilt
  \(x = pn+q+m\mathbb{Z}=pn+q+kn\mathbb{Z}=q+n\mathbb{Z}'\). Es gilt
  auch, \(k \geq 2\) wegen \(m \neq n\).

  Daraus folgt, \([q]_n\) hat mindestens zwei Urbild in \(\mathbb{Z}_m\),
  nämlich \([pn+q]_m\) für alle \(0 \leq p \leq k-1\).

\item Falls \(m = n\).

  Dann gilt \([x]_m=[x]_n\) für alle \([x]_m \in \mathbb{Z}_m\) und
  \([x]_n=[x]_m\) für alle \([x]_n \in \mathbb{Z}_n\).  Die Abbildung \(F\)
  ist dann bijektiv.
  \end{itemize}
\end{proof}

\textbf{H 4.3.i}

Diese \((G, \star)\) erfüllt den Assoziativgesetz.  Denn für alle $a, b,
c \in \mathbb{N}\( gilt \)\min \left\{  \min \left\{ a,b \right\},
  c\right\} = \min \left\{ a, \min \left\{ b, c \right\}\right\} =
\min \left\{ a,b,c \right\}$.

\vspace{4mm}

Es existiert jedoch kein neutrales Element
\(n \in \mathbb{N}\) mit \(a=\min \left\{ a,n \right\}\) für alle
\(a \in \mathbb{N}\).  Denn, für alle \(n \in \mathbb{N}\) gilt auch $n+1
\in \mathbb{N}$, weil \(\mathbb{N}\) induktiv ist.  Aus der
Anordnungsaxiomen folgt unmittelbar \(n=\min \left\{ n, n+1 \right\}\)
im Widerspruch zur Voraussetzung, dass \(a= \min \left\{ a,n \right\}\)
für alle \(a \in \mathbb{N}\).

\vspace{4mm}

Weil es kein neutrales Element \(n\) gibt, gibt es auch kein inverses
Element \(a'\) zum \(a\) mit \(a' \star a = n\).

\vspace{4mm}

\textbf{H 4.3.ii}

Diese \((G, \star)\) erfüllt den Assoziativgesetz.  Denn für alle
\(a, b, c \in \mathbb{R}\) gilt
$\sqrt{\left(\sqrt{a^2+b^2}
  \right)^2+c^2}=\sqrt{a^2+\left(\sqrt{b^2+c^2}
  \right)^2}=\sqrt{a^2+b^2+c^2}$.

\vspace{4mm}

Es existiert jedoch kein neutrales Element \(n \in \mathbb{R}\) mit
\(a=n \star a\) für alle \(a \in \mathbb{R}\).  Denn, für alle
\(a \in \mathbb{R}_{<0}\) gilt \(n \star a \geq 0\) für alle
\(n \in \mathbb{R}\).  Insbesondere gilt \(n \star a \neq a\) für alle
\(n \in \mathbb{R}\).

\vspace{4mm}

Weil es kein neutrales Element \(n\) gibt, gibt es auch kein inverses
Element \(a'\) zum \(a\) mit \(a' \star a = n\).

\vspace{4mm}
\textbf{H 4.3.iii}

Diese \((G, \star)\) erfüllt nicht den Assoziativgesetz.  Denn, für alle
\(A, B, C \in \mathcal{P}(X)\) gilt,
\begin{align*}
  (A \setminus B) \setminus C
  &= (A \cap \overline{B}) \cap \overline{C}\\
  &= A \cap \overline{B} \cap \overline{C}\\
  &= \left\{ x|x \in A \wedge x \notin B \wedge x \notin C \right\}\\
  \\
  A \setminus (B \setminus C)
  &= A \cap \overline{(B \cap \overline{C})}\\
  &= A \cap (\overline{B} \cup C)\\
  &= (A \cap \overline{B}) \cup (A \cap C)\\
  &= \left\{ x | (x \in A \wedge x \notin B) \lor (x \in A \wedge x
    \in C) \right\}.
\end{align*}

Daraus folgt, dass \((A \star B) \star C \neq A \star (B \star C)\).

\vspace{4mm}

Es existiert kein neutrales Element.  Denn, für alle $N, A \in
\mathcal{P}(X)$ gilt \(N \setminus A \neq A\).
Weil es kein neutrales Element \(N\) gibt, gibt es auch kein inverses
Element \(A'\) zum \(A\) mit \(A' \star A = N\).

\vspace{4mm}

\textbf{H 4.4.i}

Wir zeigen, dass \(F\) eine Abbildung definiert, indem wir zeigen, dass
für alle $x=([(p_1,q_1)]_{R_{\mathbb{Q}}},[(p_2,q_2)]_{R_{\mathbb{Q}}})
\in \mathbb{Z}_{\mathbb{Q}} \times \mathbb{Z}_{\mathbb{Q}}$
existiert genau ein \(y=[(p_1q_2+p_2q_1,q_1q_2)]_{R_{\mathbb{Q}}}\) mit
\(x \mapsto y\).

\vspace{4mm}

Zuerst zeigen wir, dass eine Abbildung existiert.  Es gilt
\(p_1, p_2 \in \mathbb{Z}\) und
\(q_1, q_2 \in \mathbb{Z} \setminus \left\{ 0 \right\}\).  Daraus folgt,
dass \(p_1q_2+p_2q_1 \in \mathbb{Z}\) und
\(q_1q_2 \in \mathbb{Z} \setminus \left\{ 0 \right\}\).  Damit gilt
$(p_1q_2+p_2q_1,q_1q_2) \in \mathbb{Z} \times \mathbb{Z} \setminus
\left\{ 0 \right\}$ und liegt in der Definitionsbereich von
\(R_{\mathbb{Q}}\).  Deshalb existiert die Äquivalenzklasse
\([(p_1q_2+p_2q_1,q_1q_2)]_{R_{\mathbb{Q}}}\).

\vspace{4mm}

Danach zeigen wir, dass die Abbildung eindeutig bestimmt ist.  Wegen
(Skript Satz 2.16.iii:  für zwei Äquivalenzklassen \([x]_R\), \([y]_R\)
gilt entweder \([x]_R=[y]_R\) oder \([x]_R \cap [y]_R = \emptyset\)), gilt
für alle \(x \mapsto y\) dass $y =
[(p_1q_2+p_2q_1,q_1q_2)]_{R_{\mathbb{Q}}}$.  Damit ist die Abbildung
eindeutig bestimmt.

\vspace{4mm}

\textbf{H 4.4.ii}

Wir zeigen, dass die Verknüpfung \((\mathbb{Z}_{\mathbb{Q}}, F)\) die
folgenden Eigenschaften besitzt.

\begin{itemize}
\item Es gilt \((aFb)Fc=aF(bFc)\) für alle
  \(a,b,c \in \mathbb{Z}_{\mathbb{Q}}\).
  \begin{proof}
    Seien $a = [(p_a,q_a)]_{R_{\mathbb{Q}}}, b =
    [(p_b,q_b)]_{R_{\mathbb{Q}}}, c = [(p_c,q_c)]_{R_{\mathbb{Q}}} \in
    \mathbb{Z}_{\mathbb{Q}}$ beliebig gewählt.

    Es gilt,
\begin{align*}
  (aFb)Fc
  &=
     [(p_aq_b+p_bq_a,q_aq_b)]_{R_{\mathbb{Q}}}F[(p_c,q_c)]_{R_{\mathbb{Q}}}\\
  &= [((p_aq_b+p_bq_a)q_c+p_cq_aq_b,q_aq_bq_c)]_{R_{\mathbb{Q}}}\\
  &= [(p_aq_bq_c+p_bq_aq_c+p_cq_aq_b,q_aq_bq_c)]_{R_{\mathbb{Q}}}\\
  \\
  aF(bFc)
  &=[(p_a,q_a)]_{R_{\mathbb{Q}}}F[(p_bq_c+p_cq_b,q_bq_c)]_{R_{\mathbb{Q}}}\\
  &= [(p_aq_bq_c+q_a(p_bq_c+p_cq_b),q_aq_bq_c)]_{R_{\mathbb{Q}}}\\
  &= [(p_aq_bq_c+p_bq_aq_c+p_cq_aq_b,q_aq_bq_c)]_{R_{\mathbb{Q}}}\\
\end{align*}
Damit gilt \((aFb)Fc=aF(bFc)\).
  \end{proof}
\item Es existiert ein neutrales Element
  \(n \in \mathbb{Z}_{\mathbb{Q}}\) mit \(nFa=a\) für alle
  \(a \in \mathbb{Z}_{\mathbb{Q}}\).

  \begin{proof}
    Seien dieses neutrale Element \(n = [(0,1)]_{R_{\mathbb{Q}}}\) und
    \(a = [(p_a,q_a)]_{R_{\mathbb{Q}}} \in \mathbb{Z}_{\mathbb{Q}}\)
    beliebig gewählt.  Wir zeigen, dass \(nFa=a\) gilt.
\begin{align*}
  nFa &= [(0\cdot q_a + 1\cdot p_a, 1\cdot q_a)]_{R_{\mathbb{Q}}}\\
      &= [(p_a,q_a)]_{R_{\mathbb{Q}}}
\end{align*}
Damit gilt \(nFa = a\).
  \end{proof}
\item Zu jedem \(a \in \mathbb{Z}_{\mathbb{Q}}\) existiert ein inverses
  Element \(a' \in \mathbb{Z}_{\mathbb{Q}}\) mit \(a'Fa=n\).
  \begin{proof}
    Seien \(a = [(p_a,q_a)]_{R_{\mathbb{Q}}}\) und $a' =
    [(p_a,-q_a)]_{R_{\mathbb{Q}}}$.

    Es gilt \(a'Fa=[(0,-q_a^2)]_{R_{\mathbb{Q}}}\).  Wir zeigen, dass
    \([(0,-q_a^2)]_{R_{\mathbb{Q}}} = [(0,1)]_{R_{\mathbb{Q}}}\) gilt.

    Für alle \((m_1,n_1) \in [(0,-q_a^2)]_{R_{\mathbb{Q}}}\) gilt
    \(m_1 \cdot -q_a^2 = n_1 \cdot 0 = 0\).  Wegen \(q_a \neq 0\) gilt
    \(m_1 =0\).  Für alle \((m_2,n_2) \in [(0,1)]_{R_{\mathbb{Q}}}\) gilt
    \(m_2 \cdot 1 = n_2 \cdot 0 = 0\).  Wegen \(1 \neq 0\) gilt \(m_2 =0\).
    Es gilt für alle \((m_1,n_1) \in [(0,1)]_{R_{\mathbb{Q}}}\) und alle
    \((m_2,n_2) \in [(0,-q_a^2)]_{R_{\mathbb{Q}}}\) dass \(m_1, m_2=0\)
    sein muss und
    \(n_1, n_2 \in \mathbb{Z} \setminus \left\{ 0 \right\}\) beliebig.
    Daher gilt
    \([(0,-q_a^2)]_{R_{\mathbb{Q}}} = [(0,1)]_{R_{\mathbb{Q}}}\).
  \end{proof}
\item Es gilt \(aFb=bFa\) für alle \(a,b \in \mathbb{Z}_{\mathbb{Q}}\).
  \begin{proof}
    Seien
    $a = [(p_a,q_a)]_{R_{\mathbb{Q}}}, b =
    [(p_b,q_b)]_{R_{\mathbb{Q}}} \in \mathbb{Z}_{\mathbb{Q}}$ beliebig
    gewählt.

    Es gilt \(aFb = [(p_aq_b+p_bq_a, q_aq_b)]_{R_{\mathbb{Q}}}\) und
    \(bFa = [(p_bq_a+p_aq_b, q_bq_a)]_{R_{\mathbb{Q}}}\) Weil Addition
    kommutativ ist, gilt \(aFb=bFa\).  Die Gruppe
    \((\mathbb{Z}_{\mathbb{Q}}, F)\) ist daher abelsch.
  \end{proof}
\end{itemize}

\end{document}