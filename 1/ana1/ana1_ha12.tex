%% page style
\documentclass[12pt]{extarticle}
\usepackage[margin=2cm]{geometry}
\usepackage{fancyhdr,parskip}
\pagestyle{fancy}
\usepackage[onehalfspacing]{setspace}
\setlength{\parindent}{0pt}
\lhead{\myAuthor}
\rhead{\mySubject \ \myHausaufgaben. Übungsblatt \\ \myTutor}
\renewcommand*\familydefault{\sfdefault} %% Only if the base font of the document is to be sans serif

%% language
\usepackage[utf8]{inputenc}
\usepackage{xcharter-otf}
\usepackage[ngerman]{babel}

%% default packages
\usepackage{amsmath,mathtools,fontspec,amsthm,amssymb,amsfonts,
  stmaryrd, % for the lightning symbol used in proof by contraction
  tikz,     % used to draw diagrams
}


%% metadata
\newcommand{\myAuthor}{Yuchen Guo 480788 | Meng Zhang 484981 | TB9}
\newcommand{\myHausaufgaben}{12}
\newcommand{\mySubject}{Analysis}
\newcommand{\myTutor}{Tilman}


%% custom commands
\newcommand{\beh}{\textit{Behauptung.}\ }
\newcommand{\aufgn}[1]{\textbf{Aufgabe #1.}}
\newcommand{\mg}[1]{\mathbb{#1}}

\begin{document}
\aufgn{12.1.i}

\beh Die Abbildung \(f\) ist in \(x_0\) diffbar mit $f'(x_0)
= m$.

\begin{proof}
  Zur Abkürzung setzen wir \(D := \mg{R}\).  Sei die
  Folgen \((a_n)\) in \(D_{< x_0}\) und \((b_n)\) in
  \(D_{> x_0}\) beliebig gewählt mit
  \(\lim_{n \to \infty}{a_n} = x_0\) und
  \(\lim_{n \to \infty}{b_n} = x_0\).  Zu zeigen: die
  Grenzwerte
  $\lim_{n \to \infty}{\frac{f(a_n) - f(x_0)}{a_n -
      x_0}}$ und
  $\lim_{n \to \infty}{\frac{f(b_n) - f(x_0)}{b_n -
      x_0}}$ existiert und gleich
  \(m := \lim_{x \to x_0}{f'(x)}\).

  Weil \(f\) in \(D\) stetig und auf
  \(D \setminus \left\{ x_0 \right\}\) diffbar ist, ist
  \(f\) insbesondere in \(\left[ a_n, x_0 \right]\) und
  \(\left[ x_0, b_n \right]\) stetig, sowie in
  \(\left] a_n, x_0 \right[\) und
  \(\left] x_0, b_n \right[\) diffbar.  Dann existieren
  wegen Mittelwertsatz die Stellen
  \(c_n \in \left] a_n, x_0 \right[\) und
  \(d_n \in \left] x_0, b_n \right[\) mit
\begin{align*}
  f'(c_n) &= \frac{f(a_n) - f(x_0)}{a_n - x_0}, \\
  f'(d_n) &= \frac{f(b_n) - f(x_0)}{b_n - x_0}.
\end{align*}

Die Folgen \((a_n)\) und \((b_n)\) konvergieren gegen \(x_0\)
Es gilt auch \(c_n \in \left] a_n, x_0 \right[\) und
\(d_n \in \left] x_0, b_n \right[\).  Daraus folgt, die
Folgen \((c_n)\) und \((d_n)\) konvergieren gegen \(x_0\).

Wegen der Aufgabenstellung existiert der Grenzwert $m :=
\lim_{x \to x_0}{f'(x)}$.  Weil die Folgen \((c_n)\) und
\((d_n)\) gegen \(x_0\) konvergiert, es gilt dann
\begin{align*}
  m = \lim_{n \to \infty}{f'(c_n)}, \\
  m = \lim_{n \to \infty}{f'(d_n)}.
\end{align*}

Damit ist \(f\) diffbar in \(x_0\), denn \(x_0\) ist ein
Häufungspunkt von \(D\) und der Grenzwert
\begin{align*}
f'(x_0) :=
\lim_{x \to x_0}{\frac{f(x) - f(x_0)}{x - x_0}}
\end{align*}
existiert und gleich \(m\).  Die Definition von
Differenzierbarkeit in \(x_0\) (Def 6.1.1) ist damit erfüllt.
\end{proof}

\aufgn{12.1.ii}

\beh Die Aussage in (i) ist falsch, wenn man die
Stetigkeit in \(x_0\) nicht voraussetzt.

\begin{proof}
Als Gegenbeispiel betrachten wir die folgende Abbildung
\(g \colon \mg{R} \to \mg{R}\),
\begin{align*}
  x \mapsto
\begin{cases}
  1 & \text{falls } x = 0, \\
  x & \text{sonst.}
\end{cases}
\end{align*}
Wir zeigen, dass die Abbildung \(g\) diffbar auf $\mg{R}
\setminus \left\{ 0 \right\}$ aber nicht diffbar im
Punkt \(x = 0\) ist.

Sei \((a_n)\) eine Folge mit
\(\lim_{n \to \infty}{a_n} = a\),
\(a \in \mg{R} \setminus \left\{ 0 \right\}\) und
\(a_n \ne 0\) für alle \(n \in \mg{N}\).  Damit existiert
der Grenzwert
\begin{align*}
f'(a) = \lim_{n \to \infty}{\frac{f(a_n) - f(a)}{a_n -
  a}} = 1.
\end{align*}

Sei \((b_n)\) eine Folge mit
\(\lim_{n \to \infty}{b_n} = 0\). Der Grenzwert
\begin{align*}
  \frac{f(b_n) - f(0)}{b_n -  0} = \frac{1}{0}
\end{align*}
existiert nicht.

Damit haben wir gezeigt, dass die Abbildung \(g\) diffbar
auf \(\mg{R} \setminus \left\{ 0 \right\}\) aber nicht
diffbar im Punkt \(x = 0\) ist.
\end{proof}

\aufgn{12.2.i}

\beh Ist \(f \colon \left] a, b \right[ \to \mg{R}\)
konvex, so ist \(f\) stetig.

\begin{proof}
  Wir beweisen diese Aussage mittels Kontraposition,
  d.h., eine Abbildung, die nicht stetig ist, ist nicht
  konvex.  Zur Abkürzung setzen wir
  \(D := \left] a, b \right[\).

  Angenommen, \(f\) ist nicht stetig in \(c \in D\).  Dann
  existiert wegen des \(\varepsilon\)-\(\delta\) Kriterium
  ein \(\varepsilon > 0\) sodass zu jedem \(\delta > 0\)
  existiert ein \(x \in D\) mit
  \(\left| x - c \right| < \delta\) aber
  \(\left| f(x) - f(c) \right| \ge \varepsilon\).

  O.B.d.A. sei \(0 < x - c < \delta\) und
  \(f(x) - f(c) \ge \varepsilon\). Zu zeigen: es existiert
  ein \(\lambda \in \left[ 0, 1 \right]\) mit $f((1 -
  \lambda) x + \lambda c) > (1 - \lambda) f(x) +
  \lambda f(c)$.  Diese Ungleichung gilt im Fall
  \(\lambda = \frac{1}{2}\).


\end{proof}
\aufgn{12.2.ii}

\beh Ist \(f \colon \left[ a, b \right] \to \mg{R}\)
konvex, so ist \(f\) stetig.  Diese Aussage ist falsch.

\begin{proof}
Wir widerlegen diese Aussage mit einem Gegenbeispiel.
Wir betrachten die Abbildung
\begin{align*}
f \colon \left[ 0, 1 \right] \to \mg{R}, \quad x \mapsto
\begin{cases}
  x & x \in \left[ 0, 1 \right[, \\
  2 & x = 1.
\end{cases}
\end{align*}
Die Abbildung \(f\) ist konvex aber nicht stetig.
\end{proof}

\end{document}