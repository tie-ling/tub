\documentclass[12pt]{extarticle}
\usepackage{amsmath,mathtools,fontspec,amsthm,amssymb,amsfonts,fancyhdr,color,graphicx,}
\usepackage[margin=2.5cm]{geometry}
\usepackage[utf8]{inputenc}
\usepackage{xcharter-otf}
\usepackage[ngerman]{babel}
\usepackage[onehalfspacing]{setspace}
\usepackage{tikz}
\renewcommand{\familydefault}{\sfdefault}
\pagestyle{fancy}
\setlength{\parindent}{0pt}
\lhead{Yuchen Guo 480788, Meng Zhang 484981, Chaeyoung Hong 478363}
\rhead{Analysis I 1. Übungsblatt\\Tutor*in: Tilman}
\begin{document}

\section*{Aufgabe 1.1} Für alle \(a, b \in \mathbb{R}\) gelten die
folgenden Aussagen und Rechenregeln:
\subsection*{(i)} \(A\): Es gilt \(ab = 0\).

\(B\): Es gilt \(a = 0\) oder \(b = 0\).

\begin{proof}
\(A \implies B\).
    \begin{itemize}
    \item Falls \(a = 0\), Aussage \(B\) ist dann wahr und es gilt $A
\implies B$.
    \item Falls \(a \neq 0\), sollen wir zeigen, dass \(b = 0\) gilt.

\textbf{Skript Satz 1.4.4} Die Gleichung \(a \cdot x = b\) hat genau
eine Lösung, nämlich \(x = a^{-1} \cdot b\).

\textbf{Skript Satz 1.5.1} Für alle \(a,b \in \mathbb{R}\) gilt $a \cdot
0 = 0$.

Das heißt, die Gleichung \(ab=0\) hat genau eine Lösung $b=a^{-1} \cdot
0$ (Satz 1.4.4) und \(a^{-1} \cdot 0 = 0\) (Satz 1.5.1).

Daraus folgt, dass \(b = 0\), Aussage \(B\) ist dann wahr und es gilt
\(A \implies B\).
    \end{itemize}
\(B \implies A\).

  Wir nehmen an, dass \(B\) gilt.  Es folgt aus Satz 1.5.1, dass \(A\)
  wahr ist.  Damit ist \(B \implies A\) bewiesen.

Damit ist \( A \Leftrightarrow B\) bewiesen.

\end{proof}

\textbf{Skript Satz 1.4.4} Die Gleichung \(a \cdot x = b\) hat genau
eine Lösung, nämlich \(x = a^{-1} \cdot b\).
\begin{proof}
  In dieser Aussage stecken genau genommen gleich zwei Aussagen,
  nämlich eine über die Existenz der Lösung und eine weitere über
  deren Eindeutigkeit.  Wir beweisen zunächst die Existenz, indem wir
  zeigen, dass \(a^{-1} \cdot b\) eine Lösung ist.  Es gilt:
\begin{align*}
  a \cdot (a^{-1} \cdot b) &= (a \cdot a^{-1}) \cdot b \tag*{M1} \\
                           &= 1 \cdot b \tag*{M4} \\
                           &= b \cdot 1 \tag*{M2} \\
  &= b \tag*{M3}
\end{align*}

Sei nun \(y \in \mathbb{R}\) ein weiteres Element mit der Eigenschaft
\(a \cdot y = b\).  Dann gilt:
\begin{align*}
  a^{-1} \cdot b &= a^{-1} \cdot (a \cdot y) \tag*{Definition von \(y\)}
  \\
                 &= (a^{-1} \cdot a) \cdot y \tag*{M1} \\
                 &= (a \cdot a^{-1}) \cdot y \tag*{M2} \\
                 &= 1 \cdot y \tag*{M4} \\
                 &= y \cdot 1 \tag*{M2} \\
  &= y \tag*{M3}
\end{align*}
Jede weitere Lösung \(y\) ist also identisch mit \(a^{-1} \cdot b\).
\end{proof}

\textbf{Skript Satz 1.5.1} Für alle \(a \in \mathbb{R}\) gilt $a \cdot
0 = 0$.

\begin{proof}

\begin{align*}
  a \cdot 0 &= a \cdot (0+0) \tag*{A3} \\
  &= a \cdot 0 + a \cdot 0 \tag*{D}
\end{align*}


Das heißt, dass \(a \cdot 0\) eine Lösung für
\(a \cdot 0 + x = a \cdot 0\) ist.  Nach der Definition des additiven
neutralen Elements ist aber auch \(x = 0\) eine Lösung für diese Gleichung.
Wegen der Eindeutigkeit des additiven neutralen Elements (Skript Satz
1.3.1, in Skript bewiesen) ist also \(a \cdot 0 = 0\).
\end{proof}

\subsection*{(ii)}
Um \((-a)b=a(-b)=-(ab)\) zu beweisen, sollen wir zeigen, dass
\(A \implies B \implies C \implies A\) wahr ist.


\textbf{Lemma Ü1 A1.1.ii.1}  Für alle \(a \in \mathbb{R}\) gilt, dass $(-1)
\cdot a = -a$.
\begin{proof}
\begin{align*}
  (-1) \cdot a &= (-1) \cdot a + 0 \tag*{A3}\\
               &= (-1) \cdot a + (a + (-a)) \tag*{A4}\\
               &= ((-1) \cdot a + a) + (-a) \tag*{A1} \\
               &= ((1 + (-1)) \cdot a) + (-a) \tag*{D} \\
               &= 0 \cdot a + (-a) \tag*{A4} \\
               &= a \cdot 0 + (-a) \tag*{M2} \\
               &= 0 + (-a) \tag*{Skript Satz 1.5.1} \\
               &= (-a) + 0 \tag*{A2} \\
               &= -a \tag*{A3}
\end{align*}
\end{proof}

\begin{proof}
  Zunächst zeigen wir, dass \(A \implies B\) wahr ist.
\begin{align*}
  (-a)b &= (-1) \cdot a \cdot b \tag*{Lemma Ü1 A1.1.ii.1}\\
        &= a \cdot (-1) \cdot b \tag*{M2} \\
        &= a \cdot ((-1) \cdot b) \tag*{M1} \\
  &= a \cdot (-b) \tag*{Lemma Ü1 A1.1.ii.1}
\end{align*}
\end{proof}

\begin{proof}
Wir zeigen, dass \(B \implies C\) wahr ist.
\begin{align*}
  a(-b) &= a \cdot ((-1) \cdot b) \tag*{Lemma Ü1 A1.1.ii.1} \\
        &= (-1) \cdot a \cdot b \tag*{M2} \\
        &= (-1) \cdot (ab) \tag*{M1} \\
        &= -ab \tag*{Lemma Ü1 A1.1.ii.1}
\end{align*}
\end{proof}

Analog können wir zeigen, dass \(C \implies A\) wahr ist.  Damit ist die
Behauptung bewiesen. \qed

\subsection*{(iii)}
\textbf{Lemma Ü1 A1.1.iii.1} Für alle \(a \in \mathbb{R}\) gilt, $-(-a)
= a$.

Nach Existenz des additiven Inverses, gilt \((-a) + -(-a) = 0\).  Es
gilt aber auch, \((-a) + a =0\).  Wegen der Eindeutigkeit der Addition
(analog zum Skript Satz 1.4.4, siehe oben), gilt \(-(-a) = a\). \qed
\begin{proof}
\begin{align*}
  (-a)(-b) &= ((-1) \cdot a)((-1) \cdot b) \tag*{Lemma Ü1 A1.1.ii.1}\\
           &= ((-1) \cdot (-1)) (a \cdot b) \tag*{M1, M2}\\
           &= 1 \cdot (a \cdot b) \tag*{Lemma Ü1 A1.1.iii.1} \\
  &= ab \tag*{M3}
\end{align*}
  \end{proof}

\section*{Aufgabe 1.2}
\subsection*{(i)}
\begin{proof}
Zuerst zeigen wir, dass wegen Dreiecksungleichung die folgende
Ungleichungen gilt.
\begin{align*}
\left| (x+y) + (x-y) \right| &\leq \left| x+y \right| + \left| x - y
                               \right| \tag*{Dreiecksungleichung}\\
  \left| 2x \right| &\leq \left| x+y \right| + \left| x - y \right|\\
  2\left|  x \right| &\leq \left| x+y \right| + \left| x - y \right| \tag{1}\\
\end{align*}
\textbf{Fallunterscheidung}
\begin{itemize}
\item Wenn \(\left| x \right| \geq \left| y \right|\) gilt,

  Dann aus (1) folgt, dass $\left| x \right| + \left| y \right| \leq
  \left| x+y \right| + \left| x - y \right|$.

\item Wenn \(\left| y \right| > \left| x \right|\) gilt,

  Dann wegen \(\left| y-x \right| = \left| x-y \right|\), gilt $2 \left|
    y \right| \leq \left| y+x \right| + \left| y-x \right|$, daraus
  folgt, dass $\left| x \right| + \left| y \right| \leq
  \left| x+y \right| + \left| x - y \right|$.
\end{itemize}
\end{proof}
\subsection*{(ii)}
\textbf{Fallunterscheidung}
\begin{itemize}
\item Falls \(xy > 0\), dann folgt

\begin{align*}
  \left| \frac{x}{y} + \frac{y}{x} \right|
  &= \left| \frac{x^2+y^2}{xy} \right|\\
  &= \frac{x^2+y^2}{xy} \tag*{\(xy>0\)}\\
  &= \frac{x^2+y^2+2xy-2xy}{xy} \\
  &= \frac{(x-y)^2}{xy} + 2 \\
  &\geq 2
\end{align*}

\item Falls \(xy < 0\), dann folgt

\begin{align*}
  \left| \frac{x}{y} + \frac{y}{x} \right|
  &= \left| \frac{x^2+y^2}{xy} \right|\\
  &= - \frac{x^2+y^2}{xy} \tag*{\(xy<0\)}\\
  &= - \frac{x^2+y^2+2xy-2xy}{xy} \\
  &= - \frac{(x+y)^2}{xy} + 2 \\
  &\geq 2
\end{align*}
\end{itemize}

\section*{Aufgabe 1.3}
\subsection*{(i)}
\textbf{Fallunterscheidung}
\begin{itemize}
\item \((x-2 \geq 0) \wedge (x-3 > 0) \implies x > 3\)

  Dann folgt,
\begin{align*}
  \frac{1-(x-2)}{x-3} &< \frac{1}{2}\\
  1-x+2 &< \frac{x}{2} - \frac{3}{2}\\
  x &> 3
\end{align*}

D.h., \(\{x|x \in \mathbb{R}, x > 3\}\) ist eine Menge von Lösungen.
\item \((x-2 \geq 0) \wedge (x-3 < 0) \implies 2 \leq x < 3\)

  Dann folgt,
\begin{align*}
  \frac{1-(x-2)}{3-x} &< \frac{1}{2}\\
  x &> 3
\end{align*}

Weil \(\{x|x>3\} \cap \{x|2 \leq x < 3\} = \emptyset\), gibt es keine
Lösung, die diese Voraussetzung erfüllt.
\item \((x-2 < 0) \wedge (x-3 > 0) \implies x = \emptyset\)
\item \((x-2 < 0) \wedge (x-3 < 0) \implies x < 2\)

  Dann folgt,
\begin{align*}
  \frac{1-(2-x)}{3-x} &< \frac{1}{2}\\
  x &< \frac{5}{2}
\end{align*}
D.h., \(\{x|x \in \mathbb{R}, x < 2\}\) ist eine Menge von Lösungen.
\end{itemize}

Insgesamt gilt \(\{x|x \in \mathbb{R}, x < 2 \quad \text{oder} \quad x > 3\}\) als die
Menge von Lösungen.

\subsection*{(ii)}
\textbf{Fallunterscheidung}
\begin{itemize}
\item \(x \geq -3a\) und \(x \geq a\).

  Dann erhalten wir \(4a = 2a\), weshalb \(a=0\) gilt.  Die Gleichung gilt
  dann für alle \(x \geq 0\).
\item \(x \geq -3a\) und \(x < a\).

  Dann erhalten wir \(2x =0\), weshalb \(x=0\) gilt.  Die Gleichung gilt
  dann für alle \(a > 0\).
\item \(x < -3a\) und \(x \geq a\).

  Dann erhalten wir \(x=-2a\).  Die Gleichung gilt dann für alle $-2a <
  -3a$ und \(-2a \geq a\) und \(x = -2a\).
\item \(x < -3a\) und \(x < a\).

  Dann erhalten wir \(-2a=2a\), weshalb \(a=0\).  Die Gleichung gilt dann
  für alle \(x < 0\).
\end{itemize}
\section*{Aufgabe 1.4}
\subsection*{(i)}
\(A\): Gegeben ist \(a \in \mathbb{R}\) mit der folgenden Eigenschaft: für
jedes \(\varepsilon > 0\) ist \(|a| < \varepsilon\).

\(B\): Es gilt \(a = 0\).
\begin{proof}
Wir beweisen \(A \implies B\) mit Kontraposition, nämlich $\neg B
\implies \neg A$ gilt.

Angenommen, es gilt \(a \neq 0\).  Aus der Definition des Absolutbetrags
folgt, dass \(|a| > 0\).  Wegen \textbf{Skript Satz 1.10.6},
\begin{center}
  Aus \(a < b\) folgt \(a < \frac{a+b}{2} < b\).
\end{center}

existiert eine Zahl \(x = \frac{|a|+0}{2}\) mit
\(0 < x < |a|\).  Insbesondere für \(\varepsilon = \frac{|a|}{2}\) gilt
\(\varepsilon < |a|\).

Es gilt also \(\neg A\).  Damit ist \(A \implies B\) bewiesen.
\end{proof}

\subsection*{(ii)}

Die Aussage (i) ist auch dann noch wahr, wenn man nur
\(|a| \leq \varepsilon\) für alle \(\varepsilon > 0\) fordert, denn
\(\neg B \implies \neg A\) gilt immer noch, nämlich es existiert ein
\(\varepsilon = \frac{|a|}{2}\) mit \(|a| > \varepsilon\).
\end{document}