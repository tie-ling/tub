%% page style
\documentclass[12pt]{extarticle}
\usepackage[margin=2cm]{geometry}
\usepackage{fancyhdr,parskip}
\pagestyle{fancy}
\usepackage[onehalfspacing]{setspace}
\setlength{\parindent}{0pt}
\lhead{\myAuthor}
\rhead{\mySubject \ \myHausaufgaben. Übungsblatt \\ \myTutor}
\renewcommand*\familydefault{\sfdefault} %% Only if the base font of the document is to be sans serif

%% language
\usepackage[utf8]{inputenc}
\usepackage{xcharter-otf}
\usepackage[ngerman]{babel}

%% default packages
\usepackage{amsmath,mathtools,fontspec,amsthm,amssymb,amsfonts,
  stmaryrd, % for the lightning symbol used in proof by contraction
  tikz,     % used to draw diagrams
}


%% metadata
\newcommand{\myAuthor}{Yuchen Guo 480788 | Meng Zhang 484981 | TB9}
\newcommand{\myHausaufgaben}{11}
\newcommand{\mySubject}{Analysis}
\newcommand{\myTutor}{Tilman}


%% custom commands
\newcommand{\beh}{\textit{Behauptung.}\ }
\newcommand{\aufgn}[1]{\textbf{Aufgabe #1.}}
\newcommand{\mg}[1]{\mathbb{#1}}
\newcommand{\mc}[1]{\mathcal{#1}}
\newcommand{\Real}{\operatorname{Re}}
\newcommand{\Imag}{\operatorname{Im}}
\begin{document}
\fbox{%
  \begin{minipage}{\textwidth}
\begin{itemize}
\item \textbf{Definition 3.41.3.}

Sei \(X\) eine Menge.  \(X\)
heißt abzählbar, falls es eine Surjektion \(\mg{N} \to X\)
gibt.
\item \textbf{Beispiel 1.15.}

Seien \(a, b \in \mg{R}\), $a <
b\(.  Die Intervalle \(\left[ a, b \right]\), \)\left[ a,
  -\infty \right[$, \(\left] -\infty, b \right]\) sind
abgeschlossene Intervalle.

Die Intervalle $\left]a,
  b\right[$, \(\left] a, \infty\right[\) und
\(\left]-\infty, b\right[\) sind offene Intervalle.

\item \textbf{\(\mg{Q}\) liegt dicht in \(\mg{R}\).}

Sei \(a, b \in \mg{R}\) mit \(a < b\) beliebig gewählt.  Dann
existiert ein \(q \in \mg{Q}\) sodass \(a < q < b\).

\item \textbf{H 3.4.iv}

  Zu jedem \(a \in \mg{R}\) existiert eine Folge $(q_n)_{n
  \in \mg{N}} \subseteq \mg{Q}$ mit \(\lim_{n \to \infty}{q_n}=a\).
\end{itemize}
  \end{minipage}
}

\aufgn{11.1.1}

\beh Jedes offene Intervall lässt sich als abzählbare
Vereinigung offener Intervalle mit rationale Randpunkten
darstellen.  Diese Aussage ist wahr.

\begin{proof}
Sei \(a, b \in \mg{R}\) mit \(a < b\) beliebig gewählt.  Dann ist
\(\left] a, b \right[ \subseteq \mg{R}\) ein offenes
Intervall.  Es existiert wegen (H 3.4.iv) zwei Folgen rationaler
Zahlen \((p_n), (q_n)\) mit der Eigenschaften dass (i)
\(p_n \in \left] a, b \right[\) und $q_n \in \left] a, b
\right[\( für alle \(n \in \mg{N}\) und (ii) \)\lim_{n \to
  \infty}{p_n} = a$ und \(\lim_{n \to \infty}{q_n} = b\).
Damit gilt
$\bigcup_{i \in \mg{N}}{\left]p_i, q_i\right[} \subseteq
\left] a, b \right[$.

Zu zeigen: $\left] a, b \right[ \subseteq \bigcup_{i \in
  \mg{N}}{\left]p_i, q_i\right[}$.  Diese ist auch wahr,
denn sei \(x \in \left] a, b \right[\) beliebig gewählt.
Angenommen, es gilt $x \notin \bigcup_{i \in
  \mg{N}}{\left]p_i, q_i\right[}$.  Dann es gilt
entweder \(x \le p_i\) oder \(x \ge q_i\) für alle $i \in
\mg{N}\(.  Wegen \)\lim_{n \to
  \infty}{p_n} = a$ und \(\lim_{n \to \infty}{q_n} = b\)
ist dann \(x \le a\) oder \(x \ge b\), im Widerspruch zur
Voraussetzung dass \(x \in \left] a, b \right[\).  Damit
gilt $\left] a, b \right[ = \bigcup_{i \in
  \mg{N}}{\left]p_i, q_i\right[}$.

Falls eine Randpunkt des ursprünglichen Intervalls
Unendlich ist, können wir dann
\(\bigcup_{i \in \mg{N}}{\left]p_i, i\right[}\) oder
\(\bigcup_{i \in \mg{N}}{\left]-i, q_i\right[}\) wählen.
\end{proof}

\aufgn{11.1.ii}

\beh Jedes abgeschlossene Intervall lässt sich als
abzählbare Vereinigung abgeschlossener Intervalle mit
rationalen Randpunkten darstellen.  Diese Aussage ist
falsch.

\begin{proof}
  Wir widerlegen diese Aussage mit einem Gegenbeispiel.
  Das Intervall \(\left[0, e\right]\) ist ein
  abgeschlossenes Intervall.  Angenommen, \((p_n), (q_n)\)
  sind Folgen rationaler Zahlen mit
  $\bigcup_{i \in \mg{N}}{\left[p_n,
      q_n\right]}=\left[0, e\right]$.  Insbesondere, es
  gilt
  \(e \in \bigcup_{i \in \mg{N}}{\left[p_n, q_n\right]}\)
  und
  $x \notin \bigcup_{i \in \mg{N}}{\left[p_n,
      q_n\right]}$ für alle \(x > e\).  Also,
  $e = \max \left(\bigcup_{i \in \mg{N}}{\left[p_n,
        q_n\right]}\right)$.  Daraus folgt, \(e = q_i\)
  für ein \(i \in \mg{N}\) im Widerspruch zur
  Voraussetzung dass \((q_n) \in \mg{Q}\).
\end{proof}

\aufgn{11.1.iii}

\beh Jedes offene Intervall lässt sich als abzählbare
Vereinigung abgeschlossener Intervalle mit rationalen
Randpunkten darstellen.    Diese Aussage ist wahr.

\begin{proof}
Sei \(a, b \in \mg{R}\) mit \(a < b\) beliebig gewählt.  Dann ist
\(\left[ a, b \right] \subseteq \mg{R}\) ein abgeschlossenes
Intervall.  Es existiert wegen (H 3.4.iv) zwei Folgen rationaler
Zahlen \((p_n), (q_n)\) mit der Eigenschaften dass (i)
\(p_n \in \left[ a, b \right]\) und $q_n \in \left[ a, b
\right]\( für alle \(n \in \mg{N}\) und (ii) \)\lim_{n \to
  \infty}{p_n} = a$ und \(\lim_{n \to \infty}{q_n} = b\).
Damit gilt
$\bigcup_{i \in \mg{N}}{\left[p_i, q_i\right]} \subseteq
\left[ a, b \right]$.

Zu zeigen: $\left[ a, b \right] \subseteq \bigcup_{i \in
  \mg{N}}{\left[p_i, q_i\right]}$.  Diese ist auch wahr,
denn sei \(x \in \left[ a, b \right]\) beliebig gewählt.
Angenommen, es gilt $x \notin \bigcup_{i \in
  \mg{N}}{\left[p_i, q_i\right]}$.  Dann es gilt
entweder \(x < p_i\) oder \(x > q_i\) für alle $i \in
\mg{N}\(.  Wegen \)\lim_{n \to
  \infty}{p_n} = a$ und \(\lim_{n \to \infty}{q_n} = b\)
ist dann \(x < a\) oder \(x > b\), im Widerspruch zur
Voraussetzung dass \(x \in \left[ a, b \right]\).  Damit
gilt $\left[ a, b \right] = \bigcup_{i \in
  \mg{N}}{\left[p_i, q_i\right]}$.

Falls eine Randpunkt des ursprünglichen Intervalls
Unendlich ist, können wir dann
\(\bigcup_{i \in \mg{N}}{\left[p_i, i\right]}\) oder
\(\bigcup_{i \in \mg{N}}{\left[-i, q_i\right]}\) wählen.
\end{proof}

\fbox{%
  \begin{minipage}{\textwidth}
    \textbf{Definition von offenen Mengen.}

    Eine Menge \(M \subseteq \mg{R}\) heißt offen, falls
    zu jedem \(x_0 \in M\) ein \(\varepsilon > 0\) existiert
    mit
    $\left] x_0 - \varepsilon, x_0 + \varepsilon \right[
    \subseteq M$.  Eine Menge heißt abgeschlossen, falls
    \(\mg{R} \setminus M\) offen ist.
  \end{minipage}
}

\aufgn{11.2.i}

\beh Sei \(A\) eine Teilmenge von \(\mg{R}\).  Die Menge \(A\)
ist genau dann abgeschlossen, falls für jede in \(\mg{R}\)
konvergente Folge \((a_n)\) deren Glieder ausschließlich
in \(A\) liegen auch ihr Grenzwert in \(A\) liegt.  Zur
Abkürzung setzen wir \(A^c := \mg{R} \setminus A\).

\begin{proof}
  Hinrichtung.  Abgeschlossen \(\to\)
  Folgenabgeschlossen. Angenommen, die Menge \(A\) ist
  abgeschlossen.  Wegen Definition ist dann \(A^c\) offen.
  Zu zeigen: zu jeder Folge \((a_n)\), deren Glieder
  ausschließlich in \(A\) liegen, ihr Grenzwert in \(A\)
  liegen muss.

  Wir bemühen uns um einen Widerspruchsbeweis.
  Angenommen, ihr Grenzwert \(a \in A^c\).  Weil die Menge
  \(A^c\) offen ist, existiert ein \(\varepsilon > 0\) mit
  $\left] a - \varepsilon, a + \varepsilon\right[
  \subseteq A^c$.  Weil die Folge \((a_n)\) gegen \(a\)
  konvergiert und \(\varepsilon > 0\), existiert ein
  \(N_{\varepsilon} \in \mg{N}\) sodass für alle
  \(n \ge N_{\varepsilon}\) gilt
  \(\left| a_n - a \right| < \varepsilon\).  Also, es
  existiert im Widerspruch zur (Voraussetzung dass
  \(a_n \in A\) für alle \(n \in \mg{N}\)) solche
  \(a_n \in A^c\).

  Rückrichtung.  Folgenabgeschlossen \(\to\)
  Abgeschlossen.  ???
\end{proof}

\aufgn{11.2.ii}

\beh Die Menge $\left\{ \frac{1}{n+1} \mid n \in \mg{N}
\right\} \cup \left\{ 0 \right\}$ ist abgeschlossen.

\newpage

\aufgn{11.3.i}

Es gilt wegen \(\sin(2x) = 2\sin(x)\cos(x)\) dass $f: x
\mapsto \frac{1}{2}\sin(2\sqrt{x})$.  Daraus folgt mit
der Kettenregel dass $f'(x) =
\frac{1}{2\sqrt{x}}\cos(2\sqrt{x})$.

\aufgn{11.3.ii}

Wir berechnen zuerst die Ableitung von $p \colon x
\mapsto x + \sqrt{1+x^2}\(.  Es gilt dann \)p' \colon x
\mapsto 1 + x(1+x^2)^{\frac{1}{2}}\(.  Insgesamt ist dann also \)g'
\colon x \mapsto \frac{1+x(1+x^2)^{-\frac{1}{2}}}{x + \sqrt{1+x^2}}$.

\end{document}