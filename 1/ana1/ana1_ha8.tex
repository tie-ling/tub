%% page style
\documentclass[12pt]{extarticle}
\usepackage[margin=2cm]{geometry}
\usepackage{fancyhdr,parskip}
\pagestyle{fancy}
\usepackage[onehalfspacing]{setspace}
\setlength{\parindent}{0pt}
\lhead{\myAuthor}
\rhead{\mySubject \ \myHausaufgaben. Übungsblatt \\ \myTutor}
\renewcommand*\familydefault{\sfdefault} %% Only if the base font of the document is to be sans serif

%% language
\usepackage[utf8]{inputenc}
\usepackage{xcharter-otf}
\usepackage[ngerman]{babel}

%% default packages
\usepackage{amsmath,mathtools,fontspec,amsthm,amssymb,amsfonts,
  stmaryrd, % for the lightning symbol used in proof by contraction
  tikz,     % used to draw diagrams
}


%% metadata
\newcommand{\myAuthor}{Yuchen Guo 480788 | Meng Zhang 484981 | TB9}
\newcommand{\myHausaufgaben}{8}
\newcommand{\mySubject}{Analysis}
\newcommand{\myTutor}{Tilman}


%% custom commands
\newcommand{\beh}{\textit{Behauptung.}\ }
\newcommand{\aufgn}[1]{\textbf{Aufgabe #1.}}
\newcommand{\mg}[1]{\mathbb{#1}}

\begin{document}
\aufgn{8.1.i}

\beh Die Funktion
\(f \colon \left] 0, \infty \right[ \to \mg{R}\),
\(x \mapsto \sqrt{x}\) ist gleichmäßig stetig mit $\delta
= \varepsilon^2$ aber nicht Lipschitz-stetig.

\begin{proof}
Zur Abkürzung setzen wir $D := \left] 0, \infty
\right[\(.  Sei \(x, y \in D\) mit \)\left| x - y \right| <
\delta$.  O.B.d.A sei \(x \le y\).  Dann gilt wegen
\(\delta =  \varepsilon^2\) dass
\begin{align*}
  \left| x - y \right| &= y - x < \varepsilon^2 \\
  y < x + \varepsilon^2 &\le x + \varepsilon^2 + 2
  \varepsilon \sqrt{x}\\
  y &< \left( \sqrt{x} + \varepsilon \right)^2
\end{align*}
Wegen \(0 < x < y\) und \(\varepsilon > 0\) folgt
\begin{align*}
  \sqrt{y} &< \sqrt{x} + \varepsilon \\
  \sqrt{y} - \sqrt{x} &< \varepsilon \\
  \left| f(y) - f(x) \right| &< \varepsilon
\end{align*}
Daraus folgt, dass die Funktion in Definitionsbereich
gleichmäßig stetig ist.

Die Funktion ist nicht Lipschitz-stetig.  Angenommen,
es existiert ein \(L > 0\) mit
\(\left| f(x) - f(y) \right| \le L \left| x - y \right|\)
für alle \(x, y \in D\).  O.B.d.A sei \(x \le y\).  Dann
gilt
\begin{align*}
\left| f(x) - f(y) \right| &\le L \left| x - y \right| \\
  \sqrt{y} - \sqrt{x} &\le L (y-x) \\
  \frac{1}{\sqrt{y} - \sqrt{x}} &\le L.
\end{align*}
Als \(x-y \to 0\) gilt \(L \to \infty\).
\end{proof}

\aufgn{8.1.ii}

\beh Die Funktion
\(g \colon \left]0, \infty\right[ \to \mg{R}\),
\(x \mapsto \frac{1}{\sqrt{x}}\) ist stetig aber nicht
gleichmäßig stetig.

\begin{proof}
  Die Funktion ist stetig.  Denn es gilt für alle Folge
  \((a_n) \subseteq \left]0, \infty\right[\) mit
  $\lim_{n \to \infty}{(a_n)}=p \in \left]0,
    \infty\right[$ dass
\begin{align*}
  \lim_{n \to \infty}{g(a_n)}
  = g \left(\lim_{n \to \infty}{a_n} \right) = g(p).
\end{align*}

Die Funktion ist nicht gleichmäßig stetig, denn sei
\(\delta > 0\) beliebig gewählt.  Sei \(x, y \in D\) mit
\(\left| x - y \right| < \delta\).  O.B.d.A sei
\(x \le y\).
Weiter gilt, weil \(g\) monoton fallend ist,
\begin{align*}
  \left| g(x) - g(y) \right|
  &= g(x) - g(y)  \\
  & =\frac{1}{\sqrt{x}} - \frac{1}{\sqrt{y}}.
\end{align*}

Als \(x \to 0\) gilt \(g(x) \to \infty\) aber
\(g(y) < \infty\).  Also
\(g(x) - g(y) = \infty > \varepsilon\).

Damit haben wir gezeigt, dass \(g\) gleichmäßig stetig
ist.
\end{proof}
\aufgn{8.2}

Sei die Folge \((a_n) \subseteq \mg{R}\) und
\(d \in \mg{R}\) ein Häufungspunkt von \((a_n)\) mit
\(d \notin (a_n)\).  Dann existiert eine Teilfolge
\((a_{n_k})\) von \((a_n)\) mit
\(\lim_{k \to \infty}{a_{n_k}} = d\).

\aufgn{8.2.i}

\beh Ist \(f \colon (a_n) \to \mg{R}\) gleichmäßig
stetig, so existiert eine stetige Funktion
\(F \colon (a_n) \cup \left\{ m \right\} \to \mg{R}\) mit
\(F\vert_{(a_n)}=f\).

\begin{proof}
Wir zeigen, dass solche Funktion existiert, indem wir
zeigen, dass diese Funktion
\begin{align*}
  F(x) =
\begin{cases}
  f(x) & x \in (a_n) \\
  \lim_{x \to d}{f(x)} & x = d
\end{cases}
\end{align*} stetig ist.
Wegen der gleichmäßig-Stetigkeit der Funktion $f \colon
(a_n) \to \mg{R}$ ist \(F\) stetig in \((a_n)\).  Nun
betrachten wir den Fall \(x = d\).  Zu zeigen: $\lim_{x
  \to d^-}{f(x)} = \lim_{x \to d^+}{f(x)}$.

Es gilt wegen der gleichmäßige Stetigkeit von \(f\) dass
für alle \(\varepsilon > 0\) existiert ein \(\delta > 0\)
sodass für alle \(x, y \in (a_n)\) mit
\(\left| x - y \right| < \delta\) gilt
\(\left| f(x) - f(y) \right| < \varepsilon\).

Angenommen,
\(\lim_{x \to d^-}{f(x)} \ne \lim_{x \to d^+}{f(x)}\), also
\begin{align*}
  \left| \lim_{x \to d^-}{f(x)} - \lim_{x \to d^+}{f(x)}
  \right| = h > 0.
\end{align*}
Dann ist die Funktion nicht gleichmäßig stetig, also es
existiert ein \(\varepsilon > 0\), nämlich alle
\(\varepsilon < h\), sodass für alle \(\delta > 0\)
existiert \(x, y \in (a_n)\) mit
\(\left| x - y \right| < \delta\) sodass
\(\left| f(x) - f(y) \right| \ge \varepsilon\).

Damit gilt
$\lim_{x \to d^-}{f(x)} = \lim_{x \to d^+}{f(x)} =
\lim_{x \to d}{f(x)}=F(d)$.  Die Funktion \(F\) ist in
\(d\) stetig.
\end{proof}


\aufgn{8.2.ii}

\beh Sei \(g \colon \left] a, b \right] \to \mg{R}\) stetig.  Es gilt
\begin{align*}
g \text{ gleichmäßig stetig } \iff \lim_{x \to
  a^+}{g(x)} \text{ existiert.}
\end{align*}

\begin{proof}
Wir beweisen diese Behauptung als wahr, indem wir
zeigen, dass die Hinrichtung und Rückrichtung wahr
sind.

\textit{Hinrichtung.} Sei $(p_n) \subseteq \left] a, b
\right]\( eine Folge mit \)\lim_{n \to \infty}{p_n} =
a$.  Zu zeigen: \(\lim_{n \to \infty}{g(p_n)}\)
existiert.

Sei \(\delta > 0\) beliebig gewählt. Weil die Folge
\((p_n)\) gegen \(a\) konvergiert, existiert ein
\(N_{\delta} \in \mg{N}\) sodass für alle
\(n, m \in \mg{N}_{\ge N_{\delta}}\) gilt
(O.Forster §5, Satz 1: Konvergente Folgen sind
Cauchy-Folgen)
\begin{align*}
\left| a_n - a_m \right| < \delta.
\end{align*}  Weiter gilt wegen der gleichmäßigen
Stetigkeit der Funktion \(g\) dass zu jedem $\varepsilon
> 0\( existiert ein \(\delta > 0\) sodass für alle \)a_n,
a_m \in \left]a, b\right]\( mit \)\left| a_n - a_m
\right| < \delta$ gilt
\begin{align*}
\left| g(a_n) - g(a_m) \right| < \varepsilon.
\end{align*}
Weil \(\varepsilon > 0\) beliebig gewählt war, ist die
Folge \((g(a_n))\) eine Cauchy-Folge und diese Folge
besitzt wegen der Vollständigkeits-Axiom einen
Grenzwert.  D.h., es existiert
\(\lim_{n \to \infty}{g(p_n)} \in \mg{R}\).  Daraus
folgt, \(\lim_{x \to a^+}{g(x)} \in \mg{R}\) existiert.

\textit{Rückrichtung.} Sei
\((p_n) \subseteq \left] a, b \right]\) eine Folge mit
\(\lim_{n \to \infty}{p_n} = a\). Sei \(\delta > 0\)
beliebig gewählt.  Es gilt wegen der Konvergenz von
\((p_n)\) dass
\begin{align*}
\left| a_n - a_m \right| < \delta.
\end{align*}
Wegen der Voraussetzung existiert der Grenzwert
\(\lim_{n \to \infty}{g(p_n)} \in \mg{R}\), d.h., die
Folge \((g(p_n))\) konvergiert.  Sei \(\varepsilon > 0\)
beliebig gewählt. Es gilt wegen (O.Forster
§5, Satz 1: siehe oben) dass
\begin{align*}
\left| g(a_n) - g(a_m) \right| < \varepsilon.
\end{align*}
Weil \(\delta, \varepsilon \in \mg{R}_{>0}\) beliebig
gewählt waren, existiert zu jedem \(\varepsilon > 0\) ein
\(\delta > 0\) sodass für alle $a_n, a_m \in \left] a, b
\right]$ mit \(\left| a_n - a_m \right| < \delta\) gilt
\begin{align*}
\left| g(a_n) - g(a_m) \right| < \varepsilon.
\end{align*}
Also die Funktion \(g\) ist gleichmäßig stetig.
\end{proof}

\aufgn{8.2.iii}

Falls nur die Stetigkeit von \(f\) vorausgesetzt ist, ist
die Aussage von (i) nicht wahr.  Dies zeigen wir mit
einem Gegenbeispiel.

Sei
$D:=(\mg{R} \setminus \left\{ 0 \right\}) \subseteq
\mg{R}$ und \(0 \in \mg{R}\) ein Häufungspunkt von \(D\)
mit \(0 \notin D\).  Die Funktion \(f: D \to \mg{R}\),
\(x \mapsto \frac{1}{x}\) ist stetig auf \(D\).  Sei dann
\(a \in \mg{R}\) beliebig und die Funktion
\begin{align*}
  F(x) =
\begin{cases}
  f(x) & x \in D, \\
  a & x = d.
\end{cases}
\end{align*}
Angenommen, die Funktion ist stetig in \(0\).  Dann
existiert zu jedem \(\varepsilon > 0\) ein \(\delta > 0\)
mit \(\left| f(y) - a \right| < \varepsilon\) für alle
\(\left| y-0 \right| < \delta\).  Diese ist nicht der
Fall, denn es gilt \(\varepsilon \to \infty\) wenn
\(y \to 0\).

Die Funktion \(F(x)\) ist nicht stetig in \(0\) für alle
\(a \in \mg{R}\), also nicht stetig fortsetzbar.

\aufgn{8.3}

\beh Eine stetige Funktion \(f \colon \mg{R} \to \mg{R}\)
mit \(\lim_{\left| x \right| \to \infty}{f(x)}=\infty\)
hat ein Minimum.

\begin{proof}
  Angenommen, ein Minimum existiert nicht.

  \textit{Falls 1.} Es gilt
  \(\inf (f(\mg{R})) = -\infty\). Dann existiert ein
  Folge \((a_n)\) mit
  \(\lim_{n \to \infty}f(a_n) = - \infty\).

  \textit{Falls 1.1.}  Falls \((a_n)\) divergiert
  bestimmt gegen \(\pm \infty\).  O.B.d.A sei
  \(\lim_{n \to \infty}{a_n} = \infty\).  Dann gilt im
  Widerspruch zur Voraussetzung
  (\(\lim_{\left| x \right| \to \infty}{f(x)} = \infty\))
  dass
\begin{align*}
  \lim_{n \to \infty}{f(a_n)}
  = \lim_{\left| x \right|  \to \infty}{f(x)}
  = - \infty \ne \infty.
\end{align*}

\textit{Falls 1.2.}  Falls \((a_n)\) konvergiert gegen
\(a \in \mg{R}\).  Dann ist der Funktionswert \(f\) im
Widerspruch zur Voraussetzung (Wertbereich ist
\(\mg{R}\)) nicht im Wertbereich.
\begin{align*}
  \lim_{n \to \infty}{f(a_n)}
  = - \infty = f(a) \notin \mg{R}
\end{align*}

\textit{Falls 2.} Das Infimum
\(p := \inf (f(\mg{R})) \in \mg{R}\) liegt nicht im
\(f(\mg{R})\).  Dann existiert ein Folge \((a_n)\) mit
\(\lim_{n \to \infty}f(a_n) = p\).

  \textit{Falls 2.1.}  Falls \((a_n)\) divergiert
  bestimmt gegen \(\pm \infty\).  O.B.d.A sei
  \(\lim_{n \to \infty}{a_n} = \infty\).  Dann gilt im
  Widerspruch zur Voraussetzung
  (\(\lim_{\left| x \right| \to \infty}{f(x)} = \infty\))
  dass
\begin{align*}
\lim_{x \to \infty}{f(x)} = p \ne \infty.
\end{align*}

\textit{Falls 2.2.}  Falls \((a_n)\) konvergiert gegen
\(a \in \mg{R}\).  Dann ist die Funktion \(f\) im
Widerspruch zur Voraussetzung (\(f\) stetig in \(\mg{R}\))
in \(a \in \mg{R}\) nicht stetig wegen Voraussetzung $p
\notin f(\mg{R})$ dass
\begin{align*}
  \lim_{n \to \infty}{f(a_n)} = p
  \ne f \left( \lim_{n  \to \infty}{a_n} \right).
\end{align*}
Daraus folgt, dass eine stetige und koerzitive Funktion
stets ein Minimum besitzt.
\end{proof}
\aufgn{8.4}

\beh Es gilt für alle \(\lambda_i \ge 0\) und \(x_i > 0\)
mit \(\sum_{k=1}^n{\lambda_k} = 1\) dass
\begin{align*}
  \ln \left( \sum_{i=1}^n{\lambda_ix_i} \right) \ge
  \sum_{i=1}^n{\lambda_i\ln(x_i)}.
\end{align*}

\begin{proof}
  Wegen (Satz 4.43) und (Definition 4.44) gilt, die
  Exponentialfunktion
  \(\exp \colon \mg{R} \to \left] 0 , \infty \right[\)
  ist streng monoton wachsend und stetig und deren
  Umkehrfunktion
  \(\ln \colon \left]0, \infty\right[ \to \mg{R}\) ist
  streng monoton wachsend und stetig.  Sei \((a_i)\) eine
  Folge mit \(e^{a_i}:=x_i\) für alle
  \(i \in \mg{N} \setminus \left\{ 0 \right\}\).  O.B.d.A
  sei die Folge monoton wachsend.  Zu zeigen:
\begin{align*}
  \ln \left( \sum_{i=1}^n{\lambda_ie^{a_i}} \right) \ge
  \sum_{i=1}^n{\lambda_ia_i}.
\end{align*}
  Weil die Exponentialfunktion konvex ist, d.h. für
  alle \(\lambda \in \left[0, 1\right]\) und alle $x, y
  \in \mg{R}$ gilt
\begin{align*}
  \exp(\lambda x + (1-\lambda) y)
  \le \lambda e^x + (1-\lambda)e^y.
\end{align*}
Daraus folgt,
\begin{align*}
  \exp(\lambda a_i + (1-\lambda) a_j)
  &\le \lambda e^{a_i} + (1-\lambda)e^{a_j} \\
  \lambda a_i + (1-\lambda) a_j
  &\le \ln \left( \lambda e^{a_i} + (1-\lambda)e^{a_j}  \right).
\end{align*}
Wegen  \(\sum_{k=1}^n{\lambda_k} = 1\) gilt
\begin{align*}
  \lambda_i a_i < (1-\lambda_i) a_i.
\end{align*}
Weil die Folge \((a_n)\) monoton wachsend ist, gilt
\end{proof}
\end{document}