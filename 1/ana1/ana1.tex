\documentclass[draft,a5paper]{article}
\usepackage{newtx}
\usepackage[germanb]{babel}
\usepackage[margin=2cm]{geometry}

\linespread{1.1}
\newcommand{\qed}{\(\blacksquare\)}
\newcommand{\beh}{\setlength{\parindent}{0pt}\textit{Behauptung.}\ }
\newcommand{\bew}{\setlength{\parindent}{0pt}\textit{Beweis.}\ }
\newcommand{\mg}[1]{\mathbb{#1}}
\newcommand{\mc}[1]{\mathcal{#1}}
\newcommand{\defn}[1]{\item \textbf{Definition} #1.}
\newcommand{\satz}[1]{\item \textbf{Satz} #1.}

\author{Yuchen Guo, Meng Zhang}
\date{\today}
\title{HA 1 -- Ana 2}

\begin{document}
\begin{enumerate}

\defn{Binomial-Koeffizienten}  Seien \(n, k \in \mg{N}\).
Dann gilt
\begin{align*}
\begin{pmatrix}
n \\ k
\end{pmatrix} :=
  \prod_{j = 1}^k{\frac{n-j+1}{j}}.
\end{align*}
Für \(0 \le k \le n\) gilt
\begin{align*}
\begin{pmatrix}
n \\ k
\end{pmatrix} = \frac{n!}{k! (n-k)!} =
\begin{pmatrix}
n \\ n - k
\end{pmatrix}.
\end{align*}

\satz{Binomischer Lehrsatz}  Seien \(x, y\) reelle Zahlen
und \(n\) eine natürliche Zahl.  Dann gilt
\begin{align*}
(x+y)^n = \sum_{k=0}^n{
\begin{pmatrix}
n \\ k
\end{pmatrix} x^{n-k}y^k
}.
\end{align*}

\satz{Geometrische Reihe}  Für jede reelle Zahl $x \ne
1$ und jede natürliche Zahl \(n\) gilt
\begin{align*}
\sum_{k=0}^n{x^k} = \frac{1-x^{n+1}}{1-x}.
\end{align*}

\satz{Dreiecks-Ungleichung}  Seien \(x, y \in \mg{R}\).
Dann gilt
\begin{align*}
\left| x + y \right| \le \left| x \right| + \left| y \right|.
\end{align*}

\satz{Bernoullische Ungleichung} Sei \(x \ge -1\) und
\(n \in \mg{N}\).  Dann gilt
\begin{align*}
(1+x)^n \ge 1 + nx.
\end{align*}

\defn{Konvergenz von Folgen} Sei \((a_n)_{n \in \mg{N}}\)
eine Folge reeller Zahlen.  Die Folge heißt konvergent
gegen \(a \in \mg{R}\), falls zu jedem \(\varepsilon > 0\)
existiert ein \(N_{\varepsilon} \in \mg{N}\), sodass
\(\left| a_n - a \right| < \varepsilon\) für alle
\(n \ge N_{\varepsilon}\).  In diesem Fall nennt man \(a\)
den Grenzwert der Folge und schreibt
\(\lim_{n \to \infty}{a_n} = a\).

\defn{\(\varepsilon\)-Umgebung} von \(a \in \mg{R}\) ist das
Intervall
\(\left] a- \varepsilon, a+ \varepsilon\right[\).

\defn{Beschränktheit von Folgen}  Eine Folge $(a_n)_{n
\in \mg{N}}$ reeller Zahlen heißt beschränkt, wenn es
eine reelle Konstante \(M \ge 0\) gibt, sodass für alle $n
\in \mg{N}$ gilt
\begin{align*}
\left| a_n \right| \le M.
\end{align*}  Die Folge \((a_n)\) heißt nach oben (bzw. unten)
beschränkt, wenn es eine Konstante \(K \in \mg{R}\) gibt,
sodass \(a_n \le K\) (bzw. \(a_n \ge K\)) für alle \(n\) gilt.

\satz{Summe und Produkt konvergenter Folgen}  Die
Summenfolge und die Produktfolge zwei konvergenter
Folgen sind konvergent und es gilt
\begin{align*}
\lim_{n \to \infty}{(a_n + b_n)} &= \lim_{n \to
  \infty}{a_n} + \lim_{n \to \infty}{b_n}, \\
  \lim_{n \to \infty}{(a_n b_n)} &= \lim_{n \to
  \infty}{a_n} \cdot \lim_{n \to \infty}{b_n}.
\end{align*}

\satz{Unendliche geometrische Reihe}  Die Reihe
\(\sum_{n=0}^{\infty}{x^n}\) konvergiert für alle $\left|
  x \right| < 1$ mit dem Grenzwert
\begin{align*}
\sum_{n = 0}^{\infty}{x^n} = \frac{1}{1-x}.
\end{align*}

\defn{Bestimmte Divergenz einer Folgen}  Eine Folge
\((a_n)_{n \in \mg{N}}\) reeller Zahlen heißt bestimmt
divergent gegen \(+ \infty\), falls zu jedem $K \in
\mg{R}\( ein \(N \in \mg{N}\) existiert, sodass für alle \)n
\ge N$ gilt
\begin{align*}
a_n > K.
\end{align*}

\defn{Cauchy-Folge}  Eine Folge \((a_n)_{n \in \mg{N}}\)
reeller Zahlen heißt Cauchy-Folge, falls zu jedem
\(\varepsilon > 0\) ein \(N_{\varepsilon} \in \mg{N}\)
existiert, sodass für alle \(n, m \ge N\) gilt
\begin{align*}
\left| a_n - a_m \right| < \varepsilon.
\end{align*}

\satz{Jede konvergente Folge reeller Zahlen ist eine Cauchy-Folge}

\defn{Teilfolge} Sei \((a_n)_{n \in \mg{N}}\) eine Folge
reeller Zahlen und sei \((n_k)_{k \in \mg{N}}\) eine
streng monoton wachsende Folge mit \(n_k \in \mg{N}\) für
alle \(k \in \mg{N}\).  Dann heißt die Folge
\((a_{n_k})_{k \in \mg{N}}\) eine Teilfolge von \((a_n)\).

\satz{Bolzano-Weierstraß}  Jede beschränkte Folge
\((a_n)_{n\in \mg{N}}\) reeller Zahlen besitzt eine
konvergente Teilfolge.

\defn{Häufungspunkt} Eine Zahl \(a\) heißt Häufungspunkt
einer Folge \((a_n)_{n \in \mg{N}}\), wenn es eine
Teilfolge von \((a_n)\) gibt, die gegen \(a\) konvergiert.

\satz{Jede beschränkte monotone Folge \((a_n)\) reeller
  Zahlen konvergiert}

\satz{Cauchysches Konvergenz-Kriterium für Reihen}  Sei
\((a_n)_{n \in \mg{N}}\) eine Folge reeller Zahlen.  Die
Reihe \(\sum_{n=0}^{\infty}{a_n}\) konvergiert genau dann,
falls zu jedem \(\varepsilon > 0\) ein \(N \in \mg{N}\)
existiert, sodass für alle \(n \ge m \ge N\) dass
\begin{align*}
\left| \sum_{k=m}^n{a_k} \right| < \varepsilon.
\end{align*}

\satz{Notwendiges Kriterium für die Konvergenz einer
  Reihe \(\sum_{n=0}^{\infty}{a_n}\)} ist $\lim_{n\to
  \infty}{a_n} = 0$.

\satz{Eine Reihe \(\sum_{n = 0}^{\infty}{a_n}\) mit
  \(a_n \ge 0\) für alle \(n \in \mg{N}\) konvergiert genau
  dann, wenn die Reihe (d.h. die Folge der
  Partialsummen) beschränkt ist}

\satz{Die harmonische Reihe
  \(\sum_{n=1}^{\infty}{\frac{1}{n}}\) divergiert}

\satz{Die Reihen \(\sum_{n=1}^{\infty}{\frac{1}{n^k}}\)
  konvergiert für \(k > 1\)}

\satz{Leibniz'sches Konvergenz-Kriterium für
  alternierende Reihen}  Sei $(a_n)_{n
\in \mg{N}}$ eine monoton fallende Folge nicht-negativer
Zahlen mit \(\lim_{n \to \infty}{a_n} = 0\).  Dann
konvergiert die alternierende Reihe
\begin{align*}
\sum_{n=0}^{\infty}{(-1)^n {a_n}}.
\end{align*}

\defn{Absolute Konvergenz einer Reihe}  Eine Reihe
\(\sum_{n=0}^{\infty}{a_n}\) heißt absolut konvergent,
falls die Reihe der Absolutbeträge
\(\sum_{n=0}^{\infty}{\left| a_n \right|}\) konvergiert.

\satz{Majoranten-Kriterium} Sei
\(\sum_{n=0}^{\infty}{c_n}\) eine konvergente Reihe mit
lauter nicht-negativen Gliedern und $(a_n)_{n\in
  \mg{N}}$ eine Folge sodass für alle \(n \in \mg{N}\)
gilt
\(\left| a_n \right| \le c_n\).
Dann konvergiert die Reihe \(\sum_{n=0}^{\infty}{a_n}\)
absolut.

\defn{Limes superior, Limes inferior} Sei
\((a_n)_{n \in \mg{N}}\) eine beschränkte Folge reeller
Zahlen.  Dann definiert man
\begin{align*}
  \limsup_{n \to \infty}{a_n}
  &:= \lim_{n \to \infty}
    \left( \sup \left\{ a_k \colon k \ge n \right\}
    \right), \\
  \liminf_{n \to \infty}{a_n}
  &:= \lim_{n \to \infty}
    \left( \inf \left\{ a_k \colon k \ge n \right\} \right).
\end{align*}


\satz{Wurzel-Kriterium}  Sei \((a_n)_{n \in \mg{N}}\) eine
Folge mit $a := \limsup_{n \to \infty}{\sqrt[n]{\left|
      a_n \right|}} \in \mg{R} \cup \left\{ \infty
\right\}$.  Falls \(a < 1\), so konvergiert die Reihe
\(\sum_{k=0}^{\infty}{a_k}\) absolut.  Falls \(a > 1\) oder
\(a = \infty\), so divergiert die Reihe.

\satz{Quotienten-Kriterium}  Sei \((a_n)\) eine Folge mit
\(a_n \ne 0\) für alle \(n \in \mg{N}\).  Falls $\limsup_{n
  \to \infty}{\left| \frac{a_{n+1}}{a_n} \right|} < 1$,
so konvergiert die Reihe \(\sum_{k=0}^{\infty}{a_k}\)
absolut.  Falls $\liminf_{n \to \infty}{\left|
    \frac{a_{n+1}}{a_n} \right|} > 1$, so divergiert die
Reihe \(\sum_{k=0}^{\infty}{a_k}\).

\satz{Umordnungssatz} Jede Umordnung einer absolut
konvergenter Reihe konvergiert absolut gegen denselben
Grenzwert.

\satz{Exponentialreihe}  Für jedes \(x \in \mg{R}\)
konvergiert die Exponentialreihe
\begin{align*}
\exp(x) = \sum_{n=0}^{\infty}{\frac{x^n}{n!}}
\end{align*}
absolut wegen Quotientenkriterium.

\defn{Berührpunkt}  Sei \(A \subseteq \mg{R}\) und $a
\in \mg{R}$.  Der Punkt \(a\) heißt Berührpunkt von \(A\),
falls in jeder \(\varepsilon\)-Umgebung
$U_{\varepsilon}(a) := \left] a- \varepsilon, a+
  \varepsilon\right[$ mindestens ein Punkt von \(A\) liegt.

\defn{Häufungspunkt}  Sei \(A \subseteq \mg{R}\) und $a
\in \mg{R}$.  Der Punkt \(a\) heißt Häufungspunkt von \(A\),
falls in jeder \(\varepsilon\)-Umgebung
$U_{\varepsilon}(a) := \left] a- \varepsilon, a+
  \varepsilon\right[$ unendlich viele Punkte von \(A\) liegen.

\defn{Stetigkeit von Funktionen}  Sei $f \colon D \to
\mg{R}$ eine Funktion und \(a \in D\).  Die Funktion \(f\)
heißt stetig im Punkt \(a\), falls für jede Folge
\((x_n)_{n \in \mg{N}}\), \(x_n \in D\) mit $\lim_{n \to
  \infty}{x_n} = a$ gilt
\begin{align*}
\lim_{n \to \infty}{f(x_n)} = f(a).
\end{align*}


\satz{Zwischenwertsatz}  Sei $f \colon \left[a,
  b\right] \to \mg{R}\( eine stetige Funktion mit \)f(a) <
0$ und \(f(b) > 0\) (bzw. \(f(a) > 0\) und \(f(b) < 0\)).
Dann existiert ein \(p \in \left[a,b\right]\) mit \(f(p) = 0\).

\defn{Kompaktes Intervall}  Ein kompaktes Intervall ist
ein abgeschlossenens und beschränktes Intervall
\(\left[a,b\right] \subset \mg{R}\).

\satz{von Minimum und Maximum}  Jede in einem kompakten
Intervall stetige Funktion $f \colon \left[a,b\right]
\to \mg{R}$ ist beschränkt und nimmt ihr Maximum und
Minimum an.

\satz{\(\varepsilon\)-\(\delta\) Definition der Stetigkeit}
Sei \(D \subset \mg{R}\) und \(f \colon D \to \mg{R}\) eine
Funktion.  \(f\) ist genau dann im Punkt \(p \in D\) stetig,
falls zu jedem \(\varepsilon > 0\) existiert ein $\delta
> 0\(, sodass für alle \(x \in D\) mit \)\left| x -p \right|
< \delta$ gilt \(\left| f(x) - f(p) \right| < \varepsilon\).

\satz{Gleichmäßige Stetigkeit}  Eine Funktion $f \colon
D \to \mg{R}$ heißt in \(D\) gleichmäßig stetig, falls zu
jedem \(\varepsilon > 0\) existiert ein \(\delta > 0\)
sodass für alle \(x, x' \in D\) mit $\left| x - x' \right|
< \delta\( gilt \)\left| f(x) - f(x') \right| <
\varepsilon$.

\satz{von Heine} Jede auf einem kompakten Intervall stetige
  Funktion \(f \colon \left[a,b\right] \to \mg{R}\) ist
  dort gleichmäßig stetig.

\defn{Lipschitz-Stetigkeit}  Eine auf einer Teilmenge $D
\subseteq \mg{R}\( definierte Funktion \)f \colon D \to
\mg{R}$ heißt Lipschitz-stetig mit Lipschitz-Konstante
\(L \in \mg{R}_{>0}\) falls für alle \(x, x' \in D\) gilt
\begin{align*}
\left| f(x) - f(x') \right| \le L \left| x - x' \right|.
\end{align*}

\defn{Exponentialfunktion zur Basis \(a\)}  Für \(a > 0\)
sei die Funktion \(\exp_a \colon \mg{R} \to \mg{R}\)
definiert durch
\begin{align*}
\exp_a(x) := \exp(x \ln a) = a^x = e^{x \ln a}.
\end{align*}

\defn{Cosinus, Sinus}  Für \(x \in \mg{R}\) ist
\begin{align*}
e^{ix} = \cos x + i \sin x.
\end{align*}

\satz{Additionstheoreme}  Für alle \(x, y \in \mg{R}\)
gilt
\begin{align*}
  \cos(x+y)
  &= \cos x \cos y - \sin x \sin y,\\
  \sin ( x + y )
  &= \sin x \cos y + \cos x \sin y,\\
  \cos(2x)
  &= 2\cos^2 x - 1,\\
  \sin(2x)
  &= 2\sin x \cos x.
\end{align*}

\defn{Diffbarkeit}  Sei \(D \subset \mg{R}\) und $f \colon
D \to \mg{R}$ eine Funktion.  \(f\) heißt in einem Punkt
\(x_0 \in D\) diffbar, falls \(x_0\) Häufungspunkt von \(D\) ist
und der Grenzwert
\begin{align*}
f'(x_0) := \lim_{\substack{x \to x_0\\x \in D \setminus
  \left\{ x_0 \right\}}}{\frac{f(x) - f(x_0)}{x - x_0}}
\end{align*}
existiert.

\defn{Innere Punkt}  Sei \(M \subseteq \mg{R}\) eine
Teilmenge der reeller Zahlen.  Ein Punkt \(x_0 \in M\)
heißt innerer Punkt von \(M\), falls es ein $\varepsilon >
0$ gibt, sodass \(U_{\varepsilon}(x_0) \subseteq M\) gilt.

\defn{Offene Menge} \(M\) heißt offen, falls jeder Punkt
von \(M\) ein innerer Punkt von \(M\) ist.

\defn{Abgeschlossene Menge} \(M\) heißt abgeschlossen,
falls \(\mg{R} \setminus M\) offen ist.

\defn{Lokales Maximum, Minimum}  Sei $f \colon \left] a,
  b \right[ \to \mg{R}$ eine Funktion.  Man sagt, \(f\)
habe in \(x \in \left]a, b \right[\) ein lokales Maximum
(bzw. Minimum), wenn ein \(\varepsilon > 0\) existiert,
sodass für alle \(\xi\) mit $\left| x - \xi \right| <
\varepsilon\( gilt \(f(x) \ge f(\xi)\) (bzw. \)f(x) \le
f(\xi)$).

\satz{von Rolle} Sei \(a < b\) und
\(f \colon \left[a, b\right] \to \mg{R}\) Funktion auf
\(\left[a, b\right]\) stetig und \(\left]a, b\right[\)
diffbar mit \(f(a) = f(b)\).  Dann existiert ein $\xi \in
\left]a, b\right[$ mit \(f'(\xi) = 0\).

\satz{Mittelwertsatz der Differentialrechnung}  Sei
\(a<b\) und \(f \colon \left[a, b\right] \to \mg{R}\) Funktion auf
\(\left[a, b\right]\) stetig und \(\left]a, b\right[\)
diffbar.  Dann existiert ein $\xi \in \left]a,
  b\right[$, sodass
\begin{align*}
\frac{f(b) - f(a)}{b-a} = f'(\xi).
\end{align*}

\defn{Konvexität}  Sei \(D \subset \mg{R}\) ein
Intervall.  Eine Funktion \(f \colon D \to \mg{R}\) heißt
konvex, wenn für alle \(x_1, x_2 \in D\) und alle
\(0<\lambda<1\) gilt
\begin{align*}
f(\lambda x_1 + (1-\lambda) x_2) \le \lambda f(x_1) +
  (1-\lambda) f(x_2).
\end{align*}

\satz{Sei \(D \subset \mg{R}\) ein offenes Intervall und
  \(f \colon D \to \mg{R}\) eine zweimal diffbare
  Funktion.  \(f\) ist genau dann konvex, wenn $f''(x) \ge
  0$ für alle \(x \in D\)}

\satz{Regel von de l'Hospital} Sei \(I \subseteq \mg{R}\)
ein Intervall, \(x_0 \in I\) und seien
\(f,g \colon I \setminus \left\{ x_0 \right\} \to \mg{R}\)
diffbar und \(g'(x) \ne 0\) für alle
\(x \in I \setminus \left\{ x_0 \right\}\).  Falls es gilt
\(\lim_{x \to x_0}{f(x)} = 0 = \lim_{x \to x_0}{g(x)}\)
und der Grenzwert
\(\lim_{x \to x_0}{\frac{f'(x)}{g'(x)}} =: a\) existiert,
so ist der Grenzwert
\(\lim_{x \to x_0}{\frac{f(x)}{g(x)}}\) gleich \(a\).

\defn{Taylorpolynom}  Sei \(I \subseteq \mg{R}\) ein
Intervall, \(x_0 \in I\), sowie \(f \colon I \to \mg{R}\)
eine \(n\)-mal diffbare Funktion.  Die Polynomfunktion
\(T_n \colon \mg{R} \to \mg{R}\) mit
\begin{align*}
T_n(x) = \sum_{k=0}^n{\frac{f^{(k)}(x_0)}{k!}(x-x_0)^k}
\end{align*}
für alle \(x \in \mg{R}\) heißt \(n\)-tes Taylorpolynom von
\(f\) im Entwicklungspunkt \(x_0\).  $R_n \colon I \to
\mg{R},\ x \mapsto f(x) - T_n(x)$ heißt das zu \(T_n\)
gehörige Restglied.

\satz{Taylor-Formel}  Sei \(I \subseteq \mg{R}\) ein
Intervall, \(x_0 \in I\), sowie \(f \colon I \to \mg{R}\)
eine Funktion \(n\)-mal diffbar in \(x_0\), wobei $n \ge
1$.  Für alle \(x \in I\) gilt
\begin{align*}
f(x) = \sum_{k=0}^n{\frac{f^{(k)}(x_0)}{k!}(x-x_0)^k} +
  R_n(x), \quad \text{wobei} \quad \lim_{x\to
  x_0}{\frac{R_n(x)}{(x-x_0)^n}} = 0.
\end{align*}

\satz{Hinreichendes Kriterium für lokale Extremstellen}
Sei \(D \subseteq \mg{R}\) und \(x_0 \in D\) ein innerer
Punkt von \(D\) und sei \(f \colon D \to \mg{R}\) \(n\)-mal
diffbar in \(x_0\) und es gilt für alle $1 \le m \le n -
1$ dass \(f^{(m)}(x_0) = 0\) und \(f^{(n)}(x_0) \ne 0\).
Falls \(n\) ungerade, dann hat \(f\) in \(x_0\) kein lokales
Extremum.  Falls \(n\) gerade und \(f^{(n)}(x_0) >0\), so
hat \(f\) in \(x_0\) ein striktes lokales Minimum.  Falls \(n\)
gerade und \(f^{(n)}(x_0) <0\), so hat \(f\) in \(x_0\) ein
striktes lokales Maximum.

\satz{Integrierbarkeit}  Sei $f \colon \left[ a,b
\right] \to \mg{R}$ stetig oder monoton, dann ist \(f\)
integrierbar.

\satz{Mittelwertsatz der Integralrechnung}  Gegeben sei
die stetige Funktion $f \colon \left[ a, b \right] \to
\mg{R}\(.  Dann existiert ein \)\xi \in \left[ a, b
\right]$ sodass \(\int_a^b{f(x)dx}=f(\xi) (b-a)\).


\satz{Hauptsatz der Differential- und Integralrechnung}
Seien \(I \subseteq \mg{R}\) ein Intervall, $f \colon I
\to \mg{R}$ stetig und \(a \in I\).  Dann ist die Funktion
\begin{align*}
F \colon I \to \mg{R},\ x \mapsto F(x) = \int_a^x{f(t)dt}
\end{align*}
eine Stammfunktion von \(f\).  Ist \(G \colon I \to \mg{R}\)
eine weitere Stammfunktion von \(f\), so gibt es ein $c
\in \mg{R}\(, sodass für alle \(x \in I\) gilt \)G(x) = F(x)
+ c$.

\satz{Partielle Integration}  Seien $f,g \colon \left[
  a, b \right] \to \mg{R}$ stetig diffbar.  Dann gilt
\begin{align*}
\int_a^b{f(x)g'(x)dx}=(fg)(x) \Big|_a^b - \int_a^b{f'(x)g(x)dx}.
\end{align*}

\satz{Substitutionsregel}  Sei \(I \subseteq \mg{R}\) ein
Intervall, sei \(f \colon I \to \mg{R}\) stetig und sei $g
\colon \left[ a, b \right] \to I$ stetig diffbar.  Dann
gilt
\begin{align*}
\int_a^b{f(g(x)) g'(x)}dx = \int_{g(a)}^{g(b)}{f(x)}dx.
\end{align*}

\defn{Punktweise Konvergenz von Funktionenfolgen}  Sei
\(K\) eine Menge und seien \(f_n \colon K \to \mg{C}\), $ n
\in \mg{N}$, Funktionen.  Die Folge \((f_n)\) konvergiert
punktweise gegen eine Funktion \(f \colon K \to \mg{C}\),
falls für jedes \(x \in K\) die Folge \((f_n(x))\) gegen
\(f(x)\) konvergiert, d.h., wenn gilt: zu jedem \(x \in K\)
und \(\varepsilon > 0\) existiert ein $N = N(x,
\varepsilon)\(, sodass \)\left| f_n(x) - f(x) \right| <
\varepsilon$ für alle \(n \ge N\).

\defn{Gleichmäßige Konvergenz von Funktionenfolgen}  Die
Folge \((f_n)\) konvergiert gleichmäßig gegen eine
Funktion \(f \colon K \to \mg{C}\), falls gilt: zu jedem
\(\varepsilon > 0\) existiert ein \(N = N(\varepsilon)\),
sodass \(\left| f_n(x) - f(x) \right| < \varepsilon\) für
alle \(x \in K\) und alle \(n \ge N\).

\defn{Supremumsnorm} Sei \(K\) eine Menge und \(f \colon K \to \mg{C}\)
eine Funktion.  Dann setzt man
\begin{align*}
\| f \|_K := \sup \left\{ |f(x)|\colon x \in K\right\}.
\end{align*}

\defn{Gleichmäßigen Konvergenz, Umformung mit der Supremumsnorm}  Eine
Folge \(f_n \colon K \to \mg{C}\), \(n \in \mg{N}\), von Funktionen
konvergiert genau dannn gleichmäßig auf \(K\) gegen $f \colon K \to
\mg{C}$, wenn
\begin{align*}
\lim_{n \to \infty}{\|f_n - f\|_K} = 0.
\end{align*}
Noch zu ergänzen: Uneigentliche Integrale, Konvergenz
von Funktionenfolgen. test
\end{enumerate}
\end{document}