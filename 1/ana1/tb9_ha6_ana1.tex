\documentclass[12pt]{extarticle}
\usepackage{amsmath,mathtools,fontspec,amsthm,amssymb,amsfonts,fancyhdr,color,graphicx,}
\usepackage[margin=2.5cm]{geometry}
\usepackage[utf8]{inputenc}
\usepackage{xcharter-otf}
\usepackage[ngerman]{babel}
\usepackage[onehalfspacing]{setspace}
\usepackage{parskip}
\usepackage{tikz}
\renewcommand{\familydefault}{\sfdefault}
\pagestyle{fancy}
\setlength{\parindent}{0pt}
\lhead{Yuchen Guo 480788, Meng Zhang 484981, Chaeyoung Hong 478363}
\rhead{6. Hausaufgabenblatt, Ana I\\Tilman, Gruppe TB9}
\begin{document}
\textbf{H 6.1.i}

\textit{Behauptung.}  Die Reihe konvergiert gegen \(2\).
\begin{proof}
  Wir zeigen, dass das Cauchy-Produkt konvergiert gegen \(2\), indem wir
  zeigen, dass die Reihen \(\sum_{n=0}^{\infty}{\frac{1}{2^n}}\) und
  \(\sum_{n=0}^{\infty}{\frac{1}{(k+1)(k+2)}}\) konvergiert absolut
  gegen \(2\) und \(1\).

  Die Reihe \(\sum_{n=0}^{\infty}{\frac{1}{2^n}}\)  ist eine
  geometrische Reihe mit $\left|  q \right|  = \left|  \frac{1}{2}
  \right|  < 1$ und konvergiert damit absolut gegen \(\frac{1}{1-q}=2\).

  Wir betrachten die \(n\)-te Partialsumme \(s_n\) der Reihe
  \(\sum_{n=0}^{\infty}{\frac{1}{(k+1)(k+2)}}\).  Es gilt
\begin{align*}
  s_n &= \sum_{k=0}^n{\frac{1}{(k+1)(k+2)}}\\
      &= \sum_{k=0}^n{\frac{1}{k+1}-\frac{1}{k+2}}\\
      &=
        \frac{1}{0+1}-\frac{1}{0+2}
        +\frac{1}{1+1}-\frac{1}{1+2}
        +\frac{1}{2+1}-\frac{1}{2+2}+\ldots-\frac{1}{n+2}\\
      &= 1 - \frac{1}{n+2}
\end{align*}

Die Folge \((s_n)\) hat den Grenzwert \(1\).  Damit konvergiert die Reihe
\(\sum_{n=0}^{\infty}{\frac{1}{(k+1)(k+2)}}\) gegen \(1\).

Wegen Satz 3.32 konvergiert das Cauchy-Produkt dann gegen \(2\).
\end{proof}

\textbf{H 6.1.ii}

\textit{Behauptung.}  Die Reihe \(\sum_{n=0}^{\infty}{a_n}\) konvergent.

\begin{proof}
  Die Folge \((\left|  a_n \right| )_{n \in \mathbb{N}_{\geq 1}}\) ist eine monoton
  fallende Nullfolge.  Denn, es gilt $\lim_{n \rightarrow
    \infty}{\frac{1}{\sqrt{n}}}=0$ und
\begin{align*}
  \left|  a_n \right|  - \left|  a_{n+1} \right|
  &= \frac{1}{\sqrt{n}} - \frac{1}{\sqrt{n+1}}\\
  &= \frac{\sqrt{n+1}-\sqrt{n}}{\sqrt{n(n+1)}} \\
  &> 0.
\end{align*}

Aus dem Leibniz-Kriterium folgt, dass \(\sum_{n=0}^{\infty}{a_n}\) konvergiert.
\end{proof}

\textit{Behauptung.}  Das Cauchy-Produkt der Reihe $
\sum_{n=0}^{\infty}{a_n}$ divergiert.

\begin{proof}
Es gilt,
\begin{align*}
  \sum_{k=0}^{\infty}{\left( \sum_{l=0}^k{a_{k-l}a_l} \right)}
  &= \sum_{k=0}^{\infty}{\left( \sum_{l=0}^k{a_{k-l}a_l} \right)} \\
\end{align*}

???
\end{proof}

\textbf{H 6.2}

\textbf{H 6.3.i}

\textit{Behauptung.} Die Menge \(\mathbb{A}\) der
algebraischen Zahlen ist abzählbar.

\begin{proof}
  Sei \(M_n\) die Menge aller \(n\)-elementige Teilmenge
  von \(\mathbb{Q}\).  Wir zeigen, dass die Menge \(M_n\)
  abzählbar ist, mittels vollständigen Induktion.

  \textbf{Induktionsanfang:} \(n=1\).  Die Menge
  $M_1 = \left\{ \left\{ q_0 \right\}, \left\{ q_1
    \right\}, \left\{ q_2 \right\}, \ldots \right\}$
  mit \(q_1, q_2, \ldots \in \mathbb{Q}\) ist abzählbar,
  denn es eine Bijektion
  $f \colon \left\{ q  \mid  N \in M, q \in N\right\}
  \rightarrow \mathbb{Q}$ gibt.  Die Menge der
  rationalen Zahlen ist abzählbar, deshalb \(M_1\) ist
  auch abzählbar.

  \textbf{Induktionsannahme:} Die Menge
  $M_n= \{ \underbrace{\left\{ q_a, \ldots, q_b
      \right\}}_{n \text{ Elementen}}, \ldots,
    \underbrace{\left\{ q_c, \ldots, q_d \right\}}_{n
      \text{ Elementen}} \}$ ist abzählbar.

    \textbf{Induktionsschritt:} \(n \rightarrow n+1\).
    Die Menge \(M_{n+1}\) ist auch abzählbar, weil es
    gibt nur abzählbare viele Möglichkeiten, eine Zahl
    aus \(\mathbb{Q}\) wählen und in eine Teilmenge von
    \(M_n\) hinzufügen, also die Menge
    $\{ \underbrace{ N \cup \{q_{x}\}}_{n+1 \text{ Elementen}}  \mid  q_x \in
    \mathbb{Q}, N \in M_n \}$ ist abzählbar.  Wegen (Satz 3.44:
    Die Vereinigung abzählbar vieler abzählbarer Mengen
    ist abzählbar) ist die Menge \(M_{n+1}\) abzählbar. \qed

    Weil die Menge \(M_n\) abzählbar ist, dann gibt es
    für jede Teilmenge \(N_i \in M_n\) höchstens \(n\)
    verschiedene Lösungen für die Gleichung
    \(\sum_{k=0}^n{a_kx^k}\).  Also die Lösungsmenge
    \(\left| X_{N_i}  \right| \leq n\).  Weil \(M_n\) abzählbar ist, ist die
    Vereinigung abzählbare viele Teilmenge
    \(\bigcup_{i=0}^n{X_{N_i}}\) auch abzählbar.
    Außerdem ist die Menge
    \(Y= \left\{ M_1, M_2, \ldots, M_n \right\}\) auch
    abzählbar, weil es eine Bijektion
    \(f\colon \mathbb{N} \rightarrow Y\) gibt.
    Daraus folgt, dass die Vereinigung der Lösungsmenge
    abzählbar ist.
  \end{proof}

  \textbf{H 6.3.ii}

  \textbf{H 6.4}

  \textit{Lemma.} Es gibt zu jedem $a \in \left]0,
    1\right[$ ein \(b \in \left]0, 1\right[\) mit \(b > a\).

  \begin{proof}
  Sei also die Zahl \(a \in \left]0, 1\right[\) beliebig gewählt.  Wir
zeigen, dass es eine Zahl \(b \in \left]0, 1\right[\) mit \(b > a\)
existieren muss.

Angenommen, es existiert keine Zahl \(b \in \left]0, 1\right[\) mit $b >
a\(.  Dann gilt \(b \leq a\) für alle \(b \in \left]0, 1\right[\), also \)a =
\sup \left]0, 1\right[$.  Es gilt auch, dass \(a \in \left]0, 1\right[\), daraus folgt,
dass \(a = \max \left]0, 1\right[\).  Dies ist ein Widerspruch, denn \(\left]0, 1\right[\)
ist eine offene Intervall und hat daher kein Maximum.
  \end{proof}


  \textit{Behauptung.}  Es gibt eine überabzählbare
  Menge von paarweise disjunkten offenen und nicht leeren Intervallen.

  \begin{proof}

    Es gilt, dass die Intervall \(\left]0, 1\right[\)
    überabzählbar ist.  Daraus folgt, dass die Menge
    aller Intervallen \(\left]a, 1\right[\) mit
    \(a \in \left]0, 1\right[\) überabzählbar ist, denn
    es existiert eine bijektive Abbildung
    $f\colon \left]0, 1\right[ \rightarrow M, a \mapsto
    \left]a, 1\right[$.

    O.B.d.A seien die Mengen \(A:= \left]a, 1\right[\),
    \(B:=\left]b, 1\right[\) und \(C:= \left]c, 1\right[\)
    drei Teilmengen von \(\left]0, 1\right[\) mit
    \(a < b < c\).  Dann gilt $A \setminus B = \left]a,
      b\right[\( und \)B \setminus C = \left]b,
      c\right[\(. Die Mengen \(A \setminus B\) und \)B
    \setminus C$ sind paarweise disjunkte offene und
    nicht leere Intervallen.


    Wegen Lemma existiert zu jedem
    \(a \in \left]0, 1\right[\) ein \(b\) und \(c\) dass
    \(a < b < c\).  Also, es gibt eine überabzählbare
    Menge von paarweise disjunkten offenen und nicht
    leeren Intervallen innerhalb \( \left]0, 1\right[\).
\end{proof}

\textbf{Zusatzaufgabe 2}

\textit{Behauptung.}  Die Menge \(\mathcal{I}\) aller
induktiven Teilmengen von \(\mathbb{R}\) ist
überabzählbar.

\begin{proof}
  Wir zeigen, dass für alle \(x \in \left] 0, 1 \right[\)
  dass die Mengen
  $I_x:= \mathbb{N} \cup \left\{ x+n  \mid  n \in \mathbb{N}
  \right\}$ induktiv ist, per (Definition 1.17).

  \begin{itemize}
  \item Es gilt \(0 \in I_x\) wegen $\mathbb{N} \subseteq
    I_x$.
  \item Für alle \(a \in \mathbb{R}\) mit \(a \in I_x\)
    gilt, dass \(a \in \mathbb{N}\) oder
    \(a \in \left\{ x + n  \mid  n \in \mathbb{N}\right\}\).
    Falls \(a \in \mathbb{N}\), dann gilt \(a+1 \in I_x\),
    weil die Menge der natürlichen Zahlen \(\mathbb{N}\)
    induktiv ist. Falls
    \(a \in \left\{ x + n  \mid  n \in \mathbb{N}\right\}\),
    dann gilt \(a = x + n\) und
    $a + 1 = x + (n + 1) \in \left\{ x + n  \mid  n \in
      \mathbb{N}\right\}$, weil \(n+1 \in \mathbb{N}\).
  \end{itemize}

  Die Abbildung \(x \mapsto I_x\) mit
  \(x \in \left]0, 1 \right[\) ist injektiv, denn es gilt
  für alle \(x_1, x_2 \in \left]0, 1\right[\) mit
  \(x_1 \neq x_2\) dass
  $\left\{ x_1+n \mid n \in \mathbb{N} \right\} \cap
  \left\{ x_2+n \mid n \in \mathbb{N} \right\} =
  \emptyset$ und damit \(I_{x_1} \neq I_{x_2}\).

  Wir haben gezeigt, dass es eine Injektion
  \(x \mapsto I_x\) mit \(x \in \left] 0, 1 \right[\)
  existiert und \(I_x\) induktiven Teilmengen von
  \(\mathbb{R}\) sind.  Es gilt auch, dass
  \(\left] 0, 1 \right[\) überabzählbar, damit ist auch
  \(I_x\) überabzählbar. Weil \(I_x\) eine überabzählbare
  Teilmenge von \(\mathcal{I}\) ist, ist auch die Menge
  \(\mathcal{I}\) aller induktiven Teilmengen von
  \(\mathbb{R}\) überabzählbar.

\end{proof}
\end{document}