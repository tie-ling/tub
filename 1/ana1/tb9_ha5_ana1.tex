\documentclass[12pt]{extarticle}
\usepackage{amsmath,mathtools,fontspec,amsthm,amssymb,amsfonts,fancyhdr,color,graphicx,}
\usepackage[margin=2.5cm]{geometry}
\usepackage[utf8]{inputenc}
\usepackage{xcharter-otf}
\usepackage[ngerman]{babel}
\usepackage[onehalfspacing]{setspace}
\usepackage{parskip}
\usepackage{tikz}
\renewcommand{\familydefault}{\sfdefault}
\pagestyle{fancy}
\setlength{\parindent}{0pt}
\lhead{Yuchen Guo 480788, Meng Zhang 484981, Chaeyoung Hong 478363}
\rhead{5. Hausaufgabenblatt, Ana I\\Tilman, Gruppe TB9}
\begin{document}
\textbf{H 5.1.i}

Siehe nächste Seite.

\newpage

\textbf{H 5.1.ii}

\textit{Lemma.} Die Folge \((a_k)\) ist monoton steigend.

\begin{proof}
  Wir zeigen, dass die Folge \((a_k)\) monoton steigend ist, indem wir
  zeigen, dass \(a_{k+1}-a_k=1-\frac{1}{2}a_k>0\) gilt, mittels
  vollständigen Induktion.

  \textbf{Induktionsanfang:} \(a_1-a_0=1-\frac{1}{2}a_0=1>0\).

  \textbf{Induktionsannahme:} Es gilt \(a_{k+1}-a_k=1-\frac{1}{2}a_k>0\)
  für ein festes \(k \in \mathbb{N}\).

  \textbf{Induktionsschritt:} \(k \rightarrow k+1\). Es gilt
  \(a_{k+2}-a_{k+1}=1-\frac{1}{2}a_{k+1}=\frac{3}{4}-\frac{1}{4}a_k\).
  Wegen der Induktionsvoraussetzung gilt \(\frac{1}{2}>\frac{1}{4}a_k\).
  Daraus folgt, dass \(a_{k+2}-a_{k+1}>0\).
\end{proof}

\textit{Behauptung.}  Die Reihe \(\sum_{k=1}^{\infty}{a_k}\) divergiert.

\begin{proof}
  Wir zeigen, dass die Reihe divergiert, indem wir zeigen, dass die
  Folge \((a_k)\) keine Nullfolge ist und damit (Satz 3.4: Notwendiges
  Kriterium) nicht erfüllt.

  Wir haben gezeigt, dass \((a_k)\) monoton steigend ist.  Es gilt auch,
  dass \(a_1=\frac{1}{2}\).  Damit ist \((a_k)\) keine Nullfolge und die
  Reihe \(\sum_{k=1}^{\infty}{a_k}\) divergiert.
\end{proof}

\textbf{H 5.1.iii}

\textit{Behauptung.} Die Reihe \(\sum_{k=2}^{\infty}{\frac{1}{4k^2-1}}\)
konvergiert gegen \(\frac{1}{6}\).

\begin{proof}
  Sei die Folge \(a_k := \frac{1}{4k^2-1}\), dann gilt
\begin{align*}
  a_k&=\frac{1}{4k^2-1}\\
                &=\frac{1}{2(2k-1)(2k+1)}\\
                &=\frac{1}{2} \left( \frac{1}{2k-1} - \frac{1}{2k+1} \right)
\end{align*}

Daraus folgt, dass
\begin{align*}
  \sum_{k=2}^{\infty}{a_k} = \frac{1}{2} \left( \frac{1}{3}-\frac{1}{5}+\frac{1}{5}-\frac{1}{7}+\ldots \right)
\end{align*}
Damit konvergiert die Reihe gegen \(\frac{1}{6}\).
\end{proof}

\newpage

\textbf{H 5.2.i}

\textit{Behauptung.} Sei \((a_n)\) eine monoton fallende Nullfolge
nichtnegativer reeller Zahlen.  Die Reihe \(\sum_{k=1}^{\infty}{a_k}\)
konvergiert genau dann, wenn die Reihe
\(\sum_{k=1}^{\infty}{2^ka_{2^k}}\) konvergent ist.

\begin{proof}
  Wir zeigen, die zwei Aussage äquivalent sind, indem wir zeigen, dass
  die Hinrichtung und Rückrichtung wahr sind.
  \begin{itemize}
  \item Aus der Konvergenz von  \(\sum_{k=1}^{\infty}{a_k}\) folgt die
    Konvergenz von \(\sum_{k=1}^{\infty}{2^ka_{2^k}}\).

    Aus der Monotonie von \((a_n)\) folgt, dass
\begin{align*}
2^ka_{2^k} \leq 2\sum_{j=2^{k-1}+1}^{2^k}{a_j}
\end{align*}
für alle \(k \geq 2\) gilt.  D.h., es gilt
\begin{align*}
  2^2a_{2^2} &= a_4 + a_4 + a_4 + a_4\\
             &\leq 2(a_3+a_4).
\end{align*}
Falls \(k=1\), dann gilt \(2^1a^1=a_2+a_2\leq 2(a_1+a_2)\).  Daraus folgt,
es gilt
\begin{align*}
 \sum_{k=1}^{\infty}{2^ka_{2^k}} \leq 2\sum_{k=1}^{\infty}{a_k}.
\end{align*}
Weil \(\sum_{k=1}^{\infty}{a_k}\) konvergent ist, ist auch
\(2\sum_{k=1}^{\infty}{a_k}\) konvergent und damit
\(\sum_{k=1}^{\infty}{2^ka_{2^k}}\) konvergent.
  \item Aus der Konvergenz von \(\sum_{k=1}^{\infty}{2^ka_{2^k}}\) folgt die
    Konvergenz von \(\sum_{k=1}^{\infty}{a_k}\) .

    Aus der Monotonie von \((a_n)\) folgt, dass
\begin{align*}
\sum_{j =2^k}^{2^{k+1}-1}{a_j} \leq 2^ka_{2^k} \tag{1}
\end{align*}
für alle \(k \geq 1\) gilt.  D.h., es gilt
\begin{align*}
a_2+a_3 \leq 2^1a_{2^1}.
\end{align*}
Weil \(\sum_{k=1}^{\infty}{2^ka_{2^k}}\) konvergent ist, ist dann
\(\sum_{k=1}^{\infty}{a_k}\) konvergent wegen (1).
  \end{itemize}

\end{proof}

\textbf{H 5.2.ii}
\begin{proof}

  \begin{itemize}
  \item Falls \(q \leq 0\), dann ist die Folge
    \((\frac{1}{k^q})_{k \in \mathbb{N}_{\geq 1}}\) keine Nullfolge und
    die Reihe divergiert.
  \item Falls \(q > 1\).
    Es gilt,
\begin{align*}
  \sum_{k=1}^{\infty}{2^k \frac{1}{(2^k)^q}} &= \sum_{k=1}^{\infty}{(2^k)^{(1-q)}}.
\end{align*}

Wegen \(q > 1\), es gilt auch, dass
\begin{align*}
\limsup_{n\rightarrow\infty}{\left| \frac{a_{n+1}}{a_n}
  \right|}&=\frac{(2^k)^{(1-q)}\cdot 2^{(1-q)}}{(2^k)^{(1-q)}}\\
          &= 2^{(1-q)} \\
          &< 1.
\end{align*}
Damit konvergiert die Reihe $\sum_{k=1}^{\infty}{2^k
  \frac{1}{(2^k)^q}}$ absolut.  Aus (i) folgt dann, die ursprüngliche
Reihe konvergiert.
\item Falls \(q = 1\), dann ist die Reihe die harmonische Reihe und
  divergiert.
\item Falls \(0< q <1\), dann gilt
  $\liminf_{n\rightarrow\infty}{\left| \frac{a_{n+1}}{a_n}
    \right|}=2^{(1-q)}>1$ und damit divergiert die Reihe.
  \end{itemize}
\end{proof}

\newpage

\textbf{H 5.3}

Siehe nächste Seite.

\textbf{H 5.4}

Siehe nächste Seite.

\newpage

\textbf{H 5.5.i}

Sei \((a_n)\) eine Folge mit
\(\lim_{n\rightarrow\infty}{a_n}=a\), \(a \in \mathbb{R}\) und
\(s_n:=\frac{1}{n+1}\sum_{k=0}^n{a_k}\).

\textit{Behauptung.}  Es gilt \(\lim_{n\rightarrow\infty}{s_n}=a\).

\begin{proof}
  Sei \(\varepsilon > 0\) beliebig gewählt.  es existiert ein $N_{\varepsilon} \in
  \mathbb{N}$ sodass für alle \(n \geq N_{\varepsilon}\) gilt
\begin{align*}
\left| a_n-a \right|<\varepsilon. \tag{1}
\end{align*}
Aus (1) folgt, dass
\begin{align*}
  \sum_{k=N_{\varepsilon}}^n{\left| a_k - a \right|}
  < \left( n - N_{\varepsilon} + 1 \right)\varepsilon\\
  \frac{1}{n - N_{\varepsilon} + 1}\sum_{k=N_{\varepsilon}}^n
  {\left| a_k - a \right|}  < \varepsilon \tag{2}
\end{align*}
Wegen Dreiecksungleichung und (2) folgt, dass
\begin{align*}
\left| \frac{1}{n-N_{\varepsilon}+1}
  \sum_{k=N_{\varepsilon}}^n{(a_k-a)} \right|
  \leq \frac{1}{n-N_{\varepsilon}+1}\sum_{k=N_{\varepsilon}}^n{\left|
  a_k - a \right|} < \varepsilon.
\end{align*}
Insbesondere, es gilt
\begin{align*}
  \left| \frac{1}{n-N_{\varepsilon}+1}
  \sum_{k=N_{\varepsilon}}^n{(a_k)} - a \right|  < \varepsilon.
\end{align*}
Sei \(N_b = n-N_{\varepsilon}\), dann erhalten wir
\begin{align*}
  \left| \frac{1}{N_b+1}
  \sum_{k=0}^{N_b}{(a_k)} -
  \underbrace{\frac{1}{N_b+1}
  \sum_{k=0}^{N_{\varepsilon}-1}{(a_k)}}_{\text{(3)}}
  - a \right|  < \varepsilon.
\end{align*}
Die Formel (3) konvergiert gegen \(0\) wenn
\(N_b\rightarrow\infty\). Damit konvergiert \((s_n)\) gegen \(a\).
\end{proof}

\textbf{H 5.5.ii}

\textit{Behauptung.}  Aus der Divergenz von \((a_n)\) folgt die
Divergenz von \((b_n)\).

Diese Aussage ist falsch.  Wir widerlegen diese Aussage mit einem
Gegenbeispiel.

Sei \(a_n:=(-1)^n\).  Es gilt
\begin{align*}
  \sum_{k=0}^n{a_k}=\begin{cases}
                      0 & n \text{ ungerade,}\\
                      1 & \text{sonst.}
                    \end{cases}
\end{align*}
Sei dann \(\varepsilon > 0\) beliebig gewählt und
\(N_{\varepsilon} := \lceil \frac{1}{\varepsilon} + 1 \rceil\).  Dann
gilt für alle \(n \geq N_{\varepsilon}\), \(n \in \mathbb{N}\) dass
\(\left| b_n \right| < \varepsilon\). Es gilt im Widerspruch zur
Voraussetzung, dass \((b_n)\) gegen \(0\) konvergiert.
\end{document}