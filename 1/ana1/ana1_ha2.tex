\documentclass[12pt]{extarticle}
\usepackage{amsmath,mathtools,fontspec,amsthm,amssymb,amsfonts,fancyhdr,color,graphicx,}
\usepackage[margin=2.5cm]{geometry}
\usepackage[utf8]{inputenc}
\usepackage{xcharter-otf}
\usepackage[ngerman]{babel}
\usepackage[onehalfspacing]{setspace}
\usepackage{tikz}
\renewcommand{\familydefault}{\sfdefault}
\pagestyle{fancy}
\setlength{\parindent}{0pt}
\lhead{Yuchen Guo 480788, Meng Zhang 484981, Chaeyoung Hong 478363}
\rhead{2. Hausaufgabenblatt, Ana I\\Tilman,  Fr 10-12 a.m.}
\begin{document}
\textbf{H 2.1}
\begin{proof}
  \begin{align*}
    &\left( 1-a \right)^2 \sum_{k=1}^{2022}{ka^{k-1}}\\
    &= \left( 1+a^2-2a \right) \sum_{k=1}^{2022}{ka^{k-1}}\\
    &=
      \sum_{k=1}^{2022}{ka^{k-1}}+\sum_{k=1}^{2022}{ka^{k+1}}-\sum_{k=1}^{2022}{2ka^k}\\
    &=\sum_{k=0}^{2021}{(k+1)a^k}+\sum_{k=2}^{2023}{(k-1)a^k}-\sum_{k=1}^{2022}{2ka^k}\\
    &=1+\sum_{k=1}^{2021}{(k+1)a^k}+\sum_{k=1}^{2021}{(k-1)a^k}+2021a^{2022}+2022a^{2023}-\sum_{k=1}^{2021}{2ka^k}-4044a^{2022}\\
    &=1-2023a^{2022}+2022a^{2023}
  \end{align*}
\end{proof}
\textbf{H 2.2.1}
\vspace{4mm}


\textbf{Definition 1.17}
Eine Teilmenge \(M \subseteq \mathbb{R}\) heißt \textit{induktiv},
falls gilt:
\begin{itemize}
\item \(0 \in M\).
\item Für alle \(x \in \mathbb{R}\) gilt: $x \in M \implies x+1
  \in M$.
\end{itemize}
\begin{proof}

  Wir beweisen $M = \left\{ 0 \right\} \cup \left\{ x \in \mathbb{R}
    | x \geq 1 \right\}$ als induktiv, indem wir zeigen, dass die
  Menge \(M\) die Definition von induktiver Mengen erfüllt.

  \begin{itemize}
  \item Aus der Definition folgt, \(0 \in M\).
  \item Für alle \(x \in M\), gilt \(x + 1 \in M\).

    Wir beweisen diese Aussage mittels vollständiger Induktion
    über \(x \in M\).

    \begin{itemize}
    \item Induktionsanfang \(x = 0\).

      Wegen M3 der Körperaxiomen, existiert ein neutrales Element
      von Multiplikation \(1 \in \mathbb{R}\) mit der Eigenschaft
      \(x \cdot 1 = x\) für alle \(x \in \mathbb{R}\).

      Wegen A3 der Körperaxiomen, existiert ein neutrales Element
      von Addition \(0 \in \mathbb{R}\) mit der Eigenschaft
      \(x + 0 = x\) für alle \(x \in \mathbb{R}\).

      Daraus folgt
      \begin{align*}
        x+1 &= 0+1\\
            &= 1+0 \tag*{A2}\\
            &= 1 \tag*{A3}
      \end{align*}
      Wegen \(1 \geq 1\), es gilt \(x + 1 =1 \in M\).
    \item Induktionsschritt \(x \rightarrow x + 1\)

      Für alle \(x \in \left\{ x \in \mathbb{R} | x \geq 1 \right\}\) gilt
      \begin{align*}
        x \geq 1 &\implies x+1 \geq 1+1 \tag*{O3}\\
                 &\implies x+1 > 1 \tag*{O2}\\
                 &\implies x+1 \in M \tag*{Definition von \(M\)}
      \end{align*}

      Daraus folgt, für alle \(x \in M\) gilt: \(x = 0 \in M\) und $x \in M
      \implies x+1 \in M$.
    \end{itemize}
  \end{itemize}
  Die Menge \(M\) ist daher eine induktive Menge.
\end{proof}

\textbf{Lemma H2.2.1}  Für alle \(m, n \in \mathbb{N}\) mit \(m \geq n\)
gilt \(m-n\in \mathbb{N}\).

\begin{proof}
  Wir bemühen uns eine vollständige Induktion.

  \begin{itemize}
  \item Induktionsanfang \(m=n=0\)
\begin{align*}
m-n=0-0=0+(-0)=0+0=0 \in \mathbb{N}
\end{align*}
\item Induktionsannahme: Es gelte nun \(m-n \in \mathbb{N}\) für
  bestimmte \(m, n \in \mathbb{N}\).
\item Erster Induktionsschritt \(m \rightarrow m+1\), \(n \rightarrow n\)

  Aus \((m+1)-n=(m-n)+1\) folgt, dass \((m-n)+1 \in \mathbb{N}\), weil
  \(m-n \in \mathbb{N}\) gelte.

\item Zweiter Induktionsschritt \(m \rightarrow m+1\), $n \rightarrow n
  + 1$

  Aus \((m+1)-(n+1)=m-n\) folgt, dass \((m+1)-(n+1) \in \mathbb{N}\), weil
  \(m-n \in \mathbb{N}\) gelte.

\end{itemize}

Damit ist die Behauptung bewiesen.
  \end{proof}

  \textbf{H2.2}

  \textit{Behauptung.} Es existiert kein \(p \in \mathbb{N}\) mit $n < p
  < n+1$.

  \begin{proof}
    Wir beweisen die Negation der Behauptung als falsch.


    \vspace{4mm} Sei \(p \in \mathbb{N}\) mit \(n < p < n+1\) für alle
    \(n \in \mathbb{N}\).  Es folgt die Ungleichung \(0 < p-n < 1\).


    Insbesondere, es gilt \(p-n<1\) mit \(p, n \in \mathbb{N}\) und $p >
    n$.  Nach der Lemma muss \(p-n \in \mathbb{N}\) sein mit \(p-n<1\).

    Also bekommen wir \(p-n=0\) und daher \(n=p\).  Diese ist im
    Widerspruch zur Behauptung \(n<p\).

    Damit wurde die Behauptung als wahr gezeigt.
  \end{proof}
\vspace{2mm}


  \textbf{HA 2.3.1}

  Seien \(A, B \subseteq R\) beschränkte Mengen.  Wir
  definieren: \(A+B:= \left\{ a+b | a \in A, b \in B \right\}\) und $A
  \cdot B:= \left\{ ab| a \in A, b \in B \right\}$.

\vspace{4mm}

  \textit{Behauptung HA 2.3.1.}  Es gilt \(\sup (A+B)=\sup A + \sup B\).

  Diese Behauptung ist wahr.

  \begin{proof}
Nach Definition von Supremum gilt, für alle \(a \in A\), \(a \leq \sup A\)
und für alle \(b \in B\), \(b \leq \sup B\).

Addieren die zwei Ungleichungen, bekommen wir:  für alle \(a \in A\) und
\(b \in B\), gilt \(a+b \leq \sup A + b \leq \sup A + \sup B\).

D.h., \(\sup A + \sup B\) ist eine obere Schranke der Menge \(A + B\).

\vspace{4mm}

Wir zeigen, dass \(\sup A + \sup B\) die kleinste obere Schranke der
Menge \(A + B\) ist, indem wir zeigen, dass zu jedem \(\varepsilon > 0\)
gibt es ein \(x \in A+B\) mit \(s- \varepsilon <x\).

\vspace{4mm}

Sei \(\varepsilon > 0\) beliebig gewählt.  Seien
\(p = \sup A - \frac{\varepsilon}{2}\) und
\(q = \sup B - \frac{\varepsilon}{2}\).  Aus der Definition der Suprema
folgt, dass es gibt ein \(a_0 \in A\) mit \(p < a_0 \leq \sup A\) und ein
\(b_0 \in B\) mit \(q < b_0 \leq \sup B\).

Addieren wir die beide Ungleichungen, erhalten wir
\begin{align*}
\sup A + \sup B -\varepsilon < a_0 + b_0 \leq \sup A + \sup B
\end{align*}

mit \(a_0 + b_0 \in A + B\).  Damit gilt  \(\sup A + \sup B = \sup (A+B)\).
  \end{proof}

\vspace{4mm}

  \textbf{HA 2.3.2}

  \textit{Behauptung HA 2.3.2.}  Es gilt \(\sup (A \cdot B)=\sup A \cdot \sup B\).


  Diese Aussage ist falsch.  Wir widerlegen diese Aussage mit einem
  Gegenbeispiel.

\vspace{4mm}

  \begin{proof}
  Seien \(A = [-4,-2]\) und \(B=[-6,-2]\).  Es gilt \(\sup A = -2\) und
  \(\sup B = -2\) und \(\sup (A \cdot B)=24 \neq \sup A \cdot \sup B\).

  Damit ist die Aussage falsch.
  \end{proof}

  \textbf{HA 2.4}
\begin{center}
\textit{  Bei dieser Aufgabe könnten wir leider nicht auf einen richtigen
  Lösungsweg einigen und haben wir daher mehrer Alternativen
  hingeschrieben.}
\end{center}

  \textit{Behauptung.}  Sei \(A \subseteq \mathbb{R}\) nach unten
  beschränkt und \(a \in \mathbb{R}\) beliebig gewählt.
  Es gilt
\begin{align*}
  a = \inf A \iff a = \sup \left\{ c \in \mathbb{R} | x > c
       \text{ für alle } x \in A \right\}.
\end{align*}
      \textbf{Lösungsweg von Zhang.}
\begin{proof}
  Zu zeigen, dass \(a = \inf A = \sup M \)
  \(M = \left\{ c \in \mathbb{R} | x > c, \forall x \in A \right\}\) \(\Rightarrow\) M ist nach oben beschränkt \(\Rightarrow\) M hat ein Supremum (Axiom 1.49 Vollständigkeitsaxiom), sei dieses Supremum \(\sup M\)
  \(\Rightarrow \sup M \leq \inf A\) muss gelten. Wir zeigen jetzt mithilfe Widerspruchsbeweises, dass \(\sup M < \inf A\) nicht gilt
  Gelte \(\sup M < \inf A\), so müsste es ein \(\tilde{a}\) geben, mit \(\sup M < \tilde{a} < \inf A\), und \( \tilde{a} \notin A, \tilde{a} \notin M \)
  \(\tilde{a} \notin A \Rightarrow \tilde{a} \geq x\)
  \(\tilde{a} \notin M \Rightarrow \tilde{a} < x\)
  Das ist ein Widerspruch. Daher gilt nicht
  \(\sup M < \tilde{a} < \inf A\), es kann nur sein \(\sup M = \inf A\)


  \end{proof}
      \textbf{Lösungsweg von Guo.}
  \begin{proof}

    Zur Abkürzung setzen wir $M = \left\{ c \in \mathbb{R} | x > c
       \text{ für alle } x \in A \right\}$.

      \begin{itemize}
      \item Hinrichtung:

        Aus der Definition der größten unteren Schranke folgt, dass
        \(a \leq x\) für alle \(x \in A\).  Daraus folgt unmittelbar, dass
        jedes Element \(x \in A\) eine obere Schranke der Menge
        \(N_0 := \left\{ n | n \leq a \right\}\) ist.  Es gilt, $a = \sup
        N_0$, weil für alle \(n \in N_0\) die Aussage \( \leq a\) gilt.
        Wir machen hier eine Fallunterscheidung:
         \begin{enumerate}
             \item Falls \(a = \inf A \notin A\), dann gilt für alle
             \(n \in N_0: n \leq a\) und für alle \(c \in M: c \leq a\).
             Daraus folgt, dass \(a = \sup N_0 = \sup M\).
             \item Falls \(a = \inf A \in A\), dann gilt für
             \(N_1 := \left\{ n | n < a \right\}\), dass \(a = \sup N_1\).  Es
             gilt für alle \(c \in M: c < a\).  Daraus folgt, dass
             \(a = \sup N_1 = \sup M\).
         \end{enumerate}

      \item Rückrichtung:

        Wegen \(a = \sup M\), gilt für alle \(x \in M: x \leq a\).  Daraus
        folgt unmittelbar, dass jedes \(c \in M\) eine untere Schränke
        der Menge \(N_2 := \left\{ n | n \geq a \right\}\) ist.  Es
        gilt, \(a = \inf N_2\), weil für alle \(n \in N_2\) die Aussage $n
        \geq a$ gilt.

        Falls \(a = \sup M \in A\), dann gilt für alle
        \(n \in N_2: n \geq a\) und für alle \(x \in A: x \geq a\).
        Daraus folgt, dass \(a = \inf N_2 = \inf A\).

        Falls \(a = \sup M \notin A \), dann gilt für
        \(N_3 := \left\{ n | n > a \right\}\), dass \(a = \inf N_3\). Es
        gilt für alle \(x \in A: x > a\).  Daraus folgt, dass
        \(a = \inf N_3 = \inf A\).
      \end{itemize}

    \end{proof}
\end{document}