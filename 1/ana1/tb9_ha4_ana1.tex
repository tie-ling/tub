\documentclass[12pt]{extarticle}
\usepackage{amsmath,mathtools,fontspec,amsthm,amssymb,amsfonts,fancyhdr,color,graphicx,}
\usepackage[margin=2.5cm]{geometry}
\usepackage[utf8]{inputenc}
\usepackage{xcharter-otf}
\usepackage[ngerman]{babel}
\usepackage[onehalfspacing]{setspace}
\usepackage{parskip}
\usepackage{tikz}
\renewcommand{\familydefault}{\sfdefault}
\pagestyle{fancy}
\setlength{\parindent}{0pt}
\lhead{Yuchen Guo 480788, Meng Zhang 484981, Chaeyoung Hong 478363}
\rhead{4. Hausaufgabenblatt, Ana I\\Tilman, Gruppe TB9}
\begin{document}
\textbf{H 4.1.1}

\textit{Behauptung.}   Die Folge $a_n:= \sqrt{n}(\sqrt{n+1} -
\sqrt{n})$ konvergiert gegen \(\frac{1}{2}\).


\begin{proof}

  Wir zeigen, dass diese Folge eine Grenzwert besitzt.
  \begin{itemize}
  \item \((a_n)\) ist monoton steigend.

  Zuerst zeigen wir, dass \((a_n)\) monoton steigend ist.  Es gilt,
\begin{align*}
a_{n+1} - a_n &= \sqrt{n+1}(\sqrt{n+2}-\sqrt{n}) - 1.
\end{align*}
Wir lösen die Ungleichung \(a_{n+1} - a_n > 0\), dann erhalten wir
\(\left( \frac{1}{2(n+1)} \right)^2 < 1\).  Diese Ungleichung gilt für
alle \(n \in \mathbb{N}\).
\item Die Zahl \(\frac{1}{2}\) ist eine obere Schranke von \((a_n)\).

  Lösen die Ungleichung \(a_n < \frac{1}{2}\), dann erhalten wir
  \(\frac{1}{4} > 0\).  Deshalb  gilt \(a_n < \frac{1}{2}\) für alle $n
  \in \mathbb{N}$.
\item Die Zahl \(\frac{1}{2}\) ist die kleinste obere Schranke von
  \((a_n)\).

  Sei \(a < \frac{1}{2}\) die kleinste obere Schranke von \((a_n)\).  Also
  gilt \(a_n \leq a\) für alle \(n \in \mathbb{N}\).  Wir zeigen, dass
  solche \(a\) nicht existiert.

  Wir lösen die Ungleichung \(\sqrt{n(n+1)} - n \leq a\).  Es gilt,
\begin{align*}
  \sqrt{n(n+1)} - n &\leq a\\
  n(n+1) &\leq (a+n)^2 \\
  n(n+1) &\leq a^2+n^2+2an\\
  0 &\leq a^2 + (2a -1)n\\
  (1-2a)n &\leq a^2.
\end{align*}

Falls \(1-2a > 0\), dann gilt die Ungleichung \(\sqrt{n(n+1)} - n \leq a\)
nur mit \(n \leq \frac{a^2}{1-2a}\).  Also gilt
\(\sqrt{n(n+1)} - n \leq a\) nur für endlich viele \(n \in \mathbb{N}\).

Falls \(1-2a \leq 0\), dann widerspricht die Voraussetzung $a <
\frac{1}{2}$.

  \end{itemize}

Deshalb ist \((a_n)\) eine monoton steigende Folge mit \(\frac{1}{2}\) als
deren kleinste obere Schranke.  Wegen (Skript Satz 2.22) konvergiert
\((a_n)\) gegen \(\frac{1}{2}\).
\end{proof}

\textbf{H 4.1.2}

\textit{Behauptung.}   Die Folge \(b_{n+1}:= 2b_n - c(b_n)^2\)
konvergiert gegen \(\frac{1}{c}\).

\begin{proof}

Zuerst zeigen wir mittels vollständigen Induktion, dass \(0< cb_n<1\)
gilt.

\textbf{Induktionsanfang}: \(n=0\).
Es gilt \(0 < b_0 < \frac{1}{c}\) und \(c > 0\).  Daraus folgt, $0 < c b_0
< 1$.

\textbf{Induktionsannahme}: Für ein festes \(n \in \mathbb{N}\) gilt $0
< cb_n < 1$.

\textbf{Induktionsschritt}: \(n \rightarrow n+1\).
Es gilt \(cb_{n+1}=2cb_n-(cb_n)^2\).  Diese Polynom \(2x - x^2\) ist
monoton steigend in der Intervall \(\left] 0, 1\right[\) mit den
Wertbereich \(\left] 0, 1\right[\).  Daraus folgt, $0
< cb_{n+1} < 1$.

\vspace{3mm}

Aus \(0 < cb_n < 1\) folgt, \(0 < b_n < \frac{1}{c}\).  Die Folge ist dann
nach oben durch \(\frac{1}{c}\) beschränkt.  Es gilt auch, dass
\(\frac{b_{n+1}}{b_n}=2-cb_n\) und \(1 < \frac{b_{n+1}}{b_n} < 2\).  Die
Folge \((b_n)\) ist daher monoton steigend.  Daraus folgt, dass \((b_n)\)
konvergiert.

Wir können leider nicht zeigen, dass \(\frac{1}{c}\) die kleinste obere
Schranke von \((b_n)\) ist.
\end{proof}

\textbf{H 4.2}

\textit{Behauptung.}  Die Folge \((a_{\varphi(n)})_{n \in \mathbb{N}}\)
konvergiert gegen \(a\).
\begin{proof}
  Sei \(\varepsilon > 0\) beliebig gewählt.

  \begin{itemize}
  \item   Weil \((a_n)\) gegen \(a\)
  konvergiert, existiert ein \(N_{\varepsilon} \in \mathbb{N}\) sodass
  \(|a_n - a| < \varepsilon\) für alle
  \(n \in \left\{ x \in \mathbb{N} | x \geq N_{\varepsilon} \right\}\).
\item   Weil \(\varphi: \mathbb{N} \rightarrow \mathbb{N}\) eine injektive
  Abbildung ist, existiert zu jedem \(b \in \mathbb{N}\) ein
  \(l \in \mathbb{N}\) sodass \(\varphi(n) > b\) für alle $n \in \left\{ x
    \in \mathbb{N} | x > l\right\}$.

  \textit{Beweis.}  Wir bemühen uns um einen Widerspruchsbeweis.
  Angenommen, es existiert ein \(b \in \mathbb{N}\) sodass
  \(\varphi(n) \leq b\) für alle \(l, n \in \mathbb{N}\).  Die
  Definitionsbereich von \(\varphi\) ist \(\mathbb{N}\).  Es gilt
  \(\left| \mathbb{N} \right| > b\).  Daraus folgt,
  \(\varphi: \mathbb{N} \rightarrow \left\{ 0, \ldots, b \right\}\).
  Die Abbildung ist im Widerspruch zur Voraussetzung nicht
  injektiv. \qed
\item Sei \(l_{N_{\varepsilon}} \in \mathbb{N}\) mit
  \(\varphi(n) > N_{\varepsilon}\) für alle
  \(n \in \left\{ x \in \mathbb{N} | x > l_{N_{\varepsilon}} \right\}\).
  Dann gilt \(|a_{\varphi(n)} - a| < \varepsilon\) für alle
  \(n \geq l_{N_{\varepsilon}} + 1\).  Die Folge \((a_{\varphi(n)})\)
  konvergiert deshalb gegen \(a\).
  \end{itemize}
\end{proof}

\textbf{H 4.3}

\textit{Behauptung.} \((a_n)_{n\in \mathbb{N}}\) konvergiert gegen \(a\)
\(\iff\) Jede Teilfolge von \((a_n)\) besitzt eine gegen \(a\) konvergente
Teilfolge.

\begin{proof}
  Wir beweisen die zwei Aussagen als äquivalent, indem wir zeigen,
  dass die Hinrichtung und Rückrichtung wahr sind.

  \begin{itemize}
  \item Hinrichtung.

    Seien \((a_{n_k})\) eine Teilfolge von \((a_n)\) und \((a_{n_{k_j}})\)
    eine Teilfolge von \((a_{n_k})\).  Wir benutzen (Satz Ü 4.2)
    zweimal.  Nämlich, aus \(\lim_{n\rightarrow\infty}{a_n} = a\) folgt
    \(\lim_{n\rightarrow\infty}{a_{n_k}} = a\).  Aus
    \(\lim_{n\rightarrow\infty}{a_{n_k}} = a\) folgt
    \(\lim_{n\rightarrow\infty}{a_{n_{k_l}}} = a\).
  \item Rückrichtung.

    Wir zeigen, dass \((a_n)\) eventuell keine Häufungspunkte anders als
    \(a\) besitzen darf.  Wir bemühen uns dafür einen
    Widerspruchsbeweis.

    Angenommen, \(h \neq a\) sei eine Häufungspunkt von \((a_n)\).  Aus
    der Definition von Häufungspunkt folgt, dass es eine Teilfolge
    \((a_{n_k})\) von \((a_n)\) existiert mit
    \(\lim_{n\rightarrow\infty}{a_{n_k}}=h\).

    Wegen (Satz Ü 4.2), gilt, dass alle Teilfolge von \((a_{n_k})\)
    konvergiert gegen \(h\).  Wir haben aber vorausgesetzt, dass
    mindestens eine Teilfolge von \((a_{n_k})\) gegen \(a\) konvergieren
    muss mit \(a \neq h\).  Dies ist daher ein Widerspruch, und solche
    \(h\) existiert nicht.

    Weil alle Teilfolgen von  \((a_n)\) konvergieren gegen \(a\), folgt
    aus (Satz Ü 4.2) dass \((a_n)\) konvergiert gegen \(a\).
  \end{itemize}
\end{proof}

\textbf{H 4.4.1}

Seien \((a_n)\) und \((b_n)\) beschränkte Folgen reeller Zahlen.

\textit{Behauptung.}  Es gilt
\begin{align*}
\limsup_{n\rightarrow\infty}{a_n} = -\liminf_{n\rightarrow\infty}{(-a_n)}.
\end{align*}

\begin{proof}
  Wir verwenden die Bezeichnungen
  \(A_n:= \left\{ a_k: k \geq n \right\}\) und \(s_n:= \sup A_n\).

\vspace{3mm}

  Weil \((a_n)\) beschränkt ist, ist die Folge \((s_n)\) monoton fallend.
  Die Menge \(A_n\) ist nach oben durch \(s_n\) beschränkt.

\vspace{3mm}

  Nun definieren wir \(-A_n:= \left\{ -a_k: k \geq n \right\}\).  Die
  Menge \(-A_n\) ist nach unten durch \(-s_n\) beschränkt.  Also gilt
  \(\inf(-A_n)=-s_n\).  Daraus folgt, \(-\inf(-A_n)=s_n\) für alle
  \(n \in \mathbb{N}\).

\vspace{3mm}

  Daraus folgt,
  $\limsup_{n\rightarrow\infty}{a_n} =
  -\liminf_{n\rightarrow\infty}{(-a_n)}$.
\end{proof}

\textbf{H 4.4.1}

Seien \((a_n)\) und \((b_n)\) beschränkte Folgen reeller Zahlen.

\textit{Behauptung.}
???

\end{document}