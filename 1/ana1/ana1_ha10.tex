%% page style
\documentclass[12pt]{extarticle}
\usepackage[margin=2cm]{geometry}
\usepackage{fancyhdr,parskip}
\pagestyle{fancy}
\usepackage[onehalfspacing]{setspace}
\setlength{\parindent}{0pt}
\lhead{\myAuthor}
\rhead{\mySubject \ \myHausaufgaben. Übungsblatt \\ \myTutor}
\renewcommand*\familydefault{\sfdefault} %% Only if the base font of the document is to be sans serif

%% language
\usepackage[utf8]{inputenc}
\usepackage{xcharter-otf}
\usepackage[ngerman]{babel}

%% default packages
\usepackage{amsmath,mathtools,fontspec,amsthm,amssymb,amsfonts,
  stmaryrd, % for the lightning symbol used in proof by contraction
  tikz,     % used to draw diagrams
}


%% metadata
\newcommand{\myAuthor}{Yuchen Guo 480788 | Meng Zhang 484981 | TB9}
\newcommand{\myHausaufgaben}{10}
\newcommand{\mySubject}{Analysis}
\newcommand{\myTutor}{Tilman}


%% custom commands
\newcommand{\beh}{\textit{Behauptung.}\ }
\newcommand{\aufgn}[1]{\textbf{Aufgabe #1.}}
\newcommand{\mg}[1]{\mathbb{#1}}
\newcommand{\mc}[1]{\mathcal{#1}}
\newcommand{\Real}{\operatorname{Re}}
\newcommand{\Imag}{\operatorname{Im}}
\begin{document}
\aufgn{10.2}

Es sei
$\mc{S} := \left\{ z \in \mg{C} \colon \left| z \right|
  = 1 \right\}$ und
$\mc{E} := \left\{ z \in \mg{C} \mid \exists n \in
  \mg{N} \colon z^n = 1 \right\}$

\aufgn{10.2.i}

\beh Es gilt \(\mc{E} \subseteq \mc{S}\) und zu
jedem \(z \in \mc{S}\) existiert eine Folge \((z_k)\) in
\(\mc{E}\) mit \(\lim_{k \to \infty}{z_k} = z\).

\begin{proof}
  Es gilt \(\mc{E} \subseteq \mc{S}\).  Denn sei
  \(z \in \mc{E}\) beliebig gewählt.  Wegen (Satz
  5.40) und (Satz 5.38) gibt es ein
  \(r \in \left] 0, \infty \right[\) und \(\varphi \in \mg{R}\)
  mit \(z = re^{i \varphi}\).  Wegen
  $\left| z^n \right| = \left| z \right|^n = \left|
    r^n \right| = 1$ folgt
  \(r = \left| z \right| = 1\) und \(z \in \mc{S}\).

  Zu jedem \(z \in \mc{S}\) existiert eine Folge \((z_k)\)
  in \(\mc{E}\) mit \(\lim_{k \to \infty}{z_k} = z\).  Denn
  sei \(z \in \mc{S}\) beliebig gewählt und sei die
  Lösungsmenge
  \(M_n := \left\{ w \in \mg{C} | w^n = 1\right\}\) für
  alle \(n \in \mg{N}\) definiert.  Sei \(\varepsilon > 0\)
  beliebig gewählt.  Als \(n \to \infty\) existiert in der
  Menge \(M_n\) solche \(w\) sodass $\left| z - w \right| <
  \varepsilon$ für alle \(n \ge N_{\varepsilon}\).  Daraus
  folgt \(\lim_{k \to \infty}{z_k} = z\).
\end{proof}

\aufgn{10.2.ii}

\beh Die stetige Funktionen \(f, g\) hat die Eigenschaft
\(f\big|_{\mc{E}} = g\big|_{\mc{E}}\).  Dann folgt \(f=g\).

\begin{proof}
  Weil es zu jedem \(z \in \mc{S}\) eine Folge \((z_k)\) in
  \(\mc{E}\) existiert mit \(\lim_{k \to \infty}{z_k}=z\),
  gilt insbesondere zu jedem \(z \in \mc{S}\) wegen
  \(f\big|_{\mc{E}} = g\big|_{\mc{E}}\) dass
\begin{align*}
\lim_{k \to \infty}{f(z_k)} = \lim_{k \to \infty}{g(z_k)}.
\end{align*}
Weil \(f,g\) stetig in \(\mc{S}\) sind, gilt zu jedem
\(z \in \mc{S}\) dass
\begin{align*}
  \lim_{k \to \infty}{f(z_k)}
  &= \lim_{k \to \infty}{g(z_k)} \\
  f \left( \lim_{k \to \infty}{z_k} \right)
  &= g \left( \lim_{k \to \infty}{z_k} \right) \\
 f(z) &= g(z).
\end{align*}
Daraus folgt \(f = g\).
\end{proof}

\aufgn{10.3}

Sei \(g \colon \mg{R} \to \mg{R}\) eine beschränkte
Funktion.

\aufgn{10.3.i}

\beh Die Funktion $f_1 \colon \mg{R} \to \mg{R}, \quad
f_1(x) := x \cdot g(x)$ ist stetig aber nicht
diffbar in \(x_0 = 0\).

\begin{proof}
  Zuerst zeigen wir, dass die Funktion stetig ist.  Weil
  die Funktion \(g \colon \mg{R} \to \mg{R}\) beschränkt
  ist, gibt es solche \(a, b \in \mg{R}\) sodass
  \(a \le g(x) \le b\) für alle \(x \in \mg{R}\).  Sei dann
  \(\varepsilon > 0\) beliebig gewählt und sei
  \(x < \frac{\varepsilon}{b}\).  Daraus folgt, dass
\begin{align*}
  \left| f(x) - f(0) \right|
  &= \left| x \cdot g(x) \right| \\
  &< \varepsilon.
\end{align*}
Nun zeigen wir, dass die Funktion nicht diffbar ist.
Sei die Funktion
\begin{align*}
  g(x) =
\begin{cases}
  1 &\text{falls } x \ge 0, \\
  -1 &\text{sonst.}
\end{cases}
\end{align*}
Dann ist \(g\) nach oben und nach unten beschränkt und es
gilt
\begin{align*}
f_1(x)=
\begin{cases}
  x &\text{falls } x \ge 0, \\
  -x &\text{sonst.}
\end{cases}
\end{align*}
Die Funktion \(f\) ist nicht diffbar in \(x_0 = 0\), denn es
gilt
\begin{align*}
  \lim_{x \to 0^-}{f_1(x)}
  = -1 \ne 1 = \lim_{x \to 0^+}{f_1(x)}.
\end{align*}
\end{proof}

\aufgn{10.3.ii}

\beh Die Funktion $f_2 \colon \mg{R} \to \mg{R}, \quad
f_2(x) := x^2 \cdot g(x)$ ist stetig und diffbar.
\end{document}