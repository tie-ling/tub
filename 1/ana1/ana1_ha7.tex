%% page style
\documentclass[12pt]{extarticle}
\usepackage[margin=2cm]{geometry}
\usepackage{fancyhdr,parskip}
\pagestyle{fancy}
\usepackage[onehalfspacing]{setspace}
\setlength{\parindent}{0pt}
\lhead{\myAuthor}
\rhead{\mySubject \ \myHausaufgaben. Übungsblatt \\ \myTutor}
\renewcommand*\familydefault{\sfdefault} %% Only if the base font of the document is to be sans serif

%% language
\usepackage[utf8]{inputenc}
\usepackage{xcharter-otf}
\usepackage[ngerman]{babel}

%% default packages
\usepackage{amsmath,mathtools,fontspec,amsthm,amssymb,amsfonts,
  stmaryrd, % for the lightning symbol used in proof by contraction
  tikz,     % used to draw diagrams
}


%% metadata
\newcommand{\myAuthor}{Yuchen Guo 480788 | Meng Zhang 484981 | TB9}
\newcommand{\myHausaufgaben}{7}
\newcommand{\mySubject}{Analysis}
\newcommand{\myTutor}{Tilman}


%% custom commands
\newcommand{\beh}{\textit{Behauptung.}\ }
\newcommand{\aufgn}[1]{\textbf{Aufgabe #1.}}
\newcommand{\mg}[1]{\mathbb{#1}}

\begin{document}
\aufgn{7.1.ii}

\beh Die Funktion \(h(x)\) ist stetig in $\left]a,
  b\right[\( mit \(a \in \mg{Q}\) und \)b:= \min \left\{ x
  \in \mg{Q} \mid x > a \right\}$.

\begin{proof}
  Es gilt wegen der Definition des Intervalls dass
  $\left] a, b \right[ \subseteq (\mg{R} \setminus
  \mg{Q})$.  Es gilt wegen der Definition der Funktion
  \(h\) dass \(h(x)=1-x\) für \(x \in \left]a, b\right[\).
  Die Funktion \(h(x)\) ist ein Polynomfunktion und ist
  deshalb stetig.
\end{proof}

\aufgn{7.3}

\begin{proof}
  Angenommen, es gilt \(f(\xi) \ne f(\xi + 1/2)\) für
  alle \(\xi \in \left[0,1/2\right]\).

  Falls es gilt für alle \(\xi \in \left[0,1/2\right]\)
  dass \(f(\xi) > f(\xi + 1/2)\), dann gilt insbesondere
  \(f(0) > f(1/2) > f(1)\) im Widerspruch zur
  Voraussetzung \(f(0)=f(1)\).

  Falls es gilt für alle \(\xi \in \left[0,1/2\right]\)
  dass \(f(\xi) < f(\xi + 1/2)\), dann gilt insbesondere
  \(f(0) < f(1/2) <  f(1)\) im Widerspruch zur
  Voraussetzung \(f(0)=f(1)\).

  Falls es gibt \(\xi_2<\xi_1 \in \left[ 0, 1/2 \right]\)
  mit \(\xi_2 < f(\xi_2 + 1/2)\) und
  \(f(\xi_1 + 1/2) < \xi_1\) dann es existiert(?) ein
  \(\xi \in \left[ \xi_2, \xi_1 \right]\) mit
  \(f(\xi + 1/2)=\xi\) im Widerspruch zur Voraussetzung.
\end{proof}

\aufgn{7.4}

Sei \(f \colon \left[a, b\right] \to \left[a, b \right]\)
stetig.

\aufgn{7.4.i}


\textbf{Lemma.} \beh Falls es gibt
\(x_2<x_1 \in \left[ a, b \right]\) mit
\(x_2 < f(x_2) < f(x_1) < x_1\) dann es existiert ein
\(x \in \left[ f(x_2), f(x_1) \right]\) mit \(f(x)=x\).

  \begin{proof}
    Wir beweisen die Lemma mit der
    Intervallhalbierungsmethode.  Dazu konstruieren wir
    per Induktion Intervalle $I_n = \left[ a_n, b_n
    \right]$ sodass \(I_{n+1} \subseteq I_n\), sowie
\begin{align*}
b_n - a_n = \frac{(b-a)}{2^n} \quad \text{und} \quad
  f(a_n) \leq x \leq f(b_n) \tag*{\(\star\)}
\end{align*}
für alle \(n \in \mg{N}\) gilt.

\textbf{Induktionsanfang:} \(n=0.\) Setze \(a_0:=x_2\) und
\(b_0:=x_1\).  Dann gilt \(b_0 - a_0 = b - a\) und
\(f(a_0) < x < f(b_0)\).

\textbf{Induktionsschritt:} \(n \to n+1.\)  Seien $a_n,
b_n$ bereits konstruiert, sodass \(\star\) erfüllt ist.
Setze
\begin{align*}
  \begin{cases}
    a_{n+1} = a_n, \qquad b_{n+1}=\frac{a_n+b_n}{2}
                            \quad & \text{falls } f((a_n+b_n)/2) \ge x,\\
    a_{n+1} = \frac{a_n+b_n}{2}, \qquad b_{n+1}=b_n
                                          \quad &\text{sonst}.
  \end{cases}
\end{align*}
Dann gilt \(b_{n+1}-a_{n+1}=\frac{b-a}{2^{n+1}}\) und
\(f(a_{n+1})\leq x \leq f(b_{n+1})\).  Nach dem
Intervallschachtelungsprinzip (Satz 2.37) existiert
wegen \(I_{n+1} \in I_n\) für alle \(n \in \mg{N}\) und
\(\lim_{n \to \infty}{(b_n - a_n)} = 0\) ein $x \in
\mg{R}$, sodass
\begin{align*}
\bigcap_{n \in \mg{N}}{I_n} = \left\{ x \right\}
\end{align*}
gilt und die beiden Folgen \((a_n)\) und \((b_n)\) der
Intervallränder gegen \(x\) konvergieren.  Da außerdem
\(f\) stetig ist, erhalten wir damit
\begin{align*}
f(x) = f \left( \lim_{n \to \infty}{a_n} \right) =
  \lim_{n \to \infty}{f(a_n)} \leq x \leq \lim_{n \to
  \infty}{f(b_n)} = f \left( \lim_{n \to \infty}{b_n}
  \right) = f(x),
\end{align*}
d.h. \(f(x) = x\).  Wegen \(f(a), f(b) \ne x\) und $x \in
I_0$ folgt \(x \in \left] a, b \right[\).
  \end{proof}

\aufgn{7.4.i}

\beh Es existiert ein \(x \in \left[ a, b \right]\) mit
\(f(x) = x\).

\begin{proof}
  Seien \(p, q \in \left[ a, b \right]\) mit
  $f(p):=\min \left\{ f(x) \mid x \in \left[ a, b \right]
  \right\}$ und
  $f(q):=\max \left\{ f(x) \mid x \in \left[ a, b \right]
  \right\}$. Es gilt wegen der Voraussetzung dass
\begin{align*}
a \le f(p) \le f(q) \le b.
\end{align*}
Angenommen, es gilt für alle $x \in \left[ a, b
\right]$ dass \(f(x) < x\) oder \(f(x) > x\).
Falls \(f(x) > x\) für alle \(x \in \left[ a, b \right]\).
Dann gilt insbesondere \(f(b) > b\) im Widerspruch zur
\(f(b) \in \left[ a, b \right]\).
Falls \(f(x) < x\) für alle \(x \in \left[ a, b \right]\).
Dann gilt insbesondere \(f(a) < a\) im Widerspruch zur
\(f(a) \in \left[ a, b \right]\).

Falls es gibt \(x_1 \in \left[ a, b \right]\) mit
\(f(x_1) < x_1\) und \(x_2 \in \left[ a, b \right]\) mit
\(f(x_2) > x_2\) dann es existiert wegen Lemma ein
\(x\) mit \(x_2 < x < x_1\) und \(f(x)=x\) im Widerspruch
zur Voraussetzung.

\end{proof}

\aufgn{7.4.ii}

Der Fixpunkt ist nicht immer eindeutig bestimmt.  Denn,
es seien \(x_1, x_2, x_3, x_4 \in \left[ a, b \right]\)
mit \(x_1<x_2<x_3<x_4\) und \(f(x_1)<x_1\), \(f(x_2)>x_2\),
\(f(x_3)<x_3\) und \(f(x_4)>x_4\),  dann können mindestens
zwei Fixpunkte \(x_a, x_b\) mit \(x_1<x_a<x_2\) und
\(x_3<x_b<x_4\) mittels Intervallhalbierungsmethode
gefunden werden.

\aufgn{7.4.iii}

\beh Sei \(f \colon \mg{R} \to \mg{R}\) stetig und
beschränkt.  Daraus folgt, dass \(f\) einen Fixpunkt
hat.

\begin{proof}
  Weil \(f \colon \mg{R} \to \mg{R}\) stetig und
  beschränkt ist, gibt es \(a, b \in \mg{R}\) sodass für
  alle \(x \in \mg{R}\) gilt
\begin{align*}
a \le f(x) \le b.
\end{align*}
Sei nun \(a\) die größte und \(b\) die kleinste solche
Zahl.  Angenommen, es gilt für alle \(x \in \mg{R}\) dass
\(f(x) < x\) oder \(f(x) > x\).  Falls \(f(x) > x\) für alle
\(x \in \mg{R}\).  Dann gilt insbesondere \(f(b) > b\) im
Widerspruch zur \(f(b) \le b\).  Falls \(f(x) < x\) für alle
\(x \in \mg{R}\).  Dann gilt insbesondere \(f(a) < a\) im
Widerspruch zur \(f(a) \ge a\).

Falls es gibt \(x_1 \in \mg{R}\) mit
\(f(x_1) < x_1\) und \(x_2 \in \mg{R}\) mit
\(f(x_2) > x_2\) dann es existiert wegen Lemma ein
\(x\) mit \(x_2 < x < x_1\) und \(f(x)=x\) im Widerspruch
zur Voraussetzung.

\end{proof}

\aufgn{Zusatz 3}

\beh Seien \(f, g \colon \left[a, b\right] \to \mg{R}\)
 stetige Funktionen mit \(f(x)^2=g(x)^2 > 0\) für alle
\(x \in \left[a, b\right]\).  Es gilt \(f = g\) oder \(f = -g\).

\begin{proof}
  Angenommen, es gilt \(f \ne g\) und \(f \ne -g\).  Das
  heißt, es existiert \(m, n \in \left[a, b\right]\) mit
  \(f(m) \ne g(m)\) und \(f(n) \ne -g(n)\).  Sei die
  Funktion \(h(x):=x^2\).  Diese Funktion ist eine
  Polynomfunktion und ihre Stetigkeit folgt aus (Satz
  4.8).  Die Funktionen \(h \circ f\) und \(h \circ g\)
  sind auch stetig wegen der Stetigkeit von \(f\) und \(g\).

Wegen  \(f(x)^2=g(x)^2 > 0\) für alle
\(x \in \left[a, b\right]\) gilt
\begin{align*}
  \left( \frac{f(x)}{g(x)} \right)^2 &= 1\\
  f(x) &= \pm g(x).
\end{align*}

Wegen \(f(m) \ne g(m)\) dann gilt \(f(x) = -g(x)\) für alle
\(x \in \left[a, b\right]\) im Widerspruch zur Annahme.
\end{proof}

\end{document}