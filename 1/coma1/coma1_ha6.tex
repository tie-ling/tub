\documentclass[12pt]{extarticle}
\usepackage{amsmath,mathtools,fontspec,amsthm,amssymb,amsfonts,fancyhdr,color,graphicx,}
\usepackage[margin=2.5cm]{geometry}
\usepackage[utf8]{inputenc}
\usepackage{xcharter-otf}
\usepackage[ngerman]{babel}
\usepackage[onehalfspacing]{setspace}
\usepackage{tikz}
\renewcommand{\familydefault}{\sfdefault}
\usepackage{parskip}
\pagestyle{fancy}
\setlength{\parindent}{0pt}
\lhead{Yiwen Yang 466096, Qing Wang 458040, Yuchen Guo 480788}
\rhead{CoMa 6. Übungsblatt\\Tutor: Lino Mo 4-5.30pm}
\author{HA Gruppe 103: \\Yiwen Yang 466096\\Qing Wang 458040\\Yuchen Guo 480788\\Tutor: Lino Mo 4-5.30pm}
\title{6. Übungsblatt\\\small CoMa I WS 22/23}
\date{}
\begin{document}
\maketitle
\newpage
\textbf{1. Aufgabe (a)}

\textit{Behauptung.}  Die Aussage gilt für jeden Baum auf \(n\) Knoten
\(\implies\) Die Aussage gilt für jeden einfachen, ungerichteten,
zusammenhängenden Graphen auf \(n\) Knoten.

\begin{proof}
  Aus der Definition von Baum folgt, dass ein Baum ein ungerichteter,
  zusammenhängender Graph ohne Kreise ist.

  Nun fügen wir Kanten hinzu, sodass dieser ungerichteter,
  zusammenhängender Graph nicht mehr ein Baum ist.  Wir haben aber
  vorausgesetzt, dass es einen Knoten in dem Baum gibt, dessen Abstand
  von allen anderen Knoten höchstens \(\frac{n}{2}\) ist.  Der Abstand
  kann nur durch das Hinzufügen von Kanten verkürzt werden.  Daraus
  folgt, dass die Aussage für jeden einfachen, ungerichteten,
  zusammenhängenden Graphen auf \(n\) Knoten gilt.
\end{proof}

\textbf{1. Aufgabe (b)}

\textit{Behauptung.}  Es gibt in jedem einfachen, ungerichteten,
zusammenhängenden Graphen \(G = (V, E, \Psi)\) ohne Kreise (Baum) mit \(n\) Knoten einen
Knoten, dessen Abstand von allen anderen Knoten höchstens
\(\frac{n}{2}\) ist.

\begin{proof}
  Wir bemühen uns um einen Widerspruchsbeweis.

  Falls \(n\) gerade.  Angenommen, der Abstand zwischen allen Knoten im
  Baum \(G\) ist mindestens \(\frac{n}{2}+1\).  Es gibt insgesamt $\left|
    V(G) \right|-1=n-1$ Kanten. Im Fall \(n=2\) dann gibt es nur \(2\)
  Knoten und \(1\) Kante.  Der Abstand zwischen allen Knoten ist \(1\) im
  Widerspruch zur Annahme \(\geq 2\).

  Falls \(n\) ungerade.  Angenommen, der Abstand zwischen allen Knoten
  im Baum \(G\) ist mindestens \(\lceil\frac{n}{2}\rceil\). Es gibt insgesamt $\left|
    V(G) \right|-1=n-1$ Kanten. Im Fall \(n=3\) dann gibt es nur \(3\)
  Knoten und \(2\) Kante.  Der Abstand zwischen den mittleren Punkt und
  den beiden Punkt am Rand ist \(1\) im Widerspruch zur Annahme \(\geq 2\).
\end{proof}
\end{document}
