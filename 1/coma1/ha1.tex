\documentclass[12pt]{extarticle}
\usepackage{amsmath,mathtools,fontspec,amsthm,amssymb,amsfonts,fancyhdr,color,graphicx,}
\usepackage[margin=2.5cm]{geometry}
\usepackage[utf8]{inputenc}
\usepackage{xcharter-otf}
\usepackage[ngerman]{babel}
\usepackage[onehalfspacing]{setspace}
\usepackage{tikz}
\renewcommand{\familydefault}{\sfdefault}
\pagestyle{fancy}
\setlength{\parindent}{0pt}
\lhead{Yiwen Yang 466096, Qing Wang 458040, Yuchen Guo 480788}
\rhead{CoMa 1. Übungsblatt\\Tutor: Tino Mo 4-5.30pm}
\author{HA Gruppe 103: \\Yiwen Yang 466096\\Qing Wang 458040\\Yuchen Guo 480788\\Tutor: Tino Mo 4-5.30pm}
\title{1. Übungsblatt\\\small CoMa I WS 22/23}
\date{}
\begin{document}
\maketitle
\newpage
\textbf{1. Aufgabe (a)}
Siehe oben.
\textbf{1. Aufgabe (b)}
\begin{itemize}
\item \([4]_{R}=\{4,-4\}\)
\item \([3]_{R}=\{3,-3\}\)
\item \([-2]_{R}=\{-2,2\}\)
\item \([1]_{R}=\{1,-1\}\)
\item \([0]_{R}=\{0\}\)
\end{itemize}
\textbf{2. Aufgabe}
\begin{enumerate}
\item Diese Relation ist eine Abbildung, weil es für jedes \(a\) genau
  ein \(b=|a|\) gibt.

  Diese Relation ist nicht injektiv, weil es für \(b>0\) zwei Lösungen,
  nämlich \(b=|a|\) und \(b=|-a|\) gibt.

  Diese Relation ist nicht bijektiv, weil es für \(b<0\) keine Lösung
  gibt.

  Diese Relation ist nicht bijektiv, weil diese nicht injektiv und
  nicht bijektiv ist.
\item Diese Relation ist keine Abbildung, weil es für jedes \(a>0\) zwei
  Werte gibt, nämlich \(b=a\) und \(b=-a\).
\item Diese Relation ist keine Abbildung, denn für alle ungerade \(a\)
  ist keine \(b\) in \(\mathbb{Z}\) zugeordnet.
\item Siehe unten.
\end{enumerate}
\textbf{Aufgabe 3.(a)}
\begin{proof}
  Seien \(x_1,x_2 \in X\) mit \((g \circ f)(x_1)=(g \circ f)(x_2)\).
  Aus der Injektivität von \(g\) folgt \(f(x_1)=f(x_2)\).
  Aus der Injektivität von \(f\) folgt \(x_1=x_2\).

  D.h., dass  \((g \circ f)(x_1)=(g \circ f)(x_2) \implies x_1=x_2\) und
  damit  \((g \circ f)\) injektiv ist.
\end{proof}

\textbf{Aufgabe 3.(b)}


\(A\): \(g \circ f\) bijektiv.

\(B\): \(g\) surjektiv.

\(C\): \(f\) injektiv.

Diese Aussage enthält zwei Aussagen.  Nämlich: \(A \implies B\) und $A
\implies C$.  Wir beweisen die Aussagen mit Kontraposition.

\(\neg B \implies \neg A\)
\begin{proof}
  Sei \(g\) nicht surjektiv.  Dann existiert ein Element \(z \in Z\), für
  dieses Element es kein Urbild \(y \in Y\) mit \(g(y)=z\) gibt.

  D.h., für alle \(x \in X\) gilt \(g(f(x)) \neq z\) mit \(z \in Z\).  Deshalb
  ist \(g \circ f\) nicht bijektiv.

  Damit ist \(\neg B \implies \neg A\) bewiesen. Deshalb gilt \(A \implies B\).
\end{proof}

\(\neg C \implies \neg A\)
\begin{proof}
  Sei \(f\) nicht injektiv.  Dann existiert zwei Elemente \(x_1,x_2 \in X\)
  mit \(x_1 \neq x_2\) und \(f(x_1)=f(x_2)\).

  D.h., für \(x_1 \neq x_2\) gilt es \((g \circ f)(x_1)=(g \circ f)(x_2)\),
  also \(g \circ f\) ist nicht bijektiv.

  Damit ist \(\neg C \implies \neg A\) bewiesen. Deshalb gilt \(A \implies C\).
\end{proof}

\textbf{Aufgabe 3.(c)}

Diese Aussage ist falsch, denn die Abbildung \(f\) muss nicht
unbedingt surjektiv sein, um die Voraussetzung zu erfüllen.

Wir widerlegen diese Aussage mit einem Gegenbeispiel.

Seien Funktionen \(f(x)= \left| x \right|\),
\(f: \mathbb{R} \rightarrow \mathbb{R}\) und \(g(x) = x^2\),
\(g: \mathbb{R} \rightarrow \mathbb{R}_{\geq 0}\).

Die Funktion \(f\) ist nicht injektiv und nicht surjektiv, denn für alle
\(y \in \mathbb{R}, y < 0\) gibt es keine Lösung und für alle
\(y \in \mathbb{R}, y > 0\) gibt es genau zwei Lösungen.

Die Funktion \(g\) ist surjektiv, denn für alle
\(y \in \mathbb{R}_{\geq 0}\) gibt es mindestens eine Lösung
\(x \in \mathbb{R}\), nämlich \(x=\pm \sqrt{y}\).

Die Funktion \(g \circ f\) ist surjektiv, denn für alle
\(y \in \mathbb{R}_{\geq 0}\) gibt es mindestens eine
Lösung \(x \in \mathbb{R}\) mit
\((g \circ f)(x)= y\).

Deshalb ist die Behauptung, dass \(f\) surjektiv wenn \(g \circ f\) und
\(g\) surjektiv sind, falsch.

\textbf{Aufgabe 4.(a)}


Siehe unten.

\textbf{Aufgabe 4.(b)}


Für die Mengen \(D:=\left\{ x | x \in \mathbb{Z}, x \leq -2 \right\}\)
und \(D:=\left\{ x| x \in \mathbb{Z}, x \geq -2 \right\}\) ist $f: D
\rightarrow \mathbb{N}$ genau dann bijektiv.
\end{document}
