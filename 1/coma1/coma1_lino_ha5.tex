\documentclass[12pt]{extarticle}
\usepackage{amsmath,mathtools,fontspec,amsthm,amssymb,amsfonts,fancyhdr,color,graphicx,}
\usepackage[margin=2.5cm]{geometry}
\usepackage[utf8]{inputenc}
\usepackage{xcharter-otf}
\usepackage[ngerman]{babel}
\usepackage[onehalfspacing]{setspace}
\usepackage{tikz}
\usetikzlibrary{calc,positioning,shapes.misc}
\renewcommand{\familydefault}{\sfdefault}
\usepackage{parskip}
\pagestyle{fancy}
\setlength{\parindent}{0pt}
\lhead{Yiwen Yang 466096, Qing Wang 458040, Yuchen Guo 480788}
\rhead{CoMa 5. Übungsblatt\\Tutor: Lino Mo 4-5.30pm}
\author{HA Gruppe 103: \\Yiwen Yang 466096\\Qing Wang 458040\\Yuchen Guo 480788\\Tutor: Lino Mo 4-5.30pm}
\title{5. Übungsblatt\\\small CoMa I WS 22/23}
\date{}
\begin{document}
\maketitle
\newpage
\textbf{3. Aufgabe}
\begin{itemize}
\item \((0, 2, 2, 2, 4, 4, 6)\)

  Ein Graph mit den obigen Graden existiert nicht.  Wir betrachten
  insbesondere der Knoten \(6\).  Dieser Knoten hat ingesamt \(6\) Kanten,
  aber es existiert nur \(5\) andere Knoten mit Kanten.  Also es gibt
  eine Kante, die keine Knoten verbindet.
\item \((2, 2, 1, 3, 2, 3)\)

  Ein Graph mit den obigen Graden existiert nicht.

  Falls es existert keine Kante zwischen den Knoten mit \(3\) Grade,
  dann es gibt nur eine Möglichkeit, nämlich ein Knoten \(3\)-Grades
  verbindet \((2,2,1)\) und die andere verbindet \((2,2,2)\) .  In diesem
  Fall gibt es eine Kante, die keine Knoten verbindet.
  \begin{center}
\begin{tikzpicture}
  \begin{scope}[every node/.style={circle,thick,draw}]
    \node (a2) at (2.5,2) {2};
    \node (b2) at (2.5,1) {2};
    \node (c1) at (2.5,0) {1};
    \node (d3) at (0,1.5) {3};
    \node (e2) at (2.5,-1) {2};
    \node (f3) at (0,-0.5) {3};
    \node (q)  at (0,-1.5) {?};
\end{scope}
\begin{scope}
  \path [-] (d3) edge (a2);
  \path [-] (d3) edge (b2);
  \path [-] (d3) edge (c1);
  \path [-] (f3) edge (e2);
  \path [-] (f3) edge (a2);
  \path [-] (f3) edge (b2);
  \path [-] (q) edge (e2);
\end{scope}
\end{tikzpicture}
  \end{center}
  Falls es existert eine Kante zwischen den Knoten mit \(3\) Grade, also
  prüfen wir, ob es ein Graph mit den Graden \((2, 2, 1, 2, 2, 2)\)
  existiert. Dieser Graph existiert ebenfalls nicht, denn, eine der
  Knoten \(2\)-Grades muss \((1)\) verbinden, also erhalten wir
  \((1,2,0,2,2,2)\).  Wiederholen wir noch vier Mal, dann erhalten wir
  \((0,0,0,0,0,1)\), also in diesem Fall gibt es eine Kante, die keine
  Knoten verbindet.
  \begin{center}
\begin{tikzpicture}
  \begin{scope}[every node/.style={circle,thick,draw}]
    \node (a2) at (2.5,2) {2};
    \node (b2) at (2.5,1) {2};
    \node (c1) at (2.5,0) {1};
    \node (d3) at (0,1.5) {3};
    \node (e2) at (2.5,-1) {2};
    \node (f3) at (0,-0.5) {3};
    \node (q)  at (0,-1.5) {?};
\end{scope}
\begin{scope}
  \path [-] (d3) edge (f3);
  \path [-] (a2) edge (b2);
  \path [-] (a2) edge[bend left=60] (c1);
  \path [-] (b2) edge (d3);
  \path [-] (d3) edge (e2);
  \path [-] (e2) edge (f3);
  \path [-] (q) edge (f3);
\end{scope}
\end{tikzpicture}
  \end{center}
\item \((2,5,1,2,2,2)\)
  \begin{center}
\begin{tikzpicture}
  \begin{scope}[every node/.style={circle,thick,draw}]
    \node (a5) at (0,0) {5};
    \node (b2) at (0,1) {2};
    \node (c2) at (1,0) {2};
    \node (d2) at (0.7,-1) {2};
    \node (e2) at (-0.7,-1) {2};
    \node (f1) at (-1,0) {1};
\end{scope}
\begin{scope}
\path [-] (a5) edge (b2);
\path [-] (a5) edge (c2);
\path [-] (a5) edge (d2);
\path [-] (a5) edge (e2);
\path [-] (a5) edge (f1);
\path [-] (b2) edge (c2);
\path [-] (d2) edge (e2);
\end{scope}
\end{tikzpicture}
  \end{center}
\end{itemize}

\textbf{4. Aufgabe, Teil (a)}
\begin{proof}
  Wir beweisen diese Aussage mittels Kontraposition.

  Es gelte die folgende Aussage:
  \begin{quote}
    Es gibt nur \(0\) oder \(1\) oder \(2\) Personen die sich alle
    gegenseitig die Hand geschüttelt haben, \textbf{und},
    es gibt nur \(0\) oder \(1\) oder \(2\) Personen, die sich alle
    gegenseitig nicht die Hand geschüttelt haben.
  \end{quote}

  Der Fall mit nur \(1\) Person macht kein Sinn, deshalb erhalten wir
  die folgende Aussage:
  \begin{quote}
    Es gibt nur \(0\) oder \(2\) (ein Paar) Personen die sich alle
    gegenseitig die Hand geschüttelt haben, \textbf{und}, es gibt nur
    \(0\) oder \(2\) (ein Paar) Personen, die sich alle gegenseitig nicht
    die Hand geschüttelt haben.
  \end{quote}

  Es gibt nun vier Fällen zu betrachten, nämlich, \(0\) und \(0\); \(0\) und
  \(2\); \(2\) und \(0\); \(2\) und \(2\).  In allen vier Fällen ist im
  Widerspruch zur Voraussetzung die Anzahl der Coma-Betreuer weniger
  als \(6\).
\end{proof}

\textbf{4. Aufgabe, Teil (b)}
\begin{proof}
  Wir beweisen diese Aussage mittels Kontraposition.

  Wir durchnummerieren die Betreuern mit natürliche Zahlen von \(1\) bis
  \(n\) und sei die Menge der Anzahl von Handschütteln
\begin{align*}
M:= \left\{ a_1, a_2, \ldots, a_n \right\}.
\end{align*}

Angenommen, es gilt für alle \(p,q \in \mathbb{N}_{\leq n}\) mit $p \neq
q$ dass \(a_p \neq a_q\).

O.B.d.A sei auch \(a_1 < a_2 < \ldots < a_n\).  Es gilt nun $a_n \leq n
- 1$, denn ein Person kann höchstens alle andere \(n-1\) Personen die
Hand schütteln.  Wegen Transitivität gilt \(a_{n-1} < n - 1\).  Wegen
\(a_{n-1} \in \mathbb{N}\) gilt \(a_{n-1} \leq n-2\), und so weiter.
Damit erhalten wir \(a_1 \leq 0\), also \(a_1 = 0\).

Wir bemerken, dass die Menge der Anzahl von Handschütteln ist dann
durch die im oben genannten Beschränkungen eindeutig bestimmt, also
wegen \(a_1=0\) und \(a_1 < a_2\) und \(a_2 \leq 1\) muss \(a_2 = 1\) sein.
Also die Menge ist \(\left\{ 0, 1, 2, 3, \ldots, n-1 \right\}\).  Aber
der letzter Betreuer hat eine Anzahl von \(n-1\), also alle andere
Personen die Hand geschüttelt, im Widerspruch zur \(a_1=0\).
\end{proof}
\end{document}
