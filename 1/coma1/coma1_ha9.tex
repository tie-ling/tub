%% page style
\documentclass[12pt]{extarticle}
\usepackage[margin=2cm]{geometry}
\usepackage{fancyhdr,parskip}
\pagestyle{fancy}
\usepackage[onehalfspacing]{setspace}
\setlength{\parindent}{0pt}
\lhead{\myAuthor}
\rhead{\mySubject \ \myHausaufgaben. Übungsblatt \\ \myTutor}
\renewcommand*\familydefault{\sfdefault} %% Only if the base font of the document is to be sans serif

%% language
\usepackage[utf8]{inputenc}
\usepackage{xcharter-otf}
\usepackage[ngerman]{babel}

%% default packages
\usepackage{amsmath,mathtools,fontspec,amsthm,amssymb,amsfonts,
  stmaryrd, % for the lightning symbol used in proof by contraction
  tikz,     % used to draw diagrams
}

%% metadata
\newcommand{\myAuthor}{Yiwen Yang 466096 | Qing Wang 458040 | Yuchen Guo 480788}
\newcommand{\myHausaufgaben}{9}
\newcommand{\mySubject}{CoMa}
\newcommand{\myTutor}{Lino}

%% custom commands
\newcommand{\aufgn}[1]{\textbf{Aufgabe #1.}}
\newcommand{\mg}[1]{\mathbb{#1}}
\newcommand{\lin}{\operatorname{lin}}
\newcommand{\beh}{\textit{Behauptung.}\ }
%% code listing options
\usepackage{listings}
\usepackage[lighttt]{lmodern}
\lstset{
  numbers=left,
  basicstyle=\ttfamily,
  keywordstyle=\ttfamily\bfseries,
}
\begin{document}
\aufgn{1.a}

\begin{lstlisting}[language=Python]
def insertionSort(A):
    n = len(A)
    for i in range(1, n):
        s = A[i]
        k = i
        while k > 0 and s < A[k-1]:
            A[k] = A[k-1]
            k -= 1
        A[k] = s
\end{lstlisting}

\beh Für den List \texttt{[8, 7, 6, 5, 4, 3, 2, 1]}
benötigt InsertionSort die größtmögliche Anzahl an
Vergleichen.

\begin{proof}
InsertionSort benötigt für diesen List die größtmögliche
Anzahl an Vergleichen, denn die List hat die
Eigenschaft, dass die Zahl an der \(i\)-ten Stelle kleiner als
alle Zahlen an der \(0 \le j \le i-1\)-ten Stelle und
die innere \texttt{while}-Schleife benötigt jeweils
\(i-1\)-mal Vergleichen zu terminieren.
\end{proof}

\aufgn{1.b}

\begin{lstlisting}[language=Python]
def mergeSort(A):
    if len(A) <= 1:
        return(A)
    else:
        return(merge(mergeSort(A[:len(A)//2]),
                     mergeSort(A[len(A)//2:])))
def merge(left, right):
    merged = []
    while left and right:
        if left[-1] < right[0]:
            return (left + right)
        if left[0] < right[0]:
            merged.append(left[0])
            left = left[1:]
        else:
            merged.append(right[0])
            right = right[1:]
    merged.extend(left + right)
    return merged
\end{lstlisting}

\beh Für den List \texttt{[8, 1, 7, 2, 6, 3, 5, 4]}
benötigt MergeSort die größtmögliche Anzahl an
Vergleichen.

\begin{proof}
MergeSort benötigt für diesen List die größtmögliche
Anzahl an Vergleichen, denn die Bedingung
\texttt{left[-1] < right[0]} ist niemals erfüllt und die
Laufzeit wächst daher quadratisch.
\end{proof}

\aufgn{1.c}

\begin{lstlisting}[language=Python]
def quickSort(A):
    if len(A) <= 1:
        return A
    else:
        pivot = A[0]
        B = []
        C = []
        for i in A[1:]:
            if i <= pivot:
                B.append(i)
            else:
                C.append(i)
        return quickSort(B) + [pivot] + quickSort(C)
\end{lstlisting}

\beh Für den List \texttt{[1, 2, 3, 4, 5, 6, 7, 8]}
benötigt QuickSort die größtmögliche Anzahl an
Vergleichen mit \texttt{pivot = A[0]}.

\begin{proof}
QuickSort benötigt für diesen List die größtmögliche
Anzahl an Vergleichen, denn die List ist bereits sortiert
und wir nehmen jedes Mal das erste Element der List als
Pivot-Element und der Algorithmus benötigt insgesamt
\begin{align*}
7+6+5+4+3+2+1 \text{ Mal Vergleichungen}
\end{align*}
um zu terminieren.  Die Länge der Liste reduziert jedes
Mal um Eins und Eins ist die kleinstmögliche Zahl, mit
der der Algorithmus noch terminieren kann.  Damit ist
die Anzahl der Vergleichen am größten.
\end{proof}

\newpage

\aufgn{2}

\begin{lstlisting}[language=Python]
def partition(A, left, right):
    # always take the last element in the list as pivot
    pivot = A[right]
    stored_index = left
    for i in range(left, right):
        if A[i] <= pivot:
            A[stored_index], A[i] = A[i], A[stored_index]
            stored_index += 1
    # the list is now partially sorted into two sides
    # one side all smaller than pivot, the other all larger than pivot
    # now move pivot to the place where it should be
    A[stored_index], A[right] = A[right], A[stored_index]
    # return the sorted position of pivot
    return stored_index


def kth_smallest(A, k, left, right):
    # first, partition once and check if we can get the result
    pivot_index = partition(A, left, right)
    if pivot_index - left == k - 1:
        return A[pivot_index]
    # else, search the left side of the list if k is in the left
    if pivot_index - left > k - 1:
        return kth_smallest(A, k, left, pivot_index - 1)
    # else, search the right side
    return kth_smallest(A, k - pivot_index + left - 1,
                        pivot_index + 1, right)

def kth_smallest_start(A, k):
    return kth_smallest(A, k, 0, len(A) - 1)

def find_median(A):
    return kth_smallest_start(A, (len(A) // 2) + 1)
\end{lstlisting}

\newpage
\aufgn{3.a}
\begin{lstlisting}[language=Python]
def find_min_max(A: list[int]):
    list_of_smaller_elements = []
    list_of_larger_elements = []
    for i in range(0, len(A) - 1, 2):
        if A[i] < A[i+1]:
            list_of_larger_elements.append(A[i+1])
            list_of_smaller_elements.append(A[i])
        else:
            list_of_larger_elements.append(A[i])
            list_of_smaller_elements.append(A[i+1])
    max_element = list_of_larger_elements[0]
    min_element = list_of_smaller_elements[0]
    if len(A) % 2 == 1:
        if A[-1] < min_element:
            list_of_smaller_elements.append(A[-1])
        else:
            list_of_larger_elements.append(A[-1])
    for i in list_of_larger_elements:
        if i > max_element:
            max_element = i
    for i in list_of_smaller_elements:
        if i < min_element:
            min_element = i
    return min_element, max_element
\end{lstlisting}
\aufgn{3.b}

\beh Der Algorithmus ist korrekt und die
Laufzeitfunktion \(g\) ist kleiner-gleich
$\Theta\left(3 \left \lfloor \frac{n}{2} \right \rfloor
\right)$.

\begin{proof}
  Im ersten Teil vergleichen wir paarweise die Elementen
  in der ganzen Liste, zu jedem Element \(x\) der Liste
  \texttt{list\_of\_smaller\_elements} existiert mindestens
  ein Element \(y\) der Liste
  \texttt{list\_of\_larger\_elements} mit der Eigenschaft
  \(x < y\).  Insgesamnt wird höchstens
  \(\left \lfloor \frac{n}{2} \right \rfloor + 1\)
  Vergleichen benötigt.

  Im nächsten Schritt wird jeweils das Minimum und das
  Maximum in der beiden Listen mittels einem naiven
  Algorithmus gefunden.  Dieses Schritt benötigt
  höchstens
  $\left \lfloor \frac{n}{2} \right \rfloor - 1 + \left
    \lfloor \frac{n}{2} \right \rfloor$ Vergleichungen.
  Insgesamt wird höchstens
  \(3 \left \lfloor \frac{n}{2} \right \rfloor\) Schritten
  benötigt.

  Die Korrektheit ist klar, denn das Maximum und das
  Minimum jeweils größer gleich und kleiner gleich alle
  Elemente und deshalb in der zwei Teil-Liste enthalten sind.
\end{proof}

\end{document}