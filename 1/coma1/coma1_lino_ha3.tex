\documentclass[12pt]{extarticle}
\usepackage{amsmath,mathtools,fontspec,amsthm,amssymb,amsfonts,fancyhdr,color,graphicx,}
\usepackage[margin=2.5cm]{geometry}
\usepackage[utf8]{inputenc}
\usepackage{xcharter-otf}
\usepackage[ngerman]{babel}
\usepackage[onehalfspacing]{setspace}
\usepackage{tikz}
\renewcommand{\familydefault}{\sfdefault}
\pagestyle{fancy}
\setlength{\parindent}{0pt}
\lhead{Yiwen Yang 466096, Qing Wang 458040, Yuchen Guo 480788}
\rhead{CoMa 3. Übungsblatt\\Tutor: Lino Mo 4-5.30pm}
\author{HA Gruppe 103: \\Yiwen Yang 466096\\Qing Wang 458040\\Yuchen Guo 480788\\Tutor: Lino Mo 4-5.30pm}
\title{3. Übungsblatt\\\small CoMa I WS 22/23}
\date{}
\begin{document}
\maketitle
\newpage

\textbf{H 1.a}

\textit{Behauptung.}  Aus \(b_1 > b_2\) folgt \(n_1 > n_2\).

\vspace{4mm}

Wir widerlegen diese Behauptung mit einem Beispiel.  Seien \(n_1=(1)_3\)
und \(n_2=(1)_2\).  Es gilt \(b_1 > b_2\) und im Widerspruch zur
Behauptung \(n_1=n_2=(1)_{10}\).  Damit ist die Behauptung widerlegt.

\vspace{4mm}

\textbf{H 1.b}

\textit{Behauptung.}  Aus \(n_1 > n_2\) folgt \(b_1 > b_2\).

\begin{proof}
  Es gilt
  $n_1=\sum_{i=0}^{l-1}{z_ib_1^i}=\sum_{i=0}^{l-1}{ \left(
      \frac{b_1}{b_2} \right)^iz_i b_2^i}$ und
  \(n_2=\sum_{i=0}^{l-1}{z_ib_2^i}\).  Aus \(n_1>n_2\) folgt, dass
  \(\left( \frac{b_1}{b_2} \right)^i > 1\) gilt.

  Es gilt, für alle \(0 < x \leq 1\) dass \(0 < x^n \leq 1\).  Deshalb
  muss \(\frac{b_1}{b_2}>1\) sein.  Wegen \(b_1, b_2 \in \mathbb{N}_{>0}\)
  gilt dann \(b_1>b_2\).

\end{proof}

\textbf{H 1.c}

\textit{Behauptung.}  Aus \(b_1|b_2\) folgt \(n_1|n_2\).

\begin{proof}
  Es gilt \(n_1=\sum_{i=0}^{l-1}{z_ib_1^i}\) und
  $n_2=\sum_{i=0}^{l-1}{z_ib_2^i}=\sum_{i=0}^{l-1}{
    \left( \frac{b_2}{b_1}  \right)^iz_i b_1^i}$.


  Aus \(b_1|b_2\) folgt, dass \(b_2 = kb_1\) mit \(k \in \mathbb{N}\).
  Daraus folgt, dass \(n_2=\sum_{i=0}^{l-1}{ k^i z_i b_1^i}\).

  Es gilt, \(n_2/n_1=\sum_{i=1}^{l-1}{k^i} \in \mathbb{N}\).  Damit gilt
  \(n_1|n_2\).
\end{proof}

\textbf{H 1.d}

\textit{Behauptung.}  Aus \(n_1|n_2\) folgt \(b_1|b_2\).

\vspace{4mm}

Wir widerlegen diese Behauptung mit einem Beispiel.  Seien \(n_1=(0)_2\)
und \(n_2=(0)_3\).  Es gilt \(n_1=n_2=(0)_{10}\) und daher \(n_1|n_2\).  Es
gilt aber nicht \(2|3\).

\vspace{4mm}

\textbf{H 1.e}

\textit{Behauptung.}  Es gilt \(\frac{n_1}{n_2}=\frac{b_1}{b_2}\).

\vspace{4mm}

Wir widerlegen diese Behauptung mit einem Beispiel.  Seien \(n_1=(1)_2\)
und \(n_2=(1)_3\).  Es gilt \(\frac{n_1}{n_2}=1\neq \frac{3}{2}\).

\vspace{4mm}

\textbf{H 1.f}

\textit{Behauptung.}  Sei \(b_1 > b_2\) und \(m \in \mathbb{N}\).  Dann
existiert ein \(N \in \mathbb{N}\), sodass folgendes gilt:  Ist \(l > N\)
und \(z_{l-1}\neq 0\), dann ist \(n_1 > mn_2\).

\vspace{4mm}

\begin{proof}
  Es gilt
  $n_1=\sum_{i=0}^{l-1}{z_ib_1^i}=\sum_{i=0}^{l-1}{ \left(
      \frac{b_1}{b_2} \right)^iz_i b_2^i}$ und
  \(n_2=\sum_{i=0}^{l-1}{z_ib_2^i}\).  Wegen \(b_1>b_2\) gilt
  \(\frac{b_1}{b_2}>1\).
  Seien \(m=1\), \(N=2\) und \(l=3\).  Dann gilt
\begin{align*}
n_1&=z_2\left(\frac{b_1}{b_2} \right)^2 b_2^2 +
     z_1\left(\frac{b_1}{b_2} \right) b_2 + z_0\\\\
  n_2&=z_2 b_2^2 + z_1 b_2 + z_0\\
     &=mn_2
\end{align*}
Es gilt wegen \(\frac{b_1}{b_2}>1\) dass \(n_1>mn_2\).
\end{proof}
\vspace{4mm}

\newpage

\textbf{H 2.a}

\vspace{4mm}

Siehe Aufgabe (H 1.b).

\vspace{4mm}

\textbf{H 2.b}

\vspace{4mm}

Siehe Aufgabe (H 1.a).

\vspace{4mm}

\textbf{H 2.c}

\vspace{4mm}

Aus der Voraussetzung folgt,
\begin{align*}
  (b_1^l-n_1) \mod b_1^l&=(b_2^l-n_2) \mod b_2^l.
\end{align*}
???

\vspace{4mm}

\textbf{H 3.a}

\vspace{4mm}

\vspace{4mm}

\textbf{H 3.b.i}

\vspace{4mm}

\vspace{4mm}

\textbf{H 3.b.ii}

\vspace{4mm}

\vspace{4mm}

\newpage

\textbf{H 4.a}

\begin{proof}
  Sei der Code an der \(k\)-ten Stelle nicht erkennbar.

  Es gilt $\sum_{n=0}^6{a_{2n+1}}+3\sum_{n=1}^6{a_{2n}} \equiv 0
  \pmod{10}$.  Es gilt auch, \(1 \leq a_k \leq 9\).

\vspace{4mm}
  \begin{itemize}
  \item Falls \(k\) gerade.

    Dann gilt
\begin{align*}
    \sum_{n=0}^6{a_{2n+1}}+3\sum_{n=1}^{k-1}{a_{2n}}+3\sum_{n=k+1}^{6}{a_{2n}}+3a_k
  \equiv 0 \pmod{10}\\\\
  3a_k=10p-\left( \sum_{n=0}^6{a_{2n+1}}+3\sum_{n=1}^{k-1}{a_{2n}}+3\sum_{n=k+1}^{6}{a_{2n}} \right)
\end{align*}
mit \(p \in \mathbb{N}\).  Wegen \(3a_k\) muss die Rechtsseite der
Gleichung durch \(3\) teilbar sein, die mögliche Lösungen sind also
\(\left\{ 3, 6, 9, 12, 15, 18, 21, 24, 27 \right\}\).  Wir bemerken,
dass alle mögliche Lösungen geteilt durch \(10\) einen eindeutigen Rest
besitzt, nämlich \(\left\{ 7, 4, 1, 8, 5, 2, 9, 6, 3 \right\}\).  Also,
die Werte von \(a_k\) und \(p\) ist durch den Rest von der größen Klammern
geteilt durch \(10\) eindeutig bestimmt.

  \item Falls \(k\) ungerade.

    Dann gilt
\begin{align*}
\sum_{n=0}^{k-1}{a_{2n+1}}+\sum_{n=k+1}^6{a_{2n+1}}+3\sum_{n=1}^6{a_{2n}}+a_k
  \equiv 0  \pmod{10}\\\\
  a_k=10p-\left( \sum_{n=0}^{k-1}{a_{2n+1}}+\sum_{n=k+1}^6{a_{2n+1}}+3\sum_{n=1}^6{a_{2n}} \right).
\end{align*}
mit \(p \in \mathbb{N}\).  Wegen \(1 \leq a_k \leq 9\) ist dann \(p\)
eindeutig bestimmt.  Denn, für \(p-1\) oder \(p+1\) muss entweder $a_k >
9$ oder \(a_k<1\) gelten.

  \end{itemize}
\end{proof}

\vspace{4mm}

\newpage

\textbf{H 4.b}

\begin{proof}
  Sei der Code an der \(k\)-ten und \(k+1\) Stelle vertauscht.

  Es gilt $\sum_{n=0}^6{a_{2n+1}}+3\sum_{n=1}^6{a_{2n}} \equiv 0
  \pmod{10}$.  Es gilt auch, \(1 \leq a_k \leq 9\).

Jetzt werden \(a_k\) und \(a_{k+1}\) vertauscht.  Wir zeigen, dass
\(3a_k+a_{k+1} \neq a_k+3a_{k+1}+10n\), also
\(4|a_k-a_{k+1}| \neq 10n\) für alle \(n \in \mathbb{Z}\)
gilt.

Nun betrachten wir die Fälle wenn
\(|a_k-a_{k+1}|\) gleich \(\left\{ 1,2,3,4,5,6,7,8 \right\}\) ist.

Sei \(|a_k-a_{k+1}|=1\), dann gilt \(4|a_k-a_{k+1}| =4 \neq 10n\).

Sei \(|a_k-a_{k+1}|=2\), dann gilt \(4|a_k-a_{k+1}| =8 \neq 10n\).

Sei \(|a_k-a_{k+1}|=3\), dann gilt \(4|a_k-a_{k+1}| =12 \neq 10n\).

Sei \(|a_k-a_{k+1}|=4\), dann gilt \(4|a_k-a_{k+1}| =16 \neq 10n\).

Sei \(|a_k-a_{k+1}|=5\), dann gilt $4|a_k-a_{k+1}| =20 =
10n$ mit \(n=2\).

Sei \(|a_k-a_{k+1}|=6\), dann gilt \(4|a_k-a_{k+1}| =24 \neq 10n\).

Sei \(|a_k-a_{k+1}|=7\), dann gilt \(4|a_k-a_{k+1}| =28 \neq 10n\).

Sei \(|a_k-a_{k+1}|=8\), dann gilt \(4|a_k-a_{k+1}| =32 \neq 10n\).

Damit haben wir gezeigt, dass der Code kann niemals korrekt sein, wenn
zwei ungleichen, aufeinanderfolgenden Ziffern vertauscht werden.
Außer deren Differenz ist gleich \(5\).

\end{proof}
\end{document}
