\documentclass[12pt]{extarticle}
\usepackage{amsmath,mathtools,fontspec,amsthm,amssymb,amsfonts,fancyhdr,color,graphicx,}
\usepackage[margin=2.5cm]{geometry}
\usepackage[utf8]{inputenc}
\usepackage{xcharter-otf}
\usepackage[ngerman]{babel}
\usepackage[onehalfspacing]{setspace}
\usepackage{tikz}
\renewcommand{\familydefault}{\sfdefault}
\pagestyle{fancy}
\setlength{\parindent}{0pt}
\lhead{Yiwen Yang 466096, Qing Wang 458040, Yuchen Guo 480788}
\rhead{CoMa 2. Übungsblatt\\Tutor: Tino Mo 4-5.30pm}
\author{HA Gruppe 103: \\Yiwen Yang 466096\\Qing Wang 458040\\Yuchen Guo 480788\\Tutor: Tino Mo 4-5.30pm}
\title{2. Übungsblatt\\\small CoMa I WS 22/23}
\date{}
\begin{document}
\maketitle
\newpage
\textbf{Aufgabe 1 (a)}
\begin{proof}
Weil \(f, g \in O(h)\), dann gilt von Definition
\begin{align*}
  \exists \alpha_1>0, \exists n_1 \in \mathbb{N}, \forall
  n \geq n_1&: f(n) \leq \alpha_1h(n)\\
  \exists \alpha_2>0, \exists n_2 \in \mathbb{N}, \forall
  n \geq n_2&: g(n) \leq \alpha_2h(n)
\end{align*}

Sei \(n_3=\max \left\{ n_1, n_2 \right\}\), dann gilt
\begin{align*}
  \forall n \geq n_3&: f(n) \leq \alpha_1h(n)\\
  \forall n \geq n_3&: g(n) \leq \alpha_2h(n)
\end{align*}

Multiplizieren wir die Ungleichungen jeweils mit $a, b \in
\mathbb{R}_{>0}$ und addieren wir diese, erhalten wir
\begin{align*}
  \forall n \geq n_3&:
  a\cdot f(n) + b\cdot g(n) \leq (\alpha_1\cdot a +\alpha_2 \cdot b )\cdot h(n)
\end{align*}
Setzen wir \(\alpha=\alpha_1\cdot a +\alpha_2 \cdot b\), dann erhalten
wir
\begin{align*}
  \exists \alpha > 0, \exists n_3 \in \mathbb{N},
  \forall n \geq n_3&:
  a\cdot f(n) + b\cdot g(n) \leq \alpha \cdot h(n)
\end{align*}

Damit werde die Beziehung \(a \cdot f + b \cdot g \in O(h)\) gezeigt.
\end{proof}

\textbf{Aufgabe 1 (b)}

\textit{Behauptung.}  Aus \(f \in \Omega(g)\) und \(g \in \Omega(h)\)
folgt \(f \in \Omega(h)\).

Diese Behauptung ist wahr.
\begin{proof}

  Sei
\begin{align*}
  f \in \Omega(g) &:= \left\{ f: \mathbb{N} \rightarrow \mathbb{R}_{\geq
    0}|\exists \beta_1 >0, \exists n_1 \in \mathbb{N},
  \forall n \geq n_1: f(n) \geq \beta_1 g(n)\right\}\\
  g \in \Omega(h) &:= \left\{ g: \mathbb{N} \rightarrow \mathbb{R}_{\geq
    0}|\exists \beta_2 >0, \exists n_2 \in \mathbb{N},
  \forall n \geq n_2: g(n) \geq \beta_2 h(n)\right\}\\
\end{align*}
Dann gilt
\begin{align*}
  f \in \Omega(h) &:= \left\{
                    f: \mathbb{N} \rightarrow \mathbb{R}_{\geq
                    0}|    \forall n \geq \max \left\{ n_1, n_2 \right\}:
                    f(n) \geq \beta_1 g(n) \geq \beta_1 \cdot
                    \beta_2 h(n)\right\}\\
\end{align*}
\end{proof}

\textbf{Aufgabe 1 (c)}

Siehe unten.


\textbf{Aufgabe 2 (a)}

\begin{proof}
  Wir bemühen einen Beweis mittels vollständigen Induktion.
  \begin{itemize}
  \item Induktionsanfang \(n=0\)

    Wenn es keine Gerade gibt, bleibt die Ebene ganz.  Also es gibt
    genau ein Gebiet.

    Andereseits gilt \(\frac{0^2+0+2}{2}=1\).  Daher gilt die Aussage
    für \(n=0\).
  \item Induktionsschritt \(n \rightarrow n+1\)

    Es gilt die Aussage, dass man mit \(n \in \mathbb{N}\) Geraden die
    Ebene in höchstens \(\frac{n^2+n+2}{2}\) Gebiete zerlegen kann.

    Ohne Beschränkung der Allgemeinheit betrachten wir den Fall, wo
    die Ebene mit \(n\) Geraden genau in \(\frac{n^2+n+2}{2}\) Gebiete
    geteilt wird.

    Nun fügen wir eine \((n+1)\)-ten Gerade hinzu, sodass die
    \((n+1)\)-ten Gerade alle \(n\) Geraden schneidet, und alle
    Schnittpunkte der \((n+1)\)-ten Gerade von bereits existierenden
    Schnittpunkte verschieden sind.

    Weil jedes Gebiet konvex ist, kann am meisten \(n+1\) neue Gebiete
    von der Teilung durch \((n+1)\)-ten Gerade entstehen.  Damit gilt,


\begin{align*}
f(n+1)=f(n)+n+1=\frac{n^2+n+2}{2}+n+1=\frac{(n+1)(n+2)+2}{2}
\end{align*}
  \end{itemize}
\end{proof}

\textbf{Aufgabe 2 (b)}

\begin{proof}
  Wir beweisen die Aussage, dass es unendlich viele Primzahlen gibt,
  mittels Widerspruchsbeweis.

  Angenommen, es gibt endlich viele Primzahlen:
  \(M= \left\{ p_1, p_2, p_3, \cdots, p_n \right\}\).

  Wir betrachten die Zahl $a=p_1\cdot p_2 \cdot p_3 \cdots \cdot
  p_n+1$.

  Weil \(a\) größer als alle Primzahlen ist, muss \(a\) nicht prim sein.
  Also, \(p \notin M\).

  Es gilt aber, dass \(a\) geteilt durch alle Primzahlen \(p \in M\) immer
  einen Rest von \(1\) hat.  Es gilt, \(a\) ist eine Primzahl.  Diese ist
  im Widerspruch zur Voraussetzung, dass die Menge \(M\) alle Primzahl enthält.
\end{proof}

\textbf{Aufgabe 3 (a)}
\begin{proof}
  Wir bemühen einen Beweis mittels vollständigen Induktion.

  \begin{itemize}
  \item Induktionsanfang \(n=2\)
\begin{align*}
\frac{1}{1 \cdot 2} = \frac{2-1}{2} = \frac{1}{2}
\end{align*}
\item Induktionsschritt \(n \rightarrow n+1\)
\begin{align*}
\sum_{k=2}^{n+1}{\frac{1}{(k-1)\cdot
  k}}&=\sum_{k=2}^n{\frac{1}{(k-1)\cdot k}}+\frac{1}{(n+1)\cdot n}\\
     &= \frac{n-1}{n}+\frac{1}{(n+1)n}\tag*{Induktionsvoraussetzung}\\
  &=\frac{n}{n+1}
\end{align*}
  \end{itemize}
\end{proof}


\textbf{Aufgabe 3 (b)}
\begin{proof}

  Seien \(a, b, c\) die drei Seite einer Dreieck.  Wir beweisen, dass
  die Gleichung
\begin{align*}
a^2-m^2=b^2-(c-m)^2
\end{align*}

Eine Lösung mit \(m \in \mathbb{R}_{\geq 0}\) hat.

Es gilt \(m=\frac{a^2+c^2-b^2}{2c}\).  Zu zeigen: \(a^2+c^2-b^2 \geq 0\).
\end{proof}
\newpage
\textbf{Aufgabe 4 (a)}
\begin{align*}
  &(101001)_2+(100110)_2 \\
  &= (1 \times 2^5+0 \times 2^4+1 \times 2^3+0 \times 2^2+0 \times
    2^1+1 \times 2^0)_{10} \\
  &+ (1 \times 2^5+0 \times 2^4+0 \times 2^3+1 \times 2^2+1 \times
    2^1+0 \times 2^0)_{10} \\
  &= (1 \times 2^6+0 \times 2^5+0 \times 2^4+1 \times 2^3+1 \times 2^2+1 \times
    2^1+1 \times 2^0)_{10} \\
  &= (1001111)_2
\end{align*}

\textbf{Aufgabe 4 (b)}
\begin{align*}
  &(2102)_3+(2021)_3\\
  &=(2 \times 3^3 + 1 \times 3^2 + 0 \times 3^1 + 2 \times 3^0)_{10}\\
  &+(2 \times 3^3 + 0 \times 3^2 + 2 \times 3^1 + 1 \times 3^0)_{10}\\
  &=(1 \times 3^4 +1 \times 3^3 + 2 \times 3^2 + 0 \times 3^1 + 0
    \times 3^0)_{10}\\
  &=(11200)_3
\end{align*}


\textbf{Aufgabe 4 (c)}
\begin{align*}
  &(7541)_8-(2005)_8\\
  &=(7 \times 8^3 + 5 \times 8^2 + 4 \times 8^1 + 1 \times 8^0)_{10}\\
  &-(2 \times 8^3 + 0 \times 8^2 + 0 \times 8^1 + 5 \times 8^0)_{10}\\
  &=(5 \times 8^3 + 5 \times 8^2 + 3 \times 8^1 + 4 \times 8^0)_{10}\\
  &=(5534)_8
\end{align*}


\textbf{Aufgabe 4 (d)}
\begin{align*}
  &(AB2E)_{16}-(1001)_{16}\\
  &=(10 \times 16^3 + 11 \times 16^2 + 2 \times 16^1 + 14 \times 16^0)_{10}\\
  &-(1 \times 16^3 + 0 \times 16^2 + 0 \times 16^1 + 1 \times 16^0)_{10}\\
  &=(9 \times 16^3 + 11 \times 16^2 + 2 \times 16^1 + 13 \times 16^0)_{10}\\
  &=(9B2D)_{16}
\end{align*}

\textbf{Aufgabe 4 (e)}
\begin{align*}
  &(11101)_{2}\cdot (101001)_{2}\\
  &=(1 \times 2^4 + 1 \times 2^3 + 1 \times 2^2 + 0 \times 2^1 + 1 \times 2^0)_{10}\\
  &\cdot(1 \times 2^5+ 0 \times 2^4+ 1 \times 2^3 + 0 \times 2^2 + 0 \times 2^1 + 1 \times 2^0)_{10}\\
  &=(29)_{10} \cdot (41)_{10}\\
  &=(1189)_{10}\\
  &=(1 \times 2^{10} + 1 \times 2^7 + 1 \times 2^5 + 1 \times 2^2 + 1
    \times 2^0)_{10}\\
  &=(10010100101)_2
\end{align*}
\end{document}
