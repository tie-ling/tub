%% page style
\documentclass[12pt]{extarticle}
\usepackage[margin=2cm]{geometry}
\usepackage{fancyhdr,parskip}
\pagestyle{fancy}
\usepackage[onehalfspacing]{setspace}
\setlength{\parindent}{0pt}
\lhead{\myAuthor}
\rhead{\mySubject \ \myHausaufgaben. Übungsblatt \\ \myTutor}
\renewcommand*\familydefault{\sfdefault} %% Only if the base font of the document is to be sans serif

%% language
\usepackage[utf8]{inputenc}
\usepackage{xcharter-otf}
\usepackage[ngerman]{babel}

%% default packages
\usepackage{amsmath,mathtools,fontspec,amsthm,amssymb,amsfonts,
  stmaryrd, % for the lightning symbol used in proof by contraction
  tikz,     % used to draw diagrams
}

%% metadata
\newcommand{\myAuthor}{Yiwen Yang 466096 | Qing Wang 458040 | Yuchen Guo 480788}
\newcommand{\myHausaufgaben}{7}
\newcommand{\mySubject}{CoMa}
\newcommand{\myTutor}{Lino}

%% custom commands
\newcommand{\aufgn}[1]{\textbf{Aufgabe #1.}}
\newcommand{\mg}[1]{\mathbb{#1}}
\newcommand{\lin}{\operatorname{lin}}
\newcommand{\beh}{\textit{Behauptung.}\ }
%% code listing options
\usepackage{listings}
\usepackage[lighttt]{lmodern}
\lstset{
  numbers=left,
  basicstyle=\ttfamily,
  keywordstyle=\ttfamily\bfseries,
}
\begin{document}

\aufgn{3.1}

\begin{figure}
\centering
\begin{tikzpicture}
  \begin{scope}[every node/.style={circle,thick,draw}]
    \node (1) at (0,1) {1};
    \node (2) at (0,0) {2};
    \node (3) at (0,-1) {3};
    \node (4) at (2,1) {4};
    \node (5) at (2,0) {5};
    \node (6) at (2,-1) {6};
\end{scope}
\begin{scope}
  \path [-] (1) edge[bend right=60] (2);
  \path [-] (2) edge[bend right=60] (3);
  \path [-] (3) edge (4);
  \path [-] (2) edge (5);
  \path [-] (4) edge[bend left=60] (5);
  \path [-] (5) edge[bend left=60] (6);
  \path [-] (4) edge[bend left=90] (6);
\end{scope}
\end{tikzpicture}
\caption{Original.}
\end{figure}

\begin{figure}
\centering
\begin{tikzpicture}
  \begin{scope}[every node/.style={circle,thick,draw}]
    \node (1) at (0,1) {1};
    \node (2) at (0,0) {2};
    \node (3) at (0,-1) {3};
    \node (4) at (2,1) {4};
    \node (6) at (2,-1) {6};
\end{scope}
\begin{scope}
  \path [-] (1) edge[bend right=60] (2);
  \path [-] (2) edge[bend right=60] (3);
  \path [-] (3) edge (4);
  \path [-] (2) edge (4);
  \path [-] (4) edge[bend left=60] (4);
  \path [-] (4) edge[bend left=60] (6);
  \path [-] (4) edge[bend left=90] (6);
\end{scope}
\end{tikzpicture}
\caption{Kontraktion.}
\end{figure}

\[
  \begin{pmatrix}
0 & 1 & 0 & 0 & 0 \\
    1 & 0 & 1 & 1 & 0\\

    0 & 1 & 0 & 1 & 0\\
    0 & 1 & 1 & 0 & 2\\
    0 & 0 & 0 & 2 & 0
  \end{pmatrix}
\]

\aufgn{3.2}

\(G\) zusammenhängend \(\implies\) \(G_{/e}\) zusammenhängend.
\begin{proof}
  Angenommen, \(G_{/e}\) ist nicht zusammenhängend.  Dann existiert
  mindestens ein \(v \in E_{/e}\) mit \(\delta(v) = \emptyset\).
  \begin{itemize}
  \item   Falls \(v \notin \Psi(e)\), dann ist \(G\) nicht
    zusammenhängend.
  \item   Falls \(v \in \Psi(e)\), dann gilt
  $(\Psi(e_1) \cup \Psi(e_2)) \setminus \left\{ e \right\} =
  \emptyset$.  Das heißt, die Knoten \(e_1, e_2\) besitzt keine Kante
  anders als \(e\).  Es gibt kein Weg zwischen \(e_1, e_2\) und alle
  andere Knoten in der Knotenmenge.  Dann ist \(G\) nicht
  zusammenhängend.
  \end{itemize}
\end{proof}
\(G_{/e}\) zusammenhängend \(\implies\) \(G\) zusammenhängend.

\begin{proof}
  Angenommen, \(G\) ist nicht zusammenhängend.  Es existiert ein $a, b
  \in V$ sodass es kein a-b Weg gibt.  Nach Kontraktion gibt es immer
  keinen a-b Weg.  Deshalb ist \(G_{/e}\) nicht zusammenhängend.
\end{proof}

\aufgn{3.3}

\(G\) zusammenhängend mit \(n\) Kanten \(\implies\) nach \(n\)-ten
Kantenkontraktion gibt es nur eine Knoten.

\begin{proof}
  Dann gibt es für alle Knoten \(a, b \in V\) ein a-b Weg.  Daraus folgt
  die Behauptung.
\end{proof}

Nach \(n\)-ten Kantenkontraktion gibt es nur eine Knoten. \(\implies\) \(G\)
zusammenhängend mit \(n\) Kanten.

\begin{proof}
  Angenommen, \(G\) ist zusammenhängend aber lässt sich nicht auf eine
  Knoten kontraktieren.  Dann ist \(G\) im Widerspruch zur Voraussetzung
  nicht zusammenhängend.
\end{proof}
\newpage
\aufgn{4}
\begin{lstlisting}[language=Python]
def MinMaxSort(A, sta=0, end=-1):
    if end < 0:
        end += len(A)
    if sta >= end:
        return A
    mini = maxi = A[sta]
    mi = ma = sta

    for i in range(sta+1, end+1):
        if A[i] > maxi:
            maxi, ma = A[i], i
        if A[i] < mini:
            mini, mi = A[i], i

    A[mi], A[sta] = A[sta], A[mi]

    if (ma != sta):
        A[ma], A[end] = A[end], A[ma]
    else:
        A[mi], A[end] = A[end], A[mi]
    return MinMaxSort(A, sta+1, end-1)
\end{lstlisting}
\aufgn{4.1}

\beh  Die Funktion ist korrekt.
\begin{proof}
  Wir zeigen, dass diese Funktion korrekt ist, mittels Induktion.

  \textit{Induktionsanfang.}  Eingabe ist List \texttt{A}.  Beim
  ersten Aufruf dieser Funktion gilt \texttt{sta=0} und
  \texttt{end=len(A)-1}.  Innerhalb der \texttt{for}-Schleife wird
  jeweils
  \begin{itemize}
  \item das größte Element der Variable \texttt{maxi},
  \item der Index des größte Element der Variable \texttt{ma},
  \item das kleinste Element der Variable \texttt{mini},
  \item der Index des kleinste Element der Variable \texttt{mi},
  \end{itemize}
  zugewiesen.  Anschließend werden das Element am Anfang dieses Lists
  mit dem kleinsten Element dieses ganzen Lists und das Element am
  Ende dieses Lists mit dem größten Element dieses ganzen Lists
  getauscht.  So dass es gilt
  \begin{itemize}
  \item \texttt{L[0] \(\leq\) L[i]} für alle \texttt{i} \(\leq\)
    \texttt{len[L] - 1}, und
  \item \texttt{L[-1] \(\geq\) L[i]} für alle \texttt{i} \(\leq\)
    \texttt{len[L] - 1}.
  \end{itemize}

  \textit{Induktionsannahme.}  Beim \(n\)-ten rekursiven Aufruf dieser
  Funktion, also \texttt{MinMaxSort(L, n, len(L)-n)} gilt, dass
  jeweils die ersten und die letzten \(n\) Elemente dieses Lists korrekt
  sortiert sind.

  \textit{Induktionsschritt.}  Als Eingabe \texttt{(L, n, len(L)-n)}
  \(\rightarrow\) \texttt{(L, n+1, len(L)-n-1)}.  Dann gilt wegen
  Induktionsannahme dass
  \begin{itemize}
  \item \texttt{L[n+1]} \(\geq\) \texttt{L[i]} für alle
  \(0\leq\) \texttt{i} \(\leq\) \texttt{n},
\item \texttt{L[len(L)-n-1]} \(\leq\) \texttt{L[i]} für alle
  \texttt{len(L)-n} \(\leq\) \texttt{i} \(\leq\) \texttt{len(L)}.
\end{itemize}

Analog zum ersten Aufruf gilt auch \texttt{L[len(L)-n-1]}
größer-gleich und \texttt{L[n+1]} kleiner-gleich alle Elemente
inzwischen.  Deshalb werden die Elemente korrekt sortiert.
\end{proof}

\aufgn{4.2}

\beh Sei Laufzeitfunktion dieses Funktions \(g\).  Es gilt \(g \in \Theta(n^2)\).

\begin{proof}
  Wir bemerken, dass nur die \texttt{for}-Schleife und deren rekursive
  Aufrufe eine nicht konstante Laufzeit haben.

  Die Funktion terminiert genau dann, wenn alle Elemente sortiert
  sind. O.B.d.A, sei der Lange des Lists gerade, dann benötigt die
  Funktion
\begin{align*}
\underbrace{n + (n - 2) + (n - 4) + (n - 6) + \ldots + 0}_{n/2\text{ Elemente}}
\end{align*}
Schritte zu terminieren.  Die Laufzeitfunktion ist dann
\begin{align*}
\frac{n^2}{2}-n^2-n.
\end{align*}

Damit folgt  \(g \in \Theta(n^2)\).
\end{proof}

\end{document}